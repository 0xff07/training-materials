\section{Kernel optimizations}

\begin{frame}[fragile]
\frametitle{Measure - Kernel initialization functions}
To understand which are the longest initializations, pass
\code{initcall_debug} as an argument on your kernel command line.
\begin{block}{}
\tiny
\begin{verbatim}
...
[    3.750000] calling  ov2640_i2c_driver_init+0x0/0x10 @ 1
[    3.760000] initcall ov2640_i2c_driver_init+0x0/0x10 returned 0 after 544 usecs
[    3.760000] calling  at91sam9x5_video_init+0x0/0x14 @ 1
[    3.760000] at91sam9x5-video f0030340.lcdheo1: video device registered @ 0xe0d3e340, irq = 24
[    3.770000] initcall at91sam9x5_video_init+0x0/0x14 returned 0 after 10388 usecs
[    3.770000] calling  gspca_init+0x0/0x18 @ 1
[    3.770000] gspca_main: v2.14.0 registered
[    3.770000] initcall gspca_init+0x0/0x18 returned 0 after 3966 usecs
...
\end{verbatim}
\end{block}
It is probably a good idea to increase the log buffer with
\code{CONFIG_LOG_BUF_SHIFT} in your kernel configuration. You will
also need \code{CONFIG_PRINTK_TIME} and \code{CONFIG_KALLSYMS}.
\end{frame}

\begin{frame}
\frametitle{Reducing the size}
First, we focus on reducing the size without removing features
\begin{itemize}
	\item The main mechanism is to use kernel modules
	\item Compile everything that is not needed at boot time as a
		module
	\item Two benefits: the kernel will be smaller and load faster and
		less initialization code will get executed
	\item Remove features that are not used by userland:
		\code{CONFIG_KALLSYMS}, \code{CONFIG_DEBUG_FS},
		\code{CONFIG_BUG}
	\item use features designed for embedded systems:
		\code{CONFIG_SLOB}, \code{CONFIG_EMBEDDED}
\end{itemize}
\end{frame}

\begin{frame}
\frametitle{Results}
Before 8.54s
\begin{center}
    \includegraphics[width=\textwidth]{slides/boottime-init-scripts2/timechart-initramfs.pdf}
\end{center}
After:
\begin{center}
    \includegraphics[width=\textwidth]{slides/boottime-kernel/timechart-modules.pdf}
\end{center}
Total: 6.45s.
\end{frame}

\begin{frame}
\frametitle{Kernel Compression}
Depending on the balance between your storage reading speed and your
CPU power to decompress the kernel, you will need to benchmark
different compression algorithms.

Before (gzip): 6.45s.
\begin{center}
    \includegraphics[width=\textwidth]{slides/boottime-kernel/timechart-modules.pdf}
\end{center}
After (LZO):
\begin{center}
    \includegraphics[width=\textwidth]{slides/boottime-kernel/timechart-lzo.pdf}
\end{center}
Total: 6.46s.
Conclusion: don't use LZO for now.
\end{frame}

\begin{frame}
\frametitle{Initialization}
If you can't compile a feature as a module (e.g. networking or block
subsystem), try \code{deferred_initcalls}. You kernel will not shrink
but it won't execute some initialization. Once your critical
application is ready, you can start the remaining initcalls. See
\url{http://elinux.org/Deferred_Initcalls}
\end{frame}

\begin{frame}[fragile]
\frametitle{Tuning the command line}
\begin{itemize}
	\item At each boot, the Linux kernel calibrates a delay loop (for
		the udelay function). This measures a number of loops per
		jiffy ({\em lpj}) value. You just need to measure this once! Find
		the \code{lpj} value in the kernel boot messages:
\begin{block}{}
\small
\begin{verbatim}
Calibrating delay loop... 262.96 BogoMIPS (lpj=1314816)
\end{verbatim}
\end{block}
		Now, you can use the \code{lpj=<value>} argument. This saves
		around 180 ms on ARM.
	\item The console output is actually taking a lot of time. You
		probably don't need it in production. It can be disabled by
		passing the \code{quiet} argument on the kernel command line.
		You will still be able to use \code{dmesg} to get the
		messages.
\end{itemize}
\end{frame}

\begin{frame}
  \frametitle{Multiprocessor support (SMP)}
  \begin{itemize}
	  \item SMP is quite slow to initialize
	  \item UP systems may be faster to boot
	  \item What you can try is to hotplug the other cores after your critical application has started
  \end{itemize}
\end{frame}

\setuplabframe
{Reduce kernel boot time}
{
\begin{itemize}
\item Recompile the kernel, switching to an initramfs
\item Use \code{initcall_debug} to find the biggest
      time consumers
\item Reduce the number of modules
\item Tune kernel command line parameters
\end{itemize}
}

\begin{frame}
\frametitle{Kernel Optimization results}
Before (gzip): 6.45s.
\begin{center}
    \includegraphics[width=\textwidth]{slides/boottime-kernel/timechart-modules.pdf}
\end{center}
After:
\begin{center}
    \includegraphics[width=\textwidth]{slides/boottime-kernel/timechart-final.pdf}
\end{center}
Total: 5.77s. Without losing any functionality!
\end{frame}

\begin{frame}
\frametitle{Kernel: last milliseconds (1)}
To shave off the last milliseconds, you will probably want to remove
unnecessary features:
\begin{itemize}
        \item \code{CONFIG_PRINTK=n} will have the same effect as the
\code{quiet}
                argument but you won't have any access to kernel
messages.
        \item Try \code{CONFIG_CC_OPTIMIZE_FOR_SIZE=y}. This will have
an
                impact on the performance, you will have to benchmark.
        \item Try to initialize less RAM by passing a \code{mem} value
on the
                kernel command line. The less RAM you need to
initialize, the
                faster you will boot.
        \item Module loading/unloading
        \item Block layer
        \item Network stack
        \item USB stack
\end{itemize}
\end{frame}

\begin{frame}
\frametitle{Kernel (last milliseconds (2)}
\begin{itemize}
        \item a.out format
        \item power management
        \item \code{CONFIG_SYSFS_DEPRECATED}
        \item Input: keyboards / mice / touchscreens
        \item \code{CONFIG_LEGACY_PTY_COUNT} or the
                \code{pty.legacy_count} kernel parameter
\end{itemize}
\end{frame}

