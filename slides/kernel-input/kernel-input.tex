\subsection{The input subsystem}

\begin{frame}{What is input subsystem?}
  \begin{itemize}
  \item The input subsystem takes care of all the input events coming
    from the user.
  \item Initially written to support the USB {\em HID}(Human Interface
    Device) devices, it quickly grew up to handle all kind of inputs
    (using USB or not): keyboard, mice, joystick, touchscreen, etc.
  \item The input subsystem is split in two parts:
    \begin{itemize}
    \item {\bf Device drivers}: they talk to the hardware (for example
      via USB), and provide events (keystrokes, mouse movements,
      touchscreen coordinates) to the input module
    \item {\bf Event handlers}: they get events from input and pass them
      where needed via various interfaces (most of the time through
      \code{evdev})
    \end{itemize}
  \item In userspace it is usually used by the the graphic stack such
    as {\em X.Org}, {\em Wayland} or {\em Android}.
  \end{itemize}
\end{frame}

\begin{frame}{input subsystem overview}

TODO: [nice diagram]

\end{frame}

\begin{frame}
  \frametitle{input subsystem overview}
  \begin{itemize}
  \item Kernel option \code{CONFIG_INPUT}
    \begin{itemize}
    \item \code{menuconfig INPUT}
      \begin{itemize}
      \item \code{tristate "Generic input layer (needed for keyboard, mouse, ...)"}
      \end{itemize}
    \end{itemize}
  \item Implemented in \code{drivers/input/}
    \begin{itemize}
    \item \code{input.c}, \code{input-polldev.c}, \code{evbug.c}
    \end{itemize}
  \item Implements a single character driver and defines the
    user/kernel API
    \begin{itemize}
    \item \code{include/uapi/linux/input.h}
    \end{itemize}
  \item Defines the set of operations a input driver must implement
    and helper functions for the drivers
    \begin{itemize}
    \item \code{struct input_dev} for the device driver part
    \item \code{struct input_handler} for the event handler part
    \item  \code{include/linux/input.h}
    \end{itemize}
  \end{itemize}
\end{frame}

\setuplabframe
{Expose the Nunchuk functionality to userspace}
{
  \begin{itemize}
  \item Extend the Nunchuk driver to expose the Nunchuk features to
    userspace applications, as an {\em input} device.
  \item Test the operation of the Nunchuk using sample userspace
    applications.
  \end{itemize}
}
