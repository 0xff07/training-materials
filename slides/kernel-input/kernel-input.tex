\section{The input subsystem}

\begin{frame}{What is input subsystem?}
  \begin{itemize}
  \item The input subsystem takes care of all the input events coming
    from the human user.
  \item Initially written to support the USB {\em HID} (Human
    Interface Device) devices, it quickly grew up to handle all kind
    of inputs (using USB or not): keyboard, mice, joystick,
    touchscreen, etc.
  \item The input subsystem is split in two parts:
    \begin{itemize}
    \item {\bf Device drivers}: they talk to the hardware (for example
      via USB), and provide events (keystrokes, mouse movements,
      touchscreen coordinates) to the input core
    \item {\bf Event handlers}: they get events from drivers and pass
      them where needed via various interfaces (most of the time
      through \code{evdev})
    \end{itemize}
  \item In userspace it is usually used by the graphic stack such
    as {\em X.Org}, {\em Wayland} or {\em Android's InputManager}.
  \end{itemize}
\end{frame}

\begin{frame}{Input subsystem diagram}
  \begin{center}
    \includegraphics[width=\textwidth]{slides/kernel-input/input-subsystem-diagram.pdf}
  \end{center}
\end{frame}

\begin{frame}{Input subsystem overview}
  \begin{itemize}
  \item Kernel option \code{CONFIG_INPUT}
    \begin{itemize}
    \item \code{menuconfig INPUT}
      \begin{itemize}
      \item \code{tristate "Generic input layer (needed for keyboard, mouse, ...)"}
      \end{itemize}
    \end{itemize}
  \item Implemented in \kpath{drivers/input/}
    \begin{itemize}
    \item \code{input.c}, \code{input-polldev.c}, \code{evbug.c}
    \end{itemize}
  \item Implements a single character driver and defines the
    user/kernel API
    \begin{itemize}
    \item \kpath{include/uapi/linux/input.h}
    \end{itemize}
  \item Defines the set of operations a input driver must implement
    and helper functions for the drivers
    \begin{itemize}
    \item \kstruct{input_dev} for the device driver part
    \item \kstruct{input_handler} for the event handler part
    \item  \kpath{include/linux/input.h}
    \end{itemize}
  \end{itemize}
\end{frame}

\begin{frame}[fragile]{Input subsystem API 1/3}
  \begin{itemize}
  \item An {\em input device} is described by a very long
    \kstruct{input_dev} structure, an excerpt is:
    \begin{block}{}
    \begin{minted}[fontsize=\tiny]{c}
struct input_dev {
    const char *name;
    [...]
    unsigned long evbit[BITS_TO_LONGS(EV_CNT)];
    unsigned long keybit[BITS_TO_LONGS(KEY_CNT)];
    [...]
    int (*getkeycode)(struct input_dev *dev,
                      struct input_keymap_entry *ke);
    [...]
    int (*open)(struct input_dev *dev);
    [...]
    int (*event)(struct input_dev *dev, unsigned int type,
                 unsigned int code, int value);
    [...]
};
\end{minted}
\end{block}
\item Before being used this struct must be allocated and initialized
  using: \code{struct input_dev *input_allocate_device(void);}
\item After unregistering \kstruct{input_dev}, the struct must be freed
  using: \code{void input_free_device(struct input_dev *dev);}
  \end{itemize}
\end{frame}

\begin{frame}[fragile]{Input subsystem API 2/3}
  \begin{itemize}
  \item Depending on the type of event that will be generated, the
    input bit fields \code{evbit} and \code{keybit} must be configured:
    For example, for a button we only generate
    \ksym{EV_KEY} type events, and from those only \ksym{BTN_0} events
    code:
    \begin{block}{}
    \begin{minted}[fontsize=\footnotesize]{c}
      set_bit(EV_KEY, myinput_dev.evbit);
      set_bit(BTN_0, myinput_dev.keybit);
    \end{minted}
    \end{block}
  \item \kfunc{set_bit} is an atomic operation allowing to set a particular bit 
        to \code{1} (explained later).
  \item Once the {\em input device} is allocated and filled, the
    function to register it
    is: \code{int input_register_device(struct input_dev *);}
  \item When the driver will be unloaded, the {\em input device} is
    unregistered using:
    \code{void input_unregister_device(struct input_dev *);}
  \end{itemize}
\end{frame}

\begin{frame}[fragile]{Input subsystem API 3/3}
  \begin{itemize}
  \item The events are sent by the driver to the event handler using
    \code{input_event(struct input_dev *dev, unsigned int type, unsigned int code, int value);}
    \begin{itemize}
    \item The event types are documented in \kerneldoc{input/event-codes.txt}
    \item An event is composed by one or several input data changes
      (packet of input data changes) such as the button state, the
      relative or absolute position along an axis, etc..
    \item After submitting potentially multiple events, the {\em
        input} core must be notified by calling:
      \code{ void input_sync(struct input_dev *dev)}:
    \item The input subsystem provides other wrappers such as
      \kfunc{input_report_key}, \kfunc{input_report_abs}, ...
    \end{itemize}
  \end{itemize}
\end{frame}

\begin{frame}[fragile]{Polled input subclass}
  \begin{itemize}
  \item The input subsystem provides a subclass supporting simple input
    devices that {\em do not raise interrupts} but have to be {\em
      periodically scanned or polled} to detect changes in their
    state.
  \item A {\em polled input device} is described by a
    \kstruct{input_polled_dev} structure:
    \begin{block}{}
    \begin{minted}[fontsize=\footnotesize]{c}
struct input_polled_dev {
        void *private;
        void (*open)(struct input_polled_dev *dev);
        void (*close)(struct input_polled_dev *dev);
        void (*poll)(struct input_polled_dev *dev);
        unsigned int poll_interval; /* msec */
        unsigned int poll_interval_max; /* msec */
        unsigned int poll_interval_min; /* msec */
        struct input_dev *input;
/* private: */
        struct delayed_work work;
}
    \end{minted}
    \end{block}
  \end{itemize}
\end{frame}

\begin{frame}[fragile]{Polled input subsystem API}
  \begin{itemize}
  \item Allocating/freeing the \kstruct{input_polled_dev} structure is
    done using
    \begin{itemize}
    \item \kfunc{input_allocate_polled_device}
    \item \kfunc{input_free_polled_device}
    \end{itemize}
  \item Among the handlers of the \kstruct{input_polled_dev} only the
    \code{poll()} method is mandatory, this function polls the device
    and posts input events.
  \item The fields \code{id}, \code{name}, \code{phys}, \code{bits} of
    \kstruct{input} must be initialized too.
  \item If none of the \code{poll_interval} fields are filled then the
    default poll interval is 500ms.
  \item The device registration/unregistration is done with:
    \begin{itemize}
    \item \code{input_register_polled_device(struct input_polled_dev *dev)}.
    \item \code{input_unregister_polled_device(struct input_polled_dev *dev)}
    \end{itemize}
  \end{itemize}
\end{frame}

\begin{frame}[fragile]{{\em evdev} userspace interface}
  \begin{itemize}
  \item The main userspace interface to {\em input devices} is the
    {\bf event interface}
  \item Each {\em input device} is represented as a
    \code{/dev/input/event<X>} character device
  \item A userspace application can use blocking and non-blocking
    reads, but also \code{select()} (to get notified of events) after
    opening this device.
  \item Each read will return \kstruct{input_event} structures of the
    following format:
    \begin{block}{}
    \begin{minted}[fontsize=\tiny]{c}
struct input_event {
        struct timeval time;
        unsigned short type;
        unsigned short code;
        unsigned int value;
};
\end{minted}
\end{block}
\item A very useful application for {\em input device} testing is
  \code{evtest}, from \url{http://cgit.freedesktop.org/evtest/}
  \end{itemize}
\end{frame}

\setuplabframe
{Expose the Nunchuk to userspace}
{
  \begin{itemize}
  \item Extend the Nunchuk driver to expose the Nunchuk features to
    userspace applications, as an {\em input} device.
  \item Test the operation of the Nunchuk using sample userspace
    applications.
  \end{itemize}
}
