\section{Advanced topics}

\begin{frame}{{\tt BR2\_EXTERNAL}}
  \begin{itemize}
  \item Storing your custom packages, custom configuration files and
    custom {\em defconfigs} inside the Buildroot tree may not be the
    most practical solution
    \begin{itemize}
    \item Doesn't cleanly separate open-source parts from proprietary parts
    \item Makes it harder to upgrade Buildroot
    \end{itemize}
  \item The \code{BR2_EXTERNAL} mechanism allows to store your own
    package recipes, {\em defconfigs} and other artefacts outside of
    the Buildroot tree.
  \item Specify \code{BR2_EXTERNAL} on the command line when building.
  \item Buildroot will:
    \begin{itemize}
    \item include \code{$(BR2_EXTERNAL)/Config.in} in the configuration menu
    \item include \code{$(BR2_EXTERNAL)/external.mk} in the make logic
    \item include \code{$(BR2_EXTERNAL)/configs/} in the list of {\em defconfigs}
    \end{itemize}
  \item Note: can only be used to add new packages, not to override
    existing Buildroot packages
  \end{itemize}
\end{frame}

\begin{frame}[fragile]{{\tt BR2\_EXTERNAL}: recommended structure}

  \begin{block}{}
    {\tiny
\begin{verbatim}
+-- board/
|   +-- <company>/
|       +-- <boardname>/
|           +-- linux.config
|           +-- busybox.config
|           +-- <other configuration files>
|           +-- post_build.sh
|           +-- post_image.sh
|           +-- rootfs_overlay/
|           |   +-- etc/
|           |   +-- <some file>
|           +-- patches/
|               +-- foo/
|               |   +-- <some patch>
|               +-- libbar/
|                   +-- <some other patches>
|
+-- configs/
|   +-- <boardname>_defconfig
|
+-- package/
|   +-- <company>/
|       +-- package1/
|       |    +-- Config.in
|       |    +-- package1.mk
|       +-- package2/
|           +-- Config.in
|           +-- package2.mk
|
+-- Config.in
+-- external.mk
\end{verbatim}
    }
  \end{block}

\end{frame}

\begin{frame}[fragile]{{\tt BR2\_EXTERNAL/Config.in}}

  \begin{itemize}
  \item Custom configuration options
  \item Configuration options for the \code{BR2_EXTERNAL} packages
  \item The \code{$BR2_EXTERNAL} variable is available
  \end{itemize}

  \begin{block}{Example \code{$(BR2_EXTERNAL)/Config.in}}
\begin{verbatim}
source "$BR2_EXTERNAL/package/package1/Config.in"
source "$BR2_EXTERNAL/package/package2/Config.in"
\end{verbatim}
  \end{block}

\end{frame}

\begin{frame}[fragile]{{\tt BR2\_EXTERNAL/external.mk}}

  \begin{itemize}
  \item Can include custom {\em make} logic
  \item Generally only used to include the package \code{.mk} files
  \end{itemize}

  \begin{block}{Example \code{$(BR2_EXTERNAL)/external.mk}}
\begin{minted}[fontsize=\scriptsize]{make}
include $(sort $(wildcard $(BR2_EXTERNAL)/package/*/*.mk))
\end{minted}
  \end{block}
\end{frame}

\begin{frame}[fragile]{Using {\tt BR2\_EXTERNAL}}
  \begin{itemize}
  \item Not a configuration option, only an {\bf environment variable}
    to be passed on the command line
    \begin{block}{}
\begin{verbatim}
make BR2_EXTERNAL=/path/to/external
\end{verbatim}
    \end{block}
  \item {\bf Automatically saved} in the hidden \code{.br-external}
    file in the output directory
    \begin{itemize}
    \item no need to pass \code{BR2_EXTERNAL} at every make invocation
    \item can be changed at any time by passing a new value, and
      removed by passing an empty value
    \end{itemize}
  \item Can be either an {\bf absolute} or a {\bf relative} path, but
    if relative, important to remember that it's relative to the
    Buildroot source directory
  \end{itemize}
\end{frame}

\begin{frame}{Tips for building faster}
  \begin{itemize}
  \item Build time is often an issue, so here are some tips to help
    \begin{itemize}
    \item Use fast hardware: lots of RAM, and SSD
    \item Do not use virtual machines
    \item You can enable the \code{ccache} {\em compiler cache} using
      \code{BR2_CCACHE}
    \item Use external toolchains instead of internal toolchains
    \item Learn about rebuilding only the few packages you actually
      care about
    \item Build everything locally, do not use NFS for building
    \item Remember that you can do several independent builds in
      parallel in different output directories
    \end{itemize}
  \end{itemize}
\end{frame}

\setuplabframe
{Advanced aspects}
{
  \begin{itemize}
  \item Use \code{legal-info} for legal information extraction
  \item Use \code{graph-depends} for dependency graphing
  \item Use \code{graph-build} for build time graphing
  \item Use \code{BR2_EXTERNAL} to isolate the project-specific
    changes (packages, configs, etc.)
  \end{itemize}
}
