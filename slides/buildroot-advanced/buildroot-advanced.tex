\section{Advanced topics}

\begin{frame}{{\tt BR2\_EXTERNAL}: principle}
  \begin{itemize}
  \item Storing your custom packages, custom configuration files and
    custom {\em defconfigs} inside the Buildroot tree may not be the
    most practical solution
    \begin{itemize}
    \item Doesn't cleanly separate open-source parts from proprietary parts
    \item Makes it harder to upgrade Buildroot
    \end{itemize}
  \item The \code{BR2_EXTERNAL} mechanism allows to store your own
    package recipes, {\em defconfigs} and other artefacts {\bf
      outside} of the Buildroot source tree.
  \item It is possible to use several \code{BR2_EXTERNAL} trees, to
    further separate various aspects of your project.
  \item Note: can only be used to add new packages, not to override
    existing Buildroot packages
  \end{itemize}
\end{frame}

\begin{frame}{{\tt BR2\_EXTERNAL}: example organization}

  \begin{itemize}
  \item \code{project/}
    \begin{itemize}
    \item \code{buildroot/}
      \begin{itemize}
      \item The Buildroot source code, cloned from Git, or extracted
        from a release tarball.
      \end{itemize}
    \item \code{external1/}
    \item \code{external2/}
      \begin{itemize}
      \item Two external trees
      \end{itemize}
    \item \code{output-build1/}
    \item \code{output-build2/}
      \begin{itemize}
      \item Several {\em output} directories, to build various configurations
      \end{itemize}
    \item \code{custom-app/}
    \item \code{custom-lib/}
      \begin{itemize}
      \item The source code of your custom applications and libraries.
      \end{itemize}
    \end{itemize}
  \end{itemize}
\end{frame}

\begin{frame}{{\tt BR2\_EXTERNAL}: mechanism}
  \begin{itemize}
  \item Specify, as a colon-separated list, the {\em external}
    directories in \code{BR2_EXTERNAL}
  \item Each {\em external} directory must contain:
    \begin{itemize}
    \item \code{external.desc}, which provides a name and description
    \item \code{Config.in}, configuration options that will be
      included in {\em menuconfig}
    \item \code{external.mk}, will be included in the make logic
    \end{itemize}
  \item If \code{configs} exists, it will be used when listing all
    {\em defconfigs}
  \end{itemize}
\end{frame}

\begin{frame}[fragile]{{\tt BR2\_EXTERNAL}: recommended structure}
  \footnotesize
  \begin{columns}[b]
  \column{0.5\textwidth}
  \begin{block}{}
\begin{verbatim}
+-- board/
|   +-- <company>/
|       +-- <boardname>/
|           +-- linux.config
|           +-- busybox.config
|           +-- <other configuration files>
|           +-- post_build.sh
|           +-- post_image.sh
|           +-- rootfs_overlay/
|           |   +-- etc/
|           |   +-- <some file>
|           +-- patches/
|               +-- libbar/
|                   +-- <some patches>
|
+-- configs/
|   +-- <boardname>_defconfig
|
\end{verbatim}
   \end{block}
   \column{0.5\textwidth}
   \begin{block}{}
\begin{verbatim}
+-- package/
|   +-- <company>/
|       +-- package1/
|       |    +-- Config.in
|       |    +-- package1.mk
|       +-- package2/
|           +-- Config.in
|           +-- package2.mk
|
+-- Config.in
+-- external.mk
+-- external.desc
\end{verbatim}
   \end{block}
   \end{columns}
\end{frame}

\begin{frame}[fragile]{{\tt BR2\_EXTERNAL}: {\tt external.desc}}

  \begin{itemize}
  \item File giving metadata about the {\em external tree}
  \item Mandatory
    \code{name} field, using characters in the set
    \code{[A-Za-z0-9_]}. Will be used to define
    \code{BR2_EXTERNAL_<NAME>_PATH} available in
    \code{Config.in} and \code{.mk} files, pointing
    to the external tree directory.
  \item Optional
    \code{desc} field, giving a free-form description of the external tree. Should be reasonably short.
  \item Example
    \begin{block}{}
\begin{verbatim}
name: FOOBAR
desc: Foobar Company
\end{verbatim}
    \end{block}
  \end{itemize}

\end{frame}

\begin{frame}[fragile]{{\tt BR2\_EXTERNAL}: main {\tt Config.in}}

  \begin{itemize}
  \item Custom configuration options
  \item Configuration options for the external packages
  \item The
    \code{$BR2_EXTERNAL_<NAME>_PATH} variable is available, where
    \code{NAME} is defined in \code{external.desc}
  \end{itemize}

  \begin{block}{Example \code{Config.in}}
\begin{verbatim}
source "$BR2_EXTERNAL_<NAME>_PATH/package/package1/Config.in"
source "$BR2_EXTERNAL_<NAME>_PATH/package/package2/Config.in"
\end{verbatim}
  \end{block}

\end{frame}

\begin{frame}[fragile]{{\tt BR2\_EXTERNAL}: {\tt external.mk}}

  \begin{itemize}
  \item Can include custom {\em make} logic
  \item Generally only used to include the package \code{.mk} files
  \end{itemize}

  \begin{block}{Example \code{external.mk}}
\begin{minted}[fontsize=\scriptsize]{make}
include $(sort $(wildcard $(BR2_EXTERNAL_<NAME>_PATH)/package/*/*.mk))
\end{minted}
  \end{block}
\end{frame}

\begin{frame}[fragile]{Using {\tt BR2\_EXTERNAL}}
  \begin{itemize}
  \item Not a configuration option, only an {\bf environment variable}
    to be passed on the command line
    \begin{block}{}
\begin{verbatim}
make BR2_EXTERNAL=/path/to/external1:/path/to/external2
\end{verbatim}
    \end{block}
  \item {\bf Automatically saved} in the hidden \code{.br-external.mk}
    file in the output directory
    \begin{itemize}
    \item no need to pass \code{BR2_EXTERNAL} at every make invocation
    \item can be changed at any time by passing a new value, and
      removed by passing an empty value
    \end{itemize}
  \item Can be either an {\bf absolute} or a {\bf relative} path, but
    if relative, important to remember that it's relative to the
    Buildroot source directory
  \end{itemize}
\end{frame}

\begin{frame}[fragile]{Use {\tt BR2\_EXTERNAL} in your configuration}

  \begin{itemize}
  \item In your Buildroot configuration, don't use absolute paths for
    the {\em rootfs overlay}, the {\em post-build scripts}, {\em
      global patch directories}, etc.
  \item If they are located in an external tree, you can use
    \code{$(BR2_EXTERNAL_<NAME>_PATH)} in your Buildroot configuration
    options.
  \item With the recommended structure shown before, a Buildroot
    configuration would look like:
    \begin{block}{}
{\tiny
\begin{verbatim}
BR2_GLOBAL_PATCH_DIR="$(BR2_EXTERNAL_<NAME>_PATH)/board/<company>/<boardname>/patches/"
...
BR2_ROOTFS_OVERLAY="$(BR2_EXTERNAL_<NAME>_PATH)/board/<company>/<boardname>/rootfs_overlay/"
...
BR2_ROOTFS_POST_BUILD_SCRIPT="$(BR2_EXTERNAL_<NAME>_PATH)/board/<company>/<boardname>/post_build.sh"
BR2_ROOTFS_POST_IMAGE_SCRIPT="$(BR2_EXTERNAL_<NAME>_PATH)/board/<company>/<boardname>/post_image.sh"
...
BR2_LINUX_KERNEL_USE_CUSTOM_CONFIG=y
BR2_LINUX_KERNEL_CUSTOM_CONFIG_FILE="$(BR2_EXTERNAL_<NAME>_PATH)/board/<company>/<boardname>/linux.config"
\end{verbatim}
}
    \end{block}
  \end{itemize}

\end{frame}

\begin{frame}{Package-specific targets: basics}
  \begin{itemize}
  \item Internally, each package is implemented through a number of
    package-specific {\em make targets}
    \begin{itemize}
    \item They can sometimes be useful to call directly, in certain
      situations.
    \end{itemize}
  \item The targets used in the normal build flow of a package are:
    \begin{itemize}
    \item \code{<pkg>}, fully build and install the package
    \item \code{<pkg>-source}, just download the source code
    \item \code{<pkg>-extract}, download and extract
    \item \code{<pkg>-patch}, download, extract and patch
    \item \code{<pkg>-configure}, download, extract, patch and configure
    \item \code{<pkg>-build}, download, extract, patch, configure and build
    \item \code{<pkg>-install-staging}, download, extract, patch,
      configure and do the staging installation (target packages only)
    \item \code{<pkg>-install-target}, download, extract, patch,
      configure and do the target installation (target packages only)
    \item \code{<pkg>-install}, download, extract, patch,
      configure and install
    \end{itemize}
  \end{itemize}
\end{frame}

\begin{frame}[fragile]{Package-specific targets: example (1)}

\begin{block}{}
{\tiny
\begin{verbatim}
$ make strace
>>> strace 4.10 Extracting
>>> strace 4.10 Patching
>>> strace 4.10 Updating config.sub and config.guess
>>> strace 4.10 Patching libtool
>>> strace 4.10 Configuring
>>> strace 4.10 Building
>>> strace 4.10 Installing to target
$ make strace-build
... nothing ...
$ make ltrace-patch
>>> ltrace 0896ce554f80afdcba81d9754f6104f863dea803 Extracting
>>> ltrace 0896ce554f80afdcba81d9754f6104f863dea803 Patching
$ make ltrace
>>> argp-standalone 1.3 Extracting
>>> argp-standalone 1.3 Patching
>>> argp-standalone 1.3 Updating config.sub and config.guess
>>> argp-standalone 1.3 Patching libtool
[...]
>>> ltrace 0896ce554f80afdcba81d9754f6104f863dea803 Configuring
>>> ltrace 0896ce554f80afdcba81d9754f6104f863dea803 Autoreconfiguring
>>> ltrace 0896ce554f80afdcba81d9754f6104f863dea803 Patching libtool
>>> ltrace 0896ce554f80afdcba81d9754f6104f863dea803 Building
>>> ltrace 0896ce554f80afdcba81d9754f6104f863dea803 Installing to target
\end{verbatim}}
\end{block}

\end{frame}

\begin{frame}{Package-specific targets: advanced}
  \begin{itemize}
  \item Additional useful targets
    \begin{itemize}
    \item \code{make <pkg>-show-depends}, show the package dependencies
    \item \code{make <pkg>-graph-depends}, generates a dependency graph
    \item \code{make <pkg>-dirclean}, completely remove the package
      source code directory. The next \code{make} invocation will
      fully rebuild this package.
    \item \code{make <pkg>-reinstall}, force to re-execute the
      installation step of the package
    \item \code{make <pkg>-rebuild}, force to re-execute the build and
      installation steps of the package
    \item \code{make <pkg>-reconfigure}, force to re-execute the
      configure, build and installation steps of the package.
    \end{itemize}
  \end{itemize}
\end{frame}

\begin{frame}[fragile]{Package-specific targets: example (2)}

\begin{block}{}
{\tiny
\begin{verbatim}
$ make strace
>>> strace 4.10 Extracting
>>> strace 4.10 Patching
>>> strace 4.10 Updating config.sub and config.guess
>>> strace 4.10 Patching libtool
>>> strace 4.10 Configuring
>>> strace 4.10 Building
>>> strace 4.10 Installing to target
$ ls output/build/
strace-4.10 [...]
$ make strace-dirclean
rm -Rf /home/thomas/projets/buildroot/output/build/strace-4.10
$ ls output/build/
[... no strace-4.10 directory ...]
\end{verbatim}}
\end{block}

\end{frame}

\begin{frame}[fragile]{Package-specific targets: example (3)}

\begin{block}{}
{\tiny
\begin{verbatim}
$ make strace
>>> strace 4.10 Extracting
>>> strace 4.10 Patching
>>> strace 4.10 Updating config.sub and config.guess
>>> strace 4.10 Patching libtool
>>> strace 4.10 Configuring
>>> strace 4.10 Building
>>> strace 4.10 Installing to target
$ make strace-rebuild
>>> strace 4.10 Building
>>> strace 4.10 Installing to target
$ make strace-reconfigure
>>> strace 4.10 Configuring
>>> strace 4.10 Building
>>> strace 4.10 Installing to target
\end{verbatim}}
\end{block}

\end{frame}

\begin{frame}[fragile]{{\tt make show-info}}
  \begin{columns}
    \column{0.5\textwidth}
    \begin{itemize}
    \item \code{make show-info} outputs JSON text that describes the
      current configuration: enabled packages, in which version, their
      license, tarball, dependencies, etc.
    \item Can be useful for post-processing, build analysis, license
      compliance, etc.
    \end{itemize}

    \column{0.5\textwidth}
    \begin{block}{}
      {\tiny
\begin{minted}{json}
$ make show-info | jq .
{
  "busybox": {
    "type": "target",
    "virtual": false,
    "version": "1.31.1",
    "licenses": "GPL-2.0",
    "dl_dir": "busybox",
    "install_target": true,
    "install_staging": false,
    "install_images": false,
    "downloads": [
      {
        "source": "busybox-1.31.1.tar.bz2",
        "uris": [
          "http+http://www.busybox.net/downloads",
          "http|urlencode+http://sources.buildroot.net/busybox",
        ]
      }
    ],
    "dependencies": [
      "host-skeleton",
      "host-tar",
      "skeleton",
      "toolchain"
    ],
    "reverse_dependencies": []
  },
\end{minted}
      }
    \end{block}
  \end{columns}
\end{frame}

\begin{frame}[fragile]{Understanding rebuilds (1)}
  \begin{itemize}
  \item Doing a {\bf full rebuild} is achieved using:
    \begin{block}{}
\begin{minted}{console}
$ make clean all
\end{minted}
\end{block}
\begin{itemize}
\item It will completely remove all build artefacts and restart the
  build from scratch
\end{itemize}
  \item Buildroot {\bf does not try to be smart}
    \begin{itemize}
    \item once the system has been built, if a configuration change is
      made, the next \code{make} will {\bf not apply all the changes}
      made to the configuration.
    \item being smart is very, very complicated if you want to do it
      in a reliable way.
    \end{itemize}
  \end{itemize}
\end{frame}


\begin{frame}{Understanding rebuilds (2)}
  \begin{itemize}
  \item When a package has been built by Buildroot, Buildroot keeps a
    {\bf hidden file} telling that the package has been built.
    \begin{itemize}
    \item Buildroot will therefore {\em never} rebuild that package,
      unless a {\bf full rebuild is done}, or this specific package is
      {\bf explicitly rebuilt}.
    \item Buildroot does not {\em recurse} into each package at each
      \code{make} invocation, it would be too time-consuming. So if
      you change one source file in a package, Buildroot does not know
      it.
    \end{itemize}
  \item When \code{make} is invoked, Buildroot {\bf will always}:
    \begin{itemize}
    \item Build the packages that have not been built in a previous
      build and install them to the target
    \item Cleanup the target root filesystem from useless files
    \item Run {\em post-build} scripts, copy {\em rootfs overlays}
    \item Generate the root filesystem images
    \item Run {\em post-image} scripts
    \end{itemize}
  \end{itemize}
\end{frame}

\begin{frame}{Understanding rebuilds: scenarios (1)}
  \begin{itemize}
  \item If you enable a new package in the configuration, and run
    \code{make}
    \begin{itemize}
    \item Buildroot will build it and install it
    \item However, other packages that may benefit from this package
      will not be rebuilt automatically
    \end{itemize}
  \item If you remove a package from the configuration, and run
    \code{make}
    \begin{itemize}
    \item Nothing happens. The files installed by this package are not
      removed from the target filesystem.
    \item Buildroot does not track which files are installed by which
      package
    \item Need to do a full rebuild to get the new result. Advice: do
      it only when really needed.
    \end{itemize}
  \item If you change the sub-options of a package that has already
    been built, and run \code{make}
    \begin{itemize}
    \item Nothing happens.
    \item You can force Buildroot to rebuild this package using
      \code{make <pkg>-reconfigure} or \code{make <pkg>-rebuild}.
    \end{itemize}
  \end{itemize}
\end{frame}

\begin{frame}{Understanding rebuilds: scenarios (2)}
  \begin{itemize}
  \item If you make a change to a {\em post-build} script, a {\em
      rootfs overlay} or a {\em post-image} script, and run
    \code{make}
    \begin{itemize}
    \item This is sufficient, since these parts are re-executed at
      every \code{make} invocation.
    \end{itemize}
  \item If you change a fundamental system configuration option:
    architecture, type of toolchain or toolchain configuration, init
    system, etc.
    \begin{itemize}
    \item You {\bf must do a full rebuild}
    \end{itemize}
  \item If you change some source code in
    \code{output/build/<foo>-<version>/} and issue \code{make}
    \begin{itemize}
    \item The package will not be rebuilt automatically: Buildroot has
      a {\em hidden file} saying that the package was already built.
    \item Use \code{make <pkg>-reconfigure} or \code{make <pkg>-rebuild}
    \item And remember that doing changes in
      \code{output/build/<foo>-<version>/} can only be temporary: this
      directory is removed during a \code{make clean}.
    \end{itemize}
  \end{itemize}
\end{frame}

\begin{frame}{Tips for building faster}
  \begin{itemize}
  \item Build time is often an issue, so here are some tips to help
    \begin{itemize}
    \item Use fast hardware: lots of RAM, and SSD
    \item Do not use virtual machines
    \item You can enable the \code{ccache} {\em compiler cache} using
      \code{BR2_CCACHE}
    \item Use external toolchains instead of internal toolchains
    \item Learn about rebuilding only the few packages you actually
      care about
    \item Build everything locally, do not use NFS for building
    \item Remember that you can do several independent builds in
      parallel in different output directories
    \end{itemize}
  \end{itemize}
\end{frame}

\begin{frame}{Support for top-level parallel build (1)}
  \begin{itemize}
  \item Buildroot normally builds packages {\bf sequentially}, one after the
    other.
  \item Calling Buildroot with \code{make -jX} has no effect
  \item Parallel build is used {\em within} the build of each package:
    Buildroot invokes each package build system with \code{make -jX}
    \begin{itemize}
    \item This level of parallelization is controlled by
      \code{BR2_JLEVEL}
    \item Defaults to 0, which means Buildroot auto-detects the number
      of CPUs cores
    \end{itemize}
  \item Buildroot 2020.02 has introduced {\bf experimental} support
    for top-level parallel build
    \begin{itemize}
    \item Allows to build multiple different packages in parallel
    \item Of course taking into account their dependencies
    \item Allows to better use multi-core machines
    \item Reduces build time significantly
    \end{itemize}
  \end{itemize}
\end{frame}

\begin{frame}{Support for top-level parallel build (2)}
  \begin{itemize}
  \item To use this experimental support:
    \begin{enumerate}
    \item Enable \code{BR2_PER_PACKAGE_DIRECTORIES=y}
    \item Build with \code{make -jX}
    \end{enumerate}
  \item The {\em per-package} option ensures that each package uses
    its own \code{HOST_DIR}, \code{STAGING_DIR} and \code{TARGET_DIR}
    so that different packages can be built in parallel with no
    interference
  \item See \code{$(O)/per-package/<pkg>/}
  \item Limitations
    \begin{itemize}
    \item Not yet supported by all packages, e.g {\em Qt5}
    \item Absolutely requires that packages do not overwrite/change
      files installed by other packages
    \item \code{<pkg>-reconfigure}, \code{<pkg>-rebuild},
      \code{<pkg>-reinstall} not working
    \end{itemize}
  \end{itemize}
\end{frame}

\setuplabframe
{Advanced aspects}
{
  \begin{itemize}
  \item Use \code{legal-info} for legal information extraction
  \item Use \code{graph-depends} for dependency graphing
  \item Use \code{graph-build} for build time graphing
  \item Use \code{BR2_EXTERNAL} to isolate the project-specific
    changes (packages, configs, etc.)
  \end{itemize}
}
