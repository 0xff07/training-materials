\chapterframe{Hotplugging with udev}

\begin{frame}
  \frametitle{/dev issues and limitations}
  \begin{itemize}
  \item On Red Hat 9, 18000 entries in \code{/dev}!\\
    All entries for all possible devices had to be created at system
    installation.
  \item Needed an authority to assign major numbers\\
    \url{http://lanana.org/}: Linux Assigned Names and Numbers
    Authority
  \item Not enough numbers in 2.4, limits extended in 2.6.
  \item User space neither knew what devices were present in the
    system, nor which real device corresponded to each \code{/dev}
    entry.
  \end{itemize}
\end{frame}

\begin{frame}
  \frametitle{The udev solution}
  Takes advantage of {\bf sysfs} introduced by Linux 2.6.
  \begin{itemize}
  \item Created by Greg Kroah Hartman, a huge contributor.\\
    Other key contributors: Kay Sievers, Dan Stekloff.
  \item {\em Entirely} in user space.
  \item Automatically creates and removes device entries in
    \code{/dev/} according to inserted and removed devices.
  \item Major and minor device transmitted by the kernel.
  \item Requires no change to driver code.
  \item Fast: written in C\\
    Relatively small size: \code{udevd} version 167: 127 KB
  \end{itemize}
\end{frame}

\begin{frame}[fragile]
  \frametitle{Starting udev (1)}
  \begin{itemize}
  \item At the very beginning of user space startup,
    mount the \code{/dev/} directory as a \code{tmpfs} filesystem:\\
    \code{sudo mount -t tmpfs udev /dev}
  \item \code{/dev/} is populated with static devices available in
    \code{/lib/udev/devices/}:
  \end{itemize}
  \begin{block}{}
\begin{verbatim}
$ ls -l /lib/udev/devices/
total 12
brw------- 1 root root   7, 0 2011-06-04 10:25 loop0
drwxr-xr-x 2 root root   4096 2011-06-04 10:25 net
crw------- 1 root root 108, 0 2011-06-04 10:25 ppp
drwxr-xr-x 2 root root   4096 2011-04-07 14:43 pts
drwxr-xr-x 2 root root   4096 2011-04-07 14:43 shm
\end{verbatim}
  \end{block}
\end{frame}

\begin{frame}
  \frametitle{Starting udev (2)}
  \begin{itemize}
  \item The \code{udevd} daemon is started.  It listens to {\em uevents}
    from the driver core, which are sent whenever devices are inserted
    or removed.
  \item The \code{udevd} daemon reads and parses all the rules found
    in \code{/etc/udev/rules.d/} and keeps them in memory.
  \item Whenever rules are added, removed or modified, \code{udevd}
    receives an {\em inotify} event and updates its rule-set in memory.
    \begin{itemize}
    \item The {\em inotify} mechanism lets user space programs
      subscribe to notifications of filesystem changes.
    \item When an event is received, \code{udevd} starts a process to:
      \begin{itemize}
      \item try to match the event against \code{udev} rules,
      \item create / remove device files,
      \item and run programs (to load / remove a driver, to notify user
        space...)
      \end{itemize}
    \end{itemize}
  \end{itemize}
\end{frame}

\begin{frame}[fragile]
  \frametitle{uevent message example}
  Example inserting a USB mouse
  \begin{block}{}
\tiny
\begin{verbatim}
recv(4,
     "add@/class/input/input9/mouse2\0
     ACTION=add\0
     DEVPATH=/class/input/input9/mouse2\0
     SUBSYSTEM=input\0
     SEQNUM=1064\0
     PHYSDEVPATH=/devices/pci0000:00/0000:00:1d.1/usb2/2-2/2-2:1.0\0
     PHYSDEVBUS=usb\0
     PHYSDEVDRIVER=usbhid\0
     MAJOR=13\0
     MINOR=34\0",
     2048,
     0)
= 221
\end{verbatim}
  \end{block}
\end{frame}

\begin{frame}
  \frametitle{udev rules}
  When a \code{udev} rule matching event information is found, it can be
  used:
  \begin{itemize}
  \item To define the name and path of a device file.
  \item To define the owner, group and permissions of a device file.
  \item To execute a specified program.
  \end{itemize}
  Rule files are processed in lexical order.
\end{frame}

\begin{frame}
  \frametitle{udev naming capabilities}
  Device names can be defined
  \begin{itemize}
  \item from a label or serial number,
  \item from a bus device number,
  \item from a location on the bus topology,
  \item from a kernel name,
  \item from the output of a program.
  \end{itemize}
See \url{http://www.reactivated.net/writing_udev_rules.html}
for a very complete description. See also \code{man udev}.
\end{frame}

\begin{frame}[fragile]
  \frametitle{udev naming rule examples}
  \begin{block}{}
\tiny
\begin{verbatim}
# Naming testing the output of a program
BUS=="scsi", PROGRAM="/sbin/scsi_id", RESULT=="OEM 0815", NAME="disk1"

# USB printer to be called lp_color
BUS=="usb", SYSFS{serial}=="W09090207101241330", NAME="lp_color"

# SCSI disk with a specific vendor and model number will be called boot
BUS=="scsi", SYSFS{vendor}=="IBM", SYSFS{model}=="ST336", NAME="boot%n"

# sound card with PCI bus id 00:0b.0 to be called dsp
BUS=="pci", ID=="00:0b.0", NAME="dsp"

# USB mouse at third port of the second hub to be called mouse1
BUS=="usb", PLACE=="2.3", NAME="mouse1"

# ttyUSB1 should always be called pda with two additional symlinks
KERNEL=="ttyUSB1", NAME="pda", SYMLINK="palmtop handheld"

# multiple USB webcams with symlinks to be called webcam0, webcam1, ...
BUS=="usb", SYSFS{model}=="XV3", NAME="video%n", SYMLINK="webcam%n"
\end{verbatim}
  \end{block}
\end{frame}

\begin{frame}[fragile]
  \frametitle{udev permission rule examples}
  Excerpts from \code{/etc/udev/rules.d/40-permissions.rules}
\begin{block}{}
\begin{verbatim}
# Block devices
SUBSYSTEM!="block", GOTO="block_end"
SYSFS{removable}!="1", GROUP="disk"
SYSFS{removable}=="1", GROUP="floppy"
BUS=="usb", GROUP="plugdev"
BUS=="ieee1394", GROUP="plugdev"
LABEL="block_end"

# Other devices, by name
KERNEL=="null", MODE="0666"
KERNEL=="zero", MODE="0666"
KERNEL=="full", MODE="0666"
\end{verbatim}
  \end{block}
\end{frame}

\begin{frame}
  \frametitle{Identifying device driver modules}
  \begin{center}
    \includegraphics[width=\textwidth]{slides/sysdev-udev/module-autoloading.pdf}
  \end{center}
\end{frame}

\begin{frame}
  \frametitle{Module aliases}
  \begin{itemize}
  \item \code{MODALIAS} environment variable example (USB mouse):\\
    \code{MODALIAS=usb:v046DpC03Ed2000dc00dsc00dp00ic03isc01ip02}
  \item Matching line in \code{/lib/modules/<version>/modules.alias}:\\
    \code{alias usb:v*p*d*dc*dsc*dp*ic03isc01ip02* usbmouse}
  \end{itemize}
\end{frame}

\begin{frame}[fragile]
  \frametitle{udev modprobe rule examples}
  Even module loading is done with \code{udev}!\\
  Excerpts from \code{/etc/udev/rules.d/90-modprobe.rules}
  \begin{block}{}
\footnotesize
\begin{verbatim}
ACTION!="add", GOTO="modprobe_end"

SUBSYSTEM!="ide", GOTO="ide_end"
IMPORT{program}="ide_media --export $devpath"
ENV{IDE_MEDIA}=="cdrom",RUN+="/sbin/modprobe -Qba ide-cd"
ENV{IDE_MEDIA}=="disk",RUN+="/sbin/modprobe -Qba ide-disk"
ENV{IDE_MEDIA}=="floppy", RUN+="/sbin/modprobe -Qba ide-floppy"
ENV{IDE_MEDIA}=="tape", RUN+="/sbin/modprobe -Qba ide-tape"
LABEL="ide_end"
SUBSYSTEM=="input", PROGRAM="/sbin/grepmap --udev", \
   RUN+="/sbin/modprobe -Qba $result"
# Load drivers that match kernel-supplied alias
ENV{MODALIAS}=="?*", RUN+="/sbin/modprobe -Q $env{MODALIAS}"
\end{verbatim}
  \end{block}
\end{frame}

\begin{frame}
  \frametitle{Cold-plugging}
  \begin{itemize}
  \item Issue: loosing all device events happening during kernel
    initialization, because \code{udev} is not ready yet.
  \item Solution: after starting \code{udevd}, have the kernel emit
    uevents for all devices present in \code{/sys}.
  \item This can be done by the \code{udevtrigger} utility.
  \item Strong benefit: completely transparent for user space. Legacy
    and removable devices handled and named in exactly the same way.
  \end{itemize}
\end{frame}

\begin{frame}[fragile]
  \frametitle{Debugging events - udevmonitor (1)}
  \code{udevadm monitor} visualizes the driver core events and the \code{udev} event processes.\\
  Example event sequence connecting a USB mouse:
  \begin{block}{}
\tiny
\begin{verbatim}
UEVENT[1170452995.094476] add@/devices/pci0000:00/0000:00:1d.7/usb4/4-3/4-3.2
UEVENT[1170452995.094569] add@/devices/pci0000:00/0000:00:1d.7/usb4/4-3/4-3.2/4-3.2:1.0
UEVENT[1170452995.098337] add@/class/input/input28
UEVENT[1170452995.098618] add@/class/input/input28/mouse2
UEVENT[1170452995.098868] add@/class/input/input28/event4
UEVENT[1170452995.099110] add@/class/input/input28/ts2
UEVENT[1170452995.099353] add@/class/usb_device/usbdev4.30
UDEV  [1170452995.165185] add@/devices/pci0000:00/0000:00:1d.7/usb4/4-3/4-3.2
UDEV  [1170452995.274128] add@/devices/pci0000:00/0000:00:1d.7/usb4/4-3/4-3.2/4-3.2:1.0
UDEV  [1170452995.375726] add@/class/usb_device/usbdev4.30
UDEV  [1170452995.415638] add@/class/input/input28
UDEV  [1170452995.504164] add@/class/input/input28/mouse2
UDEV  [1170452995.525087] add@/class/input/input28/event4
UDEV  [1170452995.568758] add@/class/input/input28/ts2
\end{verbatim}
\end{block}
It gives time information measured in microseconds. You can measure
time elapsed between the uevent (\code{UEVENT} line), and the completion of
the corresponding \code{udev} process (matching \code{UDEV} line).
\end{frame}

\begin{frame}[fragile]
\frametitle{Debugging events - udevmonitor (2)}
\code{udevadm monitor --env} shows the complete event environment for each line.
\begin{block}{}
\tiny
\begin{verbatim}
UDEV [1170453642.595297] add@/devices/pci0000:00/0000:00:1d.7/usb4/4-3/4-3.2/4-3.2:1.0
UDEV_LOG=3
ACTION=add
DEVPATH=/devices/pci0000:00/0000:00:1d.7/usb4/4-3/4-3.2/4-3.2:1.0
SUBSYSTEM=usb
SEQNUM=3417
PHYSDEVBUS=usb
DEVICE=/proc/bus/usb/004/031
PRODUCT=46d/c03d/2000
TYPE=0/0/0
INTERFACE=3/1/2
MODALIAS=usb:v046DpC03Dd2000dc00dsc00dp00ic03isc01ip02
UDEVD_EVENT=1
\end{verbatim}
\end{block}
\end{frame}

\begin{frame}
  \frametitle{Misc udev utilities}
  \begin{itemize}
  \item \code{udevinfo}\\
    Lets users query the \code{udev} database.
  \item \code{udevtest <sysfs_device_path>} \\
    Simulates a \code{udev} run to test the configured rules.
  \end{itemize}
\end{frame}

\begin{frame}
  \frametitle{Firmware hotplugging}
  Also implemented with \code{udev}!
  \begin{itemize}
  \item Firmware data are kept outside device drivers
    \begin{itemize}
    \item May not be legal or free enough to distribute
    \item Firmware in kernel code would occupy memory permanently,
      even if just used once.
    \end{itemize}
  \item Kernel configuration: needs to be set in \code{CONFIG_FW_LOADER}\\
    ({\em Device Drivers} $\rightarrow$ {\em Generic Driver Options}
    $\rightarrow$ {\em hotplug firmware loading support})
  \end{itemize}
\end{frame}

\begin{frame}
  \frametitle{Firmware hotplugging implementation}
  \begin{center}
    \includegraphics[width=\textwidth]{slides/sysdev-udev/firmware-hotplugging.pdf}
  \end{center}
  See \kerneldoc{firmware_class/} in the kernel sources for a
  nice overview
\end{frame}

\begin{frame}
  \frametitle{udev files}
  \begin{itemize}
  \item \code{/etc/udev/udev.conf}\\
    \code{udev} configuration file.\\
    Mainly used to configure syslog reporting priorities.\\
    Example setting: \code{udev_log="err"}
  \item \code{/lib/udev/rules.d/}\\
    Standard \code{udev} event matching rules, installed by the distribution.
  \item \code{/etc/udev/rules.d/*.rules}\\
    Local (custom) \code{udev} event matching rules. Best to modify these.
  \item \code{/lib/udev/devices/*}\\
    static \code{/dev} content (such as \code{/dev/console},
    \code{/dev/null}...).
  \item \code{/lib/udev/*}\\
    helper programs called from \code{udev} rules.
  \item \code{/dev/*}\\
    Created device files.
  \end{itemize}
\end{frame}

\begin{frame}[fragile]
  \frametitle{Kernel configuration for udev}
  \begin{itemize}
  \item Created for 2.6.19
  \item Caution: no documentation found, and not tested yet on a minimalistic system. Some settings may still be missing.
  \item Subsystems and device drivers (USB, PCI, PCMCIA...) should be added too!
  \end{itemize}
  \begin{block}{}
\tiny
\begin{verbatim}
# General setup
CONFIG_HOTPLUG=y
# Networking, networking options
CONFIG_NET=y
# Unix domain sockets
CONFIG_UNIX=y
CONFIG_NETFILTER_NETLINK=y
CONFIG_NETFILTER_NETLINK_QUEUE=y
# Pseudo filesystems
CONFIG_PROC_FS=y
CONFIG_SYSFS=y
# Needed to manage /dev
CONFIG_TMPFS=y
CONFIG_RAMFS=y
\end{verbatim}
  \end{block}
\end{frame}

\begin{frame}
  \frametitle{udev summary - typical operation}
  \begin{center}
    \includegraphics[width=0.8\textwidth]{slides/sysdev-udev/udev-overall-architecture.pdf}
  \end{center}
\end{frame}

\begin{frame}
  \frametitle{udev resources}
  \begin{itemize}
  \item Home page\\
    \url{http://kernel.org/pub/linux/utils/kernel/hotplug/udev.html}
  \item Sources\\
    \url{http://kernel.org/pub/linux/utils/kernel/hotplug/}
  \item The \code{udev} manual page:\\
    \code{man udev}
  \end{itemize}
\end{frame}

\begin{frame}
  \frametitle{mdev, the udev for embedded systems}
  \begin{itemize}
  \item {\em udev} might be too heavy-weight for some embedded
    systems, the \code{udevd} daemon staying in the background waiting for
    events.
  \item BusyBox provides a simpler alternative called \code{mdev},
    available by enabling the \code{MDEV} configuration option.
  \item \code{mdev}'s usage is documented in \code{doc/mdev.txt} in the BusyBox source code.
  \item\code{mdev}mdev is also able to load firmware to the kernel like
    \code{udev}
  \end{itemize}
\end{frame}

\begin{frame}
  \frametitle{mdev usage}
  \begin{itemize}
  \item To use \code{mdev}, the \code{proc} and \code{sysfs} filesystems must be mounted
  \item \code{mdev} must be enabled as the hotplug event manager\\
    \code{echo /sbin/mdev > /proc/sys/kernel/hotplug}
  \item Need to mount \code{/dev} as a \code{tmpfs}:\\
    \code{mount -t tmpfs mdev /dev}
  \item Tell \code{mdev} to create the \code{/dev} entries
    corresponding to the devices detected
    during boot when \code{mdev} was not running:\\
    \code{mdev -s}
  \item The behavior is specified by the \code{/etc/mdev.conf} configuration
    file, with the following format\\
    \code{<device regex> <uid>:<gid> <octal permissions> [=path] [@|$|*<command>]}
  \item Example\\
    \code{hd[a-z][0-9]* 0:3 660}
  \end{itemize}
\end{frame}

