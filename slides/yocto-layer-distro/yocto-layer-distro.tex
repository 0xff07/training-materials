\section{Distro Layers}

\subsection{Distro Layers}

\begin{frame}{Distro layers}
  \begin{center}
    \includegraphics[width=\textwidth]{slides/yocto-layer-distro/yocto-layer-distro.pdf}
  \end{center}
\end{frame}

\begin{frame}
  \frametitle{Distro layers}
  \begin{itemize}
    \item You can create a new distribution by using a Distro layer.
    \item This allows to change the defaults that are used by
      \code{Poky}.
    \item It is useful to distribute changes that have been made in
      \code{local.conf}
  \end{itemize}
\end{frame}

\begin{frame}
  \frametitle{Best practice}
  \begin{itemize}
    \item A distro layer is used to provides policy configurations for
      a custom distribution.
    \item It is a best practice to separate the distro layer from the
      custom layers you may create and use.
    \item It often contains:
      \begin{itemize}
        \item Configuration files.
        \item Specific classes.
        \item Distribution specific recipes: initialization scripts,
          splash screen packages\dots
      \end{itemize}
  \end{itemize}
\end{frame}

\begin{frame}[fragile]
  \frametitle{Creating a Distro layer}
  \begin{itemize}
    \item The configuration file for the distro layer is
      \code{conf/distro/<distro>.conf}
    \item This file must define the \code{DISTRO} variable.
    \item It is possible to inherit configuration from an existing
      distro layer.
    \item You can also use all the \code{DISTRO_*} variables.
    \item Use \code{DISTRO = "<distro>"} in \code{local.conf} to use
      your distro configuration.
  \end{itemize}
  \begin{block}{}
    \begin{minted}[fontsize=\small]{c}
require conf/distro/poky.conf

DISTRO = "distro"
DISTRO_NAME = "distro description"
DISTRO_VERSION = "1.0"

MAINTAINER = "..."
    \end{minted}
  \end{block}
\end{frame}
