\subsection{Device Files}
\begin{frame}
  \frametitle{Devices}
  \begin{itemize}
  \item One of the kernel important role is to {\bf allow applications
      to access hardware devices}
  \item In the Linux kernel, most devices are presented to user space
    applications through two different abstractions
    \begin{itemize}
    \item {\bf Character} device
    \item {\bf Block} device
    \end{itemize}
  \item Internally, the kernel identifies each device by a triplet of
    information
    \begin{itemize}
    \item {\bf Type} (character or block)
    \item {\bf Major} (typically the category of device)
    \item {\bf Minor} (typically the identifier of the device)
    \end{itemize}
  \end{itemize}
\end{frame}

\begin{frame}
  \frametitle{Types of devices}
  \begin{itemize}
  \item Block devices
    \begin{itemize}
    \item A device composed of fixed-sized blocks, that can be read
      and written to store data
    \item Used for hard disks, USB keys, SD cards, etc.
    \end{itemize}
  \item Character devices
    \begin{itemize}
    \item Originally, an infinite stream of bytes, with no beginning,
      no end, no size. The pure example: a serial port.
    \item Used for serial ports, terminals, but also sound cards,
      video acquisition devices, frame buffers
    \item Most of the devices that are not block devices are
      represented as character devices by the Linux kernel
    \end{itemize}
  \end{itemize}
\end{frame}
