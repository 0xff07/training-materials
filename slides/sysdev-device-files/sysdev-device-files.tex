\begin{frame}
  \frametitle{Devices: everything is a file}
  \begin{itemize}
  \item A very important Unix design decision was to represent most of
    the ``system objects'' as files
  \item It allows applications to manipulate all “system objects” with
    the normal file API (\code{open}, \code{read}, \code{write},
    \code{close}, etc.)
  \item So, devices had to be represented as files to the applications
  \item This is done through a special artifact called a {\bf device
      file}
  \item It is a special type of file, that associates a file name
    visible to user space applications to the triplet {\em (type,
      major, minor)} that the kernel understands
  \item All {\em device files} are by convention stored in the
    \code{/dev} directory
  \end{itemize}
\end{frame}

\begin{frame}[fragile]
\frametitle{Device files examples}

Example of device files in a Linux system

\small
\begin{verbatim}
$ ls -l /dev/ttyS0 /dev/tty1 /dev/sda1 /dev/sda2 /dev/zero
brw-rw---- 1 root disk    8,  1 2011-05-27 08:56 /dev/sda1
brw-rw---- 1 root disk    8,  2 2011-05-27 08:56 /dev/sda2
crw------- 1 root root    4,  1 2011-05-27 08:57 /dev/tty1
crw-rw---- 1 root dialout 4, 64 2011-05-27 08:56 /dev/ttyS0
crw-rw-rw- 1 root root    1,  5 2011-05-27 08:56 /dev/zero
\end{verbatim}
\normalsize

Example C code that uses the usual file API to write data to a serial port

\small
\begin{minted}{c}
int fd;
fd = open("/dev/ttyS0", O_RDWR);
write(fd, "Hello", 5);
close(fd);
\end{minted}
\end{frame}

\begin{frame}
  \frametitle{Creating device files}
  \begin{itemize}
    \item Before Linux 2.6.32, on basic Linux systems,
    the device files had to be created manually using the
    \code{mknod} command
    \begin{itemize}
    \item \code{mknod /dev/<device> [c|b] major minor}
    \item Needed root privileges
    \item Coherency between device files and devices handled by the
      kernel was left to the system developer
    \end{itemize}
  \item The \code{devtmpfs} virtual filesystem can be mounted on
    \code{/dev} and contains all the devices known to the kernel.
    The \code{CONFIG_DEVTMPFS_MOUNT} kernel configuration option
    makes the kernel mount it automatically at boot time, except
    when booting on an initramfs.
  \end{itemize}
\end{frame}
