\section{Applications}

\begin{frame}
\frametitle{strace}
\begin{itemize}
	\item Allows to trace all the system calls made by an
              application and its children. 
	\item Useful to:
	\begin{itemize}
		\item Understand how time is spent in userspace
 		\item For example, easy to find file open attempts (\code{open()}),
		      file access (\code{read()}, \code{write()}), and 
		      memory allocations (\code{mmap2()}). Can be done 
		      without any access to source code!
		\item Find the biggest time consumers
 		      (low hanging fruit)
  		\item Find unnecessary work done in applications
		      and scripts. Example: opening the same file(s)
		      multiple times, or trying to open files that
		      do not exist.
	\end{itemize}
	\item Limitation: you can't trace the \code{init} process!
\end{itemize}
\end{frame}

\begin{frame}
\frametitle{Using strace}
\begin{itemize}
	\item \code{strace} can be compiled by your build system
	\item Even easier: drop a ready-made static binary for your
	      architecture, just when you need it. See 
	      \url{http://git.free-electrons.com/users/michael-opdenacker/static-binaries/tree/strace}
	\item Recommended usage: \\
	      \code{strace -f -tt -o strace.log <program> <arguments>}
        \begin{itemize}
	      \item \code{-f}: follow child processes
	      \item \code{-tt}: display timestamps with microsecond accuracy  
	\end{itemize}
\end{itemize}
\end{frame}

\begin{frame}[fragile]
\frametitle{strace output}
\begin{block}{}
\tiny
\begin{verbatim}
     0.000625 access("/etc/ld.so.nohwcap", F_OK) = -1 ENOENT (No such file or directory)
     0.000877 open("/lib/libgstaudio-0.10.so.0", O_RDONLY|O_CLOEXEC) = -1 ENOENT (No such file or directory)
     0.335815 open("/usr/lib/libgstaudio-0.10.so.0", O_RDONLY|O_CLOEXEC) = 12
     0.000778 read(12, "\177ELF\1\1\1\0\0\0\0\0\0\0\0\0\3\0(\0\1\0\0\0(q\0\0004\0\0\0"..., 512) = 512
     0.000699 lseek(12, 171824, SEEK_SET) = 171824
     0.000541 read(12, "\0\0\0\0\0\0\0\0\0\0\0\0\0\0\0\0\0\0\0\0\0\0\0\0\0\0\0\0\0\0\0\0"..., 1040) = 1040
     0.000775 lseek(12, 171536, SEEK_SET) = 171536
     0.000535 read(12, "A8\0\0\0aeabi\0\1.\0\0\0\0057-A\0\6\n\7A\10\1\t\2\n\6\22"..., 57) = 57
     0.000678 fstat64(12, {st_mode=S_IFREG|0755, st_size=172864, ...}) = 0
     0.000714 mmap2(NULL, 204460, PROT_READ|PROT_EXEC, MAP_PRIVATE|MAP_DENYWRITE, 12, 0) = 0xb4d82000
     0.000684 mprotect(0xb4dab000, 28672, PROT_NONE) = 0
     0.000618 mmap2(0xb4db2000, 8192, PROT_READ|PROT_WRITE, MAP_PRIVATE|MAP_FIXED|MAP_DENYWRITE, 12, 0x28) = 0xb4db2000
     0.000875 close(12)                 = 0
     0.005504 munmap(0xb4ea7000, 64)    = 0
     0.004886 open("/proc/self/cmdline", O_RDONLY|O_LARGEFILE) = 12
     0.001027 fstat64(12, {st_mode=S_IFREG|0444, st_size=0, ...}) = 0
     0.000780 fcntl64(12, F_GETFL)      = 0x20000 (flags O_RDONLY|O_LARGEFILE)
     0.000520 fstat64(12, {st_mode=S_IFREG|0444, st_size=0, ...}) = 0
     0.000645 mmap2(NULL, 4096, PROT_READ|PROT_WRITE, MAP_PRIVATE|MAP_ANONYMOUS, -1, 0) = 0xb4ea7000
     0.000673 _llseek(12, 0, [0], SEEK_CUR) = 0
     0.000813 read(12, "./HomeAutomation\0-qws\0", 4096) = 22
     0.000776 read(12, "", 3072)        = 0
     0.000848 close(12)                 = 0
     0.000438 munmap(0xb4ea7000, 4096)  = 0
     0.006391 clock_gettime(CLOCK_MONOTONIC, {251, 636907995}) = 0
     0.001903 uname({sys="Linux", node="buildroot", ...}) = 0
     0.049023 socket(PF_FILE, SOCK_STREAM|SOCK_CLOEXEC|SOCK_NONBLOCK, 0) = 12
     0.000664 connect(12, {sa_family=AF_FILE, sun_path="/var/run/nscd/socket"}, 110) = -1 ENOENT (No such file or directory)
     0.000815 close(12)                 = 0
\end{verbatim}
\end{block}
\end{frame}

\begin{frame}
\frametitle{oprofile}
\begin{itemize}
	\item Two ways of working: {\em legacy} mode and {\em
              perf\_events} mode
	\item {\em legacy} mode:
	\begin{itemize}
		\item Low accuracy, use a kernel driver to profile
		\item \code{CONFIG_OPROFILE}
		\item Userspace tools: \code{opcontrol} and \code{oprofiled}
	\end{itemize}
	\item {\em perf\_events} mode:
	\begin{itemize}
		\item Uses hardware performance counters
		\item \code{CONFIG_PERF_EVENTS} and \code{CONFIG_HW_PERF_EVENTS}
		\item Userspace tools: \code{operf}
	\end{itemize}
\end{itemize}
\end{frame}

\begin{frame}[fragile]
\frametitle{oprofile: usage}
\begin{itemize}
	\item {\em legacy} mode:
	\begin{block}{}
	\begin{verbatim}
		opcontrol --vmlinux=/path/to/vmlinux # optional step
		opcontrol --start
		/my/command
		opcontrol --stop
	\end{verbatim}
	\end{block}
	\item {\em perf\_events} mode:
	\begin{block}{}
	\begin{verbatim}
		operf --vmlinux=/path/to/vmlinux /my/command
	\end{verbatim}
	\end{block}
	\item Get the results with:
	\begin{block}{}
	\begin{verbatim}
		opreport
	\end{verbatim}
	\end{block}
\end{itemize}
\end{frame}

\begin{frame}[fragile]
\frametitle{oprofile}
\begin{block}{}
\tiny
\begin{verbatim}
# opreport
Using /var/lib/oprofile/samples/ for samples directory.
CPU: CPU with timer interrupt, speed 393 MHz (estimated)
Profiling through timer interrupt
          TIMER:0|
  samples|      %|
------------------
     1540 78.3715 no-vmlinux
      105  5.3435 libQtGui.so.4.8.4
       66  3.3588 libc-2.15.so
       64  3.2570 libQtCore.so.4.8.4
       58  2.9517 ld-2.15.so
       45  2.2901 libgobject-2.0.so.0.3000.3
       37  1.8830 libgstreamer-0.10.so.0.30.0
       13  0.6616 libglib-2.0.so.0.3000.3
        9  0.4580 libQtScript.so.4.8.4
        6  0.3053 libgcc_s.so.1
        4  0.2036 libQtDeclarative.so.4.8.4
        4  0.2036 libstdc++.so.6.0.17
        3  0.1527 libpthread-2.15.so
        2  0.1018 busybox
        2  0.1018 libQtSvg.so.4.8.4
        2  0.1018 libQtWebKit.so.4.9.3
        2  0.1018 libgthread-2.0.so.0.3000.3
        1  0.0509 HomeAutomation
        1  0.0509 libQtNetwork.so.4.8.4
        1  0.0509 libphonon_gstreamer.so
\end{verbatim}
\end{block}
\end{frame}

\begin{frame}[fragile]
\frametitle{perf}
\begin{itemize}
	\item Uses hardware performance counters
	\item \code{CONFIG_PERF_EVENTS} and \code{CONFIG_HW_PERF_EVENTS}
	\item Userspace tool: \code{perf}. It is part of the kernel
		sources so it is always in sync with your kernel.
	\item Usage:
	\begin{block}{}
	\begin{verbatim}
		perf record /my/command
	\end{verbatim}
	\end{block}
	\item Get the results with:
	\begin{block}{}
	\begin{verbatim}
		perf report
	\end{verbatim}
	\end{block}
\end{itemize}
\end{frame}

\begin{frame}[fragile]
\frametitle{perf}
\begin{block}{}
\tiny
\begin{verbatim}
    20.91%  gst-launch-0.10  libavcodec.so.53.35.0        [.] 0x00000000003bdaa1                           
    15.45%  gst-launch-0.10  libgstflump3dec.so           [.] 0x0000000000014b42                           
     3.16%  gst-launch-0.10  libglib-2.0.so.0.3600.2      [.] 0x00000000000882c9                           
     2.99%  gst-launch-0.10  libc-2.17.so                 [.] __memcpy_ssse3_back                          
     2.37%  gst-launch-0.10  liboil-0.3.so.0.3.0          [.] 0x000000000004417e                           
     2.24%  gst-launch-0.10  libgobject-2.0.so.0.3600.2   [.] g_type_value_table_peek                      
     1.53%  gst-launch-0.10  libc-2.17.so                 [.] vfprintf                                     
     1.37%  gst-launch-0.10  libgstreamer-0.10.so.0.30.0  [.] 0x0000000000026fd8                           
     1.29%  gst-launch-0.10  ld-2.17.so                   [.] do_lookup_x                                  
     0.99%  gst-launch-0.10  libpthread-2.17.so           [.] pthread_mutex_lock                           
     0.98%  gst-launch-0.10  libgobject-2.0.so.0.3600.2   [.] g_type_check_value                           
     0.93%  gst-launch-0.10  libgstavi.so                 [.] 0x00000000000119f9                           
     0.88%  gst-launch-0.10  libgstreamer-0.10.so.0.30.0  [.] gst_value_list_get_type                      
     0.85%  gst-launch-0.10  libc-2.17.so                 [.] __random                                     
     0.66%  gst-launch-0.10  [kernel.kallsyms]            [k] clear_page_c_e                               
     0.62%  gst-launch-0.10  [kernel.kallsyms]            [k] try_to_wake_up                               
     0.61%  gst-launch-0.10  [kernel.kallsyms]            [k] page_fault                                   
     0.58%  gst-launch-0.10  libgobject-2.0.so.0.3600.2   [.] g_type_is_a                                  
     0.57%  gst-launch-0.10  libc-2.17.so                 [.] __strcmp_sse42                               
     0.57%  gst-launch-0.10  [kernel.kallsyms]            [k] radix_tree_lookup_element                    
     0.57%  gst-launch-0.10  libc-2.17.so                 [.] malloc                                       
     0.57%  gst-launch-0.10  libc-2.17.so                 [.] _int_malloc                                  
     0.55%  gst-launch-0.10  libgobject-2.0.so.0.3600.2   [.] g_type_check_instance_is_a                   
     0.53%  gst-launch-0.10  [kernel.kallsyms]            [k] __ticket_spin_lock                           
     0.53%  gst-launch-0.10  libgobject-2.0.so.0.3600.2   [.] g_type_check_value_holds                     
     0.53%  gst-launch-0.10  libgstffmpeg.so              [.] 0x000000000001e40c                           
     0.51%  gst-launch-0.10  libgstreamer-0.10.so.0.30.0  [.] gst_structure_id_get_value                   
     0.50%  gst-launch-0.10  libc-2.17.so                 [.] _IO_default_xsputn                           
     0.50%  gst-launch-0.10  [kernel.kallsyms]            [k] tg_load_down                                 
\end{verbatim}
\end{block}
\end{frame}

\begin{frame}
\frametitle{Toolchains}
Try to use an optimized toolchain for your platform. That is something
you'll have to benchmark near the end of your project.
\begin{itemize}
        \item gcc version 4.6.3 (Sourcery CodeBench Lite 2012.03-57: 9.63s
        \item gcc version 4.7.3 20130226 (prerelease) (crosstool-NG
        linaro-1.13.1-4.7-2013.03-20130313 - Linaro GCC 2013.03): 9.26s
\end{itemize}
Results may vary depending on the size/features of your root filesystem.
Here, we mainly benchmark the kernel performance as the root filesystem
doesn't take much time. Note: Don't compare with a Buildroot toolchain as
those are using {\em glibc}/{\em eglibc} and not {\em uClibc} and then
the initramfs has a quite different size. Compiling the kernel with a
Linaro toolchain and using a Buildroot generated root filesystem yields
the same results as compiling everything with Buildroot.
\end{frame}

\begin{frame}
\frametitle{Linker optimizations}
Group application code used at startup
\begin{itemize}
        \item Find the functions called during startup, for example using
                the \code{-finstrument-functions} gcc option.
        \item Create a custom linker script to reorder those functions in
                the call order. You can acheive that by putting each function
                in their own section using the \code{-ffunction-sections} gcc
                option.
        \item Particularly useful for flash storage with rather big MTD
                read blocks. As the whole read blocks are read, you end up
                reading unnecessary data.
\end{itemize}
See \url{http://blogs.linux.ie/caolan/2007/04/24/controlling-symbol-ordering/}
\end{frame}

\begin{frame}
\frametitle{Prelink}
\begin{itemize}
\item Prelinking reduces the time needed to start an executable
\item It is extensively used on Android
\item It has to be configured to know which libraries needs to be
      prelinked and will assign a fixed address for each available
      symbol thus removing the need to relocate symbols when starting
      an executable.
\item Be careful of security implications, as executable code is
      always loaded at the same address.
\item Code and paper at
      \url{http://people.redhat.com/jakub/prelink/}
\end{itemize}
\end{frame}

\setuplabframe
{Reducing time starting applications}
{
\begin{itemize}
\item Trace and profile the main application with \code{strace}
\end{itemize}
}

