\subsection{Add a New Product}

\begin{frame}
  \frametitle{Defining new products}
  \begin{itemize}
  \item Devices are well supported by the Android build system. It allows
    to build multiple devices with the same source tree, to have a
    per-device configuration, etc.
  \item All the product definitions should be put in
    \code{device/<company>/<device>}
  \item The entry point is the \code{AndroidProducts.mk} file, which
    should define the \code{PRODUCT_MAKEFILES} variable
  \item This variable defines where the actual product definitions are
    located.
  \item It follows such an architecture because you can have several
    products using the same device
  \item If you want your product to appear in the \code{lunch} menu, you need
    to create a \code{vendorsetup.sh} file in the \code{device} directory,
    with the right calls to \code{add_lunch_combo}
  \end{itemize}
\end{frame}

\begin{frame}
  \frametitle{Product, devices and boards}
  \begin{center}
    \includegraphics[height=0.8\textheight]{slides/android-build-system-product/product-definition.pdf}
  \end{center}
\end{frame}

\begin{frame}[fragile]
  \frametitle{Minimal Product Declaration}
\begin{minted}[fontsize=\small]{make}
$(call inherit-product, build/target/product/generic.mk)

PRODUCT_NAME := full_MyDevice
PRODUCT_DEVICE := MyDevice
PRODUCT_MODEL := Full flavor of My Brand New Device
\end{minted}
\end{frame}

\begin{frame}[fragile]
  \frametitle{Copy files to the target}
\begin{minted}[fontsize=\small]{make}
$(call inherit-product, build/target/product/generic.mk)

PRODUCT_COPY_FILES += \
  device/mybrand/mydevice/vold.fstab:system/etc/vold.fstab

PRODUCT_NAME := full_MyDevice
PRODUCT_DEVICE := MyDevice
PRODUCT_MODEL := Full flavor of My Brand New Device
\end{minted}
\end{frame}

\begin{frame}[fragile]
  \frametitle{Add a package to the build for this product}
\begin{minted}[fontsize=\small]{make}
$(call inherit-product, build/target/product/generic.mk)

PRODUCT_PACKAGES += FooBar

PRODUCT_COPY_FILES += \
  device/mybrand/mydevice/vold.fstab:system/etc/vold.fstab

PRODUCT_NAME := full_mydevice
PRODUCT_DEVICE := mydevice
PRODUCT_MODEL := Full flavor of My Brand New Device
\end{minted}
\end{frame}

\begin{frame}
  \frametitle{Overlays}
  \begin{itemize}
  \item This is a mechanism used by products to override resources
    already defined in the source tree, without modifying the original
    code
  \item This is used for example to change the wallpaper for one
    particular device
  \item Use the \code{DEVICE_PACKAGE_OVERLAYS} or
    \code{PRODUCT_PACKAGE_OVERLAYS} variables that you set to a path
    to a directory in your device folder
  \item This directory should contain a structure similar to the
    source tree one, with only the files that you want to override
  \end{itemize}
\end{frame}


\begin{frame}[fragile]
  \frametitle{Add a device overlay}
\begin{minted}[fontsize=\small]{make}
$(call inherit-product, build/target/product/generic.mk)

PRODUCT_PACKAGES += FooBar

PRODUCT_COPY_FILES += \
  device/mybrand/mydevice/vold.fstab:system/etc/vold.fstab

DEVICE_PACKAGE_OVERLAYS := device/mybrand/mydevice/overlay

PRODUCT_NAME := full_mydevice
PRODUCT_DEVICE := mydevice
PRODUCT_MODEL := Full flavor of My Brand New Device
\end{minted}
\end{frame}

\begin{frame}
  \frametitle{Board Definition}
  \begin{itemize}
  \item You will also need a \code{BoardConfig.mk} file along with the
    product definition
  \item While the product definition was mostly about the build system
    in itself, the board definition is more about the hardware
  \item However, this is poorly documented and sometimes ambiguous so
    you will probably have to dig into the \code{build/core/Makefile}
    at some point to see what a given variable does
  \end{itemize}
\end{frame}

\begin{frame}[fragile]
  \frametitle{Minimal Board Definition}
\begin{minted}{make}
TARGET_NO_BOOTLOADER := true
TARGET_NO_KERNEL := true
TARGET_CPU_ABI := armeabi
HAVE_HTC_AUDIO_DRIVER := true
BOARD_USES_GENERIC_AUDIO := true

USE_CAMERA_STUB := true
\end{minted}
\end{frame}

\begin{frame}
  \frametitle{Other Board Variables 1/2}
  \begin{itemize}
  \item \code{TARGET_ARCH_VARIANT}
    \begin{itemize}
    \item Variant of the selected architecture (for example
      \code{armv7-a-neon} for most Cortex-A8 and A9 CPUs)
    \end{itemize}
  \item \code{TARGET_EXTRA_CFLAGS}
    \begin{itemize}
    \item Extra C compiler flags to use during the whole build
    \end{itemize}
  \item \code{TARGET_CPU_SMP}
    \begin{itemize}
    \item Does the CPU have multiple cores?
    \end{itemize}
  \item \code{TARGET_USERIMAGES_USE_EXT4}
    \begin{itemize}
    \item We want to use ext4 as filesystem for our generated
      partitions
    \end{itemize}
  \item \code{BOARD_SYSTEMIMAGE_PARTITION_SIZE}
    \begin{itemize}
    \item Size of the system partitions in bytes.
    \end{itemize}
  \end{itemize}
\end{frame}

\begin{frame}
  \frametitle{Other Board Variables 2/2}
  \begin{itemize}
  \item \code{BOARD_NAND_PAGE_SIZE}
    \begin{itemize}
    \item For NAND flash, size of the pages as given by the datasheet
    \end{itemize}
  \item \code{TARGET_NO_RECOVERY}
    \begin{itemize}
    \item We don't want to build the recovery image
    \end{itemize}
  \item \code{BOARD_KERNEL_CMDLINE}
    \begin{itemize}
    \item Boot arguments of the kernel
    \end{itemize}
  \end{itemize}
\end{frame}

\begin{frame}[fragile]
  \frametitle{Kernel Integration into Android}
  \begin{itemize}
  \item Android is just a user-space software stack, the build system
    isn't designed to build the kernel
  \item However, there is some facilities to integrate a precompiled
    kernel kernel into the Android images
  \item To do so, you need to:
    \begin{itemize}
    \item In \code{BoardConfig.mk}
      \begin{itemize}
      \item Remove \code{TARGET_NO_KERNEL} if set
      \item Set \code{BOARD_KERNEL_BASE} to the load address of your
        kernel
      \end{itemize}
    \item In your device Makefile, have something like
\begin{minted}[fontsize=\scriptsize]{make}
ifeq ($(TARGET_PREBUILT_KERNEL),)
LOCAL_KERNEL := device/ti/panda/kernel
else
LOCAL_KERNEL := $(TARGET_PREBUILT_KERNEL)
endif

PRODUCT_COPY_FILES := \
        $(LOCAL_KERNEL):kernel
\end{minted}
    \end{itemize}
  \end{itemize}
\end{frame}
