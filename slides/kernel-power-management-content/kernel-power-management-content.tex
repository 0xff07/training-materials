\begin{frame}
  \frametitle{PM building blocks}
  \begin{itemize}
  \item Several power management \emph{building blocks}
    \begin{itemize}
    \item Clock framework
    \item Suspend and resume
    \item CPUidle
    \item Runtime power management
    \item Power domains
    \item Frequency and voltage scaling
    \end{itemize}
  \item Independent \emph{building blocks} that can be improved
    gradually during development
  \end{itemize}
\end{frame}

\begin{frame}
  \frametitle{Clock framework (1)}
  \begin{itemize}
  \item Generic framework to manage clocks used by devices in the
    system
  \item Allows to reference count clock users and to shutdown the
    unused clocks to save power
  \item Simple API described in \kfile{include/linux/clk.h}.
    \begin{itemize}
    \item \kfunc{clk_get} to get a reference to a clock
    \item \kfunc{clk_enable} to start the clock
    \item \kfunc{clk_disable} to stop the clock
    \item \kfunc{clk_put} to free the clock source
    \item \kfunc{clk_get_rate} to get the current rate
    \end{itemize}
  \end{itemize}
\end{frame}


\begin{frame}{Clock framework (2)}
  The common clock framework
  \begin{itemize}
  \item Allows to declare the available clocks and their association
    to devices in the Device Tree
  \item Provides a {\em debugfs} representation of the clock tree
  \item Is implemented in \kdir{drivers/clk}
  \end{itemize}
\end{frame}

\begin{frame}{Diagram overview of the common clock framework}
  \begin{center}
    \includegraphics[width=\textwidth]{slides/kernel-power-management-content/clock-framework.pdf}
\end{center}
\end{frame}

\begin{frame}{Clock framework (3)}
  The interface of the CCF divided into two halves:
  \begin{itemize}
  \item Common Clock Framework core
    \begin{itemize}
    \item Common definition of \kstruct{clk}
    \item Common implementation of the \code{clk.h} API (defined in
      \kfile{drivers/clk/clk.c})
    \item \kstruct{clk_ops}: operations invoked by the clk API
      implementation
    \item Not supposed to be modified when adding a new driver
    \end{itemize}
  \item Hardware-specific
    \begin{itemize}
    \item Callbacks registered with \kstruct{clk_ops} and the
      corresponding hardware-specific structures
    \item Has to be written for each new hardware clock
    \item Example: \kfile{drivers/clk/mvebu/clk-cpu.c}
    \end{itemize}
  \end{itemize}
\end{frame}

\begin{frame}{Clock framework (4)}
  Hardware clock operations: device tree
  \begin{itemize}
  \item The \textbf{device tree} is the \textbf{mandatory way} to
    declare a clock and to get its resources, as for any other
    driver using DT we have to:
    \begin{itemize}
    \item \textbf{Parse} the device tree to \textbf{setup} the
      clock: the resources but also the properties are retrieved.
    \item Declare the \textbf{compatible} clocks and associate each
      to an \textbf{initialization} function using
      \kfunc{CLK_OF_DECLARE}
    \item Example: \kfile{arch/arm/boot/dts/armada-xp.dtsi} and
	\kfile{drivers/clk/mvebu/armada-xp.c}
    \end{itemize}
  \end{itemize}
  See our presentation about the Clock Framework for more details: \url{https://frama.link/MYevkXcD}
\end{frame}

\begin{frame}
  \frametitle{Suspend and resume}
  \begin{itemize}
  \item Infrastructure in the kernel to support suspend and resume
  \item System on Chip hooks
    \begin{itemize}
    \item \code{prepare()}, \code{enter()}, \code{finish()},
      \code{valid()} in a \kstruct{platform_suspend_ops} structure
    \item Registered using the \kfunc{suspend_set_ops} function
    \item See \kfile{arch/arm/mach-at91/pm.c}
    \end{itemize}
  \item Device driver hooks
    \begin{itemize}
    \item \code{pm} operations (\code{suspend()} and
      \code{resume()} hooks) in the
      \kstruct{device_driver} as a \kstruct{dev_pm_ops}
      structure in (\kstruct{platform_driver}, \kstruct{usb_driver}, etc.)
    \item See \kfile{drivers/net/ethernet/cadence/macb_main.c}
    \end{itemize}
  \end{itemize}
\end{frame}

\begin{frame}
  \frametitle{Triggering suspend}
  \begin{itemize}
  \item \kstruct{suspend_ops} functions are called by the
    \kfunc{enter_state} function.
  \item \kfunc{enter_state} also takes care of executing the suspend
    and resume functions for your devices.
  \item The execution of this function can be triggered from
    user space. To suspend to RAM:
    \begin{itemize}
    \item \code{echo mem > /sys/power/state}
    \end{itemize}
  \item Can also use the s2ram program from
    \url{http://suspend.sourceforge.net/}
  \item Read \kfile{kernel/power/suspend.c}
  \end{itemize}
\end{frame}

\begin{frame}
  \frametitle{Saving power in the idle loop}
  \begin{itemize}
  \item The idle loop is what you run when there's nothing left to run
    in the system.
  \item Implemented in all architectures in
    \code{arch/<arch>/kernel/process.c}
  \item Example to read: look for \code{cpu_idle} in
    \kfile{arch/arm/kernel/process.c}
  \item Each ARM cpu defines its own \code{arch_idle} function.
  \item The CPU can run power saving HLT instructions, enter NAP mode,
    and even disable the timers (tickless systems).
  \item See also \url{https://en.wikipedia.org/wiki/Idle_loop}
  \end{itemize}
\end{frame}

\begin{frame}
  \frametitle{Managing idle}
  Adding support for multiple idle levels
  \begin{itemize}
  \item Modern CPUs have several sleep states offering different
    power savings with associated wake up latencies
  \item The {\em dynamic tick} feature allows to remove the
    periodic tick to save power, and to know when the next event is
    scheduled, for smarter sleeps.
  \item CPUidle infrastructure to change sleep states
    \begin{itemize}
    \item Platform-specific driver defining sleep states and
      transition operations
    \item Platform-independent governors (ladder and menu)
    \item Available in particular for x86/ACPI and most ARM SoCs
    \item See \kerneldoctext{cpuidle/} in kernel sources.
    \end{itemize}
  \end{itemize}
\end{frame}

\begin{frame}
  \frametitle{PowerTOP}
  \url{https://01.org/powertop/}
  \begin{itemize}
  \item With dynamic ticks, allows to fix parts of kernel code and
    applications that wake up the system too often.
  \item PowerTOP allows to track the worst offenders
  \item Now available on ARM cpus implementing CPUidle
  \item Also gives you useful hints for reducing power.
  \end{itemize}
\end{frame}

\begin{frame}
  \frametitle{Runtime power management}
  \begin{itemize}
  \item Managing per-device idle, each device being managed by its
        device driver independently from others.
  \item According to the kernel configuration interface: \emph{Enable
      functionality allowing I/O devices to be put into energy-saving
      (low power) states at run time (or autosuspended) after a
      specified period of inactivity and woken up in response to a
      hardware-generated wake-up event or a driver's request.}
  \item New hooks must be added to the drivers:
    \code{runtime_suspend()}, \code{runtime_resume()},
    \code{runtime_idle()} in the \kstruct{dev_pm_ops}
    structure in \kstruct{device_driver}.
  \item API and details on \kerneldoctext{power/runtime_pm.txt}
  \item See \kfile{drivers/net/ethernet/cadence/macb_main.c} again.
  \end{itemize}
\end{frame}

\begin{frame}
  \frametitle{Generic PM Domains (genpd)}
  \begin{itemize}
  \item Generic infrastructure to implement power domains based on
        Device Tree descriptions, allowing to group devices by the
        physical power domain they belong to.
        This sits at the same level as bus type for calling PM hooks.
  \item All the devices in the same PD get the same state at the same time.
  \item Specifications and examples available at \kerneldoctext{devicetree/bindings/power/power_domain.txt}
  \item Driver example: \kfile{drivers/soc/rockchip/pm_domains.c}
        (\kfunc{rockchip_pd_power_on}, \kfunc{rockchip_pd_power_off},
        \kfunc{rockchip_pm_add_one_domain}...)
  \item DT example: look for \kcompat{rockchip\%2Cpx30-power-controller}{rockchip,px30-power-controller}
        (\kfile{arch/arm64/boot/dts/rockchip/px30.dtsi}) and find PD definitions
        and corresponding devices.
  \item See Kevin Hilman's talk at Kernel Recipes 2017: \url{https://youtu.be/SctfvoskABM}
  \end{itemize}
\end{frame}

\begin{frame}
  \frametitle{Frequency and voltage scaling (1)}
  Frequency and voltage scaling possible through the
  \code{cpufreq} kernel infrastructure.
  \begin{itemize}
  \item Generic infrastructure: \kfile{drivers/cpufreq/cpufreq.c} and
    \kfile{include/linux/cpufreq.h}
  \item Generic governors, responsible for deciding frequency and
    voltage transitions
    \begin{itemize}
    \item \code{performance}: maximum frequency
    \item \code{powersave}: minimum frequency
    \item \code{ondemand}: measures CPU consumption to adjust frequency
    \item \code{conservative}: often better than
      \code{ondemand}. Only increases frequency gradually when the
      CPU gets loaded.
    \item \code{userspace}: leaves the decision to a user space
      daemon.
    \end{itemize}
  \item This infrastructure can be controlled from
    \code{/sys/devices/system/cpu/cpu<n>/cpufreq/}
  \end{itemize}
\end{frame}

\begin{frame}
  \frametitle{Frequency and voltage scaling (2)}
  \begin{itemize}
  \item CPU drivers in \kdir{drivers/cpufreq}.  Example:
    \kfile{drivers/cpufreq/omap-cpufreq.c}
  \item Must implement the operations of the \code{cpufreq_driver}
    structure and register them using \code{cpufreq_register_driver()}
    \begin{itemize}
    \item \code{init()} for initialization
    \item \code{exit()} for cleanup
    \item \code{verify()} to verify the user-chosen policy
    \item \code{setpolicy()} or \code{target()} to actually perform
      the frequency change
    \end{itemize}
  \item See \kerneldoctext{cpu-freq/} for useful explanations
  \end{itemize}
\end{frame}

\begin{frame}
  \frametitle{Regulator framework}
  \begin{itemize}
  \item Modern embedded hardware have hardware responsible for voltage
    and current regulation
  \item The regulator framework allows to take advantage of this
    hardware to save power when parts of the system are unused
    \begin{itemize}
    \item A consumer interface for device drivers (i.e. users)
    \item Regulator driver interface for regulator drivers
    \item Machine interface for board configuration
    \item sysfs interface for user space
    \end{itemize}
  \item See \kerneldoctext{power/regulator/} in kernel sources.
  \end{itemize}
\end{frame}

\begin{frame}
  \frametitle{BSP work for a new board}
  In case you just need to create a BSP for your board, and your
  CPU already has full PM support, you should just need to:
  \begin{itemize}
  \item Create clock definitions and bind your devices to them.
  \item Implement PM handlers (suspend, resume) in the drivers for
    your board specific devices.
  \item Implement runtime PM handlers in your drivers.
  \item Implement board specific power management if needed (mainly
    battery management)
  \item Implement regulator framework hooks for your board if
    needed.
  \item Attach on-board devices to PM domains if needed
  \item All other parts of the PM infrastructure should be already
    there: suspend / resume, cpuidle, cpu frequency and voltage
    scaling, PM domains.
  \end{itemize}
\end{frame}

\begin{frame}
  \frametitle{Useful resources}
  \begin{itemize}
  \item \kerneldochtmldir{power} in kernel documentation.
    \begin{itemize}
    \item Will give you many useful details.
    \end{itemize}
  \item Introduction to kernel power management, Kevin Hilman (Kernel Recipes 2015)
    \begin{itemize}
      \item \url{https://www.youtube.com/watch?v=juJJZORgVwI}
    \end{itemize}
  \item Tips and ideas for prolonging battery life (Amit Kucheria)
    \begin{itemize}
    \item \url{https://j.mp/fVdxKh}
    \end{itemize}
  \end{itemize}
\end{frame}
