\subsection{Cross-compiling the kernel}

\begin{frame}
  \frametitle{Cross-compiling the kernel}
  When you compile a Linux kernel for another CPU architecture
  \begin{itemize}
  \item Much faster than compiling natively, when the target system is
    much slower than your GNU/Linux workstation.
  \item Much easier as development tools for your GNU/Linux
    workstation are much easier to find.
  \item To make the difference with a native compiler, cross-compiler
    executables are prefixed by the name of the target system,
    architecture and sometimes
    library. Examples:\\
    \small
    \code{mips-linux-gcc}, the prefix is \code{mips-linux-}\\
    \code{arm-linux-gnueabi-gcc}, the prefix is \code{arm-linux-gnueabi-}
  \end{itemize}
\end{frame}

\begin{frame}
  \frametitle{Specifying cross-compilation (1)}

  The CPU architecture and cross-compiler prefix are defined through
  the \code{ARCH} and \code{CROSS_COMPILE} variables in the toplevel
  Makefile.

  \begin{itemize}
  \item \code{ARCH} is the name of the architecture. It is defined by
    the name of the subdirectory in \code{arch/} in the kernel sources
    \begin{itemize}
    \item Example: \code{arm} if you want to compile a kernel for
          the \code{arm} architecture.
    \end{itemize}
  \item \code{CROSS_COMPILE} is the prefix of the cross compilation
    tools
    \begin{itemize}
    \item Example: \code{arm-linux-} if your compiler is \code{arm-linux-gcc}
    \end{itemize}
  \end{itemize}
\end{frame}

\begin{frame}
  \frametitle{Specifying cross-compilation (2)}

  Two solutions to define \code{ARCH} and \code{CROSS_COMPILE}:

  \begin{itemize}
  \item Pass \code{ARCH} and \code{CROSS_COMPILE} on the \code{make}
    command line: \\
    \code{make ARCH=arm CROSS_COMPILE=arm-linux- ...} \\
    Drawback: it is easy to forget to pass these variables when
    you run any \code{make} command, causing your build and
    configuration to be screwed up.
  \item Define \code{ARCH} and \code{CROSS_COMPILE} as environment
    variables: \\
    \code{export ARCH=arm} \\
    \code{export CROSS_COMPILE=arm-linux-} \\
    Drawback: it only works inside the current
    shell or terminal. You could put these settings in a file
    that you source every time you start working on the project.
    If you only work on a single architecture with always the
    same toolchain, you could even put these settings in your
    \code{~/.bashrc} file to make them permanent and visible from
    any terminal.
  \end{itemize}
\end{frame}

\begin{frame}
  \frametitle{Predefined configuration files}
  \begin{itemize}
  \item Default configuration files available, per board or per-CPU
    family
    \begin{itemize}
    \item They are stored in \code{arch/<arch>/configs/}, and are
      just minimal \code{.config} files
    \item This is the most common way of configuring a kernel for
      embedded platforms
    \end{itemize}
  \item Run \code{make help} to find if one is available for your
    platform
  \item To load a default configuration file, just run\\
    \code{make acme_defconfig}
    \begin{itemize}
    \item This will overwrite your existing \code{.config} file!
    \end{itemize}
  \item To create your own default configuration file
    \begin{itemize}
    \item \code{make savedefconfig}, to create a minimal
      configuration file
    \item \code{mv defconfig arch/<arch>/configs/myown_defconfig}
    \end{itemize}
  \end{itemize}
\end{frame}

\begin{frame}
  \frametitle{Configuring the kernel}
  \begin{itemize}
  \item After loading a default configuration file, you can adjust the
    configuration to your needs with the normal \code{xconfig},
    \code{gconfig} or \code{menuconfig} interfaces
  \item As the architecture is different from your host architecture
    \begin{itemize}
    \item Some options will be different from the native configuration
      (processor and architecture specific options, specific drivers,
      etc.)
    \item Many options will be identical (filesystems, network
      protocols, architecture-independent drivers, etc.)
    \end{itemize}
  \end{itemize}
\end{frame}

\begin{frame}
  \frametitle{Device Tree}
  \begin{itemize}
  \item Many embedded architectures have a lot of non-discoverable
    hardware.
  \item Depending on the architecture, such hardware is either
    described using C code directly within the kernel, or using a
    special hardware description language in a {\em Device Tree}.
  \item ARM, PowerPC, OpenRISC, ARC, Microblaze are examples of
    architectures using the Device Tree.
  \item A {\em Device Tree Source}, written by kernel developers,
    is compiled into a binary {\em Device Tree Blob}, passed at boot
    time to the kernel.
    \begin{itemize}
    \item There is one different Device Tree for each board/platform
      supported by the kernel, available in
      \code{arch/arm/boot/dts/<board>.dtb}.
    \end{itemize}
  \item The bootloader must load both the kernel image and the Device
    Tree Blob in memory before starting the kernel.
  \end{itemize}
\end{frame}

\begin{frame}
  \frametitle{Building and installing the kernel}
  \begin{itemize}
  \item Run \code{make}
  \item Copy the final kernel image to the target storage
    \begin{itemize}
    \item can be \code{uImage}, \code{zImage}, \code{vmlinux},
      \code{bzImage} in \code{arch/<arch>/boot}
    \item copying the Device Tree Blob might be necessary as well,
      they are available in \code{arch/<arch>/boot/dts}
    \end{itemize}
  \item \code{make install} is rarely used in embedded development, as the
    kernel image is a single file, easy to handle
    \begin{itemize}
    \item It is however possible to customize the make install
      behaviour in \code{arch/<arch>/boot/install.sh}
    \end{itemize}
  \item \code{make modules_install} is used even in embedded
    development, as it installs many modules and description files
    \begin{itemize}
    \item \code{make INSTALL_MOD_PATH=<dir>/ modules_install}
    \item The \code{INSTALL_MOD_PATH} variable is needed to install
      the modules in the target root filesystem instead of your host
      root filesystem.
    \end{itemize}
  \end{itemize}
\end{frame}

\begin{frame}
  \frametitle{Booting with U-Boot}
  \begin{itemize}
  \item Recent versions of U-Boot can boot the \code{zImage} binary.
  \item Older versions require a special kernel image format:
        \code{uImage}
    \begin{itemize}
    \item \code{uImage} is generated from \code{zImage} using the
      \code{mkimage} tool. It is done automatically by the kernel
      \code{make uImage} target.
    \item On some ARM platforms, \code{make uImage} requires passing a
      \code{LOADADDR} environment variable, which indicates at which
      physical memory address the kernel will be executed.
    \end{itemize}
  \item In addition to the kernel image, U-Boot can also pass a
    {\em Device Tree Blob} to the kernel.
  \item The typical boot process is therefore:
    \begin{enumerate}
    \item Load \code{zImage} or \code{uImage} at address X in memory
    \item Load \code{<board>.dtb} at address Y in memory
    \item Start the kernel with \code{bootz X - Y} or
      \code{bootm X - Y}\\
      The \code{-} in the middle indicates no {\em initramfs}
    \end{enumerate}
  \end{itemize}
\end{frame}

\begin{frame}
  \frametitle{Kernel command line}
  \begin{itemize}
  \item In addition to the compile time configuration, the kernel
    behaviour can be adjusted with no recompilation using the {\bf
      kernel command line}
  \item The kernel command line is a string that defines various
    arguments to the kernel
    \begin{itemize}
    \item It is very important for system configuration
    \item \code{root=} for the root filesystem (covered later)
    \item \code{console=} for the destination of kernel messages
    \item Many more exist. The most important ones are documented
          in \kerneldoc{kernel-parameters.txt} in kernel sources.
    \end{itemize}
  \item This kernel command line is either
    \begin{itemize}
    \item Passed by the bootloader. In U-Boot, the contents of the
      \code{bootargs} environment variable is automatically passed to the
      kernel
    \item Built into the kernel, using the \code{CONFIG_CMDLINE} option.
    \end{itemize}
  \end{itemize}
\end{frame}
