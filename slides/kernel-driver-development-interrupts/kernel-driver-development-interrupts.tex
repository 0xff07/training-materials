\subsection{Interrupt Management}

\begin{frame}[fragile]
  \frametitle{Registering an interrupt handler 1/2}
  When handling multiple devices, the {\em managed} API is preferred:
  \begin{minted}[fontsize=\small]{c}
int devm_request_irq(struct device *dev,
                     unsigned int irq,
                     irq_handler_t handler,
                     unsigned long irq_flags,
                     const char *devname,
                     void *dev_id);
  \end{minted}
  \begin{itemize}
  \item \code{device} for automatic freeing at device or module
        release time.
  \item \code{irq} is the requested IRQ channel. For platform
        devices, use \kfunc{platform_get_irq} to retrieve the
        interrupt number.
  \item \code{handler} is a pointer to the IRQ handler
  \item \code{irq_flags} are option masks (see next slide)
  \item \code{devname} is the registered name (for
	\code{/proc/interrupts})
  \item \code{dev_id} is an opaque pointer. It can typically
        be used to pass a pointer to a per-device data structure.
        It cannot be \code{NULL} as it is used as an identifier for
        freeing interrupts on a shared line.
  \end{itemize}
\end{frame}

\begin{frame}[fragile]
  \frametitle{Releasing an interrupt handler}
  \begin{minted}{c}
void devm_free_irq(struct device *dev,
                   unsigned int irq, void *dev_id);
  \end{minted}
  \begin{itemize}
  \item Explicitly release an interrupt handler. Done automatically
        in normal situations.
  \end{itemize}
  Defined in \kpath{include/linux/interrupt.h}
\end{frame}

\begin{frame}
  \frametitle{Registering an interrupt handler 2/2}
  Main \code{irq_flags} bit values\\
  (can be combined, \code{0} when no flags are needed):
  \begin{itemize}
     \item \ksym{IRQF_SHARED}
     \begin{itemize}
        \item The interrupt channel can be shared by several
        devices.
	\item When an interrupt is received, all the interrupt
	handlers registered on the same interrupt line are called.
	\item This requires a hardware status register telling whether
        an IRQ was raised or not.
     \end{itemize}
  \end{itemize}
\end{frame}

\begin{frame}
  \frametitle{Interrupt handler constraints}
  \begin{itemize}
  \item No guarantee in which address space the system will be in when
    the interrupt occurs: can't transfer data to and from user space.
  \item Interrupt handler execution is managed by the CPU, not by the
    scheduler.  Handlers can't run actions that may sleep, because
    there is nothing to resume their execution. In particular, need to
    allocate memory with \ksym{GFP_ATOMIC}.
  \item Interrupt handlers are run with all interrupts disabled on
    the local CPU (see \url{http://lwn.net/Articles/380931}).
    Therefore, they have to complete their job quickly
    enough, to avoiding blocking interrupts for too long.
  \end{itemize}
\end{frame}

\begin{frame}[fragile]
  \frametitle{/proc/interrupts on Raspberry Pi 2 (ARM, Linux 4.1)}
\begin{block}{}
  \tiny
\begin{verbatim}
           CPU0       CPU1       CPU2       CPU3
 16:          0          0          0          0   ARMCTRL  16 Edge bcm2708_fb dma
 32: 1660065960          0          0          0   ARMCTRL  32 Edge dwc_otg, dwc_otg_pcd, dwc_otg_hcd:usb1
 49:          0          0          0          0   ARMCTRL  49 Edge 3f200000.gpio:bank0
 50:          0          0          0          0   ARMCTRL  50 Edge 3f200000.gpio:bank1
 65:      77339          0          0          0   ARMCTRL  65 Edge 3f00b880.mailbox
 66:          2          0          0          0   ARMCTRL  66 Edge VCHIQ doorbell
 75:          1          0          0          0   ARMCTRL  75 Edge
 77:     825261          0          0          0   ARMCTRL  77 Edge DMA IRQ
 82:     819926          0          0          0   ARMCTRL  82 Edge mmc0
 83:          6          0          0          0   ARMCTRL  83 Edge uart-pl011
 96:          0          0          0          0   ARMCTRL  96 Edge arch_timer
 97:   45040705   26523650   16191929   47339273   ARMCTRL  97 Edge arch_timer
FIQ:              usb_fiq
IPI0:          0          0          0          0  CPU wakeup interrupts
IPI1:          0          0          0          0  Timer broadcast interrupts
IPI2:   34944338   35870609   37410637   12127900  Rescheduling interrupts
IPI3:          9         10          9         11  Function call interrupts
IPI4:          3          3          1          1  Single function call interrupts
IPI5:          0          0          0          0  CPU stop interrupts
IPI6:          0          0          0          0  IRQ work interrupts
IPI7:          0          0          0          0  completion interrupts
Err:          0

\end{verbatim}
\end{block}
  \footnotesize
  Note: interrupt numbers shown on the left-most column are virtual
  numbers when the Device Tree is used. The real physical interrupt
  numbers can be seen in \code{/sys/kernel/debug/irq_domain_mapping}.
\end{frame}

\begin{frame}[fragile]
  \frametitle{Interrupt handler prototype}
  \begin{itemize}
  \item \mint{c}+irqreturn_t foo_interrupt(int irq, void *dev_id)+
    \begin{itemize}
    \item \code{irq}, the IRQ number
    \item \code{dev_id}, the per-device pointer that was
      passed to \kfunc{devm_request_irq}
    \end{itemize}
  \item Return value
    \begin{itemize}
    \item \ksym{IRQ_HANDLED}: recognized and handled interrupt
    \item \ksym{IRQ_NONE}: the interrupt was not triggered by a device
     managed by this driver. Useful to share interrupt channels and/or
     report spurious interrupts to the kernel.
    \end{itemize}
  \end{itemize}
\end{frame}

\begin{frame}
  \frametitle{Typical interrupt handler's job}
  \begin{itemize}
  \item Acknowledge the interrupt to the device (otherwise no more
    interrupts will be generated, or the interrupt will keep firing
    over and over again)
  \item Read/write data from/to the device
  \item Wake up any process waiting for such data, typically on a
    per-device wait queue:\\
    \code{wake_up_interruptible(&device_queue);}
\end{itemize}
\end{frame}

\begin{frame}[fragile]
  \frametitle{Threaded interrupts}
  In 2.6.30, support for threaded interrupts has been added to
  the Linux kernel
  \begin{itemize}
  \item The interrupt handler is executed inside a thread.
  \item Allows to block during the interrupt handler, which is often
        needed for I2C/SPI devices as the interrupt handler needs to
        communicate with them.
  \item Allows to set a priority for the interrupt handler
        execution, which is useful for real-time usage of Linux
  \end{itemize}
  \begin{minted}{c}
int devm_request_threaded_irq(
    struct device *dev,
    unsigned int irq,
    irq_handler_t handler, irq_handler_t thread_fn
    unsigned long flags, const char *name, void *dev);
  \end{minted}
  \begin{itemize}
  \item \code{handler}, ``hard IRQ'' handler
  \item \code{thread_fn}, executed in a thread
  \end{itemize}
\end{frame}

\begin{frame}
  \frametitle{Top half and bottom half processing}
  Splitting the execution of interrupt handlers in 2 parts
  \begin{itemize}
  \item Top half
    \begin{itemize}
    \item This is the real interrupt handler, which should complete
      as quickly as possible since all interrupts are disabled.
      It takes the data out of the device and if substantial
      post-processing is needed, schedule a bottom half to handle it. 
    \end{itemize}
  \item Bottom half
    \begin{itemize}
    \item Is the general Linux name for various mechanisms which
      allow to postpone the handling of interrupt-related
      work. Implemented in Linux as softirqs, tasklets or
      workqueues.
    \end{itemize}
  \end{itemize}
\end{frame}

\begin{frame}
  \frametitle{Top half and bottom half diagram}
  \begin{center}
    \includegraphics[width=\textwidth]{slides/kernel-driver-development-interrupts/thread-halves.pdf}
  \end{center}
\end{frame}

\begin{frame}
  \frametitle{Softirqs}
  \begin{itemize}
  \item Softirqs are a form of bottom half processing
  \item The softirqs handlers are executed with all interrupts
    enabled, and a given softirq handler can run simultaneously on
    multiple CPUs
  \item They are executed once all interrupt handlers have completed,
    before the kernel resumes scheduling processes, so sleeping is not
    allowed.
  \item The number of softirqs is fixed in the system, so softirqs are
    not directly used by drivers, but by complete kernel subsystems
    (network, etc.)
  \item The list of softirqs is defined in
    \kpath{include/linux/interrupt.h}: \code{HI}, \code{TIMER},
    \code{NET_TX}, \code{NET_RX}, \code{BLOCK}, \code{BLOCK_IOPOLL},
    \code{TASKLET}, \code{SCHED}, \code{HRTIMER}, \code{RCU}
  \item The \code{HI} and \code{TASKLET} softirqs are used to execute
    tasklets
  \end{itemize}
\end{frame}

\begin{frame}
  \frametitle{Tasklets}
  \begin{itemize}
  \item Tasklets are executed within the \code{HI} and \code{TASKLET}
    softirqs. They are executed with all interrupts enabled, but a
    given tasklet is guaranteed to execute on a single CPU at a time.
  \item Tasklets are typically created with the \kfunc{tasklet_init}
    function, when your driver manages multiple devices, otherwise
    statically with \kfunc{TASKLET_INIT}. A tasklet is simply
    implemented as a function. Tasklets can easily
    be used by individual device drivers, as opposed to softirqs.
  \item The interrupt handler can schedule tasklet execution with:
    \begin{itemize}
    \item \kfunc{tasklet_schedule} to get it executed in the
      \code{TASKLET} softirq
    \item \kfunc{tasklet_hi_schedule} to get it executed in the
      \code{HI} softirq (higher priority)
    \end{itemize}
  \end{itemize}
\end{frame}

\begin{frame}[fragile]
  \frametitle{Tasklet Example: simplified atmel\_serial.c 1/2}
\begin{minted}[fontsize=\small]{c}
/* The tasklet function */
static void atmel_tasklet_func(unsigned long data) {
        struct uart_port *port = (struct uart_port *)data;
        [...]
}

/* Registering the tasklet */
init function(...) {
        [...]
        tasklet_init(&atmel_port->tasklet,
            atmel_tasklet_func, (unsigned long)port);
        [...]
}
\end{minted}
\end{frame}

\begin{frame}[fragile]
  \frametitle{Tasklet Example: simplified atmel\_serial.c 2/2}
\begin{minted}[fontsize=\small]{c}
/* Removing the tasklet */
cleanup function(...) {
        [...]
        tasklet_kill(&atmel_port->tasklet);
        [...]
}

/* Triggering execution of the tasklet */
somewhere function(...) {
        tasklet_schedule(&atmel_port->tasklet);
}
\end{minted}
\end{frame}

\begin{frame}
  \frametitle{Workqueues}
  \begin{itemize}
  \item Workqueues are a general mechanism for deferring work. It is
    not limited in usage to handling interrupts.
  \item The function registered as workqueue is executed in a thread,
    which means:
    \begin{itemize}
    \item All interrupts are enabled
    \item Sleeping is allowed
    \end{itemize}
  \item A workqueue is registered with \kfunc{INIT_WORK} and typically
    triggered with \kfunc{queue_work}
  \item The complete API, in \kpath{include/linux/workqueue.h} provides
    many other possibilities (creating its own workqueue threads,
    etc.)
  \end{itemize}
\end{frame}

\begin{frame}
  \frametitle{Interrupt management summary}
  \begin{itemize}
  \item Device driver
    \begin{itemize}
    \item In the \code{probe()} function, for each device,
      use \kfunc{devm_request_irq} to register an interrupt handler
      for the device's interrupt channel.
    \end{itemize}
  \item Interrupt handler
    \begin{itemize}
    \item Called when an interrupt is raised.
    \item Acknowledge the interrupt
    \item If needed, schedule a per-device tasklet taking care of handling
      data.
    \item Wake up processes waiting for the data on a per-device queue
    \end{itemize}
  \item Device driver
    \begin{itemize}
    \item In the \code{remove()} function, for each device, the
      interrupt handler is automatically unregistered.
    \end{itemize}
  \end{itemize}
\end{frame}
