\section{Writing recipes - advanced}

\subsection{Extending a recipe}

\begin{frame}
  \frametitle{Introduction to recipe extensions}
  \begin{itemize}
    \item It is a good practice {\bf not} to modify recipes available
          in Poky.
    \item But it is sometimes useful to modify an existing recipe, to
          apply a custom patch for example.
    \item The BitBake \emph{build engine} allows to modify a recipe by
          extending it.
    \item Multiple extensions can be applied to a recipe.
  \end{itemize}
\end{frame}

\begin{frame}
  \frametitle{Introduction to recipe extensions}
  \begin{itemize}
    \item Metadata can be changed, added or appended.
    \item Tasks can be added or appended.
    \item Operators are used extensively, to add, append, prepend or
          assign values.
  \end{itemize}
\end{frame}

\begin{frame}{Extend a recipe}
  \begin{center}
    \includegraphics[width=\textwidth]{slides/yocto-recipe-advanced/yocto-recipe-advanced-extend.pdf}
  \end{center}
\end{frame}

\begin{frame}
  \frametitle{Extend a recipe}
  \begin{itemize}
    \item The recipe extensions end in \code{.bbappend}
    \item Append files must have the same root name as the recipe they
          extend.
    \begin{itemize}
      \item \code{example_0.1.bbappend} applies to
            \code{example_0.1.bb}
    \end{itemize}
    \item Append files are {\bf version specific}. If the recipe is
          updated to a newer version, the append files must also be
          updated.
    \item If adding new files, the path to their directory must
          be prepended to the \code{FILESEXTRAPATHS} variable.
    \begin{itemize}
      \item Files are looked up in paths referenced in
            \code{FILESEXTRAPATHS}, from left to right.
      \item Prepending a path makes sure it has priority over the recipe's
            one. This allows to override recipes' files.
    \end{itemize}
  \end{itemize}
\end{frame}

\begin{frame}
  \frametitle{Extend a recipe: compatibility}
  \begin{itemize}
    \item When using a Yocto Project release {\bf older than 1.5}, the
      Metadata revision number must explicitly be incremented in each
      append file.
    \item The revision number is stored in the \code{PRINC} variable.
    \item At the end of the recipe, you must increment it:
      \begin{itemize}
        \item \code{PRINC := "${@int(PRINC) + 1"}}
      \end{itemize}
    \item Since version 1.5, \code{PRINC} is automatically taken care
      of unless you are building on multiple machines. In that case,
      use the PR server, with \code{bitbake-prserv}
  \end{itemize}
\end{frame}

\subsection{Append file example}

\begin{frame}[fragile]
  \frametitle{Hello world append file}
  \begin{block}{}
    \begin{minted}{c}
FILESEXTRAPATHS_prepend = "${THISDIR}/files:"

SRC_URI += "file://custom-modification-0.patch \
            file://custom-modification-1.patch \
            "
    \end{minted}
  \end{block}
\end{frame}

\subsection{Advanced recipe configuration}

\begin{frame}
  \frametitle{Advanced configuration}
  \begin{itemize}
    \item In the real word, more complex configurations are often needed
          because recipes may:
    \begin{itemize}
      \item Have dependencies
      \item Provide virtual packages
      \item Inherit generic functions from classes
    \end{itemize}
  \end{itemize}
\end{frame}

\begin{frame}
  \frametitle{Add dependencies}
  \begin{itemize}
    \item A package can have dependencies during the build or at
          runtime. To reflect these requirements in the recipe, two
          variables are used:
    \begin{description}
      \item[DEPENDS] List of the package build-time dependencies.
      \item[RDEPENDS] List of the package runtime
        dependencies. Must be package specific (e.g. with
        \code{_${PN}}).
    \end{description}
    \item \code{DEPENDS = "package_b"}: the local \code{do_configure}
      task depends on the \code{do_populate_sysroot} task of package
      b.
    \item \code{RDEPENDS_${PN} = "package_b"}: the local
      \code{do_build} task depends on the
      \code{do_package_write_<archive-format>} task of package b.
  \end{itemize}
\end{frame}

\begin{frame}
  \frametitle{Providing virtual packages}
  \begin{itemize}
    \item BitBake allows to use virtual names instead of the actual
          package name. We saw a use case with \emph{package
          variants}.
    \item The virtual name is specified through the \code{PROVIDES}
          variable.
    \item Several packages can provide the same virtual name. Only one
          will be built and installed into the generated image.
    \item \code{PROVIDES = "virtual/kernel"}
  \end{itemize}
\end{frame}

\subsection{Classes}

\begin{frame}
  \frametitle{Introduction to classes}
  \begin{itemize}
    \item Classes provide an abstraction to common code, which can be
          re-used in multiple packages.
    \item Common tasks do not have to be re-developed!
    \item Any metadata and task which can be put in a recipe can be
          used in a class.
    \item Classes extension is \code{.bbclass}
    \item Classes are located in the \code{classes} folder of a layer.
    \item Packages can use this common code by inheriting a class:
    \begin{itemize}
      \item \code{inherits <class>}
    \end{itemize}
  \end{itemize}
\end{frame}

\begin{frame}
  \frametitle{Common classes}
  \begin{itemize}
    \item Common classes can be found in \code{meta/classes/}
    \begin{itemize}
      \item \code{base.bbclass}
      \item \code{kernel.bbclass}
      \item \code{autotools.bbclass}
      \item \code{update-alternatives.bbclass}
      \item \code{useradd.bbclass}
      \item \dots
    \end{itemize}
  \end{itemize}
\end{frame}

\begin{frame}
  \frametitle{The base class 1/2}
  \begin{itemize}
    \item Every recipe inherits the base class automatically.
    \item Contains a set of basic common tasks to fetch, unpack or
          compile packages.
    \item Inherits other common classes, providing:
    \begin{itemize}
      \item Mirrors definitions: \code{DEBIAN_MIRROR},
            \code{GNU_MIRROR}, \code{KERNELORG_MIRROR}\dots
      \item The ability to filter patches by \code{SRC_URI}
      \item Some tasks: \code{clean}, \code{listtasks} or
            \code{fetchall}.
    \end{itemize}
  \end{itemize}
\end{frame}

\begin{frame}
  \frametitle{The base class 2/2}
  \begin{itemize}
    \item Defines \code{oe_runmake}, using \code{EXTRA_OEMAKE} to use
      custom arguments.
    \item In Poky, \code{EXTRA_OEMAKE} defaults to \code{-e
      MAKEFLAGS=}.
    \item The \code{-e} option to give variables taken from the
      environment precedence over variables from makefiles.
      \begin{itemize}
        \item Upstream libraries or softwares often embed their own
          \code{Makefile}.
        \item Helps not using hardcoded \code{CC} or \code{CFLAGS}
          variables in makefiles.
      \end{itemize}
  \end{itemize}
\end{frame}

\begin{frame}
  \frametitle{The kernel class}
  \begin{itemize}
    \item Used to build Linux kernels.
    \item Defines tasks to configure, compile and install a kernel and
          its modules.
    \item The kernel is divided into several packages: \code{kernel},
          \code{kernel-base}, \code{kernel-dev},
          \code{kernel-modules}\dots
    \item Automatically provides the virtual package
          \code{virtual/kernel}.
    \item Configuration variables are available:
    \begin{itemize}
      \item \code{KERNEL_IMAGETYPE}, defaults to \code{zImage}
      \item \code{KERNEL_EXTRA_ARGS}
      \item \code{INITRAMFS_IMAGE}
    \end{itemize}
  \end{itemize}
\end{frame}

\begin{frame}
  \frametitle{The autotools class}
  \begin{itemize}
    \item Defines tasks and metadata to handle packages using the
          autotools build system (autoconf, automake and libtool):
    \begin{itemize}
      \item \code{do_configure}: generates the configure script using
            \code{autoreconf} and loads it with standard arguments or
            cross-compilation.
      \item \code{do_compile}: runs \code{make}
      \item \code{do_install}: runs \code{make install}
    \end{itemize}
    \item Extra configuration parameters can be passed with
          \code{EXTRA_OECONF}.
    \item Compilation flags can be added thanks to the
          \code{EXTRA_OEMAKE} variable.
  \end{itemize}
\end{frame}

\begin{frame}[fragile]
  \frametitle{Example: use the autotools class}
  \begin{block}{}
    \begin{minted}[fontsize=\tiny]{c}
DESCRIPTION = "Print a friendly, customizable greeting"
HOMEPAGE = "https://www.gnu.org/software/hello/"
PRIORITY = "optional"
SECTION = "examples"
LICENSE = "GPLv3"

SRC_URI = "${GNU_MIRROR}/hello/hello-${PV}.tar.gz"
SRC_URI[md5sum] = "67607d2616a0faaf5bc94c59dca7c3cb"
SRC_URI[sha256sum] = "ecbb7a2214196c57ff9340aa71458e1559abd38f6d8d169666846935df191ea7"
LIC_FILES_CHKSUM = "file://COPYING;md5=d32239bcb673463ab874e80d47fae504"

inherits autotools
    \end{minted}
  \end{block}
\end{frame}

\begin{frame}
  \frametitle{The update-alternative class}
  \begin{itemize}
    \item Allows to install multiple binaries having the same
          functionality, avoiding conflicts by renaming the binaries.
    \item Four variables are used to configure the class:
    \begin{description}
      \item[ALTERNATIVE\_NAME] The name of the binary.
      \item[ALTERNATIVE\_LINK] The path of the resulting binary.
      \item[ALTERNATIVE\_PATH] The path of the real binary.
      \item[ALTERNATIVE\_PRIORITY] The alternative priority.
    \end{description}
    \item The command with the highest priority will be used.
  \end{itemize}
\end{frame}

\begin{frame}
  \frametitle{The useradd class}
  \begin{itemize}
    \item This class helps to add users to the resulting image.
    \item Adding custom users is required by many services to avoid
          running them as root.
    \item \code{USERADD_PACKAGES} must be defined when the
          \code{useradd} class is inherited. Defines the list of
          packages which needs the user.
    \item Users and groups will be created before the packages using
          it perform their \code{do_install}.
    \item At least one of the two following variables must be set:
    \begin{itemize}
      \item \code{USERADD_PARAM}: parameters to pass to
            \code{useradd}.
      \item \code{GROUPADD_PARAM}: parameters to pass to
            \code{groupadd}.
    \end{itemize}
  \end{itemize}
\end{frame}

\begin{frame}[fragile]
  \frametitle{Example: use the useradd class}
  \begin{block}{}
    \begin{minted}[fontsize=\tiny]{c}
DESCRIPTION = "useradd class usage example"
PRIORITY = "optional"
SECTION = "examples"
LICENSE = "MIT"

SRC_URI = "file://file0"
LIC_FILES_CHKSUM = "file://${COREBASE}/meta/files/common-licenses/MIT;md5=0835ade698e0bc..."

inherits useradd

USERADD_PACKAGES = "${PN}"
USERADD_PARAM = "-u 1000 -d /home/user0 -s /bin/bash user0"

do_install() {
    install -m 644 file0 ${D}/home/user0/
    chown user0:user0 ${D}/home/user0/file0
}
    \end{minted}
  \end{block}
\end{frame}

\subsection{Binary packages}

\begin{frame}
  \frametitle{Specifics for binary packages}
  \begin{itemize}
    \item It is possible to install binaries into the generated root
      filesystem.
    \item Set the \code{LICENSE} to \code{CLOSED}.
    \item Use the \code{do_install} task to copy the binaries into the
      root file system.
  \end{itemize}
\end{frame}

\subsection{Debugging recipes}

\begin{frame}[fragile]
  \frametitle{Debugging recipes}
  \begin{itemize}
    \item For each task, logs are available in the \code{temp}
      directory in the work folder of a recipe.
    \item A development shell, exporting the full environment can be
      used to debug build failures:
      \begin{block}{}
        \begin{minted}{console}
$ bitbake -c devshell <recipe>
        \end{minted}
      \end{block}
    \item To understand what a change in a recipe implies, you can
      activate build history in \code{local.conf}:
      \begin{block}{}
        \begin{minted}{c}
INHERIT += "buildhistory"
BUILDHISTORY_COMMIT = "1"
        \end{minted}
      \end{block}
      Then use the \code{buildhistory-diff} tool to examine
      differences between two builds.
      \begin{itemize}
        \item \code{./scripts/buildhistory-diff}
      \end{itemize}
  \end{itemize}
\end{frame}
