\section{Images}
\subsection{Introduction to images}

\begin{frame}
  \frametitle{Overview 1/3}
  \begin{itemize}
    \item An \code{image} is the top level recipe and is used
      alongside the \code{machine} definition.
    \item Whereas the \code{machine} describes the hardware used and
      its capabilities, the \code{image} is architecture agnostic and
      defines how the root filesystem is built, with what packages.
    \item By default, several images are provided in Poky:
      \begin{itemize}
        \item \code{meta*/recipes*/images/*.bb}
      \end{itemize}
  \end{itemize}
\end{frame}

\begin{frame}
  \frametitle{Overview 2/3}
  \begin{itemize}
    \item Common images are:
      \begin{description}
        \item[core-image-base] Console-only image, with full support
          of the hardware.
        \item[core-image-minimal] Small image, capable of booting a
          device.
        \item[core-image-minimal-dev] Small image with extra debug
          symbols, tools and libraries.
        \item[core-image-x11] Image with basic X11 support.
        \item[core-image-rt] \code{core-image-minimal} with real time
          tools and test suite.
      \end{description}
  \end{itemize}
\end{frame}

\begin{frame}
  \frametitle{Overview 3/3}
  \begin{itemize}
    \item An \code{image} is no more than a recipe.
    \item It has a description, a license and inherits the
      \code{core-image} class.
  \end{itemize}
\end{frame}

\begin{frame}
  \frametitle{Organization of an image recipe}
  \begin{itemize}
    \item Some special configuration variables are used to describe an
      image:
      \begin{description}
        \item[IMAGE\_BASENAME] The name of the output image files.
          Defaults to \code{${PN}}.
        \item[IMAGE\_INSTALL] List of packages and package groups to
          install in the generated image.
        \item[IMAGE\_ROOTFS\_SIZE] The final root filesystem size.
        \item[IMAGE\_FEATURES] List of features to enable in the
          image.
        \item[IMAGE\_FSTYPES] List of formats the OpenEmbedded build
          system will use to create images.
        \item[IMAGE\_LINGUAS] List of the locales to be supported in
          the image.
        \item[IMAGE\_PKGTYPE] Package type used by the build system.
          One of \code{deb}, \code{rpm}, \code{ipk} and \code{tar}.
        \item[IMAGE\_POSTPROCESS\_COMMAND] Shell commands to run at
          post process.
      \end{description}
  \end{itemize}
\end{frame}

\begin{frame}[fragile]
  \frametitle{Example of an image}
  \begin{block}{}
    \begin{minted}{c}
require recipes-core/images/core-image-minimal.bb

DESCRIPTION = "Example image"

IMAGE_INSTALL += " ninvaders"

IMAGE_FSTYPES = "tar.bz2 cpio squashfs"

IMAGE_PKGTYPE = "deb"

LICENSE = "MIT"
    \end{minted}
  \end{block}
\end{frame}
\subsection{Package groups}

\begin{frame}
  \frametitle{Overview}
  \begin{itemize}
    \item \code{Package groups} are a way to group packages by
      functionality or common purpose.
    \item Package groups are used in image recipes to help building
      the list of packages to install.
    \item They can be found under
      \code{meta*/recipes-core/packagegroups/}
    \item A package group is yet another recipe.
    \item The prefix \code{packagegroup-} is always used.
  \end{itemize}
\end{frame}

\begin{frame}
  \frametitle{Common package groups}
  \begin{itemize}
    \item packagegroup-core-boot
    \item packagegroup-core-buildessential
    \item packagegroup-core-nfs
    \item packagegroup-core-ssh-dropbear
    \item packagegroup-core-ssh-openssh
    \item packagegroup-core-tools-debug
    \item packagegroup-core-tools-profile
  \end{itemize}
\end{frame}

\begin{frame}[fragile]
  \frametitle{Example}
  \code{meta/recipes-core/packagegroups/packagegroup-core-nfs.bb}:
  \begin{block}{}
    \begin{minted}{c}
DESCRIPTION = "NFS package groups"
LICENSE = "MIT"
PR = "r2"

inherit packagegroup

PACKAGES = "${PN}-server"

SUMMARY_${PN}-server = "NFS server"
RDEPENDS_${PN}-server = "\
    nfs-utils \
    nfs-utils-client \
    "
    \end{minted}
  \end{block}
\end{frame}
