\section{I/O Memory and Ports}

\begin{frame}
  \frametitle{Port I/O vs. Memory-Mapped I/O}
  \begin{itemize}
  \item Memory-Mapped I/O (MMIO)
    \begin{itemize}
    \item Same address bus to address memory and I/O devices
    \item Access to the I/O devices using regular instructions
    \item Most widely used I/O method across the different
      architectures supported by Linux
    \end{itemize}
  \item Port I/O (PIO)
    \begin{itemize}
    \item Different address spaces for memory and I/O devices
    \item Uses a special class of CPU instructions to access I/O
      devices
    \item Example on x86: IN and OUT instructions
    \end{itemize}
  \end{itemize}
\end{frame}

\begin{frame}
  \frametitle{MMIO vs PIO}
  \begin{center}
    \includegraphics[width=\textwidth]{slides/kernel-driver-development-io-memory/mmio-vs-pio.pdf}
  \end{center}
\end{frame}

\begin{frame}[fragile]
  \frametitle{Requesting I/O ports}
  \begin{itemize}
  \item Tells the kernel which driver is using which I/O ports
  \item Allows to prevent other drivers from using the same I/O ports,
    but is purely voluntary.
  \item
\begin{minted}{c}
struct resource *request_region(
     unsigned long start,
     unsigned long len,
     char *name);
\end{minted}
  \item Tries to reserve the given region and returns \code{NULL} if unsuccessful.
  \item \mint{c}+request_region(0x0170, 8, "ide1");+
  \item
\begin{minted}{c}
void release_region(
    unsigned long start,
    unsigned long len);
\end{minted}
\end{itemize}
\end{frame}

\begin{frame}[fragile]
  \frametitle{/proc/ioports example (x86)}
{\small
\begin{verbatim}
0000-001f : dma1
0020-0021 : pic1
0040-0043 : timer0
0050-0053 : timer1
0070-0077 : rtc
0080-008f : dma page reg
00a0-00a1 : pic2
00c0-00df : dma2
00f0-00ff : fpu
0170-0177 : ide1
01f0-01f7 : ide0
0376-0376 : ide1
03f6-03f6 : ide0
03f8-03ff : serial
0800-087f : 0000:00:1f.0
...
\end{verbatim}
}
\end{frame}

\begin{frame}[fragile]
  \frametitle{Accessing I/O ports}
  \begin{itemize}
  \item Functions to read/write bytes (\code{b}), word (\code{w}) and longs (\code{l}) to I/O ports:
    \begin{itemize}
    \item \mint{c}+unsigned in[bwl](unsigned long port)+
    \item \mint{c}+void out[bwl](value, unsigned long port)+
    \end{itemize}
  \item And the strings variants: often more efficient than the
    corresponding C loop, if the processor supports such operations!
    \begin{itemize}
    \item \mint{c}+void ins[bwl](unsigned port, void *addr,+
      \mint{c}+unsigned long count)+
    \item \mint{c}+void outs[bwl](unsigned port, void *addr,+
      \mint{c}+unsigned long count)+
    \end{itemize}
  \item Examples
    \begin{itemize}
    \item read 8 bits
      \begin{itemize}
      \item \mint{c}|oldlcr = inb(baseio + UART_LCR)|
      \end{itemize}
    \item write 8 bits
      \begin{itemize}
      \item \mint{c}|outb(MOXA_MUST_ENTER_ENCHANCE, baseio + UART_LCR)|
      \end{itemize}
    \end{itemize}
  \end{itemize}
\end{frame}

\begin{frame}[fragile]
  \frametitle{Requesting I/O memory}
  \begin{itemize}
  \item Functions equivalent to \kfunc{request_region} and
    \kfunc{release_region}, but for I/O memory.
  \item
\begin{minted}{c}
struct resource *request_mem_region(
    unsigned long start,
    unsigned long len,
    char *name);
\end{minted}
  \item
\begin{minted}{c}
void release_mem_region(
    unsigned long start,
    unsigned long len);
  \end{minted}
\end{itemize}
\end{frame}

\begin{frame}[fragile]
  \frametitle{/proc/iomem example - ARM (Raspberry Pi, Linux 4.19)}
{\footnotesize
\begin{verbatim}
00000000-3b3fffff : System RAM
  00008000-00bfffff : Kernel code
  00d00000-00e6ad0f : Kernel data
3f006000-3f006fff : dwc_otg
3f007000-3f007eff : dma@7e007000
3f00a000-3f00a023 : watchdog@7e100000
3f00b840-3f00b87b : mailbox@7e00b840
3f00b880-3f00b8bf : mailbox@7e00b880
3f100000-3f100113 : watchdog@7e100000
3f101000-3f102fff : cprman@7e101000
3f104000-3f10400f : rng@7e104000
3f200000-3f2000b3 : gpio@7e200000
3f201000-3f2011ff : serial@7e201000
  3f201000-3f2011ff : serial@7e201000
3f202000-3f2020ff : mmc@7e202000
3f212000-3f212007 : thermal@7e212000
3f215000-3f215007 : aux@7e215000
3f980000-3f98ffff : dwc_otg
\end{verbatim}
}
\end{frame}

\begin{frame}[fragile]
  \frametitle{Mapping I/O memory in virtual memory}
  \begin{itemize}
  \item Load/store instructions work with virtual addresses
  \item To access I/O memory, drivers need to have a virtual address
    that the processor can handle, because I/O memory is not mapped by
    default in virtual memory.
  \item The \code{ioremap} function satisfies this need:
\begin{minted}{c}
#include <asm/io.h>

void __iomem *ioremap(phys_addr_t phys_addr,
    unsigned long size);
void iounmap(void __iomem *addr);
\end{minted}
  \item Caution: check that \kfunc{ioremap} doesn't return a \code{NULL} address!
  \end{itemize}
\end{frame}

\begin{frame}[fragile]
  \frametitle{ioremap()}
  \begin{center}
    \includegraphics[height=0.75\textheight]{slides/kernel-driver-development-io-memory/ioremap.pdf}\\
    \code{ioremap(0xAFFEBC00, 4096) = 0xCDEFA000}
  \end{center}
\end{frame}

\begin{frame}[fragile]
  \frametitle{Managed API}
  Using \kfunc{request_mem_region} and \kfunc{ioremap} in device
  drivers is now deprecated. You should use the below "managed"
  functions instead, which simplify driver coding and error handling:
  \begin{itemize}
  \item \kfunc{devm_ioremap}
  \item \kfunc{devm_iounmap}
  \item \kfunc{devm_ioremap_resource}
        \begin{itemize}
	\item Takes care of both the request and remapping operations!
	\item Example (\kfile{drivers/crypto/atmel-sha.c}):
	\begin{block}{}
	\begin{minted}{c}
sha_res = platform_get_resource(pdev, IORESOURCE_MEM, 0);
...
sha_dd->io_base = devm_ioremap_resource(&pdev->dev, sha_res);
	\end{minted}
	\end{block}{}
	\end{itemize}
  \item \kfunc{devm_platform_ioremap_resource}
  \end{itemize}
\end{frame}

\begin{frame}[fragile]
  \frametitle{Accessing MMIO devices}
  \begin{itemize}
  \item Directly reading from or writing to addresses returned by
    \kfunc{ioremap} (\emph{pointer dereferencing}) may not work on some
    architectures.
  \item To do PCI-style, little-endian accesses, conversion being done
    automatically
\begin{minted}{c}
unsigned read[bwl](void *addr);
void write[bwl](unsigned val, void *addr);
\end{minted}
  \item To do raw access, without endianness conversion
\begin{minted}{c}
unsigned __raw_read[bwl](void *addr);
void __raw_write[bwl](unsigned val, void *addr);
\end{minted}
  \item Example
    \begin{itemize}
    \item 32 bits write
\begin{minted}{c}
__raw_writel(1 << KS8695_IRQ_UART_TX,
    membase + KS8695_INTST);
\end{minted}
    \end{itemize}
  \end{itemize}
\end{frame}

\begin{frame}
  \frametitle{Avoiding I/O access issues}
  \begin{itemize}
  \item Caching on I/O ports or memory already disabled
  \item Use the \kfunc{writel}/\kfunc{readl} macros,
        they do the right thing for your architecture
  \item The compiler and/or CPU can reorder memory accesses, which
    might cause trouble for your devices is they expect one register
    to be read/written before another one.
    \begin{itemize}
    \item Memory barriers are available to prevent this reordering
    \item \kfunc{rmb} is a read memory barrier, prevents reads to
      cross the barrier
    \item \kfunc{wmb} is a write memory barrier
    \item \kfunc{mb} is a read-write memory barrier
    \end{itemize}
  \item Starts to be a problem with CPUs that reorder instructions and
    SMP.
  \item See \kerneldoctext{memory-barriers.txt} for details
  \end{itemize}
\end{frame}

\begin{frame}
  \frametitle{/dev/mem}
  \begin{itemize}
  \item Used to provide user space applications with direct access to
    physical addresses.
  \item Usage: open \code{/dev/mem} and read or write at given offset.
    What you read or write is the value at the corresponding physical
    address.
  \item Used by applications such as the X server to write directly to
    device memory.
  \item On \code{x86}, \code{arm}, \code{arm64},
    \code{powerpc}, \code{s390} and \code{unicore32}:
    \kconfig{CONFIG_STRICT_DEVMEM} option to restrict \code{/dev/mem}
    to non-RAM addresses, for security reasons (Linux 4.20 status).
\end{itemize}
\end{frame}
