\section{Introduction to the I2C subsystem}

\begin{frame}{What is I2C ?}
  \begin{itemize}
  \item A very commonly used low-speed devices to connect on-board
    devices to the processor.
  \item Uses only two wires: SDA for the data, SCL for the clock.
  \item It is a master/slave bus: only the master can initiate
    transactions, and slaves can only reply to transactions initiated
    by masters.
  \item In a Linux system, the I2C controller embedded in the
    processor is typically the master, controlling the bus.
  \item Each slave device is identified by an unique I2C address. Each
    transaction initiated by the master contains this address, which
    allows the relevant slave to recognize that it should reply to
    this particular transaction.
  \end{itemize}
\end{frame}

\begin{frame}{An I2C bus example}
  \begin{center}
    \includegraphics[width=\textwidth]{slides/kernel-i2c/i2c-bus.pdf}
  \end{center}
\end{frame}

\begin{frame}{The I2C subsystem}
  \begin{itemize}
  \item Like all bus subsystems, the I2C subsystem is responsible for:
    \begin{itemize}
    \item Providing an API to implement I2C controller drivers
    \item Providing an API to implement I2C device drivers, in kernel space
    \item Providing an API to implement I2C device drivers, in user space
    \end{itemize}
  \item The core of the I2C subsystem is located in
    \code{drivers/i2c}.
  \item The I2C controller drivers are located in
    \code{drivers/i2c/busses}.
  \item The I2C device drivers are located throughout
    \code{drivers/}, depending on the type of device (ex:
    \code{drivers/input} for input devices).
  \end{itemize}
\end{frame}

\begin{frame}{Registering an I2C device driver}
  \begin{itemize}
  \item Like all bus subsystems, the I2C subsystem defines a
    \code{struct i2c_driver} that inherits from
    \code{struct device_driver}, and which must be instantiated and
    registered by each I2C device driver.
    \begin{itemize}
    \item As usual, this structure points to the \code{->probe()} and
      \code{->remove()} functions.
    \item It also contains an \code{id_table} field that must point to a
      list of {\em device IDs} (which is a list of tuples containing a
      string and some private driver data). It is used for non-DT based
      probing of I2C devices.
    \end{itemize}
  \item The \code{i2c_add_driver()} and \code{i2c_del_driver()} functions
    are used to register/unregister the driver.
  \item If the driver doesn't do anything else in its init/exit
    function, then it is advised to use the \code{module_i2c_driver()}
    macro instead.
  \end{itemize}
\end{frame}

\begin{frame}[fragile]{Registering an I2C device driver: example}
  \begin{block}{}
  \begin{minted}[fontsize=\tiny]{c}
static const struct i2c_device_id <driver>_id[] = {
        { "<device-name>", 0 },
        { }
};
MODULE_DEVICE_TABLE(i2c, <driver>_id);

#ifdef CONFIG_OF
static const struct of_device_id <driver>_dt_ids[] = {
        { .compatible = "<vendor>,<device-name>", },
        { }
};
MODULE_DEVICE_TABLE(of, <driver>_dt_ids);
#endif

static struct i2c_driver <driver>_driver = {
        .probe          = <driver>_probe,
        .remove         = <driver>_remove,
        .id_table       = <driver>_id,
        .driver = {
                .name   = "<driver-name>",
                .owner  = THIS_MODULE,
                .of_match_table = of_match_ptr(<driver>_dt_ids),
        },
};

module_i2c_driver(<driver>_driver);
\end{minted}
\end{block}
\end{frame}

\begin{frame}{Registering an I2C device: non-DT}
  \begin{itemize}
  \item On non-DT platforms, the \code{i2c_board_info} structure
    allows to describe how an I2C device is connected to a board.
  \item Such structures are normally defined with the
    \code{I2C_BOARD_INFO} helper macro.
    \begin{itemize}
    \item Takes as argument the device name and the slave address of
      the device on the bus.
    \end{itemize}
  \item An array of such structures is registed on a per-bus basis
    using \code{i2c_register_board_info()}, when the platform is
    initialized.
  \end{itemize}
\end{frame}

\begin{frame}[fragile]{Registering an I2C device, non-DT example}
  \begin{block}{}
    \begin{minted}[fontsize=\tiny]{c}
static struct i2c_board_info <board>_i2c_devices[] __initdata = {
        {
                I2C_BOARD_INFO("cs42l51", 0x4a),
        },
};

void board_init(void)
{
        /*
         * Here should be the registration of all devices, including
         * the I2C controller device.
         */

        i2c_register_board_info(0, <board>_i2c_devices,
                                ARRAY_SIZE(<board>_i2c_devices));

        /* More devices registered here */
}
    \end{minted}
  \end{block}
\end{frame}

\begin{frame}{Registering an I2C device, in the DT}
  \begin{itemize}
  \item In the Device Tree, the I2C controller device is typically
    defined in the \code{.dtsi} file that describes the processor.
    \begin{itemize}
    \item Normally defined with \code{status = "disabled"}.
    \end{itemize}
  \item At the board/platform level:
    \begin{itemize}
    \item the I2C controller device is enabled
      (\code{status = "okay"})
    \item the I2C bus frequency is defined, using the
      \code{clock-frequency} property.
    \item the I2C devices on the bus are described as childs of the
      I2C controller node, where the \code{reg} property gives the I2C
      slave address on the bus.
    \end{itemize}
  \end{itemize}
\end{frame}

\begin{frame}[fragile]{Registering an I2C device, DT example (1/2)}
  \begin{block}{Definition of the I2C controller, .dtsi file}
    \begin{minted}[fontsize=\footnotesize]{perl}
i2c@7000c000 {
        compatible = "nvidia,tegra20-i2c";
        reg = <0x7000c000 0x100>;
        interrupts = <GIC_SPI 38 IRQ_TYPE_LEVEL_HIGH>;
        #address-cells = <1>;
        #size-cells = <0>;
        clocks = <&tegra_car TEGRA20_CLK_I2C1>,
                 <&tegra_car TEGRA20_CLK_PLL_P_OUT3>;
        clock-names = "div-clk", "fast-clk";
        status = "disabled";
};
    \end{minted}
  \end{block}
\end{frame}

\begin{frame}[fragile]{Registering an I2C device, DT example (2/2)}
  \begin{block}{Definition of the I2C device, .dts file}
    \begin{minted}[fontsize=\footnotesize]{perl}
i2c@7000c000 {
        status = "okay";
        clock-frequency = <400000>;

        alc5632: alc5632@1e {
                compatible = "realtek,alc5632";
                reg = <0x1e>;
                gpio-controller;
                #gpio-cells = <2>;
        };
};
    \end{minted}
  \end{block}
\end{frame}

\begin{frame}{\code{probe()} and \code{remove()}}

  \begin{itemize}
  \item The \code{->probe()} function is responsible for initializing
    the device and registering it in the appropriate kernel
    framework. It receives as argument:
    \begin{itemize}
    \item A \code{struct i2c_client} pointer, which represents the I2C
      device itself. This structure inherits from \code{struct
        device}.
    \item A \code{struct i2c_device_id} pointer, which points to the
      I2C device ID entry that matched the device that is being
      probed.
    \end{itemize}
  \item The \code{->remove()} function is responsible for
    unregistering the device from the kernel framework and shut it
    down. It receives as argument:
    \begin{itemize}
    \item The same \code{struct i2c_client} pointer that was passed as
      argument to \code{->probe()}
    \end{itemize}
  \end{itemize}
\end{frame}

\begin{frame}[fragile]{Probe/remove example}
  \begin{block}{}
    \begin{minted}[fontsize=\footnotesize]{c}
static int <driver>_probe(struct i2c_client *client,
                          const struct i2c_device_id *id)
{
        /* initialize device */
        /* register to a kernel framework */

        i2c_set_clientdata(client, <private data>);
        return 0;
}

static int <driver>_remove(struct i2c_client *client)
{
        <private data> = i2c_get_clientdata(client);
        /* unregister device from kernel framework */
        /* shut down the device */
        return 0;
}
    \end{minted}
  \end{block}
\end{frame}

\begin{frame}{Communicating with the I2C device: raw API}
  The most {\bf basic API} to communicate with the I2C device provides
  function to either send or receive data:
  \begin{itemize}
  \item
    \code{int i2c_master_send(struct i2c_client *client, const char *buf, int count);}\\Sends
    the contents of \code{buf} to the client.
  \item
    \code{int i2c_master_recv(struct i2c_client *client, char *buf, int count);}\\Receives
    \code{count} bytes from the client, and store them into
    \code{buf}.
  \end{itemize}
\end{frame}

\begin{frame}{Communicating with the I2C device: message transfer}
  The message transfer API allows to describe {\bf transfers} that
  consists of several {\bf messages}, with each message being a
  transaction in one direction:
  \begin{itemize}
  \item
    \code{int i2c_transfer(struct i2c_adapter *adap, struct i2c_msg *msg, int num);}
  \item The \code{i2c_adapter} pointer can be found by using
    \code{client->adapter}
  \item The \code{struct i2c_msg} defines the length, location, and
    direction of the message.
  \end{itemize}
\end{frame}

\begin{frame}[fragile]{I2C: message transfer example}
\begin{block}{}
  \begin{minted}[fontsize=\footnotesize]{c}
struct i2c_msg msg[2];
int error;
u8 start_reg;
u8 buf[10];

msg[0].addr = client->addr;
msg[0].flags = 0;
msg[0].len = 1;
msg[0].buf = &start_reg;
start_reg = 0x10;

msg[1].addr = client->addr;
msg[1].flags = I2C_M_RD;
msg[1].len = sizeof(buf);
msg[1].buf = buf;

error = i2c_transfer(client->adapter, msg, 2);
\end{minted}
\end{block}
\end{frame}

\begin{frame}{SMBus calls}
  \begin{itemize}
  \item SMBus is a subset of the I2C protocol.
  \item It defines a standard set of transactions, for example to read
    or write a register into a device.
  \item Linux provides SMBus functions that {\em should be used} when
    possible instead of the raw API, if the I2C device uses this
    standard type of transactions.
  \item Example: the \code{i2c_smbus_read_byte_data()} function allows
    to read one byte of data from a device register.
    \begin{itemize}
    \item It does the following operations:
      \code{S Addr Wr [A] Comm [A] S Addr Rd [A] [Data] NA P}
    \item Wihch means it first writes a one byte data command ({\em
        Comm}), and then reads back one byte of data ({\em [Data]}).
    \end{itemize}
  \end{itemize}
\end{frame}

\begin{frame}{List of SMBus functions}
  \begin{itemize}
  \item Read/write one byte
    \begin{itemize}
      \tiny
    \item \code{s32 i2c_smbus_read_byte(const struct i2c_client *client);}
    \item \code{s32 i2c_smbus_write_byte(const struct i2c_client *client, u8 value);}
    \end{itemize}
  \item Write a command byte, and read or write one byte
    \begin{itemize}
      \tiny
    \item \code{s32 i2c_smbus_read_byte_data(const struct i2c_client *client, u8 command);}
    \item \code{s32 i2c_smbus_write_byte_data(const struct i2c_client *client, u8 command, u8 value);}
    \end{itemize}
  \item Write a command byte, and read or write one word
    \begin{itemize}
      \tiny
    \item \code{s32 i2c_smbus_read_word_data(const struct i2c_client *client, u8 command);}
    \item \code{s32 i2c_smbus_write_word_data(const struct i2c_client *client, u8 command, u16 value);}
    \end{itemize}
  \item Write a command byte, and read or write a block of data (max
    32 bytes)
    \begin{itemize}
      \tiny
    \item \code{s32 i2c_smbus_read_block_data(const struct i2c_client *client, u8 command, u8 *values);}
    \item \code{s32 i2c_smbus_write_block_data(const struct i2c_client *client, u8 command, u8 length, const u8 *values);}
    \end{itemize}
  \item Write a command byte, and read or write a block of data (no
    limit)
    \begin{itemize}
      \tiny
    \item \code{s32 i2c_smbus_read_i2c_block_data(const struct i2c_client *client, u8 command, u8 length, u8 *values);}
    \item \code{s32 i2c_smbus_write_i2c_block_data(const struct i2c_client *client, u8 command, u8 length, const u8 *values);}
    \end{itemize}
  \end{itemize}
\end{frame}