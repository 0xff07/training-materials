\section{BSP Layers}

\subsection{Introduction to BSP layers in the Yocto Project}

\begin{frame}{BSP layers}
  \begin{center}
    \includegraphics[width=\textwidth]{slides/yocto-bsp/yocto-bsp-overview.pdf}
  \end{center}
\end{frame}

\begin{frame}
  \frametitle{Overview}
  \begin{itemize}
    \item BSP layers are device specific layers. They hold metadata
      with the purpose of supporting specific hardware devices.
    \item BSP layers describe the hardware features and often provide
      a custom kernel and bootloader with the required modules and
      drivers.
    \item BSP layers can also provide additional software, designed
      to take advantage of the hardware features.
    \item As a layer, it is integrated into the build system as we
      previously saw.
    \item A good practice is to name it \code{meta-<bsp_name>}.
  \end{itemize}
\end{frame}

\begin{frame}
  \frametitle{BSP layers Specifics}
  \begin{itemize}
    \item BSP layers are a subset of the layers.
    \item In addition to package recipes and build tasks, they often
      provide:
      \begin{itemize}
        \item Hardware configuration files (\code{machines}).
        \item Bootloader, kernel and display support and
          configuration.
        \item Pre-built user binaries.
      \end{itemize}
  \end{itemize}
\end{frame}

\subsection{Generating a new BSP layer}

\begin{frame}
  \frametitle{Creating a new BSP 1/3}
  \begin{itemize}
    \item A dedicated command is provided to create BSP layers:
      \code{yocto-bsp}.
    \item As for the layers, \code{meta-} is automatically prepended
      to the BSP layer's name.
    \item \code{yocto-bsp create <name> <karch>}
    \item \code{karch} stands for "kernel architecture". You can dump
      a list of the available ones by running: \code{yocto-bsp list
      karch}.
    \item \code{yocto-bsp create felabs arm}
  \end{itemize}
\end{frame}

\begin{frame}
  \frametitle{Creating a new BSP 2/3}
  \begin{itemize}
    \item \code{yocto-bsp} will prompt a few questions to help
      configure the kernel, bootloader and X support if needed.
    \item You will also need to choose compiler tuning (cortexa9,
      cortexa15, cortexm3, cortexm5\dots).
    \item And enable some functionalities (keyboard and mouse
      support).
  \end{itemize}
\end{frame}

\begin{frame}
  \frametitle{Creating a new BSP 3/3}
  \begin{itemize}
    \item A new layer is created, named \code{meta-<bsp_name>} and
      contains the following information:
      \begin{description}
        \item[binary/] Contains bootable images or build filesystem,
          if needed.
        \item[conf/layer.conf] The BSP layer's configuration.
        \item[conf/machine/] Holds the machine configuration files.
          One is created by default: \code{<bsp_name>.conf}
        \item[recipes-*] A few recipes are created, thanks to the user
          input gathered by the \code{yocto-bsp} command.
        \item[README] The layer's documentation. This file
          \emph{needs} to be updated.
      \end{description}
  \end{itemize}
\end{frame}

\subsection{Hardware configuration files}

\begin{frame}
  \frametitle{Overview 1/2}
  \begin{itemize}
    \item A layer provides one machine file (hardware configuration
      file) per machine it supports.
    \item These configuration files are stored under
      \code{meta-<bsp_name>/conf/machine/*.conf}
    \item The file names correspond to the values set in the
      \code{MACHINE} configuration variable.
      \begin{itemize}
        \item \code{meta-ti/conf/machine/beaglebone.conf}
        \item \code{MACHINE = "beaglebone"}
      \end{itemize}
    \item Each machine should be described in the \code{README} file
      of the BSP.
  \end{itemize}
\end{frame}

\begin{frame}
  \frametitle{Overview 2/2}
  \begin{itemize}
    \item The hardware configuration file contains configuration
      variables related to the architecture and to the machine
      features.
    \item Some other variables help customize the kernel image or the
      filesystems used.
  \end{itemize}
\end{frame}

\begin{frame}
  \frametitle{Machine configuration}
  \begin{description}
    \item[TARGET\_ARCH] The architecture of the device being built.
    \item[PREFERRED\_PROVIDER\_virtual/kernel] The default kernel.
    \item[MACHINE\_FEATURES] List of hardware features provided by the
      machine, e.g. \code{usbgadget usbhost screen wifi}
    \item[SERIAL\_CONSOLE] Speed and device for the serial console to
      attach. Passed to the kernel as the \code{console} parameter,
      e.g. \code{115200 ttyS0}
    \item[KERNEL\_IMAGETYPE] The type of kernel image to build, e.g.
      \code{zImage}
  \end{description}
\end{frame}

\begin{frame}
  \frametitle{\code{MACHINE_FEATURES}}
  \begin{itemize}
    \item Lists the hardware features provided by the machine.
    \item These features are used by package recipes to enable or
      disable functionalities.
    \item Some packages are automatically added to the resulting root
    filesystem depending on the feature list.
    \item The feature \code{bluetooth}:
      \begin{itemize}
        \item Adds the \code{bluez} daemon to be built and added to
          the image.
        \item Enables bluetooth support in \code{ConnMan}.
      \end{itemize}
  \end{itemize}
\end{frame}

\begin{frame}[fragile]
  \frametitle{\code{conf/machine/include/cfa10036.inc}}
  \begin{block}{}
    \begin{minted}[fontsize=\footnotesize]{sh}
# Common definitions for cfa-10036 boards
include conf/machine/include/mxs-base.inc

SOC_FAMILY = "mxs:mx28:cfa10036"

PREFERRED_PROVIDER_virtual/kernel ?= "linux-cfa"
IMAGE_BOOTLOADER = "barebox"
BAREBOX_BINARY = "barebox"
IMXBOOTLETS_MACHINE = "cfa10036"
KERNEL_IMAGETYPE = "zImage"
KERNEL_DEVICETREE = "imx28-cfa10036.dtb"

# we need the kernel to be installed in the final image
IMAGE_INSTALL_append = " kernel-image kernel-devicetree"

SDCARD_ROOTFS ?= "${DEPLOY_DIR_IMAGE}/${IMAGE_NAME}.rootfs.ext3"
IMAGE_FSTYPES ?= "tar.bz2 ext3 barebox.mxsboot-sdcard sdcard"

SERIAL_CONSOLE = "115200 ttyAMA0"
MACHINE_FEATURES = "usbgadget usbhost vfat"
    \end{minted}
  \end{block}
\end{frame}

\begin{frame}[fragile]{\code{conf/machine/cfa10057.conf}}
  \begin{block}{}
    \begin{minted}[fontsize=\tiny]{sh}
#@TYPE: Machine
#@NAME: Crystalfontz CFA-10057
#@SOC: i.MX28
#@DESCRIPTION: Machine configuration for CFA-10057, also called CFA-920
#@MAINTAINER: Alexandre Belloni <alexandre.belloni@free-electrons.com>

include conf/machine/include/cfa10036.inc

KERNEL_DEVICETREE += "imx28-cfa10057.dtb"

MACHINE_FEATURES += "touchscreen"
    \end{minted}
  \end{block}
\end{frame}

\subsection{Formfactor}

\begin{frame}
  \frametitle{Overview}
  \begin{itemize}
    \item The \code{yocto-bsp} command generates a \code{formfactor} recipe.
    \item \code{recipes-bsp/formfactor/formfactor_0.0.bbappend}
    \item \code{formfactor} is a recipe providing information
      about the hardware that is not described by other sources such
      as the kernel.
    \item This configuration is defined in the recipe in:
      \code{recipes-bsp/formfactor/formfactor/<machine>/machconfig}
    \item Default values are defined in:
      \code{meta/recipes-bsp/formfactor/files/config}
  \end{itemize}
\end{frame}

\begin{frame}[fragile]
  \frametitle{Formfactor example}
  \begin{block}{}
    \begin{minted}{sh}
HAVE_TOUCHSCREEN=1
HAVE_KEYBOARD=1

DISPLAY_CAN_ROTATE=0
DISPLAY_ORIENTATION=0
DISPLAY_WIDTH_PIXELS=640
DISPLAY_HEIGHT_PIXELS=480
DISPLAY_BPP=16
DISPLAY_DPI=150
DISPLAY_SUBPIXEL_ORDER=vrgb 
    \end{minted}
  \end{block}
\end{frame}

\subsection{Bootloader}

\begin{frame}
  \frametitle{Default bootloader 1/2}
  \begin{itemize}
    \item By default the bootloader used is the mainline version of
      \code{U-Boot}, with a fixed version (per Poky release).
    \item All the magic is done in
      \code{meta/recipes-bsp/u-boot/u-boot.inc}
    \item Some configuration variables used by the U-Boot recipe can
      be customized, in the machine file.
  \end{itemize}
\end{frame}

\begin{frame}
  \frametitle{Default bootloader 2/2}
  \begin{description}
    \item[SPL\_BINARY] If an SPL is built, describes the name of the
      output binary. Defaults to an empty string.
    \item[UBOOT\_SUFFIX] \code{bin} (default) or \code{img}.
    \item[UBOOT\_MACHINE] The target used to build the configuration.
    \item[UBOOT\_ENTRYPOINT] The bootloader entry point.
    \item[UBOOT\_LOADADDRESS] The bootloader load address.
    \item[UBOOT\_MAKE\_TARGET] Make target when building the
      bootloader.  Defaults to \code{all}.
  \end{description}
\end{frame}

\begin{frame}
  \frametitle{Customize the bootloader}
  \begin{itemize}
    \item By default no recipe is added to customize the bootloader.
    \item It is possible to do so by creating an extended recipe and
      to append extra metadata to the original one.
    \item This works well when using a mainline version of U-Boot.
    \item Otherwise it is possible to create a custom recipe.
      \begin{itemize}
        \item Try to still use
          \code{meta/recipes-bsp/u-boot/u-boot.inc}
      \end{itemize}
  \end{itemize}
\end{frame}

\subsection{Kernel}

\begin{frame}
  \frametitle{Linux kernel recipes in Yocto}
  \begin{itemize}
    \item There are basically two ways of compiling a kernel in the
      Yocto Project:
      \begin{itemize}
        \item By using the \code{linux-yocto} packages, provided in
          Poky.
        \item By using a fully custom kernel recipe.
      \end{itemize}
    \item The kernel used is selected in the machine file thanks to:
      \code{PREFERRED_PROVIDER_virtual/kernel}
    \item Its version if defined with:
      \code{PREFERRED_VERSION_<kernel_provider>}
  \end{itemize}
\end{frame}

\begin{frame}
  \frametitle{Linux Yocto 1/4}
  \begin{itemize}
    \item \code{linux-yocto} is a generic set of recipes for building
      mainline Linux kernel images.
    \item The \code{yocto-bsp} tool creates basic appended recipes to
      allow to extend the \code{linux-yocto} ones.
      \begin{itemize}
        \item \code{meta-<bsp_name>/recipes-kernel/linux/linux-yocto_*.bbappend}
      \end{itemize}
    \item \code{PREFERRED_PROVIDER_virtual/kernel = "linux-yocto"}
    \item \code{PREFERRED_VERSION_linux-yocto = "3.14\%"}
  \end{itemize}
\end{frame}

\begin{frame}[fragile]
  \frametitle{Linux Yocto 2/4}
  \begin{itemize}
    \item Like other appended recipes, patches can be added by filling
      \code{SRC_URI} with \code{.patch} and/or \code{.diff} files.
    \item The kernel configuration must also be provided, and the file
      containing it must be called \code{defconfig}.
      \begin{itemize}
        \item This can be generated from a Linux source tree, by using
          \code{make savedefconfig}
        \item The configuration can be split in several files, by
          using the \code{.cfg} extension. It is the best practice
          when adding new features:
          \begin{block}{}
            \begin{minted}{sh}
SRC_URI += "file://defconfig        \
            file://nand-support.cfg \
            file://ethernet-support.cfg"
            \end{minted}
          \end{block}
      \end{itemize}
  \end{itemize}
\end{frame}

\begin{frame}
  \frametitle{Linux Yocto 3/4}
  \begin{itemize}
    \item Configuration fragments can be generated directly with the
      \code{bitbake} command:
      \begin{enumerate}
        \item Configure the kernel following its recipe
          instructions: \code{bitbake -c kernel_configme linux-yocto}
        \item Edit the configuration:
          \code{bitbake -c menuconfig linux-yocto}
        \item Save the configuration differences:
          \code{bitbake -c diffconfig linux-yocto}
          \begin{itemize}
            \item The differences will be saved at
              \code{$WORKDIR/fragment.cfg}
          \end{itemize}
      \end{enumerate}
    \item After integrating configuration fragments into the appended
      recipe, you can check everything is fine by running:
      \code{bitbake -c kernel_configcheck -f linux-yocto}
  \end{itemize}
\end{frame}

\begin{frame}
  \frametitle{Linux Yocto 4/4}
  \begin{itemize}
    \item Another way of configuring \code{linux-yocto} is by using
      \emph{Advanced Metadata}.
    \item It is a powerful way of splitting the configuration and the
      patches into several pieces.
    \item It is designed to provide a very configurable kernel.
    \item The full documentation can be found at
      \url{https://www.yoctoproject.org/docs/1.6.1/kernel-dev/kernel-dev.html\#kernel-dev-advanced}
  \end{itemize}
\end{frame}

\begin{frame}
  \frametitle{Linux Yocto: Kernel Metadata 1/4}
  \begin{itemize}
    \item Kernel Metadata is a way to organize and to split the
      kernel configuration and patches in little pieces each providing
      support for one feature.
    \item Two main configuration variables help taking advantage of
      this:
      \begin{description}
        \item[LINUX\_KERNEL\_TYPE] \code{standard} (default),
          \code{tiny} or \code{preempt-rt}
          \begin{itemize}
            \item \code{standard}: generic Linux kernel policy.
            \item \code{tiny}: bare minimum configuration, for small
              kernels.
            \item \code{preempt-rt}: applies the \code{PREEMPT_RT}
              patch.
          \end{itemize}
        \item[KERNEL\_FEATURES] List of features to enable. Features
          are sets of patches and configuration fragments.
      \end{description}
  \end{itemize}
\end{frame}

\begin{frame}
  \frametitle{Linux Yocto: Kernel Metadata 2/4}
  \begin{itemize}
    \item Kernel Metadata can be stored in the \code{linux-yocto}
      recipe space.
    \item It must be under \code{$FILESEXTRAPATHS}. A best practice
      is to follow this directory hierarchy:
      \begin{description}
        \item[bsp/]
        \item[cfg/]
        \item[features/]
        \item[ktypes/]
        \item[patches/]
      \end{description}
    \item Kernel Metadata are divided into 3 file types:
      \begin{itemize}
        \item Description files, ending in \code{.scc}
        \item Configuration fragments
        \item Patches
      \end{itemize}
  \end{itemize}
\end{frame}

\begin{frame}[fragile]
  \frametitle{Linux Yocto: Kernel Metadata 3/4}
  \begin{itemize}
    \item Kernel Metadata description files have their own syntax,
      used to describe the feature provided and which patches and
      configuration fragments to use.
    \item Simple example, \code{features/smp.scc}
      \begin{block}{}
        \begin{minted}{sh}
define KFEATURE_DESCRIPTION "Enable SMP"

kconf hardware smp.cfg
patch smp-support.patch
        \end{minted}
      \end{block}
    \item To integrate the feature into the kernel image:
      \code{KERNEL_FEATURES += "features/smp.scc"}
  \end{itemize}
\end{frame}

\begin{frame}
  \frametitle{Linux Yocto: Kernel Metadata 4/4}
  \begin{itemize}
    \item \code{.scc} syntax description:
      \begin{description}
        \item[\code{branch <ref>}] Create a new branch relative to the
          current one.
        \item[\code{define}] Defines variables.
        \item[\code{include <scc file>}] Include another description
          file. Parsed inline.
        \item[\code{kconf [hardware|non-hardware] <cfg file>}] Queues
          a configuration fragment, to merge it into Linux's
          \code{.config}
        \item[\code{git merge <branch>}] Merge \code{branch} into the
          current git branch.
        \item[\code{patch <patch file>}] Applies \code{patch file} to
          the current git branch.
      \end{description}
  \end{itemize}
\end{frame}
