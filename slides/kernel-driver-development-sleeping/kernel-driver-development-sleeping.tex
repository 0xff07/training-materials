\subsection{Sleeping}

\begin{frame}
  \frametitle{Sleeping}
  \begin{center}
    \includegraphics[width=\textwidth]{slides/kernel-driver-development-sleeping/sleeping.pdf}\\
    Sleeping is needed when a process (user space or kernel space) is
    waiting for data.
  \end{center}
\end{frame}

\begin{frame}[fragile]
  \frametitle{How to sleep 1/3}
  \begin{itemize}
  \item Must declare a wait queue, which will be used to store the list of threads
        waiting for an event
  \item Dynamic queue declaration:
      \begin{itemize}
      \item Typically one queue per device managed by the driver
      \item It's convenient to embed the wait queue inside a per-device data
        structure.
      \item Example from \kfile{drivers/net/ethernet/marvell/mvmdio.c}:
\begin{minted}{c}
struct orion_mdio_dev {
        ...
        wait_queue_head_t smi_busy_wait;
};
struct orion_mdio_dev *dev;
...
init_waitqueue_head(&dev->smi_busy_wait);
\end{minted}
    \end{itemize}
    \item Static queue declaration:
      \begin{itemize}
      \item Using a global variable when a global resource is sufficient
      \item \mint{c}+DECLARE_WAIT_QUEUE_HEAD(module_queue);+
      \end{itemize}
  \end{itemize}
\end{frame}

\begin{frame}[fragile]
  \frametitle{How to sleep 2/3}
  Several ways to make a kernel process sleep
  \begin{itemize}
  \item \mint{c}+void wait_event(queue, condition);+
    \begin{itemize}
    \item Sleeps until the task is woken up and the given C
      expression is true. Caution: can't be interrupted (can't kill
      the user space process!)
    \end{itemize}
  \item \mint{c}+int wait_event_killable(queue, condition);+
    \begin{itemize}
    \item Can be interrupted, but only by a \emph{fatal} signal
      (\ksym{SIGKILL}). Returns \code{-}\ksym{ERESTARTSYS} if interrupted.
    \end{itemize}
  \item \mint{c}+int wait_event_interruptible(queue, condition);+
    \begin{itemize}
    \item Can be interrupted by any signal. Returns
      \code{-}\ksym{ERESTARTSYS} if interrupted.
    \end{itemize}
  \end{itemize}
\end{frame}

\begin{frame}[fragile]
  \frametitle{How to sleep 3/3}
  \begin{itemize}
  \item \mint{c}+int wait_event_timeout(queue, condition, timeout);+
    \begin{itemize}
    \item Also stops sleeping when the task is woken up and the
      timeout expired. Returns \code{0} if the timeout elapsed, non-zero if
      the condition was met.
    \end{itemize}
  \item \begin{minted}{c}
int wait_event_interruptible_timeout(queue,
                         condition, timeout);
  \end{minted}
    \begin{itemize}
    \item Same as above, interruptible. Returns \code{0} if the timeout
      elapsed, \code{-}\ksym{ERESTARTSYS} if interrupted, positive value if
      the condition was met.
    \end{itemize}
  \end{itemize}
\end{frame}

\begin{frame}[fragile]
\frametitle{How to Sleep - Example}
\begin{minted}{c}
ret = wait_event_interruptible
        (sonypi_device.fifo_proc_list,
         kfifo_len(sonypi_device.fifo) != 0);

if (ret)
        return ret;
\end{minted}
\end{frame}

\begin{frame}
  \frametitle{Waking up!}
  Typically done by interrupt handlers when data sleeping
  processes are waiting for become available.
  \begin{itemize}
  \item \code{wake_up(&queue);}
    \begin{itemize}
    \item Wakes up all processes in the wait queue
    \end{itemize}
  \item \code{wake_up_interruptible(&queue);}
    \begin{itemize}
    \item Wakes up all processes waiting in an interruptible sleep
      on the given queue
    \end{itemize}
  \end{itemize}
\end{frame}

\begin{frame}
  \frametitle{Exclusive vs. non-exclusive}
  \begin{itemize}
  \item \kfunc{wait_event_interruptible} puts a task in a
    non-exclusive wait.
    \begin{itemize}
    \item All non-exclusive tasks are woken up by \kfunc{wake_up} /
      \kfunc{wake_up_interruptible}
    \end{itemize}
  \item \kfunc{wait_event_interruptible_exclusive} puts a task in an
    exclusive wait.
    \begin{itemize}
    \item \kfunc{wake_up} / \kfunc{wake_up_interruptible} wakes up
      all non-exclusive tasks and only one exclusive task
    \item \kfunc{wake_up_all} / \kfunc{wake_up_interruptible_all}
      wakes up all non-exclusive and all exclusive tasks
    \end{itemize}
  \item Exclusive sleeps are useful to avoid waking up multiple tasks
    when only one will be able to ``consume'' the event.
  \item Non-exclusive sleeps are useful when the event can ``benefit''
    to multiple tasks.
  \end{itemize}
\end{frame}

\begin{frame}[fragile]
  \frametitle{Sleeping and waking up - Implementation}
  \begin{columns}
    \column{0.5\textwidth}
      \includegraphics[height=0.8\textheight]{slides/kernel-driver-development-sleeping/wait-event.pdf}\\
    \column{0.5\textwidth}
       The scheduler doesn't keep evaluating the sleeping condition!
       \begin{itemize}
       \item \mint{c}+wait_event(queue, cond);+
         The process is put in the \ksym{TASK_UNINTERRUPTIBLE} state.
       \item \mint{c}+wake_up(&queue);+
         All processes waiting in \code{queue} are woken up, so they get
         scheduled later and have the opportunity to evaluate the
         condition again and go back to sleep if it is not met.
       \end{itemize}
       See \kfile{include/linux/wait.h} for implementation details.
  \end{columns}
\end{frame}
