\subsection{The U-boot bootloader}

\begin{frame}
  \frametitle{U-Boot}
  U-Boot is a typical free software project
  \begin{itemize}
  \item License: GPLv2 (same as Linux)
  \item Freely available at \url{https://www.denx.de/wiki/U-Boot}
  \item Documentation available at
    \url{https://www.denx.de/wiki/U-Boot/Documentation}
  \item The latest development source code is available in a Git
    repository:
    \url{https://gitlab.denx.de/u-boot/u-boot}
  \item Development and discussions happen around an open mailing-list
    \url{https://lists.denx.de/pipermail/u-boot/}
  \item Since the end of 2008, it follows a fixed-interval release
    schedule. Every two months, a new version is released. Versions
    are named \code{YYYY.MM}.
\end{itemize}
\end{frame}

\begin{frame}
  \frametitle{U-Boot configuration}
  \begin{itemize}
  \item Get the source code from the website, and uncompress it
  \item The \code{configs/} directory contains one configuration file
    for each supported board
    \begin{itemize}
    \item It defines the CPU type, the peripherals and their configuration, the
      memory mapping, the U-Boot features that should be compiled in, etc.
    \end{itemize}
  \item Note: U-Boot is migrating from board configuration defined in
    header files (\code{include/configs/}) to {\em defconfig} like in
    the Linux kernel (\code{configs/})
    \begin{itemize}
    \item Not all boards have been converted to the new configuration
      system.
    \item Older U-Boot releases provided by hardware vendors may not
      yet use this new configuration system.
    \end{itemize}
  \end{itemize}
\end{frame}

\begin{frame}[fragile]
  \frametitle{U-Boot configuration file}
\begin{block}{CHIP\_defconfig}
{\tiny
\begin{verbatim}
CONFIG_ARM=y
CONFIG_ARCH_SUNXI=y
CONFIG_MACH_SUN5I=y
CONFIG_DRAM_TIMINGS_DDR3_800E_1066G_1333J=y
# CONFIG_MMC is not set
CONFIG_USB0_VBUS_PIN="PB10"
CONFIG_VIDEO_COMPOSITE=y
CONFIG_DEFAULT_DEVICE_TREE="sun5i-r8-chip"
CONFIG_SPL=y
CONFIG_SYS_EXTRA_OPTIONS="CONS_INDEX=2"
# CONFIG_CMD_IMLS is not set
CONFIG_CMD_DFU=y
CONFIG_CMD_USB_MASS_STORAGE=y
CONFIG_AXP_ALDO3_VOLT=3300
CONFIG_AXP_ALDO4_VOLT=3300
CONFIG_USB_MUSB_GADGET=y
CONFIG_USB_GADGET=y
CONFIG_USB_GADGET_DOWNLOAD=y
CONFIG_G_DNL_MANUFACTURER="Allwinner Technology"
CONFIG_G_DNL_VENDOR_NUM=0x1f3a
CONFIG_G_DNL_PRODUCT_NUM=0x1010
CONFIG_USB_EHCI_HCD=y
\end{verbatim}}
\end{block}
\end{frame}

\begin{frame}
  \frametitle{Configuring and compiling U-Boot}
  \begin{itemize}
  \item U-Boot must be configured before being compiled
    \begin{itemize}
    \item \code{make BOARDNAME_defconfig}
    \item Where \code{BOARDNAME} is the name of a configuration, as
      visible in the \code{configs/} directory.
    \item You can then run \code{make menuconfig} to further customize
      U-Boot's configuration!
    \end{itemize}
  \item Make sure that the cross-compiler is available in \code{PATH}
  \item Compile U-Boot, by specifying the cross-compiler prefix.\\
    Example, if your cross-compiler executable is \code{arm-linux-gcc}:\\
    \code{make CROSS_COMPILE=arm-linux-}
  \item The main result is a \code{u-boot.bin} file, which is the
    U-Boot image. Depending on your specific platform, there may be
    other specialized images: \code{u-boot.img}, \code{u-boot.kwb},
    \code{MLO}, etc.
  \end{itemize}
\end{frame}

\begin{frame}
  \frametitle{Installing U-Boot}
  U-Boot must usually be installed in flash memory to be
  executed by the hardware. Depending on the hardware, the
  installation of U-Boot is done in a different way:
  \begin{itemize}
  \item The CPU provides some kind of specific boot monitor with
    which you can communicate through serial port or USB using a
    specific protocol
  \item The CPU boots first on removable media (MMC) before booting
    from fixed media (NAND). In this case, boot from MMC to reflash
    a new version
  \item U-Boot is already installed, and can be used to flash a new
    version of U-Boot. However, be careful: if the new version of
    U-Boot doesn't work, the board is unusable
  \item The board provides a JTAG interface, which allows to write
    to the flash memory remotely, without any system running on the
    board. It also allows to rescue a board if the bootloader
    doesn't work.
  \end{itemize}
\end{frame}

\begin{frame}[fragile]
  \frametitle{U-boot prompt}
  \begin{itemize}
  \item Connect the target to the host through a serial console.
  \item Power-up the board. On the serial console, you will see
        something like:
  \end{itemize}
\tiny
\begin{verbatim}
U-Boot 2016.05 (May 17 2016 - 12:41:15 -0400)

CPU: SAMA5D36
Crystal frequency:       12 MHz
CPU clock        :      528 MHz
Master clock     :      132 MHz
DRAM:  256 MiB
NAND:  256 MiB
MMC:   mci: 0

In:    serial
Out:   serial
Err:   serial
Net:   gmac0

Hit any key to stop autoboot:  0
=>
\end{verbatim}
\normalsize
  \begin{itemize}
  \item The U-Boot shell offers a set of commands. We will study the
    most important ones, see the documentation for a complete
    reference or the \code{help} command.
  \end{itemize}
\end{frame}

\begin{frame}[fragile]
  \frametitle{Information commands}

\begin{block}{Flash information (NOR and SPI flash)}
{\tiny
\begin{verbatim}
U-Boot> flinfo
DataFlash:AT45DB021
Nb pages: 1024
Page Size: 264
Size= 270336 bytes
Logical address: 0xC0000000
Area 0: C0000000 to C0001FFF (RO) Bootstrap
Area 1: C0002000 to C0003FFF Environment
Area 2: C0004000 to C0041FFF (RO) U-Boot
\end{verbatim}}
\end{block}

\begin{block}{NAND flash information}
{\tiny
\begin{verbatim}
U-Boot> nand info
Device 0: nand0, sector size 128 KiB
  Page size      2048 b
  OOB size         64 b
  Erase size   131072 b
\end{verbatim}}
\end{block}

\begin{block}{Version details}
{\tiny
\begin{verbatim}
U-Boot> version
U-Boot 2016.05 (May 17 2016 - 12:41:15 -0400)
\end{verbatim}}
\end{block}

\end{frame}

\begin{frame}
  \frametitle{Important commands (1)}
  \begin{itemize}
  \item The exact set of commands depends on the U-Boot configuration
  \item \code{help} and \code{help command}
  \item \code{ext2load}, loads a file from an ext2 filesystem to RAM
    \begin{itemize}
    \item And also \code{ext2ls} to list files, \code{ext2info} for
      information
    \end{itemize}
  \item \code{fatload}, loads a file from a FAT filesystem to RAM
    \begin{itemize}
    \item And also \code{fatls} and \code{fatinfo}
    \end{itemize}
  \item \code{tftp}, loads a file from the network to RAM
  \item \code{ping}, to test the network
  \item \code{boot}, runs the default boot command, stored in
    \code{bootcmd}
  \item \code{bootz <address>}, starts a kernel image loaded at the
    given address in RAM
  \end{itemize}
\end{frame}

\begin{frame}
  \frametitle{Important commands (2)}
  \begin{itemize}
  \item \code{loadb}, \code{loads}, \code{loady}, load a file from the
    serial line to RAM
  \item \code{usb}, to initialize and control the USB subsystem,
    mainly used for USB storage devices such as USB keys
  \item \code{mmc}, to initialize and control the MMC subsystem, used
    for SD and microSD cards
  \item \code{nand}, to erase, read and write contents to NAND flash
  \item \code{erase}, \code{protect}, \code{cp}, to erase, modify
    protection and write to NOR flash
  \item \code{md}, displays memory contents. Can be useful to check the
    contents loaded in memory, or to look at hardware registers.
  \item \code{mm}, modifies memory contents. Can be useful to modify
    directly hardware registers, for testing purposes.
\end{itemize}
\end{frame}

\begin{frame}
  \frametitle{Environment variables: principle}
  \begin{itemize}
  \item U-Boot can be configured through environment variables
    \begin{itemize}
    \item Some specific environment variables affect the behavior of
      the different commands
    \item Custom environment variables can be added, and used in
      scripts
    \end{itemize}
  \item Environment variables are loaded from flash to RAM at U-Boot
    startup, can be modified and saved back to flash for persistence
  \item There is a dedicated location in flash (or in MMC storage)
    to store the U-Boot environment, defined in the board configuration file
  \end{itemize}
\end{frame}

\begin{frame}
  \frametitle{Environment variables commands (2)}
  Commands to manipulate environment variables:
  \begin{itemize}
    \item \code{printenv}\\
      Shows all variables
    \item \code{printenv <variable-name>}\\
      Shows the value of a variable
    \item \code{setenv <variable-name> <variable-value>}\\
      Changes the value of a variable, only in RAM
    \item \code{editenv <variable-name>}\\
      Edits the value of a variable, only in RAM
    \item \code{saveenv}\\
      Saves the current state of the environment to flash
  \end{itemize}
\end{frame}

\begin{frame}[fragile]
\frametitle{Environment variables commands - Example}
\begin{verbatim}
u-boot # printenv
baudrate=19200
ethaddr=00:40:95:36:35:33
netmask=255.255.255.0
ipaddr=10.0.0.11
serverip=10.0.0.1
stdin=serial
stdout=serial
stderr=serial
u-boot # printenv serverip
serverip=10.0.0.1
u-boot # setenv serverip 10.0.0.100
u-boot # saveenv
\end{verbatim}
\end{frame}

\begin{frame}
  \frametitle{Important U-Boot env variables}
  \begin{itemize}
  \item \code{bootcmd}, specifies the commands that U-Boot will
    automatically execute at boot time after a configurable delay
    (\code{bootdelay}), if the process is not interrupted
  \item \code{bootargs}, contains the arguments passed to the Linux
    kernel, covered later
  \item \code{serverip}, the IP address of the server that U-Boot will
    contact for network related commands
  \item \code{ipaddr}, the IP address that U-Boot will use
  \item \code{netmask}, the network mask to contact the server
  \item \code{ethaddr}, the MAC address, can only be set once
  \item \code{autostart}, if set to \code{yes}, U-Boot automatically
    starts an image after loading it in memory (\code{tftp},
    \code{fatload}...)
  \item \code{filesize}, the size of the latest copy to memory
    (from \code{tftp}, \code{fatload}, \code{nand read}...)
  \end{itemize}
\end{frame}

\begin{frame}
  \frametitle{Scripts in environment variables}
  \begin{itemize}
  \item Environment variables can contain small scripts, to execute
    several commands and test the results of commands.
    \begin{itemize}
    \item Useful to automate booting or upgrade processes
    \item Several commands can be chained using the \code{;} operator
    \item Tests can be done using
      \code{if command ; then ... ; else ... ; fi}
    \item Scripts are executed using \code{run <variable-name>}
    \item You can reference other variables using
      \code{${variable-name}}
    \end{itemize}
  \item Example
    \begin{itemize}
    \item \code{setenv mmc-boot 'if fatload mmc 0 80000000
      boot.ini; then source; else if fatload mmc 0 80000000 zImage;
      then run mmc-do-boot; fi; fi'}
  \end{itemize}
\end{itemize}
\end{frame}

\begin{frame}
  \frametitle{Transferring files to the target}
  \begin{itemize}
  \item U-Boot is mostly used to load and boot a kernel image, but it
    also allows to change the kernel image and the root filesystem
    stored in flash.
  \item Files must be exchanged between the target and the development
    workstation. This is possible:
    \begin{itemize}
    \item Through the network if the target has an Ethernet
      connection, and U-Boot contains a driver for the Ethernet
      chip. This is the fastest and most efficient solution.
    \item Through a USB key, if U-Boot supports the USB controller of
      your platform
    \item Through a SD or microSD card, if U-Boot supports the MMC
      controller of your platform
    \item Through the serial port
    \end{itemize}
  \end{itemize}
\end{frame}

\begin{frame}
  \frametitle{TFTP}
  \begin{itemize}
  \item Network transfer from the development workstation to U-Boot
    on the target takes place through TFTP
    \begin{itemize}
    \item {\em Trivial File Transfer Protocol}
    \item Somewhat similar to FTP, but without authentication and over
      UDP
    \end{itemize}
  \item A TFTP server is needed on the development workstation
    \begin{itemize}
    \item \code{sudo apt install tftpd-hpa}
    \item All files in \code{/var/lib/tftpboot} are then visible
      through TFTP
    \item A TFTP client is available in the \code{tftp-hpa} package,
      for testing
    \end{itemize}
  \item A TFTP client is integrated into U-Boot
    \begin{itemize}
    \item Configure the \code{ipaddr} and \code{serverip} environment
      variables
    \item Use \code{tftp <address> <filename>} to load a file
    \end{itemize}
  \end{itemize}
\end{frame}
