\subsection{Linux Code and Device Drivers}

\begin{frame}
  \frametitle{Programming language}
  \begin{itemize}
  \item Implemented in C like all Unix systems. (C was created to
    implement the first Unix systems)
  \item A little Assembly is used too:
    \begin{itemize}
    \item CPU and machine initialization, exceptions
    \item Critical library routines.
    \end{itemize}
  \item No C++ used, see \url{http://www.tux.org/lkml/\#s15-3}
  \item All the code compiled with gcc
    \begin{itemize}
    \item Many gcc specific extensions used in the kernel code, any
      ANSI C compiler will not compile the kernel
    \item A few alternate compilers are supported (Intel and Marvell)
    \item See
      \url{http://gcc.gnu.org/onlinedocs/gcc-4.9.0/gcc/C-Extensions.html}
    \end{itemize}
  \end{itemize}
\end{frame}

\begin{frame}
  \frametitle{No C library}
  \begin{itemize}
  \item The kernel has to be standalone and can't use user space code.
  \item User space is implemented on top of kernel services, not the
    opposite.
  \item Kernel code has to supply its own library implementations
    (string utilities, cryptography, uncompression ...)
  \item So, you can't use standard C library functions in kernel code.
    (\code{printf()}, \code{memset()}, \code{malloc()},...).
  \item Fortunately, the kernel provides similar C functions for your
    convenience, like \kfunc{printk}, \kfunc{memset},
    \kfunc{kmalloc}, ...
  \end{itemize}
\end{frame}

\begin{frame}
  \frametitle{Portability}
  \begin{itemize}
  \item The Linux kernel code is designed to be portable
  \item All code outside \kpath{arch/} should be portable
  \item To this aim, the kernel provides macros and functions to
    abstract the architecture specific details
    \begin{itemize}
    \item Endianness
      \begin{itemize}
      \item \kfunc{cpu_to_be32}
      \item \kfunc{cpu_to_le32}
      \item \kfunc{be32_to_cpu}
      \item \kfunc{le32_to_cpu}
      \end{itemize}
    \item I/O memory access
    \item Memory barriers to provide ordering guarantees if needed
    \item DMA API to flush and invalidate caches if needed
    \end{itemize}
  \end{itemize}
\end{frame}

\begin{frame}
  \frametitle{No floating point computation}
  \begin{itemize}
  \item Never use floating point numbers in kernel code. Your code may
    be run on a processor without a floating point unit (like on
    certain ARM CPUs).
  \item Don't be confused with floating point related configuration
    options
    \begin{itemize}
    \item They are related to the emulation of floating point
      operation performed by the user space applications, triggering
      an exception into the kernel.
    \item Using soft-float, i.e. emulation in user space, is however
      recommended for performance reasons.
    \end{itemize}
  \end{itemize}
\end{frame}

\begin{frame}
  \frametitle{No stable Linux internal API}
  \begin{itemize}
  \item The internal kernel API to implement kernel code can undergo
    changes between two releases.
  \item In-tree drivers are updated by the developer proposing the API
    change: works great for mainline code.
  \item An out-of-tree driver compiled for a given version may no
    longer compile or work on a more recent one.
  \item See \kerneldoc{stable_api_nonsense.txt} in kernel sources for
    reasons why.
  \item Of course, the kernel to userspace API does not change (system
    calls, \code{/proc}, \code{/sys}), as it would break existing
    programs.
  \end{itemize}
\end{frame}

\begin{frame}
  \frametitle{Kernel memory constraints}
  \begin{itemize}
  \item No memory protection
  \item Accessing illegal memory locations result in (often fatal)
    kernel oopses.
  \item Fixed size stack (8 or 4 KB). Unlike in user space, there's no
    way to make it grow.
  \item Kernel memory can't be swapped out (for the same reasons).
  \end{itemize}
\end{frame}

\begin{frame}
  \frametitle{Linux kernel licensing constraints}
  \begin{itemize}
  \item The Linux kernel is licensed under the GNU General Public
    License version 2
    \begin{itemize}
    \item This license gives you the right to use, study, modify and
      share the software freely
    \end{itemize}
  \item However, when the software is redistributed, either modified
    or unmodified, the GPL requires that you redistribute the software
    under the same license, with the source code
    \begin{itemize}
    \item If modifications are made to the Linux kernel (for example
      to adapt it to your hardware), it is a derivative work of the
      kernel, and therefore must be released under GPLv2
    \item The validity of the GPL on this point has already been
      verified in courts
    \end{itemize}
  \item However, you're only required to do so
    \begin{itemize}
    \item At the time the device starts to be distributed
    \item To your customers, not to the entire world
    \end{itemize}
  \end{itemize}
\end{frame}

\begin{frame}
  \frametitle{Proprietary code and the kernel}
  \begin{itemize}
  \item It is illegal to distribute a binary kernel that includes
    statically compiled proprietary drivers
  \item The kernel modules are a gray area: are they derived works of
    the kernel or not?
    \begin{itemize}
    \item The general opinion of the kernel community is that
      proprietary drivers are bad: \url{http://j.mp/fbyuuH}
    \item From a legal point of view, each driver is probably a
      different case
    \item Is it really useful to keep your drivers secret?
    \end{itemize}
  \item There are some examples of proprietary drivers, like the
    Nvidia graphics drivers
    \begin{itemize}
    \item They use a wrapper between the driver and the kernel
    \item Unclear whether it makes it legal or not
    \end{itemize}
  \end{itemize}
\end{frame}

\begin{frame}
  \frametitle{Advantages of GPL drivers}
  \begin{itemize}
  \item You don't have to write your driver from scratch. You can
    reuse code from similar free software drivers.
  \item You could get free community contributions, support, code
    review and testing, though this generally only happens with code
    submitted for the mainline kernel.
  \item Your drivers can be freely and easily shipped by others (for
    example by Linux distributions or embedded Linux build systems).
  \item Pre-compiled drivers work with only one kernel version and one
    specific configuration, making life difficult for users who want
    to change the kernel version.
  \item Legal certainty, you are sure that a GPL driver is fine from a
    legal point of view.
  \end{itemize}
\end{frame}

\begin{frame}
  \frametitle{Advantages of in-tree kernel drivers}
  \begin{itemize}
  \item Once your sources are accepted in the mainline tree, they are
    maintained by people making changes.
  \item Near cost-free maintenance, security fixes and improvements.
  \item Easy access to your sources by users.
  \item Many more people reviewing your code.
  \end{itemize}
\end{frame}

\begin{frame}
  \frametitle{User space device drivers 1/3}
  \begin{itemize}
  \item In some cases, it is possible to implement device drivers in
    user space!
  \item Can be used when
    \begin{itemize}
    \item The kernel provides a mechanism that allows userspace
      applications to directly access the hardware.
    \item There is no need to leverage an existing kernel subsystem
      such as the networking stack or filesystems.
    \item There is no need for the kernel to act as a ``multiplexer''
      for the device: only one application accesses the device.
    \end{itemize}
  \end{itemize}
\end{frame}

\begin{frame}
  \frametitle{User space device drivers 2/3}
  \begin{itemize}
  \item Possibilities for userspace device drivers:
    \begin{itemize}
    \item USB with {\em libusb}, \url{http://www.libusb.org/}
    \item SPI with {\em spidev}, \kerneldoc{Documentation/spi/spidev}
    \item I2C with {\em i2cdev}, \kerneldoc{Documentation/i2c/dev-interface}
    \item Memory-mapped devices with {\em UIO}, including interrupt
      handling, \url{http://free-electrons.com/kerneldoc/latest/DocBook/uio-howto/}
    \end{itemize}
  \item Certain classes of devices (printers, scanners, 2D/3D graphics
    acceleration) are typically handled partly in kernel space, partly
    in user space.
  \end{itemize}
\end{frame}

\begin{frame}
  \frametitle{User space device drivers 3/3}
  \begin{itemize}
  \item Advantages
    \begin{itemize}
    \item No need for kernel coding skills. Easier to reuse code
      between devices.
    \item Drivers can be written in any language, even Perl!
    \item Drivers can be kept proprietary.
    \item Driver code can be killed and debugged. Cannot crash the
      kernel.
    \item Can be swapped out (kernel code cannot be).
    \item Can use floating-point computation.
    \item Less in-kernel complexity.
    \item Potentially higher performances, especially for
      memory-mapped devices, thanks to the avoidance of system calls.
    \end{itemize}
  \item Drawbacks
    \begin{itemize}
    \item Less straightforward to handle interrupts.
    \item Increased interrupt latency vs. kernel code.
    \end{itemize}
  \end{itemize}
\end{frame}
