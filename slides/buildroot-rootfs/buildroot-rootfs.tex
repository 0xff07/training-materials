\setbeamerfont{block title}{size=\scriptsize}

\section{Root filesystem in Buildroot}

\begin{frame}{Overall rootfs construction steps}
  \begin{center}
    \includegraphics[width=\textwidth]{slides/buildroot-rootfs/overall-steps.pdf}
  \end{center}
\end{frame}

\begin{frame}{Root filesystem skeleton}
  \begin{itemize}
  \item The base of a Linux root filesystem: Unix directory hierarchy,
    a few configuration files and scripts in \code{/etc}. No programs
    or libraries.
  \item First thing to get copied to \code{$(TARGET_DIR)} at the
    beginning of the build.
  \item By default (\code{BR2_ROOTFS_SKELETON_DEFAULT=y}), the one in
    \code{system/skeleton} is used.
  \item A custom {\em skeleton} can be used, through the
    \code{BR2_ROOTFS_SKELETON_CUSTOM} and
    \code{BR2_ROOTFS_SKELETON_CUSTOM_PATH} options.
    \begin{itemize}
    \item Not recommended though: the {\em skeleton} is only copied
      once at the beginning of the build, and the base is usually good
      for most projects.
    \item Use {\em rootfs overlays} or {\em post-build scripts} for
      customization.
    \end{itemize}
  \end{itemize}
\end{frame}

\begin{frame}{Installation of packages}
  \begin{itemize}
  \item All the target packages will be built: Busybox, Qt, OpenSSH,
    and the thousands of other possible packages.
  \item Most of them will install files in \code{$(TARGET_DIR)}:
    programs, libraries, fonts, data files, configuration files, etc.
  \item This is really the step that will bring in the root filesystem
    the vast majority of the files.
  \item Covered in more details in the section about creating your own
    Buildroot packages.
  \end{itemize}
\end{frame}

\begin{frame}{Cleanup step}
  \begin{itemize}
  \item Once all packages have been installed, a cleanup step is
    executed to reduce the size of the root filesystem.
  \item It mainly involves:
    \begin{itemize}
    \item Remove header files, pkg-config files, CMake files, static
      libraries, man pages, documentation.
    \item Stripping all the programs and libraries using \code{strip},
      to remove unneeded information. Depends on
      \code{BR2_ENABLE_DEBUG} and \code{BR2_STRIP_*} options.
    \item Additional specific clean up steps: clean up unneeded Python
      files when Python is used, etc. See \code{TARGET_FINALIZE_HOOKS}
      in the Buildroot code.
    \end{itemize}
  \end{itemize}
\end{frame}

\begin{frame}[fragile]{Root filesystem overlay}

  \begin{itemize}

  \item To customize the contents of your root filesystem, to add
    configuration files, scripts or any other file, one possible
    solution is to use a {\bf root filesystem overlay}.

  \item A {\em root filesystem overlay} is simply a directory whose
    contents will be {\bf copied over the root filesystem}, after all
    packages have been installed. Overwriting files is allowed.

  \item The option \code{BR2_ROOTFS_OVERLAY} contains a
    space-separated list of overlay paths.

\begin{block}{}
{\small
\begin{verbatim}
$ grep ^BR2_ROOTFS_OVERLAY .config
BR2_ROOTFS_OVERLAY="board/myproject/rootfs-overlay"
$ find -type f board/myproject/rootfs-overlay
board/myproject/rootfs-overlay/etc/ssh/sshd_config
board/myproject/rootfs-overlay/etc/init.d/S99myapp
\end{verbatim}}
\end{block}

  \end{itemize}

\end{frame}

\begin{frame}{Post-build scripts}
  \begin{itemize}
  \item Sometimes a {\em root filesystem overlay} is not sufficient:
    you can use {\bf post-build scripts}.
  \item Can be used to {\bf customize existing files}, {\bf remove
      unneeded files} to save space, add {\bf new files that are
      generated dynamically} (build date, etc.)
  \item Executed before the root filesystem image is created. Can be
    written in any language, shell scripts are often used.
  \item \code{BR2_ROOTFS_POST_BUILD_SCRIPT} contains a space-separated
    list of post-build script paths.
  \item \code{$(TARGET_DIR)} path passed as first argument, additional
    arguments can be passed in the \code{BR2_ROOTFS_POST_SCRIPT_ARGS}
    option.
  \item Various environment variables are available:
    \begin{itemize}
      \item \code{BR2_CONFIG}, path to the Buildroot .config file
      \item \code{HOST_DIR}, \code{STAGING_DIR}, \code{TARGET_DIR},
        \code{BUILD_DIR}, \code{BINARIES_DIR}, \code{BASE_DIR}
      \end{itemize}
  \end{itemize}
\end{frame}

\begin{frame}[fragile]{Post-build script: example}

\begin{block}{board/myproject/post-build.sh}
\begin{minted}[fontsize=\scriptsize]{bash}
#!/bin/sh
TARGET_DIR=$1
BOARD_DIR=board/myproject/

# Generate a file identifying the build (git commit and build date)
echo $(git describe) $(date +%Y-%M-%d-%H:%m:%S) > \
    $TARGET_DIR/etc/build-id

# Create /applog mountpoint, and adjust /etc/fstab
mkdir -p $TARGET_DIR/applog
grep -q "^/dev/mtdblock7" $TARGET_DIR/etc/fstab || \
    echo "/dev/mtdblock7\t\t/applog\tjffs2\tdefaults\t\t0\t0" >> \
    $TARGET_DIR/etc/fstab

# Remove unneeded files
rm -rf $TARGET_DIR/usr/share/icons/bar
\end{minted}
\end{block}

\begin{block}{Buildroot configuration}
{\scriptsize
\begin{verbatim}
BR2_ROOTFS_POST_BUILD_SCRIPT="board/myproject/post-build.sh"
\end{verbatim}}
\end{block}

\end{frame}

\begin{frame}{Generating the filesystem images}

  \begin{itemize}
  \item In the \code{Filesystem images} menu, you can select which
    filesystem image formats to generate.
  \item To generate those images, Buildroot will generate a shell
    script that:
    \begin{itemize}
    \item {\bf Changes the owner} of all files to \code{0:0} (root user)
    \item Takes into account the global {\bf permission and device tables},
      as well as the per-package ones.
    \item Takes into account the {\bf global and per-package users
        tables}.
    \item Runs the {\bf filesystem image generation utility}, which
      depends on each filesystem type (\code{genext2fs},
      \code{mkfs.ubifs}, \code{tar}, etc.)
    \end{itemize}
  \item This script is executed using a tool called {\em fakeroot}
    \begin{itemize}
    \item Allows to fake being root so that permissions and ownership
      can be modified, device files can be created, etc.
    \end{itemize}
  \end{itemize}
\end{frame}

\begin{frame}[fragile]{Permission table}
  \begin{itemize}
  \item By default, all files are owned by the \code{root} user, and
    the permissions with which they are installed in
    \code{$(TARGET_DIR)} are preserved.
  \item To customize the ownership or the permission of installed
    files, one can create one or several {\bf permission tables}
  \item \code{BR2_ROOTFS_DEVICE_TABLE} contains a space-separated list
    of permission table files. The option name contains {\em device}
    for backward compatibility reasons only.
  \item The \code{system/device_table.txt} file is used by default.
  \item Packages can also specify their own permissions. See the {\em
      Advanced package aspects} section for details.
  \end{itemize}

\begin{block}{Permission table example}
{\tiny
\begin{verbatim}
#<name>    <type>  <mode>  <uid>   <gid>   <major> <minor> <start> <inc>   <count>
/dev       d       755     0       0       -       -       -       -       -
/tmp       d       1777    0       0       -       -       -       -       -
/var/www   d       755     33      33      -       -       -       -       -
\end{verbatim}
}
\end{block}

\end{frame}

\begin{frame}[fragile]{Device table}

  \begin{itemize}
  \item When the system is using a static \code{/dev}, one may need to
    create additional {\em device nodes}
  \item Done using one or several {\bf device tables}
  \item \code{BR2_ROOTFS_STATIC_DEVICE_TABLE} contains a
    space-separated list of device table files.
  \item The \code{system/device_table_dev.txt} file is used by
    default.
  \item Packages can also specify their own device files. See the {\em
      Advanced package aspects} section for details.
  \end{itemize}

\begin{block}{Device table example}
{\tiny
\begin{verbatim}
# <name>        <type>  <mode>  <uid>   <gid>   <major> <minor> <start> <inc>   <count>
/dev/mem        c       640     0       0       1       1       0       0       -
/dev/kmem       c       640     0       0       1       2       0       0       -
/dev/i2c-       c       666     0       0       89      0       0       1       4
\end{verbatim}
}
\end{block}

\end{frame}

\begin{frame}[fragile]{Users table}
  \begin{itemize}
  \item One may need to add specific Unix users and groups in addition
    to the one available in the default skeleton.
  \item \code{BR2_ROOTFS_USERS_TABLES} is a space-separated list of
    user tables.
  \item Packages can also specify their own users. See the {\em
      Advanced package aspects} section for details.
  \end{itemize}

\begin{block}{Users table example}
{\tiny
\begin{verbatim}
# <username> <uid> <group> <gid> <password> <home>    <shell> <groups>    <comment>
foo          -1    bar     -1    !=blabla   /home/foo /bin/sh alpha,bravo Foo user
test         8000  wheel   -1    =          -         /bin/sh -           Test user
\end{verbatim}
}
\end{block}

\end{frame}

\begin{frame}{Post-image scripts}
  \begin{itemize}
  \item Once all the filesystem images have been created, at the very
    end of the build, {\bf post-image} scripts are called.
  \item They allow to do any custom action at the end of the
    build. For example:
    \begin{itemize}
    \item Extract the root filesystem to do NFS booting
    \item Generate a final firmware image
    \item Start the flashing process
    \end{itemize}
  \item \code{BR2_ROOTFS_POST_IMAGE_SCRIPT} is a space-separated list
    of {\em post-image} scripts to call.
  \end{itemize}
\end{frame}

\begin{frame}{Init mechanism}
  \begin{itemize}
  \item Buildroot supports multiple {\em init} implementations:
    \begin{itemize}
    \item {\bf Busybox init}, the default. Simplest solution.
    \item {\bf sysvinit}, the old style featureful {\em init}
      implementation
    \item {\bf systemd}, the new generation init system
    \end{itemize}
  \item Selecting the {\em init} implementation in the \code{System
      configuration} menu will:
    \begin{itemize}
    \item Ensure the necessary packages are selected
    \item Make sure the appropriate init scripts or configuration
      files are installed by packages. See {\em Advanced package
        aspects} for details.
    \end{itemize}
  \end{itemize}
\end{frame}

\begin{frame}{{\tt /dev} management method}
  \begin{itemize}
  \item Buildroot supports four methods to handle the \code{/dev}
    directory:
    \begin{itemize}
    \item Using {\bf devtmpfs}. \code{/dev} is managed by the kernel
      {\em devtmpfs}, which creates device files
      automatically. Requires kernel 2.6.32+. Default option.
    \item Using {\bf static /dev}. This is the old way of doing
      \code{/dev}, not very practical.
    \item Using {\bf mdev}. \code{mdev} is part of Busybox and can run
      custom actions when devices are added/removed. Requires {\em
        devtmpfs} kernel support.
    \item Using {\bf eudev}. Forked from \code{systemd}, allows to run
      custom actions. Requires {\em devtmpfs} kernel support.
    \end{itemize}
  \item When {\em systemd} is used, the only option is {\em udev} from
    {\em systemd} itself.
  \end{itemize}
\end{frame}

\begin{frame}{Other customization options}
  \begin{itemize}
  \item There are various other options to customize the root
    filesystem:
    \begin{itemize}
    \item {\bf getty} options, to run a login prompt on a serial port
      or screen
    \item {\bf hostname} and {\bf banner} options
    \item {\bf DHCP network} on one interface (for more complex
      setups, use an {\em overlay})
    \item {\bf root password}
    \item {\bf timezone} installation and selection
    \end{itemize}
  \end{itemize}
\end{frame}

\begin{frame}{Deploying the images}
  \begin{itemize}
  \item By default, Buildroot simply stores the different images in
    \code{$(O)/images}
  \item It is up to the user to deploy those images to the target
    device.
  \item Possible solutions:
    \begin{itemize}
    \item For removable storage (SD card, USB keys):
      \begin{itemize}
      \item manually create the partitions and extract the root
        filesystem as a tarball to the appropriate partition.
      \item use a tool like \code{genimage} to create a complete image
        of the media, including all partitions
      \end{itemize}
    \item For NAND flash:
      \begin{itemize}
      \item Transfer the image to the target, and flash it.
      \end{itemize}
    \item NFS booting
    \item initramfs
    \end{itemize}
  \end{itemize}
\end{frame}

\begin{frame}{Deploying the image: NFS booting}
  \begin{itemize}
  \item Many people try to use \code{$(O)/target} directly for NFS booting
    \begin{itemize}
    \item This cannot work, due to permissions/ownership being
      incorrect
    \item Clearly explained in the
      \code{THIS_IS_NOT_YOUR_ROOT_FILESYSTEM} file.
    \end{itemize}
  \item Generate a tarball of the root filesystem
  \item Use \code{sudo tar -C /nfs -xf output/images/rootfs.tar} to
    prepare your NFS share.
  \end{itemize}
\end{frame}

\begin{frame}{Deploying the image: initramfs}
  \begin{itemize}

  \item Another common use case is to use an {\em initramfs}, i.e a
    root filesystem fully in RAM.
    \begin{itemize}
    \item Convenient for small filesystems, fast booting or kernel
      development
    \end{itemize}
  \item Two solutions:
    \begin{itemize}
    \item \code{BR2_TARGET_ROOTFS_CPIO=y} to generate a {\em cpio}
      archive, that you can load from your bootloader next to the
      kernel image.
    \item \code{BR2_TARGET_ROOTFS_INITRAMFS=y} to directly include the
      {\em initramfs} inside the kernel image. Only available when the
      kernel is built by Buildroot.
    \end{itemize}
  \end{itemize}
\end{frame}

\setuplabframe
{Root filesystem construction}
{
  \begin{itemize}
  \item Explore the build output
  \item Customize the root filesystem using a rootfs overlay
  \item Use a post-build script
  \item Customize the kernel with patches and additional configuration
    options
  \item Add more packages
  \item Use defconfig files and out of tree build
  \end{itemize}
}
