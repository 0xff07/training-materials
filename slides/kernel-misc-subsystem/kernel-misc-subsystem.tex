\section{The misc subsystem}

\begin{frame}{Why a {\em misc} subsystem?}
  \begin{itemize}
  \item The kernel offers a large number of {\bf frameworks} covering
    a wide range of device types: input, network, video, audio,
    etc.
    \begin{itemize}
    \item These frameworks allow to factorize common functionality
      between drivers and offer a consistent API to user space
      applications.
    \end{itemize}
  \item However, there are some devices that {\bf really do not fit in any
    of the existing frameworks}.
    \begin{itemize}
    \item Highly customized devices implemented in a FPGA, or other
      weird devices for which implementing a complete framework is not
      useful.
    \end{itemize}
  \item The drivers for such devices could be implemented directly as
    raw {\em character drivers} (with \kfunc{cdev_init} and
    \kfunc{cdev_add}).
  \item But there is a subsystem that makes this work a little bit
    easier: the {\bf misc subsystem}.
    \begin{itemize}
    \item It is really only a {\bf thin layer} above the {\em character
        driver} API.
    \item Another advantage is that devices are integrated in the Device
        Model (device files appearing in {\em devtmpfs}, which you don't
	have with raw character devices).
    \end{itemize}
  \end{itemize}
\end{frame}

\begin{frame}{Misc subsystem diagram}
  \begin{center}
    \includegraphics[width=\textwidth]{slides/kernel-misc-subsystem/misc-subsystem-diagram.pdf}
  \end{center}
\end{frame}

\begin{frame}[fragile]{Misc subsystem API (1/2)}
  \begin{itemize}
  \item The misc subsystem API mainly provides two functions, to
    register and unregister {\bf a single} {\em misc device}:
    \begin{itemize}
    \item \code{int misc_register(struct miscdevice * misc);}
    \item \code{void misc_deregister(struct miscdevice *misc);}
    \end{itemize}
  \item A {\em misc device} is described by a \kstruct{miscdevice}
    structure:
    \begin{minted}[fontsize=\footnotesize]{c}
struct miscdevice  {
        int minor;
        const char *name;
        const struct file_operations *fops;
        struct list_head list;
        struct device *parent;
        struct device *this_device;
        const char *nodename;
        umode_t mode;
};
\end{minted}
  \end{itemize}
\end{frame}

\begin{frame}[fragile]{Misc subsystem API (2/2)}
  The main fields to be filled in \kstruct{miscdevice} are:
  \begin{itemize}
  \item \code{minor}, the minor number for the device, or
    \ksym{MISC_DYNAMIC_MINOR} to get a minor number automatically
    assigned.
  \item \code{name}, name of the device, which will be used to create
    the device node if {\em devtmpfs} is used.
  \item \code{fops}, pointer to a \kstruct{file_operations} structure, that
    describes which functions implement the {\em read}, {\em write},
    {\em ioctl}, etc. operations.
  \end{itemize}
\end{frame}

\begin{frame}{User space API for misc devices}
  \begin{itemize}
  \item {\em misc devices} are regular character devices
  \item The operations they support in user space depends on the
    operations the kernel driver implements:
    \begin{itemize}
    \item The \code{open()} and \code{close()} system calls to
      open/close the device.
    \item The \code{read()} and \code{write()} system calls to
      read/write to/from the device.
    \item The \code{ioctl()} system call to call some driver-specific
      operations.
    \end{itemize}
  \end{itemize}
\end{frame}

\setuplabframe
{Output-only serial port driver}
{
  \begin{itemize}
  \item Extend the driver started in the previous lab by registering
    it into the {\em misc} subsystem.
  \item Implement serial output functionality through the {\em misc}
    subsystem.
  \item Test serial output using user space applications.
  \end{itemize}
}
