\begin{frame}
  \frametitle{proc virtual filesystem}
  \begin{itemize}
  \item The \code{proc} virtual filesystem exists since the beginning of
    Linux
  \item It allows
    \begin{itemize}
    \item The kernel to expose statistics about running processes in
      the system
    \item The user to adjust at runtime various system parameters
      about process management, memory management, etc.
    \end{itemize}
  \item The \code{proc} filesystem is used by many standard userspace
    applications, and they expect it to be mounted in /proc
  \item Applications such as \code{ps} or \code{top} would not work
    without the \code{proc} filesystem
  \item Command to mount \code{/proc}:\\
    \code{mount -t proc nodev /proc}
  \item \code{Documentation/filesystems/proc.txt} in the kernel
    sources
  \item \code{man proc}
  \end{itemize}
\end{frame}

\begin{frame}
  \frametitle{proc contents}
  \begin{itemize}
  \item One directory for each running process in the system
    \begin{itemize}
    \item \code{/proc/<pid>}
    \item \code{cat /proc/3840/cmdline}
    \item It contains details about the files opened by the process,
      the CPU and memory usage, etc.
    \end{itemize}
  \item \code{/proc/interrupts}, \code{/proc/devices},
    \code{/proc/iomem}, \code{/proc/ioports} contain general
    device-related information
  \item \code{/proc/cmdline} contains the kernel command line
  \item \code{/proc/sys} contains many files that can be written to to
    adjust kernel parameters
    \begin{itemize}
    \item They are called {\em sysctl}. See
      \code{Documentation/sysctl/} in kernel sources.
    \item Example\\
      \code{echo 3 > /proc/sys/vm/drop_caches}
    \end{itemize}
  \end{itemize}
\end{frame}

\begin{frame}[fragile]
  \frametitle{sysfs filesystem}
  \begin{itemize}
  \item The \code{sysfs} filesystem is a feature integrated in the 2.6
    Linux kernel
  \item It allows to represent in userspace the vision that the kernel
    has of the buses, devices and drivers in the system
  \item It is useful for various userspace applications that need to
    list and query the available hardware, for example udev or mdev
  \item All applications using sysfs expect it to be mounted in the
    \code{/sys} directory
  \item Command to mount \code{/sys}:\\
    \code{mount -t sysfs nodev /sys}
  \item
\begin{verbatim}
$ ls /sys/
block bus class dev devices firmware
fs kernel modulepower
\end{verbatim}
  \end{itemize}
\end{frame}

\begin{frame}
  \frametitle{Basic applications}
  \begin{itemize}
  \item In order to work, a Linux system needs at least a few
    applications
  \item An \code{init} application, which is the first userspace
    application started by the kernel after mounting the root
    filesystem
    \begin{itemize}
    \item The kernel tries to run \code{/sbin/init}, \code{/bin/init},
      \code{/etc/init} and \code{/bin/sh}.
    \item If none of them are found, the kernel panics and the boot
      process is stopped.
    \item The init application is responsible for starting all other
      userspace applications and services
    \end{itemize}
  \item Usually a shell, to allow a user to interact with the system
  \item Basic Unix applications, to copy files, move files, list files
    (commands like \code{mv}, \code{cp}, \code{mkdir}, \code{cat},
    etc.)
  \item Those basic components have to be integrated into the root
    filesystem to make it usable
  \end{itemize}
\end{frame}

\begin{frame}
  \frametitle{Overall booting process}
  \begin{center}
    \includegraphics[width=0.7\textwidth]{slides/sysdev-root-filesystem-part2/overall-boot-sequence.pdf}
  \end{center}
\end{frame}
