\subsection{JNI}

\begin{frame}
  \frametitle{Whole Android Stack}
  \begin{center}
    \includegraphics[height=0.85\textheight]{slides/android-native-layer-jni/android-stack-jni.pdf}
  \end{center}
\end{frame}

\begin{frame}
  \frametitle{What is JNI?}
  \begin{itemize}
  \item A Java framework to call and be called by native applications
    written in other languages
  \item Mostly used for:
    \begin{itemize}
    \item Writing Java bindings to C/C++ libraries
    \item Accessing platform-specific features
    \item Writing high-performance sections
    \end{itemize}
  \item It is used extensively across the Android user space to
    interface between the Java Framework and the native daemons
  \item Since Gingerbread, you can develop apps in a purely native
    way, possibly calling Java methods through JNI
  \end{itemize}
\end{frame}

\begin{frame}[fragile]
  \frametitle{C Code}
\begin{minted}[fontsize=\footnotesize]{c}
#include "jni.h"

JNIEXPORT void JNICALL Java_com_example_Print_print(JNIEnv *env,
                                                    jobject obj,
                                                    jstring javaString)
{
    const char *nativeString = (*env)->GetStringUTFChars(env,
                                                         javaString,
                                                         0);
    printf("%s", nativeString);
    (*env)->ReleaseStringUTFChars(env, javaString, nativeString);
}
\end{minted}
\end{frame}

\begin{frame}[fragile]
  \frametitle{JNI arguments}
  \begin{itemize}
  \item Function prototypes are following the template:
\begin{minted}[fontsize=\footnotesize]{c}
JNIEXPORT jstring JNICALL Java_ClassName_MethodName
          (JNIEnv*, jobject)
\end{minted}
  \item \code{JNIEnv} is a pointer to the JNI Environment that we will
    use to interact with the virtual machine and manipulate Java objects
    within the native methods
  \item \code{jobject} contains a pointer to the calling object. It is very
    similar to \code{this} in C++
  \end{itemize}
\end{frame}

\begin{frame}
  \frametitle{Types}
  \begin{itemize}
  \item There is no direct mapping between C Types and JNI types
  \item You must use the JNI primitives to convert one to his equivalent
  \item However, there are a few types that are directly mapped, and
    thus can be used directly without typecasting:
  \end{itemize}
  \begin{center}
    \begin{tabular}{|c|c|}
      \hline
      Native Type & JNI Type\\
      \hline
      unsigned char & jboolean\\
      \hline
      signed char & jbyte\\
      \hline
      unsigned short & jchar\\
      \hline
      short & jshort\\
      \hline
      long & jint\\
      \hline
      long long & jlong\\
      \hline
      float & jfloat\\
      \hline
      double & jdouble\\
      \hline
    \end{tabular}
  \end{center}
\end{frame}

\begin{frame}[fragile]
  \frametitle{Java Code}
\begin{minted}[fontsize=\footnotesize]{java}
package com.example;

class Print
{
        private static native void print(String str);

        public static void main(String[] args)
        {
                Print.print("HelloWorld!");
        }

        static
        {
                System.loadLibrary("print");
        }
}
\end{minted}
\end{frame}

\begin{frame}[fragile]
  \frametitle{Calling a method of a Java object from C}
\begin{minted}[fontsize=\footnotesize]{c}
JNIEXPORT void JNICALL Java_ClassName_Method(JNIEnv *env,
                                             jobject obj)
{
  jclass cls = (*env)->GetObjectClass(env, obj);
  jmethodID hello = (*env)->GetMethodID(env,
                                        cls,
                                        "hello",
                                        "(V)V");
  if (!hello)
    return;
  (*env)->CallVoidMethod(env, obj, hello);
}
\end{minted}
\end{frame}

\begin{frame}[fragile]
  \frametitle{Instantiating a Java object from C}
\begin{minted}[fontsize=\footnotesize]{c}
JNIEXPORT jobject JNICALL Java_ClassName_Method(JNIEnv *env,
                                                jobject obj) 
{
    jclass cls = env->FindClass("java/util/ArrayList");
    jmethodID init = env->GetMethodID(cls,
                                      "<init>",
                                      "()V");
    jobject array = env->NewObject(cls, init);

    return array;
}
\end{minted}
\end{frame}
