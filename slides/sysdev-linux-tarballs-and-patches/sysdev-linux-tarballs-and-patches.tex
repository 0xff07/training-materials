\begin{frame}
  \frametitle{Getting Linux sources}
  \begin{itemize}
  \item {\bf Full tarballs}
    \begin{itemize}
    \item Contain the complete kernel sources: long to download and
      uncompress, but must be done at least once
    \item Example:\\
      \footnotesize
      \url{https://kernel.org/pub/linux/kernel/v4.x/linux-4.20.13.tar.xz}
      \normalsize
    \item Extract command:\\
      \footnotesize
      \code{tar xf linux-4.20.13.tar.xz}
      \normalsize
    \end{itemize}
  \item {\bf Incremental patches between versions}
    \begin{itemize}
    \item It assumes you already have a base version and you apply the
      correct patches in the right order. Quick to download and apply
    \item Examples:\\
      \scriptsize
      \url{https://kernel.org/pub/linux/kernel/v4.x/patch-4.20.xz}\\(4.19 to 4.20)\\
      \url{https://kernel.org/pub/linux/kernel/v4.x/patch-4.20.13.xz}\\(4.20 to 4.20.13)
    \end{itemize}
  \item All previous kernel versions are available in
    \url{https://kernel.org/pub/linux/kernel/}
  \end{itemize}
\end{frame}

\begin{frame}[fragile]
  \frametitle{Patch}
  \begin{itemize}
  \item A patch is the difference between two source trees
    \begin{itemize}
    \item Computed with the \code{diff} tool, or with more elaborate
      version control systems
    \end{itemize}
  \item They are very common in the open-source community.\\
        See \url{https://en.wikipedia.org/wiki/Diff}
  \item Excerpt from a patch:
  \end{itemize}
\footnotesize
\begin{verbatim}
diff -Nru a/Makefile b/Makefile
--- a/Makefile 2005-03-04 09:27:15 -08:00
+++ b/Makefile 2005-03-04 09:27:15 -08:00
@@ -1,7 +1,7 @@
 VERSION = 2
 PATCHLEVEL = 6
 SUBLEVEL = 11
-EXTRAVERSION =
+EXTRAVERSION = .1
 NAME=Woozy Numbat

 # *DOCUMENTATION*
\end{verbatim}
\end{frame}

\begin{frame}[fragile]
  \frametitle{Contents of a patch}
\small
  \begin{itemize}
  \item One section per modified file, starting with a header
\scriptsize
\begin{verbatim}
diff -Nru a/Makefile b/Makefile
--- a/Makefile 2005-03-04 09:27:15 -08:00
+++ b/Makefile 2005-03-04 09:27:15 -08:00
\end{verbatim}
\small
  \item One sub-section ({\em hunk}) per modified part of the file, starting with
    a header with the starting line number and the number of lines the
    change hunk applies to
\scriptsize
\begin{verbatim}
@@ -1,7 +1,7 @@
\end{verbatim}
\small
  \item Three lines of context before the change
\scriptsize
\begin{verbatim}
 VERSION = 2
 PATCHLEVEL = 6
 SUBLEVEL = 11
\end{verbatim}
\small
  \item The change itself
\scriptsize
\begin{verbatim}
-EXTRAVERSION =
+EXTRAVERSION = .1
\end{verbatim}
\small
    \item Three lines of context after the change
\scriptsize
\begin{verbatim}
 NAME=Woozy Numbat

 # *DOCUMENTATION*
\end{verbatim}
    \end{itemize}
\end{frame}

\begin{frame}
  \frametitle{Using the patch command}
  The \code{patch} command:
  \begin{itemize}
  \item Takes the patch contents on its standard input
  \item Applies the modifications described by the patch into the
    current directory
  \end{itemize}
  \code{patch} usage examples:
  \begin{itemize}
  \item \code{patch -p<n> < diff_file}
  \item \code{cat diff_file | patch -p<n>}
  \item \code{xzcat diff_file.xz | patch -p<n>}
  \item \code{zcat diff_file.gz | patch -p<n>}
  \item Notes:
    \begin{itemize}
    \item \code{n}: number of directory levels to skip in the file paths
    \item You can reverse apply a patch with the \code{-R} option
    \item You can test a patch with \code{--dry-run} option
    \end{itemize}
  \end{itemize}
\end{frame}

\begin{frame}[fragile]
  \frametitle{Applying a Linux patch}
  \begin{itemize}
  \item Two types of Linux patches:
        \begin{itemize}
	\item Either to be applied to the previous stable version\\
	      (from \code{x.<y-1>} to \code{x.y})
	\item Or implementing fixes to the current stable version\\
	      (from \code{x.y} to \code{x.y.z})
	\end{itemize}
  \item Can be downloaded in \code{gzip} or \code{xz} (much
    smaller) compressed files.
  \item Always produced for \code{n=1}\\
    (that's what everybody does... do it too!)
  \item Need to run the \code{patch} command inside the {\bf toplevel}
    kernel source directory
  \item Linux patch command line example:\\
\begin{verbatim}
cd linux-4.19
xzcat ../patch-4.20.xz | patch -p1
xzcat ../patch-4.20.13.xz | patch -p1
cd ..; mv linux-4.19 linux-4.20.13
\end{verbatim}
  \end{itemize}
\end{frame}
