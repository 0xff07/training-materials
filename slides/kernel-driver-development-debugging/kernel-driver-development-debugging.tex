\subsection{Debugging and tracing}

\begin{frame}
  \frametitle{Debugging Using Messages}
  \begin{itemize}
  \item Three APIs are available
    \begin{itemize}
    \item The old \code{printk()}, no longer recommended for new debugging
      messages
    \item The \code{pr_*()} family of functions : \code{pr_emerg()},
      \code{pr_alert()}, \code{pr_crit()}, \code{pr_err()},
      \code{pr_warning()}, \code{pr_notice()}, \code{pr_info()},
      \code{pr_cont()} and the special \code{pr_debug()}
      \begin{itemize}
      \item They take a classic format string with arguments
      \item defined in \code{include/linux/printk.h}
      \end{itemize}
    \item The \code{dev_*()} family of functions : \code{dev_emerg()},
      \code{dev_alert()}, \code{dev_crit()}, \code{dev_err()},
      \code{dev_warning()}, \code{dev_notice()}, \code{dev_info()} and
      the special \code{dev_dbg()}
      \begin{itemize}
      \item They take a pointer to \code{struct device} as first
        argument (covered later), and then a format string with
        arguments
      \item defined in \code{include/linux/device.h}
      \item To be used in drivers integrated with the Linux device
        model
      \end{itemize}
    \end{itemize}
  \end{itemize}
\end{frame}

\begin{frame}
  \frametitle{pr\_debug() and dev\_dbg()}
  \begin{itemize}
  \item When the kernel is compiled with \code{CONFIG_DEBUG}, all
    those messages are compiled and printed at the debug level with
    \code{printk}
  \item When the kernel is compiled with \code{CONFIG_DYNAMIC_DEBUG},
    then those messages can dynamically be enabled on a per-file,
    per-module or per-message basis
    \begin{itemize}
    \item See \code{Documentation/dynamic-debug-howto.txt} for details
    \item Very powerful feature to only get the debug messages you're
      interested in.
    \end{itemize}
  \item When neither \code{CONFIG_DEBUG} nor
    \code{CONFIG_DYNAMIC_DEBUG} are enabled, those messages are not
    compiled in.
  \end{itemize}
\end{frame}

\begin{frame}
  \frametitle{Configuring The Priority}
  \begin{itemize}
  \item Each message is associated to a priority, ranging from 0 for
    emergency to 7 for debug.
  \item All the messages, regardless of their priority, are stored in
    the kernel log ring buffer
    \begin{itemize}
    \item Typically accessed using the \code{dmesg} command
    \end{itemize}
  \item Some of the messages may appear on the console, depending on
    their priority and the configuration of
    \begin{itemize}
    \item The \code{loglevel} kernel parameter, which defines the
      priority above which messages are displayed on the console. See
      \code{Documentation/kernel-parameters.txt} for details.
    \item The value of \code{/proc/sys/kernel/printk}, which allows to
      change at runtime the priority above which messages are
      displayed on the console. See
      \code{Documentation/sysctl/kernel.txt} for details.
    \end{itemize}
  \end{itemize}
\end{frame}

\begin{frame}
  \frametitle{DebugFS}
  \begin{itemize}
  \item A virtual filesystem to export debugging information to
    user-space.
    \begin{itemize}
    \item Kernel configuration: \code{DEBUG_FS}
      \begin{itemize}
      \item \code{Kernel hacking -> Debug Filesystem}
      \end{itemize}
    \item The debugging interface disappears when Debugfs is
      configured out.
    \item You can mount it as follows:
      \begin{itemize}
      \item \code{sudo mount -t debugfs none /sys/kernel/debug}
      \end{itemize}
    \item First described on \url{http://lwn.net/Articles/115405/}
    \item API documented in the Linux Kernel Filesystem API:
      \begin{itemize}
      \item \url{http://free-electrons.com/kerneldoc/latest/DocBook/filesystems/index.html}
      \end{itemize}
    \end{itemize}
  \end{itemize}
\end{frame}

\begin{frame}[fragile]
  \frametitle{DebugFS API}
  \begin{itemize}
  \item Create a sub-directory for your driver:
    \begin{itemize}
    \item \mint{c}+struct dentry *debugfs_create_dir(const char *name,+
      \mint{c}+struct dentry *parent);+
      \begin{itemize}
      \item \code{u} for decimal representation
      \item \code{x} for hexadecimal representation
      \end{itemize}
    \end{itemize}
  \item Expose an integer as a file in debugfs:
    \begin{itemize}
    \item \mint{c}+struct dentry *debugfs_create_{u,x}{8,16,32}+
      \mint{c}+(const char *name, mode_t mode, struct dentry *parent,+
      \mint{c}+u8 *value);+
    \end{itemize}
  \item Expose a binary blob as a file in debugfs:
    \begin{itemize}
    \item \mint{c}+struct dentry *debugfs_create_blob(const char *name,+
      \mint{c}+mode_t mode, struct dentry *parent,+
      \mint{c}+struct debugfs_blob_wrapper *blob);+
    \end{itemize}
  \item Also possible to support writable debugfs file or customize
    the output using the more generic \code{debugfs_create_file()}
    function.
  \end{itemize}
\end{frame}

\begin{frame}
  \frametitle{Deprecated Debugging Mechanisms}
  \begin{itemize}
  \item Some additional debugging mechanisms, whose usage is now
    considered deprecated
    \begin{itemize}
    \item Adding special \code{ioctl()} commands for debugging
      purposes. debugfs is preferred.
    \item Adding special entries in the proc filesystem. debugfs is
      preferred.
    \item Adding special entries in the sysfs filesystem. debugfs is
      preferred.
    \item Using \code{printk()}. The \code{pr_*()} and \code{dev_*()}
      functions are preferred.
    \end{itemize}
  \end{itemize}
\end{frame}

\begin{frame}
  \frametitle{Using Magic SysRq}
  \begin{itemize}
  \item Allows to run multiple debug / rescue commands even when the
    kernel seems to be in deep trouble
    \begin{itemize}
    \item On PC: \code{[Alt]} + \code{[SysRq]} + \code{<character>}
    \item On embedded: break character on the serial line +
      \code{<character>}
    \end{itemize}
  \item Example commands:
    \begin{itemize}
    \item \code{n}: makes RT processes nice-able.
    \item \code{t}: shows the kernel stack of all sleeping processes
    \item \code{w}: shows the kernel stack of all running processes
    \item \code{b}: reboot the system
    \item You can even register your own!
    \end{itemize}
  \item Detailed in \code{Documentation/sysrq.txt}
  \end{itemize}
\end{frame}

\begin{frame}
  \frametitle{kgdb - A Kernel Debugger}
  \begin{itemize}
  \item The execution of the kernel is fully controlled by \code{gdb}
    from another machine, connected through a serial line.
  \item Can do almost everything, including inserting breakpoints in
    interrupt handlers.
  \item Feature included in standard Linux since 2.6.26 (x86 and
    sparc). arm, mips and ppc support merged in 2.6.27.
  \end{itemize}
\end{frame}

\begin{frame}
  \frametitle{Using kgdb 1/2}
  \begin{itemize}
  \item Details available in the kernel documentation:
    \url{http://free-electrons.com/kerneldoc/latest/DocBook/kgdb/}
  \item Recommended to turn on \code{CONFIG_FRAME_POINTER} to aid in
    producing more reliable stack backtraces in gdb.
  \item You must include a kgdb I/O driver. One of them is kgdb over
    serial console (\code{kgdboc}: kgdb over console, enabled by
    \code{CONFIG_KGDB_SERIAL_CONSOLE})
  \item Configure \code{kgdboc} at boot time by passing to the kernel:
    \begin{itemize}
    \item \code{kgdboc=<tty-device>,<bauds>}.
    \item For example: \code{kgdboc=ttyS0,115200}
    \end{itemize}
  \end{itemize}
\end{frame}

\begin{frame}
  \frametitle{Using kgdb 2/2}
  \begin{itemize}
  \item Then also pass \code{kgdbwait} to the kernel: it makes kgdb
    wait for a debugger connection.
  \item Boot your kernel, and when the console is initialized,
    interrupt the kernel with \code{Alt} + \code{SyrRq} + \code{g}.
  \item On your workstation, start gdb as follows:
    \begin{itemize}
    \item \code{gdb ./vmlinux}
    \item \code{(gdb) set remotebaud 115200}
    \item \code{(gdb) target remote /dev/ttyS0}
    \end{itemize}
  \item Once connected, you can debug a kernel the way you would debug
    an application program.
  \end{itemize}
\end{frame}

\begin{frame}
  \frametitle{Debugging with a JTAG Interface}
  \begin{itemize}
  \item Two types of JTAG dongles
    \begin{itemize}
    \item Those offering a gdb compatible interface, over a serial
      port or an Ethernet connexion. Gdb can directly connect to them.
    \item Those not offering a gdb compatible interface are generally
      supported by OpenOCD (Open On Chip Debugger):
      \url{http://openocd.sourceforge.net/}
      \begin{itemize}
      \item OpenOCD is the bridge between the gdb debugging language
        and the JTAG-dongle specific language
      \item See the very complete documentation:
        \url{http://openocd.sourceforge.net/documentation/online-docs/}
      \item For each board, you'll need an OpenOCD configuration file
        (ask your supplier)
      \end{itemize}
    \end{itemize}
  \item See very useful details on using Eclipse / gcc / gdb / OpenOCD
    on Windows:
    \begin{itemize}
    \item \url{http://www2.amontec.com/sdk4arm/ext/jlynch-tutorial-20061124.pdf}
    \item \url{http://www.yagarto.de/howto/yagarto2/}
    \end{itemize}
  \end{itemize}
\end{frame}

\begin{frame}
  \frametitle{More Kernel Debugging Tips}
  \begin{itemize}
  \item Enable \code{CONFIG_KALLSYMS_ALL}
    \begin{itemize}
    \item \code{General Setup -> Configure standard kernel features}
    \item To get oops messages with symbol names instead of raw addresses
    \item This obsoletes the \code{ksymoops} tool
    \end{itemize}
  \item If your kernel doesn't boot yet or hangs without any message,
    you can activate the low-level debugging option
    (\code{Kernel Hacking} section, only available on \code{arm} and
    \code{unicore32}): \code{CONFIG_DEBUG_LL=y}
  \item Techniques to locate the C instruction which caused an oops
    \begin{itemize}
    \item \url{http://kerneltrap.org/node/3648}
    \end{itemize}
  \item More about kernel debugging in the free Linux Device Drivers book
    \begin{itemize}
    \item \url{http://lwn.net/images/pdf/LDD3/ch04.pdf}
    \end{itemize}
  \end{itemize}
\end{frame}

\begin{frame}
  \frametitle{Kernel Crash Analysis with kexec/kdump}
  \begin{columns}
    \column{0.6\textwidth}
    \begin{itemize}
    \item \code{kexec} system call: makes it possible to call a new
      kernel, without rebooting and going through the BIOS / firmware.
    \item Idea: after a kernel panic, make the kernel automatically
      execute a new, clean kernel from a reserved location in RAM, to
      perform post-mortem analysis of the memory of the crashed
      kernel.
    \item See \code{Documentation/kdump/kdump.txt} in the kernel
      sources for details.
    \end{itemize}
    \column{0.4\textwidth}
    \includegraphics[width=\textwidth]{slides/kernel-driver-development-debugging/kexec.pdf}
  \end{columns}
\end{frame}

\begin{frame}
  \frametitle{Tracing with SystemTap}
  \begin{itemize}
  \item \url{http://sourceware.org/systemtap/}
    \begin{itemize}
    \item Infrastructure to add instrumentation to a running kernel:
      trace functions, read and write variables, follow pointers,
      gather statistics...
    \item Eliminates the need to modify the kernel sources to add
      one's own instrumentation to investigated a functional or
      performance problem.
    \item Uses a simple scripting language.
    \item Several example scripts and probe points are available.
    \item Based on the \code{Kprobes} instrumentation infrastructure.
    \item See \code{Documentation/kprobes.txt} in kernel sources.
    \item Now supported on most popular CPUs.
    \end{itemize}
  \end{itemize}
\end{frame}



\begin{frame}[fragile]
  \frametitle{SystemTap Script Example 1/3}
\begin{verbatim}
#! /usr/bin/env stap
# Using statistics and maps to examine kernel memory
# allocations

global kmalloc

probe kernel.function("__kmalloc") {
    kmalloc[execname()] <<< $size
}

# Exit after 10 seconds
probe timer.ms(10000) {
    exit()
}
\end{verbatim}
\end{frame}

\begin{frame}[fragile]
  \frametitle{SystemTap Script Example 2/3}
{\footnotesize
\begin{verbatim}
probe end {
    foreach ([name] in kmalloc) {
        printf("Allocations for %s\n", name)
        printf("Count:     %d allocations\n", @count(kmalloc[name]))
        printf("Sum:       %d Kbytes\n", @sum(kmalloc[name])/1024)
        printf("Average:   %d bytes\n", @avg(kmalloc[name]))
        printf("Min:       %d bytes\n", @min(kmalloc[name]))
        printf("Max:       %d bytes\n", @max(kmalloc[name]))
        print("\nAllocations by size in bytes\n")
        print(@hist_log(kmalloc[name]))
        printf("-------------------------------------------\n\n")
    }
}
\end{verbatim}
}
\end{frame}

\begin{frame}[fragile]
  \frametitle{SystemTap Script Example 3/3}
{\small
\begin{verbatim}
#! /usr/bin/env stap
# Logs each file read performed by each process

probe kernel.function ("vfs_read")
{
    dev_nr = $file->f_dentry->d_inode->i_sb->s_dev
    inode_nr = $file->f_dentry->d_inode->i_ino
    printf ("%s(%d) %s 0x%x/%d\n",
         execname(), pid(), probefunc(), dev_nr, inode_nr)
}
\end{verbatim}
}
Nice tutorial on \url{http://sources.redhat.com/systemtap/tutorial.pdf}
\end{frame}

\begin{frame}
  \frametitle{Kernel Markers}
  \begin{itemize}
  \item Capability to add static markers to kernel code.
  \item Almost no impact on performance, until the marker is
    dynamically enabled, by inserting a probe kernel module.
  \item Useful to insert trace points that won't be impacted by
    changes in the Linux kernel sources.
  \item See marker and probe example in \code{samples/markers} in the
    kernel sources.
  \item See \url{http://en.wikipedia.org/wiki/Kernel_marker}
  \end{itemize}
\end{frame}

\begin{frame}
  \frametitle{LTTng}
  \begin{itemize}
  \item \url{http://lttng.org}
    \begin{itemize}
    \item The successor of the Linux Trace Toolkit (LTT)
    \item Toolkit allowing to collect and analyze tracing information
      from the kernel, based on kernel markers and kernel tracepoints.
    \item So far, based on kernel patches, but doing its best to use
      in-tree solutions, and to be merged in the future.
    \item Very precise timestamps, very little overhead.
    \item Useful documentation on \url{http://lttng.org/documentation}
    \end{itemize}
  \end{itemize}
\end{frame}

\begin{frame}
  \frametitle{LTTV}
  \begin{itemize}
  \item Viewer for LTTng traces
    \begin{itemize}
    \item Support for huge traces (tested with 15 GB ones)
    \item Can combine multiple tracefiles in a single view.
    \item Graphical or text interface
    \end{itemize}
    \item See \url{http://lttng.org/files/lttv-doc/user_guide/}
  \end{itemize}
\end{frame}
