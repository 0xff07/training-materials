\section{Kernel Debugging}

\begin{frame}
  \frametitle{Debugging Using Messages}
  \begin{itemize}
  \item Three APIs are available
    \begin{itemize}
    \item The old \code{printk()}, no longer recommended for new debugging
      messages
    \item The \code{pr_*()} family of functions: \code{pr_emerg()},
      \code{pr_alert()}, \code{pr_crit()}, \code{pr_err()},
      \code{pr_warning()}, \code{pr_notice()}, \code{pr_info()},
      \code{pr_cont()} \\
      and the special \code{pr_debug()} (see next page)
      \begin{itemize}
      \item They take a classic format string with arguments
      \item Defined in \code{include/linux/printk.h}
      \end{itemize}
    \item The \code{dev_*()} family of functions: \code{dev_emerg()},
      \code{dev_alert()}, \code{dev_crit()}, \code{dev_err()},
      \code{dev_warning()}, \code{dev_notice()}, \code{dev_info()} \\
      and the special \code{dev_dbg()} (see next page)
      \begin{itemize}
      \item They take a pointer to \code{struct device} as first
        argument (covered later), and then a format string with
        arguments
      \item Defined in \code{include/linux/device.h}
      \item To be used in drivers integrated with the Linux device
        model
      \end{itemize}
    \end{itemize}
  \end{itemize}
\end{frame}

\begin{frame}
  \frametitle{pr\_debug() and dev\_dbg()}
  \begin{itemize}
  \item When the driver is compiled with \code{DEBUG} defined, all
    those messages are compiled and printed at the debug level.
    \code{DEBUG} can be defined by {\tt \#define DEBUG} at the
    beginning of the driver, or using
    \code{ccflags-$(CONFIG_DRIVER) += -DDEBUG} in the \code{Makefile}
  \item When the kernel is compiled with \code{CONFIG_DYNAMIC_DEBUG},
    then those messages can dynamically be enabled on a per-file,
    per-module or per-message basis
    \begin{itemize}
    \item See \kerneldoc{dynamic-debug-howto.txt} for details
    \item Very powerful feature to only get the debug messages you're
      interested in.
    \end{itemize}
  \item When \code{DEBUG} is not defined and
    \code{CONFIG_DYNAMIC_DEBUG} is not enabled, those messages are not
    compiled in.
  \end{itemize}
\end{frame}

\begin{frame}
  \frametitle{Configuring The Priority}
  \begin{itemize}
  \item Each message is associated to a priority, ranging from \code{0} for
    emergency to \code{7} for debug.
  \item All the messages, regardless of their priority, are stored in
    the kernel log ring buffer
    \begin{itemize}
    \item Typically accessed using the \code{dmesg} command
    \end{itemize}
  \item Some of the messages may appear on the console, depending on
    their priority and the configuration of
    \begin{itemize}
    \item The \code{loglevel} kernel parameter, which defines the
      priority above which messages are displayed on the console. See
      \kerneldoc{kernel-parameters.txt} for details.
    \item The value of \code{/proc/sys/kernel/printk}, which allows to
      change at runtime the priority above which messages are
      displayed on the console. See
      \kerneldoc{sysctl/kernel.txt} for details.
    \end{itemize}
  \end{itemize}
\end{frame}

\begin{frame}
  \frametitle{DebugFS}
  \begin{itemize}
  \item A virtual filesystem to export debugging information to
    user-space.
    \begin{itemize}
    \item Kernel configuration: \code{DEBUG_FS}
      \begin{itemize}
      \item \code{Kernel hacking -> Debug Filesystem}
      \end{itemize}
    \item The debugging interface disappears when Debugfs is
      configured out.
    \item You can mount it as follows:
      \begin{itemize}
      \item \code{sudo mount -t debugfs none /sys/kernel/debug}
      \end{itemize}
    \item First described on \url{http://lwn.net/Articles/115405/}
    \item API documented in the Linux Kernel Filesystem API:
      \begin{itemize}
      \item \kerneldoc{DocBook/filesystems/}
      \end{itemize}
    \end{itemize}
  \end{itemize}
\end{frame}

\begin{frame}[fragile]
  \frametitle{DebugFS API}
  \begin{itemize}
  \item Create a sub-directory for your driver:
    \begin{itemize}
    \item \mint{c}+struct dentry *debugfs_create_dir(const char *name,+
      \mint{c}+struct dentry *parent);+
    \end{itemize}
  \item Expose an integer as a file in DebugFS:
    \begin{itemize}
    \item \mint{c}+struct dentry *debugfs_create_{u,x}{8,16,32}+
      \mint{c}+(const char *name, mode_t mode, struct dentry *parent,+
      \mint{c}+u8 *value);+
      \begin{itemize}
      \item \code{u} for decimal representation
      \item \code{x} for hexadecimal representation
      \end{itemize}
    \end{itemize}
  \item Expose a binary blob as a file in DebugFS:
    \begin{itemize}
    \item \mint{c}+struct dentry *debugfs_create_blob(const char *name,+
      \mint{c}+mode_t mode, struct dentry *parent,+
      \mint{c}+struct debugfs_blob_wrapper *blob);+
    \end{itemize}
  \item Also possible to support writable DebugFS files or customize
    the output using the more generic \code{debugfs_create_file()}
    function.
  \end{itemize}
\end{frame}

\begin{frame}
  \frametitle{Deprecated Debugging Mechanisms}
  \begin{itemize}
  \item Some additional debugging mechanisms, whose usage is now
    considered deprecated
    \begin{itemize}
    \item Adding special \code{ioctl()} commands for debugging
      purposes. DebugFS is preferred.
    \item Adding special entries in the \code{proc} filesystem. DebugFS is
      preferred.
    \item Adding special entries in the \code{sysfs} filesystem. DebugFS is
      preferred.
    \item Using \code{printk()}. The \code{pr_*()} and \code{dev_*()}
      functions are preferred.
    \end{itemize}
  \end{itemize}
\end{frame}

\begin{frame}
  \frametitle{Using Magic SysRq}
  \begin{itemize}
  \item Allows to run multiple debug / rescue commands even when the
    kernel seems to be in deep trouble
    \begin{itemize}
    \item On PC: \code{[Alt]} + \code{[SysRq]} + \code{<character>}
    \item On embedded: break character on the serial line +
      \code{<character>}
    \end{itemize}
  \item Example commands:
    \begin{itemize}
    \item \code{n}: makes RT processes nice-able.
    \item \code{w}: shows the kernel stack of all sleeping processes
    \item \code{t}: shows the kernel stack of all running processes
    \item \code{b}: reboot the system
    \item You can even register your own!
    \end{itemize}
  \item Detailed in \kerneldoc{sysrq.txt}
  \end{itemize}
\end{frame}

\begin{frame}
  \frametitle{kgdb - A Kernel Debugger}
  \begin{itemize}
  \item The execution of the kernel is fully controlled by \code{gdb}
    from another machine, connected through a serial line.
  \item Can do almost everything, including inserting breakpoints in
    interrupt handlers.
  \item Feature supported for the most popular CPU architectures
  \end{itemize}
\end{frame}

\begin{frame}
  \frametitle{Using kgdb 1/2}
  \begin{itemize}
  \item Details available in the kernel documentation:
    \kerneldoc{DocBook/kgdb/}
  \item Recommended to turn on \code{CONFIG_FRAME_POINTER} to aid in
    producing more reliable stack backtraces in \code{gdb}.
  \item You must include a kgdb I/O driver. One of them is \code{kgdb} over
    serial console (\code{kgdboc}: \code{kgdb} over console, enabled by
    \code{CONFIG_KGDB_SERIAL_CONSOLE})
  \item Configure \code{kgdboc} at boot time by passing to the kernel:
    \begin{itemize}
    \item \code{kgdboc=<tty-device>,<bauds>}.
    \item For example: \code{kgdboc=ttyS0,115200}
    \end{itemize}
  \end{itemize}
\end{frame}

\begin{frame}
  \frametitle{Using kgdb 2/2}
  \begin{itemize}
  \item Then also pass \code{kgdbwait} to the kernel: it makes
    \code{kgdb} wait for a debugger connection.
  \item Boot your kernel, and when the console is initialized,
    interrupt the kernel with \code{Alt} + \code{SyrRq} + \code{g}.
  \item On your workstation, start \code{gdb} as follows:
    \begin{itemize}
    \item \code{gdb ./vmlinux}
    \item \code{(gdb) set remotebaud 115200}
    \item \code{(gdb) target remote /dev/ttyS0}
    \end{itemize}
  \item Once connected, you can debug a kernel the way you would debug
    an application program.
  \end{itemize}
\end{frame}

\begin{frame}
  \frametitle{Debugging with a JTAG Interface}
  \begin{itemize}
  \item Two types of JTAG dongles
    \begin{itemize}
    \item Those offering a \code{gdb} compatible interface, over a
      serial port or an Ethernet connexion. \code{gdb} can directly
      connect to them.
    \item Those not offering a gdb compatible interface are generally
      supported by OpenOCD (Open On Chip Debugger):
      \url{http://openocd.sourceforge.net/}
      \begin{itemize}
      \item OpenOCD is the bridge between the gdb debugging language
        and the JTAG interface of the target CPU.
      \item See the very complete documentation:
        \url{http://openocd.sourceforge.net/documentation/online-docs/}
      \item For each board, you'll need an OpenOCD configuration file
        (ask your supplier)
      \end{itemize}
    \end{itemize}
  \end{itemize}
   \begin{center}
     \includegraphics[width=\textwidth]{slides/kernel-driver-development-debugging/jtag.pdf}
   \end{center}
\end{frame}

\begin{frame}
  \frametitle{More Kernel Debugging Tips}
  \begin{itemize}
  \item Enable \code{CONFIG_KALLSYMS_ALL}
    \begin{itemize}
    \item \code{General Setup -> Configure standard kernel features}
    \item To get oops messages with symbol names instead of raw addresses
    \item This obsoletes the \code{ksymoops} tool
    \end{itemize}
  \item If your kernel doesn't boot yet or hangs without any message,
    you can activate early debugging options (\code{CONFIG_DEBUG_LL} and
    \code{CONFIG_EARLYPRINTK}), and add \code{earlyprintk} to the kernel
    command line.
  \item Techniques to locate the C instruction which caused an oops:
    \url{http://j.mp/18oMRHx}
  \end{itemize}
\end{frame}

\begin{frame}
  \frametitle{Kernel Crash Analysis with kexec/kdump}
  \begin{columns}
    \column{0.6\textwidth}
    \begin{itemize}
    \item \code{kexec} system call: makes it possible to call a new
      kernel, without rebooting and going through the BIOS / firmware.
    \item Idea: after a kernel panic, make the kernel automatically
      execute a new, clean kernel from a reserved location in RAM, to
      perform post-mortem analysis of the memory of the crashed
      kernel.
    \item See \kerneldoc{kdump/kdump.txt} in the kernel
      sources for details.
    \end{itemize}
    \column{0.4\textwidth}
    \includegraphics[width=\textwidth]{slides/kernel-driver-development-debugging/kexec.pdf}
  \end{columns}
\end{frame}
