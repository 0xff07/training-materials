\subsection{DMA}

\begin{frame}
  \frametitle{DMA Integration}
  \begin{center}
    \includegraphics[width=\textwidth]{slides/kernel-driver-development-dma/dma-integration.pdf}
  \end{center}
\end{frame}

\begin{frame}
  \frametitle{Constraints with a DMA}
  \begin{itemize}
  \item A DMA deals with physical addresses, so:
    \begin{itemize}
    \item Programming a DMA requires retrieving a physical address at
      some point (virtual addresses are usually used)
    \item The memory accessed by the DMA shall be physically
      contiguous
    \end{itemize}
  \item The CPU can access memory through a data cache
    \begin{itemize}
    \item Using the cache can be more efficient (faster accesses to
      the cache than the bus)
    \item But the DMA does not access to the CPU cache, so one need to
      take care of cache coherency (cache content vs memory content)
    \item Either flush or invalidate the cache lines corresponding to
      the buffer accessed by DMA and processor at strategic times
    \end{itemize}
  \end{itemize}
\end{frame}

\begin{frame}
  \frametitle{DMA Memory Constraints}
  \begin{itemize}
  \item Need to use contiguous memory in physical space.
  \item Can use any memory allocated by \code{kmalloc} (up to 128 KB)
    or \code{__get_free_pages} (up to 8MB).
  \item Can use block I/O and networking buffers, designed to support
    DMA.
  \item Can not use vmalloc memory (would have to setup DMA on each
    individual physical page).
  \end{itemize}
\end{frame}

\begin{frame}[fragile]
  \frametitle{Reserving Memory for DMA}
  \begin{itemize}
  \item To make sure you've got enough RAM for big DMA transfers...
  \item Example assuming you have 32 MB of RAM, and need 2 MB for DMA:
    \begin{itemize}
    \item Boot your kernel with \code{mem=30}
    \item The kernel will just use the first 30 MB of RAM.
    \item Driver code can now reclaim the 2 MB left:
      \begin{minted}{c}
dmabuf = ioremap (
    0x1e00000, /* Start: 30 MB */
    0x200000   /* Size: 2 MB */
);
      \end{minted}
    \item You can also use \code{mem=} to reserve specific RAM areas
      for specific devices
      (DSP, video device...).
    \item Panda board example:
      \begin{itemize}
      \item \code{mem=456M@0x80000000}, \code{mem=512M@0xA0000000}
      \end{itemize}
    \end{itemize}
  \end{itemize}
\end{frame}

\begin{frame}
  \frametitle{Memory Synchronization Issues}
  \begin{itemize}
  \item Memory caching could interfere with DMA
    \begin{itemize}
    \item Before DMA to device
      \begin{itemize}
      \item Need to make sure that all writes to DMA buffer are
        committed.
      \end{itemize}
    \item After DMA from device
      \begin{itemize}
      \item Before drivers read from DMA buffer, need to make sure
        that memory caches are flushed.
      \end{itemize}
    \item Bidirectional DMA
      \begin{itemize}
      \item Need to flush caches before and after the DMA transfer.
      \end{itemize}
    \end{itemize}
  \end{itemize}
\end{frame}

\begin{frame}
  \frametitle{Linux DMA API}
  \begin{itemize}
  \item The kernel DMA utilities can take care of:
    \begin{itemize}
    \item Either allocating a buffer in a cache coherent area,
    \item Or making sure caches are flushed when required,
    \item Managing the DMA mappings and IOMMU (if any).
    \item See \code{Documentation/DMA-API.txt} for details about the
      Linux DMA generic API.
    \item Most subsystems (such as PCI or USB) supply their own DMA
      API, derived from the generic one. May be sufficient for most
      needs.
    \end{itemize}
  \end{itemize}
\end{frame}

\begin{frame}
  \frametitle{Coherent or Streaming DMA Mappings}
  \begin{itemize}
  \item Coherent mappings
    \begin{itemize}
    \item The kernel allocates a suitable buffer and sets the mapping
      for the driver.
    \item Can simultaneously be accessed by the CPU and device.
    \item So, has to be in a cache coherent memory area.
    \item Usually allocated for the whole time the module is loaded.
    \item Can be expensive to setup and use on some platforms.
    \end{itemize}
  \item Streaming mappings
    \begin{itemize}
    \item The kernel just sets the mapping for a buffer provided by
      the driver.
    \item Use a buffer already allocated by the driver.
    \item Mapping set up for each transfer. Keeps DMA registers free
      on the hardware.
    \item Some optimizations also available.
    \item The recommended solution.
    \end{itemize}
  \end{itemize}
\end{frame}

\begin{frame}[fragile]
  \frametitle{Allocating Coherent Mappings}
  \begin{itemize}
  \item The kernel takes care of both buffer allocation and mapping
  \end{itemize}
\begin{minted}[fontsize=\small]{c}
#include <asm/dma-mapping.h>

void *                       /* Output: buffer address */
    dma_alloc_coherent(
         struct device *dev, /* device structure */
         size_t size,        /* Needed buffer size in bytes */
         dma_addr_t *handle, /* Output: DMA bus address */
         gfp_t gfp           /* Standard GFP flags */
);

void dma_free_coherent(struct device *dev,
    size_t size, void *cpu_addr, dma_addr_t handle);
\end{minted}
\end{frame}

\begin{frame}[fragile]
  \frametitle{Setting up streaming mappings}
  \begin{itemize}
  \item Works on buffers already allocated by the driver
  \end{itemize}
\begin{minted}[fontsize=\small]{c}
#include <linux/dmapool.h>

dma_addr_t dma_map_single(
      struct device *,        /* device structure */
      void *,                 /* input: buffer to use */
      size_t,                 /* buffer size */
      enum dma_data_direction /* Either DMA_BIDIRECTIONAL,
                               * DMA_TO_DEVICE or
                               * DMA_FROM_DEVICE */
);

void dma_unmap_single(struct device *dev, dma_addr_t handdle,
    size_t size, enum dma_data_direction dir);
\end{minted}
\end{frame}

\begin{frame}
  \frametitle{DMA Streaming Mapping Notes}
  \begin{itemize}
  \item When the mapping is active: only the device should access the
    buffer (potential cache issues otherwise).
  \item The CPU can access the buffer only after unmapping! Use
    locking to prevent CPU access to the buffer.
  \item Another reason: if required, this API can create an
    intermediate bounce buffer (used if the given buffer is not usable
    for DMA).
  \item The Linux API also supports scatter / gather DMA streaming
    mappings.
  \end{itemize}
\end{frame}

\begin{frame}
  \frametitle{DMA summary}
  \begin{itemize}
  \item Most drivers can use the specific API provided by their
    subsystem: USB, PCI, SCSI... Otherwise they can use the Linux
    generic API:
  \item Coherent mappings
    \begin{itemize}
    \item DMA buffer allocated by the kernel
    \item Set up for the whole module life
    \item Can be expensive. Not recommended.
    \item Let both the CPU and device access the buffer at the same
      time.
    \item Main functions:
      \begin{itemize}
      \item \code{dma_alloc_coherent}
      \item \code{dma_free_coherent}
      \end{itemize}
    \end{itemize}
  \item Streaming mappings
    \begin{itemize}
    \item DMA buffer allocated by the driver
    \item Set up for each transfer
    \item Cheaper. Saves DMA registers.
    \item Only the device can access the buffer when the mapping is
      active.
    \item Main functions:
      \begin{itemize}
      \item \code{dma_map_single}
      \item \code{dma_unmap_single}
      \end{itemize}
    \end{itemize}
  \end{itemize}
\end{frame}
