\subsection{DMA}

\begin{frame}
  \frametitle{DMA Integration}
  \begin{center}
    \includegraphics[width=\textwidth]{slides/kernel-driver-development-dma/dma-integration.pdf}
  \end{center}
\end{frame}

\begin{frame}
  \frametitle{Peripheral DMA}
  \begin{center}
    \includegraphics[height=0.7\textheight]{slides/kernel-driver-development-dma/peripheral-dma.pdf}
  \end{center}
\end{frame}

\begin{frame}
  \frametitle{DMA Controllers}
  \begin{center}
    \includegraphics[width=\textwidth]{slides/kernel-driver-development-dma/dma-controller.pdf}
  \end{center}
\end{frame}

\begin{frame}
  \frametitle{DMA descriptors}
  \begin{center}
    \includegraphics[width=\textwidth]{slides/kernel-driver-development-dma/dma-descriptors.pdf}
  \end{center}
\end{frame}

\subsection{DMA Usage}

\begin{frame}
  \frametitle{Constraints with a DMA}
  \begin{itemize}
  \item A DMA deals with physical addresses, so:
    \begin{itemize}
    \item Programming a DMA requires retrieving a physical address at
      some point (virtual addresses are usually used)
    \item The memory accessed by the DMA shall be physically
      contiguous
    \end{itemize}
  \item The CPU can access memory through a data cache
    \begin{itemize}
    \item Using the cache can be more efficient (faster accesses to
      the cache than the bus)
    \item But the DMA does not access to the CPU cache, so one need to
      take care of cache coherency (cache content vs memory content)
    \item Either flush or invalidate the cache lines corresponding to
      the buffer accessed by DMA and processor at strategic times
    \end{itemize}
  \end{itemize}
\end{frame}

\begin{frame}
  \frametitle{DMA Memory Constraints}
  \begin{itemize}
  \item Need to use contiguous memory in physical space.
  \item Can use any memory allocated by \kfunc{kmalloc} (up to 128 KB)
    or \kfunc{__get_free_pages} (up to 8MB).
  \item Can use block I/O and networking buffers, designed to support
    DMA.
  \item Can not use \kfunc{vmalloc} memory (would have to setup DMA on each
    individual physical page).
  \end{itemize}
\end{frame}

\begin{frame}
  \frametitle{Memory Synchronization Issues}
  Memory caching could interfere with DMA
  \begin{itemize}
  \item Before DMA to device
    \begin{itemize}
    \item Need to make sure that all writes to DMA buffer are
      committed.
    \end{itemize}
  \item After DMA from device
    \begin{itemize}
    \item Before drivers read from DMA buffer, need to make sure
      that memory caches are flushed.
    \end{itemize}
  \item Bidirectional DMA
    \begin{itemize}
    \item Need to flush caches before and after the DMA transfer.
    \end{itemize}
  \end{itemize}
\end{frame}

\begin{frame}
  \frametitle{Linux DMA API}
  The kernel DMA utilities can take care of:
  \begin{itemize}
  \item Either allocating a buffer in a cache coherent area,
  \item Or making sure caches are flushed when required,
  \item Managing the DMA mappings and IOMMU (if any).
  \item See \kerneldoctext{DMA-API.txt} for details about the
    Linux DMA generic API.
  \item Most subsystems (such as PCI or USB) supply their own DMA
    API, derived from the generic one. May be sufficient for most
    needs.
  \end{itemize}
\end{frame}

\begin{frame}
  \frametitle{Coherent or Streaming DMA Mappings}
  \begin{itemize}
  \item Coherent mappings
    \begin{itemize}
    \item The kernel allocates a suitable buffer and sets the mapping
      for the driver.
    \item Can simultaneously be accessed by the CPU and device.
    \item So, has to be in a cache coherent memory area.
    \item Usually allocated for the whole time the module is loaded.
    \item Can be expensive to setup and use on some platforms.
    \end{itemize}
  \item Streaming mappings
    \begin{itemize}
    \item The kernel just sets the mapping for a buffer provided by
      the driver.
    \item Use a buffer already allocated by the driver.
    \item Mapping set up for each transfer. Keeps DMA registers free
      on the hardware.
    \item The recommended solution.
    \end{itemize}
  \end{itemize}
\end{frame}

\begin{frame}[fragile]
  \frametitle{Allocating Coherent Mappings}
  The kernel takes care of both buffer allocation and mapping
\begin{minted}[fontsize=\small]{c}
#include <asm/dma-mapping.h>

void *                       /* Output: buffer address */
    dma_alloc_coherent(
         struct device *dev, /* device structure */
         size_t size,        /* Needed buffer size in bytes */
         dma_addr_t *handle, /* Output: DMA bus address */
         gfp_t gfp           /* Standard GFP flags */
);

void dma_free_coherent(struct device *dev,
    size_t size, void *cpu_addr, dma_addr_t handle);
\end{minted}
\end{frame}

\begin{frame}[fragile]
  \frametitle{Setting up streaming mappings}
  Works on buffers already allocated by the driver
\begin{minted}[fontsize=\small]{c}
#include <linux/dmapool.h>

dma_addr_t dma_map_single(
      struct device *,        /* device structure */
      void *,                 /* input: buffer to use */
      size_t,                 /* buffer size */
      enum dma_data_direction /* Either DMA_BIDIRECTIONAL,
                               * DMA_TO_DEVICE or
                               * DMA_FROM_DEVICE */
);

void dma_unmap_single(struct device *dev, dma_addr_t handdle,
    size_t size, enum dma_data_direction dir);
\end{minted}
\end{frame}

\begin{frame}
  \frametitle{DMA Streaming Mapping Notes}
  \begin{itemize}
  \item When the mapping is active: only the device should access the
    buffer (potential cache issues otherwise).
  \item The CPU can access the buffer only after unmapping!
  \item Another reason: if required, this API can create an
    intermediate bounce buffer (used if the given buffer is not usable
    for DMA).
  \item The Linux API also supports scatter / gather DMA streaming
    mappings.
  \end{itemize}
\end{frame}

\subsection{DMA transfers}

\begin{frame}
  \frametitle{Starting DMA transfers}
  \begin{itemize}
  \item If the device you're writing a driver for is doing peripheral
    DMA, no external API is involved.
  \item If it relies on an external DMA controller, you'll need to
    \begin{itemize}
    \item Ask the hardware to use DMA, so that it will drive its
      request line
    \item Use Linux DMAEngine framework, especially its slave API
    \end{itemize}
  \end{itemize}
\end{frame}

\begin{frame}
  \frametitle{DMAEngine Slave API}
  \begin{itemize}
  \item In order to start a DMA transfer, you need to call the
    following functions from your driver
    \begin{enumerate}
    \item Request a channel for exclusive use with
      \kfunc{dma_request_channel}, or one of its variants
    \item Configure it for our use case, by filling a
      \kstruct{dma_slave_config} structure with various parameters
      (source and destination adresses, accesses width, etc.) and
      passing it as an argument to \kfunc{dmaengine_slave_config}
    \item Start a new transaction with
      \kfunc{dmaengine_prep_slave_single} or
      \kfunc{dmaengine_prep_slave_sg}
    \item Put the transaction in the driver pending queue using
      \kfunc{dmaengine_submit}
    \item And finally ask the driver to process all pending
      transactions using \kfunc{dma_async_issue_pending}
    \end{enumerate}
  \item Of course, this needs to be done in addition to the DMA
    mapping seen previously
  \item Some frameworks abstract it away from you, such as {\em SPI}
    and {\em ASoC}
  \end{itemize}
\end{frame}
