\subsection{Bootloader optimizations}
\begin{frame}
\frametitle{Bootloader}
Usually, bootloaders include many features needed only for
development.
\begin{itemize}
	\item There may be different bootloaders for your board. Try
		those !
	\item Assess what features you really need. Do you need to
		upgrade from the bootloader ?
	\item Remove the boot delay! For \code{u-boot}, then have a look
		at the \code{CONFIG_ZERO_BOOTDELAY_CHECK} option, documented in
		\code{doc/README.autoboot} that will allow you to still enter
		the bootloader shell.
	\item Maybe you can skip the bootloader (AT91 example: 
                \url{http://free-electrons.com/blog/starting-linux-directly-from-at91bootstrap3/}).
\end{itemize}
\end{frame}

\begin{frame}
\frametitle{Results}
Before: 5.77s
\begin{center}
    \includegraphics[width=\textwidth]{slides/boottime-kernel/timechart-final.pdf}
\end{center}
After:
\begin{center}
    \includegraphics[width=\textwidth]{slides/boottime-bootloader/timechart-barebox.pdf}
\end{center}
Total: 4.67s.
\end{frame}

\begin{frame}
\frametitle{Compression}
Let's try LZO compression for the kernel again:
Before (gzip): 4.67s
\begin{center}
    \includegraphics[width=\textwidth]{slides/boottime-bootloader/timechart-barebox.pdf}
\end{center}
After (LZO):
\begin{center}
    \includegraphics[width=\textwidth]{slides/boottime-bootloader/timechart-barebox-lzo.pdf}
\end{center}
Total: 4.59s.
\end{frame}

\begin{frame}[fragile]
\frametitle{Features}
Now, let's limit the features of the bootloader.\\
We still keep a way to interact with it when a GPIO has a given value.
For example, using the \code{gpio_direction_input} and
\code{gpio_get_value} commands in a script that would then start an
upgrade or boot a rescue kernel.
\begin{block}{}
\begin{verbatim}
gpio_get_value 42
if [ $? = 0 ]; then
    kdev="/dev/nand0.krescue.bb"
fi
\end{verbatim}
\end{block}
\end{frame}

\begin{frame}
\frametitle{Results}
Before: 4.59s
\begin{center}
    \includegraphics[width=\textwidth]{slides/boottime-bootloader/timechart-barebox-lzo.pdf}
\end{center}
After:
\begin{center}
    \includegraphics[width=\textwidth]{slides/boottime-bootloader/timechart-barebox-final.pdf}
\end{center}
Total: 3.07s.
\end{frame}

\begin{frame}
\frametitle{Results}
Note: the kernel didn't actually change but we don't get a message on
the serial line exactly at the time we switch from the bootloader to
the kernel.

Warning: Sometimes, the kernel is relying on the bootloader to
initialize the hardware (pinmuxing, clocks, ...) so be careful when
removing features.
\end{frame}

\begin{frame}
\frametitle{Results - Simplifying bootloader and kernel}

Before: 3.07s
\begin{center}
    \includegraphics[width=0.8\textwidth]{slides/boottime-bootloader/timechart-barebox-final.pdf}
\end{center}
After:
\begin{center}
    \includegraphics[width=0.8\textwidth]{slides/boottime-bootloader/timechart-kernel.pdf}
\end{center}
Total: 2.57s.
\end{frame}


\begin{frame}[fragile]
\frametitle{Removing the bootloader}
You can also try to boot directly from \code{AT91bootstrap} to the
Linux kernel,thus removing the second stage. But you will lose the
main advantages of using \code{barebox}.  It is using the CPU caches
while loading and decompressing the kernel.

It is quite easy with AT91bootstrap3. You just need to configure it
with one of the \code{linux} or \code{linux_dt} configuration:
\begin{block}{}
\begin{verbatim}
make at91sama5d3xeknf_linux_dt_defconfig
make
\end{verbatim}
\end{block}
See \url{http://free-electrons.com/blog/at91bootstrap-linux/}
\end{frame}

\begin{frame}
\frametitle{Removing the bootloader}
Before: 3.07s
\begin{center}
    \includegraphics[width=\textwidth]{slides/boottime-bootloader/timechart-barebox-final.pdf}
\end{center}
Using \code{AT91bootstrap} to boot the Linux kernel:
\begin{center}
    \includegraphics[width=\textwidth]{slides/boottime-bootloader/timechart-at91.pdf}
\end{center}
Total: 3.94s.
\end{frame}

\setuplabframe
{Reduce bootloader time}
{
\begin{itemize}
\item Reduce boot time by using the Barebox bootloader
\item Optimize Barebox
\end{itemize}
}

