\subsection{Loadable Kernel Modules}

\begin{frame}[fragile]
  \frametitle{Hello Module 1/2}
\begin{minted}[fontsize=\scriptsize]{c}
/* hello.c */
#include <linux/init.h>
#include <linux/module.h>
#include <linux/kernel.h>

static int __init hello_init(void)
{
  pr_alert("Good morrow");
  pr_alert("to this fair assembly.\n");
  return 0;
}

static void __exit hello_exit(void)
{
  pr_alert("Alas, poor world, what treasure");
  pr_alert("hast thou lost!\n");
}

module_init(hello_init);
module_exit(hello_exit);
MODULE_LICENSE("GPL");
MODULE_DESCRIPTION("Greeting module");
MODULE_AUTHOR("William Shakespeare");  
\end{minted}
\end{frame}

\begin{frame}[fragile]
  \frametitle{Hello Module 2/2}
\begin{minted}[fontsize=\small]{c}
\end{minted}

\begin{itemize}
\item \code{__init}
  \begin{itemize}
  \item removed after initialization (static kernel or module.)
  \end{itemize}
\item \code{__exit}
  \begin{itemize}
  \item discarded when module compiled statically into the kernel.
  \end{itemize}
\item Example available on
  \url{http://free-electrons.com/doc/c/hello.c}
\end{itemize}
\end{frame}

\begin{frame}
  \frametitle{Hello Module Explanations}
  \begin{itemize}
  \item Headers specific to the Linux kernel: \code{linux/xxx.h}
    \begin{itemize}
    \item No access to the usual C library, we're doing kernel
      programming
    \end{itemize}
  \item An initialization function
    \begin{itemize}
    \item Called when the module is loaded, returns an error code (0
      on success, negative value on failure)
    \item Declared by the \code{module_init()} macro: the name of the
      function doesn't matter, even though \code{<modulename>_init()}
      is a convention.
    \end{itemize}
  \item A cleanup function
    \begin{itemize}
    \item Called when the module is unloaded
    \item Declared by the \code{module_exit()} macro.
    \end{itemize}
  \item Metadata information declared using \code{MODULE_LICENSE()},
    \code{MODULE_DESCRIPTION()} and \code{MODULE_AUTHOR()}
  \end{itemize}
\end{frame}

\begin{frame}
  \frametitle{Symbols Exported to Modules 1/2}
  \begin{itemize}
  \item From a kernel module, only a limited number of kernel
    functions can be called
  \item Functions and variables have to be explicitly exported by the
    kernel to be visible from a kernel module
  \item Two macros are used in the kernel to export functions and
    variables:
    \begin{itemize}
    \item \code{EXPORT_SYMBOL(symbolname)}, which exports a function
      or variable to all modules
    \item \code{EXPORT_SYMBOL_GPL(symbolname)}, which exports a
      function or variable only to GPL modules
    \end{itemize}
  \item A normal driver should not need any non-exported function.
  \end{itemize}
\end{frame}

\begin{frame}
  \frametitle{Symbols exported to modules 2/2}
  \begin{center}
    \includegraphics[height=0.85\textheight]{slides/kernel-driver-development-modules/exported-symbols.pdf}
  \end{center}
\end{frame}

\begin{frame}
  \frametitle{Module License}
  \begin{itemize}
  \item Several usages
    \begin{itemize}
    \item Used to restrict the kernel functions that the module can
      use if it isn't a GPL licensed module
      \begin{itemize}
      \item Difference between \code{EXPORT_SYMBOL()} and
        \code{EXPORT_SYMBOL_GPL()}
      \end{itemize}
    \item Used by kernel developers to identify issues coming from
      proprietary drivers, which they can't do anything about
      (“Tainted” kernel notice in kernel crashes and oopses).
    \item Useful for users to check that their system is 100\% free
      (check \code{/proc/sys/kernel/tainted})
    \end{itemize}
  \item Values
    \begin{itemize}
    \item GPL compatible (see \code{include/linux/license.h}:
      \code{GPL}, \code{GPL v2}, \code{GPL and additional rights},
      \code{Dual MIT/GPL}, \code{Dual BSD/GPL}, \code{Dual MPL/GPL}
    \item \code{Proprietary}
    \end{itemize}
  \end{itemize}
\end{frame}

\begin{frame}
  \frametitle{Compiling a Module}
  \begin{itemize}
  \item Two solutions
    \begin{itemize}
    \item \emph{Out of tree}
      \begin{itemize}
      \item When the code is outside of the kernel source tree, in a
        different directory
      \item Advantage: Might be easier to handle than modifications to
        the kernel itself
      \item Drawbacks: Not integrated to the kernel
        configuration/compilation process, needs to be built
        separately, the driver cannot be built statically
      \end{itemize}
    \item Inside the kernel tree
      \begin{itemize}
      \item Well integrated into the kernel configuration/compilation
        process
      \item Driver can be built statically if needed
      \end{itemize}
    \end{itemize}
  \end{itemize}
\end{frame}

\begin{frame}[fragile]
  \frametitle{Compiling an out-of-tree Module 1/2}
  \begin{itemize}
  \item The below Makefile should be reusable for any single-file
    out-of-tree Linux module
  \item The source file is \code{hello.c}
  \item Just run make to build the hello.ko file
  \end{itemize}
\begin{minted}{make}
ifneq ($(KERNELRELEASE),)
obj-m := hello.o
else
KDIR := /path/to/kernel/sources

all:
<tab>$(MAKE) -C $(KDIR) M=`pwd` modules
endif
\end{minted}

\begin{itemize}
\item For \code{KDIR}, you can either set
  \begin{itemize}
  \item full kernel source directory (configured and compiled)
  \item or just kernel headers directory (minimum needed)
  \end{itemize}
\end{itemize}
\end{frame}

\begin{frame}
  \frametitle{Compiling an out-of-tree Module 2/2}
  \begin{center}
    \includegraphics[height=0.4\textheight]{slides/kernel-driver-development-modules/out-of-tree.pdf}
  \end{center}
  \begin{itemize}
  \item The module Makefile is interpreted with \code{KERNELRELEASE}
    undefined, so it calls the kernel Makefile, passing the module
    directory in the \code{M} variable
  \item the kernel Makefile knows how to compile a module, and thanks
    to the \code{M} variable, knows where the Makefile for our module
    is. The module Makefile is interpreted with \code{KERNELRELEASE}
    defined, so the kernel sees the \code{obj-m} definition.
  \end{itemize}
\end{frame}

\begin{frame}
  \frametitle{Modules and Kernel Version}
  \begin{itemize}
  \item To be compiled, a kernel module needs access to the kernel
    headers, containing the functions, types and constants definitions
  \item Two solutions
    \begin{itemize}
    \item Full kernel sources
    \item Only kernel headers (\code{linux-headers-*} packages in
      Debian/Ubuntu distributions)
    \end{itemize}
  \item The sources or headers must be configured
    \begin{itemize}
    \item Many macros or functions depend on the configuration
    \end{itemize}
  \item A kernel module compiled against version X of kernel headers
    will not load in kernel version Y
    \begin{itemize}
    \item modprobe/insmod will say \emph{Invalid module format}
    \end{itemize}
  \end{itemize}
\end{frame}

\begin{frame}[fragile]
  \frametitle{New Driver in Kernel Sources 1/2}
  \begin{itemize}
  \item To add a new driver to the kernel sources:
    \begin{itemize}
    \item Add your new source file to the appropriate source
      directory. Example: \code{drivers/usb/serial/navman.c}
    \item Single file drivers in the common case, even if the file is
      several thousand lines of code. Only really big drivers are
      split in several files or have their own directory.
    \item Describe the configuration interface for your new driver by
      adding the following lines to the Kconfig file in this
      directory:
    \end{itemize}
{\footnotesize
\begin{verbatim}
config USB_SERIAL_NAVMAN
        tristate "USB Navman GPS device"
        depends on USB_SERIAL
        help
          To compile this driver as a module, choose M
          here: the module will be called navman.
\end{verbatim}
}
  \end{itemize}
\end{frame}

\begin{frame}
  \frametitle{New Driver in Kernel Sources 2/2}
  \begin{itemize}
  \item Add a line in the Makefile file based on the Kconfig setting:
    \code{obj-$(CONFIG_USB_SERIAL_NAVMAN) += navman.o}
  \item It tells the kernel build system to build navman.c when the
    \code{USB_SERIAL_NAVMAN} option is enabled. It works both if
    compiled statically or as a module.
    \begin{itemize}
    \item Run \code{make xconfig} and see your new options!
    \item Run \code{make} and your new files are compiled!
    \item See \code{Documentation/kbuild/} for details and more
      elaborate examples like drivers with several source files, or
      drivers in their own subdirectory, etc.
    \end{itemize}
  \end{itemize}
\end{frame}

\begin{frame}
  \frametitle{How To Create Linux Patches}
  \begin{itemize}
  \item The old school way
    \begin{itemize}
    \item Before making your changes, make sure you have two kernel
      trees: \code{cp -a linux-2.6.37/ linux-2.6.37-patch/}
    \item Make your changes in \code{linux-2.6.37-patch/}
    \item Run \code{make distclean} to keep only source files.
    \item Create a patch file:
      \code{diff -Nur linux-2.6.37/ linux-2.6.37-patch/ > patchfile}
    \item Not practical, does not scale to multiple patches
    \end{itemize}
  \item The new school ways
    \begin{itemize}
    \item Use quilt (tool to manage a stack of patches)
    \item Use Git (revision control system used by the Linux kernel
      developers)
    \end{itemize}
  \end{itemize}
\end{frame}

\begin{frame}[fragile]
  \frametitle{Hello Module with Parameters 1/2}
\begin{minted}[fontsize=\small]{c}
/* hello_param.c */
#include <linux/init.h>
#include <linux/module.h>
#include <linux/moduleparam.h>

MODULE_LICENSE("GPL");

/* A couple of parameters that can be passed in: how many
   times we say hello, and to whom */

static char *whom = "world";
module_param(whom, charp, 0);

static int howmany = 1;
module_param(howmany, int, 0);

\end{minted}
\end{frame}

\begin{frame}[fragile]
  \frametitle{Hello Module with Parameters 2/2}
\begin{minted}[fontsize=\small]{c}
static int __init hello_init(void)
{
  int i;
  for (i = 0; i < howmany; i++)
    pr_alert("(%d) Hello, %s\n", i, whom);
  return 0;
}

static void __exit hello_exit(void)
{
  pr_alert("Goodbye, cruel %s\n", whom);
}

module_init(hello_init);
module_exit(hello_exit);
\end{minted}
Thanks to Jonathan Corbet for the example!

Example available on \url{http://free-electrons.com/doc/c/hello_param.c}
\end{frame}

\begin{frame}[fragile]
  \frametitle{Declaring a module parameter}

\begin{minted}[fontsize=\small]{c}
#include <linux/moduleparam.h>

module_param(
    name, /* name of an already defined variable */
    type, /* either byte, short, ushort, int, uint, long, ulong,
             charp, or bool.(checked at compile time!) */ 
    perm  /* for /sys/module/<module_name>/parameters/<param>,
             0: no such module parameter value file */
);

/* Example */
int irq=5;
module_param(irq, int, S_IRUGO);
\end{minted}
Modules parameter arrays are also possible with
\code{module_param_array()}, but they are less common.
\end{frame}
