\subsection{Basics}
\begin{frame}
  \frametitle{Build Systems}
  \begin{itemize}
  \item Build systems are designed to meet several goals:
    \begin{itemize}
    \item Integrate all the software components, both third-party and
      in-house into a working image
    \item Be able to easily reproduce a given build
    \end{itemize}
  \item Usually, they build software using the existing building system shipped with
    each component
  \item Several solutions: \emph{Yocto}, \emph{Buildroot},
    \emph{ptxdist}.
  \item Google came up with its own solution for Android, that never relies on
    other build systems, except for \emph{GNU/Make}
    \begin{itemize}
    \item It allows to rely on very few tools, and to
      control every software component in a consistent way.
    \item But it also means that when you have to import a new
      component, you have to rewrite the whole Makefile to build it
    \end{itemize}
  \end{itemize}
\end{frame}

\begin{frame}[fragile]
  \frametitle{First compilation}
\begin{minted}{console}
$ source build/envsetup.sh
$ lunch
You're building on Linux

Lunch menu... pick a combo:
     1. generic-eng
     2. simulator
     3. full_passion-userdebug
     4. full_crespo-userdebug

Which would you like? [generic-eng]
$ make
$ make showcommands
\end{minted}
\end{frame}

\begin{frame}
  \frametitle{Output}
  \begin{itemize}
  \item All the output is generated in the
    \code{out/} directory, outside of the source code directory 
  \item This directory contains mostly two subdirectories:
    \code{host/} and \code{target/}
  \item These directories contain all the objects files compiled
    during the build process: \code{.o} files for C/C++ code,
    \code{.jar} files for Java libraries, etc
  \item It is an interesting feature, since it keeps all the generated
    stuff separate from the source code, and we can easily clean without
    side effects
  \item It also generates the system images in the
    \code{out/target/product/<device_name>/} directory
  \end{itemize}
\end{frame}

\begin{frame}
  \frametitle{Cleaning}
  \begin{itemize}
  \item Cleaning is almost as easy as \code{rm -rf out/}
  \item \code{make clean} deletes all generated files for the current combo.
  \item \code{make clobber} deletes all generated files for all combos.
  \item \code{make installclean} removes only the parts that needs to
    be rebuilt when doing a configuration change. It is useful when
    you work with several products to avoid doing a full rebuild each
    time you change from one to the other
  \end{itemize}
\end{frame}
