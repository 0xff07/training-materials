\section{Kernel frameworks for device drivers}

\begin{frame}
  \frametitle{Kernel and Device Drivers}
  \begin{columns}
    \column{0.5\textwidth} In Linux, a driver is always interfacing
    with:
    \begin{itemize}
    \item a {\bf framework} that allows the driver to expose the
      hardware features to user space applications.
    \item a {\bf bus infrastructure}, part of the device model, to
      detect/communicate with the hardware.
    \end{itemize}
    This section focuses on the {\em kernel frameworks}, while the
    {\em device model} was covered earlier in this training.
    \column{0.5\textwidth}
    \includegraphics[height=0.8\textheight]{slides/kernel-frameworks/driver-architecture.pdf}
  \end{columns}
\end{frame}

\subsection{User space vision of devices}

\begin{frame}
  \frametitle{Types of devices} Under Linux, there are essentially
  three types of devices:
  \begin{itemize}
  \item {\bf Network devices}. They are represented as network
    interfaces, visible in user space using \code{ifconfig}.
  \item {\bf Block devices}. They are used to provide user space
    applications access to raw storage devices (hard disks, USB
    keys). They are visible to the applications as {\em device files}
    in \code{/dev}.
  \item {\bf Character devices}. They are used to provide user space
    applications access to all other types of devices (input, sound,
    graphics, serial, etc.). They are also visible to the applications
    as {\em device files} in \code{/dev}.
  \end{itemize}
  $\rightarrow$ Most devices are {\em character devices}, so we will
  study these in more details.
\end{frame}

\begin{frame}
  \frametitle{Major and minor numbers}
  \begin{itemize}
  \item Within the kernel, all block and character devices are
    identified using a {\em major} and a {\em minor} number.
  \item The {\em major number} typically indicates the family of the
    device.
  \item The {\em minor number} typically indicates the number of the
    device (when they are for example several serial ports)
  \item Most major and minor numbers are statically allocated, and
    identical across all Linux systems.
  \item They are defined in \kerneldoc{devices.txt}.
  \end{itemize}
\end{frame}
