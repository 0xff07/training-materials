\begin{frame}
  \frametitle{Porting the Linux kernel}
  \begin{itemize}
  \item The Linux kernel supports a lot of different CPU architectures
  \item Each of them is maintained by a different group of
    contributors
    \begin{itemize}
    \item See the \code{MAINTAINERS} file for details
    \end{itemize}
  \item The organization of the source code and the methods to port
    the Linux kernel to a new board are therefore very
    architecture-dependent
  \item For example, PowerPC and ARM are very different
    \begin{itemize}
    \item PowerPC relies on device trees to describe hardware details
    \item ARM relies on source code only, but the migration to device
      tree is in progress
    \end{itemize}
  \item This presentation is focused on the ARM architecture only
  \end{itemize}
\end{frame}

\begin{frame}
  \frametitle{Architecture, CPU and Machine}
  \begin{itemize}
  \item In the source tree, each architecture has its own directory
    \begin{itemize}
    \item \code{arch/arm} for the ARM architecture
    \end{itemize}
  \item This directory contains generic ARM code
    \begin{itemize}
    \item \code{boot}, \code{common}, \code{configs}, \code{kernel},
      \code{lib}, \code{mm}, \code{nwfpe}, \code{vfp},
      \code{oprofile}, \code{tools}
    \end{itemize}
  \item And many directories for different SoC families
    \begin{itemize}
    \item \code{mach-*} directories: \code{mach-pxa} for PXA CPUs,
      \code{mach-imx} for Freescale iMX CPUs, etc.
    \item Each of these directories contain
      \begin{itemize}
      \item Support for the SoC family (GPIO, clocks, pinmux, power
        management, interrupt controller, etc.)
      \item Support for several boards using this SoC
      \end{itemize}
    \end{itemize}
  \item Some CPU types share some code, in directories named
    \code{plat-*}
  \end{itemize}
\end{frame}

\begin{frame}
  \frametitle{Source Code for Calao USB A9263}
  \begin{itemize}
  \item Taking the case of the Calao USB A9263 board, which uses a
    AT91SAM9263 CPU.
  \item \code{arch/arm/mach-at91}
    \begin{itemize}
    \item AT91 generic code
      \begin{itemize}
      \item \code{clock.c}
      \item \code{leds.c}
      \item \code{irq.c}
      \item \code{pm.c}
      \end{itemize}
    \item CPU-specific code for the AT91SAM9263
      \begin{itemize}
      \item \code{at91sam9263.c}
      \item \code{at91sam926x_time.c}
      \item \code{at91sam9263_devices.c}
      \end{itemize}
    \item Board specific code
      \begin{itemize}
      \item \code{board-usb-a9263.c}
      \end{itemize}
    \end{itemize}
  \item For the rest of this presentation, we will focus on board
    support only
  \end{itemize}
\end{frame}

\begin{frame}[fragile]
  \frametitle{Configuration}
  \begin{itemize}
  \item A configuration option must be defined for the board, in
    \code{arch/arm/mach-at91/Kconfig}
{\scriptsize
\begin{verbatim}
config MACH_USB_A9263
        bool "CALAO USB-A9263"
        depends on ARCH_AT91SAM9263
        help
          Select this if you are using a Calao Systems USB-A9263.
          <http://www.calao-systems.com>
\end{verbatim}
}
  \item This option must depend on the CPU type option corresponding
    to the CPU used in the board
    \begin{itemize}
    \item Here the option is \code{ARCH_AT91SAM9263}, defined in the
      same file
    \end{itemize}
  \item A default configuration file for the board can optionally be
    stored in \code{arch/arm/configs/}. For our board, it's
    \code{at91sam9263_defconfig}
  \end{itemize}
\end{frame}

\begin{frame}[fragile]
  \frametitle{Compilation}
  \begin{itemize}
  \item The source files corresponding to the board support must be
    associated with the configuration option of the board
\begin{verbatim}
obj-$(CONFIG_MACH_USB_A9263) += board-usb-a9263.o
\end{verbatim}
  \item This is done in \code{arch/arm/mach-at91/Makefile}
\begin{verbatim}
obj-y           := irq.o gpio.o
obj-$(CONFIG_AT91_PMC_UNIT)   += clock.o
obj-y                         += leds.o
obj-$(CONFIG_PM)              += pm.o
obj-$(CONFIG_AT91_SLOW_CLOCK) += pm_slowclock.o
\end{verbatim}
  \item The Makefile also tells which files are compiled for every AT91 CPU
  \item And which files for our particular CPU, the AT91SAM9263
{\footnotesize
\begin{verbatim}
obj-$(CONFIG_ARCH_AT91SAM9263) += at91sam9263.o
at91sam926x_time.o at91sam9263_devices.o sam9_smc.o
\end{verbatim}
}
\end{itemize}
\end{frame}

\begin{frame}[fragile]
  \frametitle{Machine Structure}
  \begin{itemize}
  \item Each board is defined by a machine structure
    \begin{itemize}
    \item The word \emph{machine} is quite confusing since every
      \code{mach-*} directory contains several machine definitions, one for
      each board using a given CPU type
    \end{itemize}
  \item For the Calao board, at the end of
    \code{arch/arm/mach-at91/board-usb-a926x.c}
\begin{minted}[fontsize=\footnotesize]{c}
MACHINE_START(USB_A9263, "CALAO USB_A9263")
    /* Maintainer: calao-systems */
    .phys_io = AT91_BASE_SYS,
    .io_pg_offst = (AT91_VA_BASE_SYS >> 18) & 0xfffc,
    .boot_params = AT91_SDRAM_BASE + 0x100,
    .timer = &at91sam926x_timer,
    .map_io = ek_map_io,
    .init_irq = ek_init_irq,
    .init_machine = ek_board_init,
MACHINE_END
\end{minted}
\end{itemize}
\end{frame}

\begin{frame}
  \frametitle{Machine Structure Macros}
  \begin{itemize}
  \item \code{MACHINE_START} and \code{MACHINE_END}
    \begin{itemize}
    \item Macros defined in \code{arch/arm/include/asm/mach/arch.h}
    \item They are helpers to define a \code{struct machine_desc}
      structure stored in a specific ELF section
    \item Several \code{machine_desc} structures can be defined in a
      kernel, which means that the kernel can support several boards.
    \item The right structure is chosen at boot time
    \end{itemize}
  \end{itemize}
\end{frame}

\begin{frame}[fragile]
  \frametitle{Machine Type Number}
  \begin{itemize}
  \item In the ARM architecture, each board type is identified by a
    machine type number
  \item The latest machine type numbers list can be found at
    \url{http://www.arm.linux.org.uk/developer/machines/download.php}
  \item A copy of it exists in the kernel tree in
    \code{arch/arm/tools/mach-types}
    \begin{itemize}
    \item For the Calao board
      \begin{itemize}
      \item \code{usb_a9263 MACH_USB_A9263 USB_A9263 1710}
      \end{itemize}
    \end{itemize}
  \item At compile time, this file is processed to generate a header
    file, \code{include/asm-arm/mach-types.h}
    \begin{itemize}
    \item For the Calao board
      \begin{itemize}
      \item \mint{c}+#define MACH_TYPE_USB_A9263 1710+
      \end{itemize}
    \item And a few other macros in the same file
    \end{itemize}
  \end{itemize}
\end{frame}

\begin{frame}[fragile]
  \frametitle{Machine Type Number}
  \begin{itemize}
  \item The machine type number is set in the \code{MACHINE_START()}
    definition
    \begin{itemize}
    \item \mint{c}+MACHINE_START(USB_A9263, "CALAO USB_A9263")+
    \end{itemize}
  \item At run time, the machine type number of the board on which the
    kernel is running is passed by the bootloader in register r1
  \item Very early in the boot process
    (\code{arch/arm/kernel/head.S}), the kernel calls
    \code{__lookup_machine_type} in
    \code{arch/arm/kernel/head-common.S}
  \item \code{__lookup_machine_type} looks at all the
    \code{machine_desc} structures of the special ELF section
    \begin{itemize}
    \item If it doesn't find the requested number, prints a message
      and stops
    \item If found, it knows the machine descriptions and continues
      the boot process
    \end{itemize}
  \end{itemize}
\end{frame}

\begin{frame}
  \frametitle{Early Debugging and Boot Parameters}
  \begin{itemize}
  \item Early debugging
    \begin{itemize}
    \item \code{phys_io} is the physical address of the I/O space
    \item \code{io_pg_offset} is the offset in the page table to remap
      the I/O space
    \item These are used when \code{CONFIG_DEBUG_LL} is enabled to
      provide very early debugging messages on the serial port
    \end{itemize}
  \item Boot parameters
    \begin{itemize}
    \item \code{boot_params} is the location where the bootloader has
      left the boot parameters (the kernel command line)
    \item The bootloader can override this address in register
      \code{r2}
    \item See also \code{Documentation/arm/Booting} for the details of
      the environment expected by the kernel when booted
    \end{itemize}
  \end{itemize}
\end{frame}

\begin{frame}
  \frametitle{System Timer}
  \begin{itemize}
  \item The timer field points to a \code{struct sys_timer} structure,
    that describes the system timer
    \begin{itemize}
    \item Used to generate the periodic tick at HZ frequency to call
      the scheduler periodically
    \end{itemize}
  \item On the Calao board, the system timer is defined by the
    \code{at91sam926x_timer} structure in \code{at91sam926x_time.c}
  \item It contains the interrupt handler called at HZ frequency
  \item It is integrated with the \code{clockevents} and the
    \code{clocksource} infrastructures
    \begin{itemize}
    \item See \code{include/linux/clocksource.h} and
      \code{include/linux/clockchips.h} for details
    \end{itemize}
  \end{itemize}
\end{frame}

\begin{frame}
  \frametitle{map\_io()}
  \begin{itemize}
  \item The \code{map_io()} function points to \code{ek_map_io()},
    which
    \begin{itemize}
    \item Initializes the CPU using \code{at91sam9263_initialize()}
      \begin{itemize}
      \item Map I/O space
      \item Register and initialize the clocks
      \end{itemize}
    \item Configures the debug serial port and set the console to be
      on this serial port
    \item Called at the very beginning of the C code execution
      \begin{itemize}
      \item \code{init/main.c}: \code{start_kernel()}
      \item \code{arch/arm/kernel/setup.c}: \code{setup_arch()}
      \item \code{arch/arm/mm/mmu.c}: \code{paging_init()}
      \item \code{arch/arm/mm/mmu.c}: \code{devicemaps_init()}
      \item \code{mdesc->map_io()}
      \end{itemize}
    \end{itemize}
  \end{itemize}
\end{frame}

\begin{frame}
  \frametitle{init\_irq()}
  \begin{itemize}
  \item \code{init_irq()} to initialize the IRQ hardware specific
    details
  \item Implemented by \code{ek_init_irq()}, which calls
    \code{at91sam9263_init_interrupts()} in \code{at91sam9263.c},
    which mainly calls \code{at91_aic_init()} in \code{irq.c}
    \begin{itemize}
    \item Initialize the interrupt controller, assign the priorities
    \item Register the IRQ chip (\code{irq_chip} structure) to the
      kernel generic IRQ infrastructure, so that the kernel knows how
      to ack, mask, unmask the IRQs
    \end{itemize}
  \item Called a little bit later than \code{map_io()}
    \begin{itemize}
    \item \code{init/main.c}: \code{start_kernel()}
    \item \code{arch/arm/kernel/irq.c}: \code{init_IRQ()}
    \item \code{init_arch_irq()} (equal to \code{mdesc->init_irq})
    \end{itemize}
  \end{itemize}
\end{frame}

\begin{frame}
  \frametitle{init\_machine()}
  \begin{itemize}
  \item \code{init_machine()} completes the initialization of the
    board by registering all platform devices
  \item Called by \code{customize_machines()} in
    \code{arch/arm/kernel/setup.c}
  \item This function is an \code{arch_initcall} (list of functions
    whose address is stored in a specific ELF section, by levels)
  \item At the end of kernel initialization, just before running the
    first userspace program init:
    \begin{itemize}
    \item \code{init/main.c}: \code{kernel_init()}
    \item \code{init/main.c}: \code{do_basic_setup()}
    \item \code{init/main.c}: \code{do_initcalls()}
    \item Calls all initcalls, level by level
    \end{itemize}
  \end{itemize}
\end{frame}

\begin{frame}
  \frametitle{init\_machine() for Calao}
  \begin{itemize}
  \item For the Calao board, implemented in \code{ek_board_init()}
    \begin{itemize}
    \item Registers serial ports, USB host, USB device, SPI, Ethernet,
      NAND flash, 2IC, buttons and LEDs
    \item Uses \code{at91_add_device_*()} helpers, defined in
      \code{at91sam9263_devices.c}
    \item These helpers call \code{platform_device_register()} to
      register the different \code{platform_device} structures defined
      in the same file
    \item For some devices, the board specific code does the
      registration itself (buttons) or passes board-specific data to
      the registration helper (USB host and device, NAND, Ethernet,
      etc.)
    \end{itemize}
  \end{itemize}
\end{frame}

\begin{frame}
  \frametitle{Drivers}
  \begin{itemize}
  \item The \code{at91sam9263_devices.c} file doesn't implement the
    drivers for the platform devices
  \item The drivers are implemented at different places of the kernel
    tree
  \item For the Calao board
    \begin{itemize}
    \item USB host, driver \code{at91_ohci},
      \code{drivers/usb/host/ohci-at91.c}
    \item USB device, driver \code{at91_udc},
      \code{drivers/usb/gadget/at91_udc.c}
    \item Ethernet, driver \code{macb}, \code{drivers/net/macb.c}
    \item NAND, driver \code{atmel_nand},
      \code{drivers/mtd/nand/atmel_nand.c}
    \item I2C on GPIO, driver \code{i2c-gpio},
      \code{drivers/i2c/busses/i2c-gpio.c}
    \item SPI, driver \code{atmel_spi}, \code{drivers/spi/atmel_spi.c}
    \item Buttons, driver \code{gpio-keys},
      \code{drivers/input/keyboard/gpio_keys.c}
    \end{itemize}
  \item All these drivers are selected by the default configuration
    file
  \end{itemize}
\end{frame}

\begin{frame}
  \frametitle{New Directions in the ARM Architecture}
  \begin{itemize}
  \item The ARM architecture is migrating to the device tree
    \begin{itemize}
    \item \emph{The Device Tree is a data structure for describing
        hardware}
    \item Instead of describing the hardware in C, a special data
      structure, external to the kernel is used
    \item Allows to more easily port the kernel to newer platforms and
      to make a single kernel image support multiple platforms
    \end{itemize}
  \item The ARM architecture is being consolidated
    \begin{itemize}
    \item The \emph{clock} API is being converted to a proper
      framework, with drivers in \code{drivers/clk}
    \item The GPIO support is being converted as proper GPIO drivers
      in \code{drivers/gpio}
    \item The pin muxing support is being converted as drivers in
      \code{drivers/pinctrl}
    \end{itemize}
  \end{itemize}
\end{frame}

\begin{frame}[fragile]
\frametitle{Board Device Tree Example: tegra-harmony.dts}
{\tiny
\begin{verbatim}
/dts-v1/;
/memreserve/ 0x1c000000 0x04000000;
/include/ "tegra20.dtsi"
/ {
    model = "NVIDIA Tegra2 Harmony evaluation board";
    compatible = "nvidia,harmony", "nvidia,tegra20";
    chosen {
        bootargs = "vmalloc=192M video=tegrafb console=ttyS0,115200n8";
    };

    memory@0 {
        reg = < 0x00000000 0x40000000 >;
    };

    i2c@7000c000 {
        clock-frequency = <400000>;

        codec: wm8903@1a {
            compatible = "wlf,wm8903";
            reg = <0x1a>;
            interrupts = < 347 >;

            gpio-controller;
            #gpio-cells = <2>;

            /* 0x8000 = Not configured */
            gpio-cfg = < 0x8000 0x8000 0 0x8000 0x8000 >;
        };
    };
    [...]
};
\end{verbatim}
}
\end{frame}

\begin{frame}
  \frametitle{Device Tree Usage}
  \begin{itemize}
  \item The \emph{device tree source} (\code{.dts}) is compiled into a
    \emph{device tree blob} (\code{.dtb}) using a \emph{device tree
      compiler} (\code{.dtc})
    \begin{itemize}
    \item The \code{dtb} is an efficient binary data structure
    \item The \code{dtb} is either appended to the kernel image, or
      better, passed by the bootloader to the kernel
    \end{itemize}
  \item At runtime, the kernel parses the device tree to find out
    \begin{itemize}
    \item which devices are present
    \item what drivers are needed
    \item which parameters should be used to initialize the devices
    \end{itemize}
  \item On ARM, device tree support is only beginning
  \end{itemize}
\end{frame}

\begin{frame}
  \frametitle{Porting to a New Board: Advise}
  \begin{itemize}
  \item Porting Linux to a new board is easy, when Linux already
    supports the evaluation kit / development board for your CPU.
  \item Most work has already been done and it is just a matter of
    customizing devices instantiated on your boards and their
    settings.
  \item Therefore, look for how the development board is supported, or
    at least for a similar board with the same CPU.
  \item For example, review the (few) differences between the Calao
    \code{qil-a9260} board and Atmel's \code{sam9260} Evaluation Kit:
    \begin{itemize}
    \item \code{meld board-sam9260ek.c board-qil-a9260.c}
    \end{itemize}
  \item Similarly, you will find very few differences in U-boot
    between code for a board and for the corresponding evaluation
    board.
  \end{itemize}
\end{frame}
