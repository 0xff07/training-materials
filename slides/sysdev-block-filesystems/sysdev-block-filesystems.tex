\section{Block filesystems}

\begin{frame}
  \frametitle{Block vs. flash}
  \begin{itemize}
  \item Storage devices are classified in two main types: {\bf block
      devices} and {\bf flash devices}
    \begin{itemize}
    \item They are handled by different subsystems and different
      filesystems
    \end{itemize}
  \item {\bf Block devices} can be read and written to on a per-block
    basis, without erasing.
    \begin{itemize}
    \item Hard disks, floppy disks, RAM disks
    \item USB keys, Compact Flash, SD card: these are based on
      flash storage, but have an integrated controller that emulates a block
      device, managing and erasing flash sectors in a transparent way.
    \end{itemize}
  \item {\bf Raw flash devices} are driven by a controller on the
      SoC. They can be read, but writing requires erasing, and often occurs
      on a larger size than the “block” size.
    \begin{itemize}
    \item NOR flash, NAND flash
    \end{itemize}
  \end{itemize}
\end{frame}

\begin{frame}[fragile]
  \frametitle{Block device list}
  \begin{itemize}
  \item The list of all block devices available in the system can be
    found in \code{/proc/partitions}\\
\begin{verbatim}
$ cat /proc/partitions
major minor #blocks name

 179        0    3866624 mmcblk0
 179        1      73712 mmcblk0p1
 179        2    3792896 mmcblk0p2
   8        0  976762584 sda
   8        1    1060258 sda1
   8        2  975699742 sda2
\end{verbatim}
  \item And also in \code{/sys/block/}
  \end{itemize}
\end{frame}

\begin{frame}
  \frametitle{Traditional block filesystems}
  Traditional filesystems
  \begin{itemize}
  \item Can be left in a non-coherent state after a system crash or
    sudden poweroff, which requires a full filesystem check after
    reboot.
  \item \code{ext2}: traditional Linux filesystem\\
    (repair it with \code{fsck.ext2})
  \item \code{vfat}: traditional Windows filesystem\\
    (repair it with \code{fsck.vfat} on GNU/Linux or Scandisk on
    Windows)
  \end{itemize}
\end{frame}

\begin{frame}
  \frametitle{Journaled filesystems}
  \begin{columns}
    \column{0.4\textwidth}
    \begin{itemize}
    \item Designed to stay in a correct state even after system crashes
      or a sudden poweroff
    \item All writes are first described in the journal before being
      committed to files
    \end{itemize}
    \column{0.6\textwidth}
    \includegraphics[width=\textwidth]{slides/sysdev-block-filesystems/journal.pdf}
  \end{columns}
\end{frame}

\begin{frame}
  \frametitle{Filesystem recovery after crashes}
  \begin{columns}
    \column{0.6\textwidth}
    \includegraphics[width=\textwidth]{slides/sysdev-block-filesystems/journal-recovery.pdf}
    \column{0.4\textwidth}
    \begin{itemize}
    \item Thanks to the journal, the filesystem is never left in a
      corrupted state
    \item Recently saved data could still be lost
    \end{itemize}
  \end{columns}
\end{frame}

\begin{frame}
  \frametitle{Journaled block filesystems}
  Journaled filesystems
  \begin{itemize}
  \item \code{ext3}: \code{ext2} with journal extension\\
    \code{ext4}: newest version in the family with many improvements.
  \item \code{Btrfs} (``Butter FS'')\\
    The next generation. Great performance. Now used in major
    GNU/Linux distros.
  \item The Linux kernel supports many other filesystems:
    \code{reiserFS}, \code{JFS}, \code{XFS}, etc.  Each of them have
    their own characteristics, but are more oriented towards server or
    scientific workloads.
  \item It's easy to switch filesystems. The best is to try each
    and find out which yields the best performance on your own system.
  \end{itemize}
  We recommend \code{ext2} for very small partitions ($<$ 5 MB),
  because other filesystems need too much space for metadata
  (\code{ext3} and \code{ext4} need about 1 MB for a 4 MB partition).
\end{frame}

\begin{frame}
  \frametitle{Creating ext2/ext3/ext4 volumes}
  \begin{itemize}
  \item To create an empty ext2/ext3/ext4 filesystem on a block device or
    inside an already-existing image file
    \begin{itemize}
    \item \code{mkfs.ext2 /dev/hda3}
    \item \code{mkfs.ext3 /dev/sda2}
    \item \code{mkfs.ext4 /dev/sda3}
    \item \code{mkfs.ext2 disk.img}
    \end{itemize}
  \item To create a filesystem image from a directory containing all
    your files and directories
    \begin{itemize}
    \item Use the \code{genext2fs} tool, from the package of the same name
    \item \code{genext2fs -d rootfs/ rootfs.img}
    \item Your image is then ready to be transferred to your block
      device
    \end{itemize}
  \end{itemize}
\end{frame}

\begin{frame}
  \frametitle{Mounting filesystem images}
  \begin{itemize}
  \item Once a filesystem image has been created, one can access and
    modifies its contents from the development workstation, using the
    {\bf loop} mechanism
  \item Example:\\
    \code{genext2fs -d rootfs/ rootfs.img}\\
    \code{mkdir /tmp/tst}\\
    \code{mount -t ext2 -o loop rootfs.img /tmp/tst}
  \item In the \code{/tmp/tst} directory, one can access and modify
    the contents of the \code{rootfs.img} file.
  \item This is possible thanks to \code{loop}, which is a kernel
    driver that emulates a block device with the contents of a file.
  \item Do not forget to run \code{umount} before using the filesystem
    image!
  \end{itemize}
\end{frame}

\begin{frame}
  \frametitle{F2FS}
  \url{http://en.wikipedia.org/wiki/F2FS}
  \begin{itemize}
  \item Filesystem optimized for block devices based on NAND flash
  \item Available in the mainline Linux kernel
  \item Benchmarks: best performer on flash devices most of the time: \\
        See \url{http://lwn.net/Articles/520003/}
  \item Technical details: \url{http://lwn.net/Articles/518988/}
  \end{itemize}
\end{frame}

\begin{frame}
  \frametitle{Squashfs}
  Squashfs: \url{http://squashfs.sourceforge.net}
  \begin{itemize}
  \item Read-only, compressed filesystem for block devices. Fine for
    parts of a filesystem which can be read-only (kernel, binaries...)
  \item Great compression rate and read access performance
  \item Used in most live CDs and live USB distributions
  \item Supports LZO compression for better performance on embedded
    systems with slow CPUs (at the expense of a slightly degraded
    compression rate)
  \item Now supports the XZ algorithm, for a much better compression
        rate, at the expense of higher CPU usage and time.
  \end{itemize}
  Benchmarks: (roughly 3 times smaller than ext3, and 2-4 times faster)\\
  \url{http://elinux.org/Squash_Fs_Comparisons}
\end{frame}

\begin{frame}
  \frametitle{Squashfs - How to use}
  \begin{itemize}
  \item Need to install the \code{squashfs-tools} package
  \item Creation of the image
    \begin{itemize}
    \item On your workstation, create your filesystem image:\\
      \code{mksquashfs rootfs/ rootfs.sqfs}
    \item Caution: if the image already exists remove it first,\\
      or use the \code{-noappend} option.
    \end{itemize}
  \item Installation of the image
    \begin{itemize}
    \item Let's assume your partition on the target is in
      \code{/dev/sdc1}
    \item Copy the filesystem image on the device\\
      \code{dd if=rootfs.sqfs of=/dev/sdc1}\\
      Be careful when using \code{dd} to not overwrite the incorrect
      partition!
    \end{itemize}
  \item Mount your filesystem:\\
    \code{mount -t squashfs /dev/sdc1 /mnt/root}
  \end{itemize}
\end{frame}

\begin{frame}
  \frametitle{tmpfs}

  Not a block filesystem of course!

  Perfect to store temporary data in RAM: system log files, connection
  data, temporary files...

  \begin{itemize}
  \item \code{tmpfs} configuration: \code{File systems -> Pseudo filesystems}\\
    Lives in the Linux file cache. Doesn't waste RAM: unlike ramdisks, no need
    to copy files to the file cache, grows and shrinks to accommodate stored files.
    Saves RAM: can swap out pages to disk when needed.
  \item How to use: choose a name to distinguish the various tmpfs
    instances you could have. Examples:\\
    \code{mount -t tmpfs varrun /var/run}\\
    \code{mount -t tmpfs udev /dev}
  \end{itemize}
  See \kerneldoc{filesystems/tmpfs.txt} in kernel sources.
\end{frame}

\begin{frame}
  \frametitle{Mixing read-only and read-write filesystems}
  \begin{columns}
    \column{0.7\textwidth}
    Good idea to split your block storage into:
    \begin{itemize}
    \item A compressed read-only partition (\code{Squashfs})\\
      Typically used for the root filesystem (binaries, kernel...).\\
      Compression saves space. Read-only access protects your system
      from mistakes and data corruption.
    \item A read-write partition with a journaled filesystem (like \code{ext3})\\
      Used to store user or configuration data.\\
      Guarantees filesystem integrity after power off or crashes.
    \item Ram storage for temporary files (\code{tmpfs})
    \end{itemize}
    \column{0.3\textwidth}
    \includegraphics[width=\textwidth]{slides/sysdev-block-filesystems/mixing-filesystems.pdf}
  \end{columns}
\end{frame}

\begin{frame}
  \frametitle{Issues with flash-based block storage}
  \begin{itemize}
  \item Flash storage made available only through a block interface.
  \item Hence, no way to access a low level flash interface
    and use the Linux filesystems doing wear leveling.
  \item No details about the layer (Flash Translation Layer) they
    use. Details are kept as trade secrets, and may hide poor
    implementations.
  \item Can use {\em flashbench}
        (\url{https://github.com/bradfa/flashbench}) to find out
        the erase block size and optimize filesystem formating.
  \item Not knowing about the wear leveling algorithm,
        it is highly recommended to limit the number of writes
        to these devices.
  \end{itemize}
\end{frame}

\setuplabframe
{Block filesystems}
{
  \begin{itemize}
  \item Creating partitions on your block storage
  \item Booting your system with a mix of filesystems: SquashFS for
    the root filesystem (including applications), ext3 for
    configuration and user data, and tmpfs for
    temporary system files.
  \end{itemize}
}
