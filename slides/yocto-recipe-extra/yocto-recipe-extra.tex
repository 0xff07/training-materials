\section{Writing recipes - going further}

\subsection{Splitting packages}

\begin{frame}
  \frametitle{Benefits}
  \begin{itemize}
    \item Packages can be split.
    \item Useful when a single remote repository provides multiple
      binaries or libraries.
    \item The list of packages to provide is defined by the
      \code{PACKAGES} variable.
  \end{itemize}
\end{frame}

\begin{frame}[fragile]
  \frametitle{Example}
  \begin{itemize}
    \item The \code{kexec tools} provides \code{kexec} and \code{kdump}:
    \begin{block}{}
    \begin{minted}{sh}
require kexec-tools.inc
export LDFLAGS = "-L${STAGING_LIBDIR}"
EXTRA_OECONF = " --with-zlib=yes"

SRC_URI[md5sum] = "b9f2a3ba0ba9c78625ee7a50532500d8"
SRC_URI[sha256sum] = "..."

PACKAGES =+ "kexec kdump"

FILES_kexec = "${sbindir}/kexec"
FILES_kdump = "${sbindir}/kdump"
    \end{minted}
    \end{block}
  \end{itemize}
\end{frame}

\subsection{Packages features}

\begin{frame}
  \frametitle{Benefits}
  \begin{itemize}
    \item Features can be built depending on the needs.
    \item This allows to avoid compiling all features in a software
      component when only a few are required.
    \item A good example is \code{ConnMan}: Bluetooth support
      is built only if there is Bluetooth on the target.
    \item The \code{PACKAGECONFIG} variable is used to configure the
      build on a per feature granularity, for packages.
  \end{itemize}
\end{frame}

\begin{frame}
  \frametitle{\code{PACKAGECONFIG}}
  \begin{itemize}
    \item \code{PACKAGECONFIG} takes the list of features to enable.
    \item \code{PACKAGECONFIG[feature]} takes up to four arguments,
      separated by commas:
      \begin{enumerate}
        \item Argument used by the configuration task if the feature
          is enabled (\code{EXTRA_OECONF}).
        \item Argument added to \code{EXTRA_OECONF} if the feature is
          disabled.
        \item Additional build dependency (\code{DEPENDS}), if enabled.
        \item Additional runtime dependency (\code{RDEPENDS}), if enabled.
      \end{enumerate}
    \item Unused arguments can be omitted or left blank.
  \end{itemize}
\end{frame}

\begin{frame}[fragile]
  \frametitle{Example: from \code{ConnMan}}
  \begin{block}{}
  \begin{minted}{sh}
PACKAGECONFIG ??= "wifi openvpn"

PACKAGECONFIG[wifi] = "--enable-wifi,                 \
                       --disable-wifi,                \
                       wpa-supplicant,                \
                       wpa-supplicant"
PACKAGECONFIG[bluez] = "--enable-bluetooth,           \
                        --disable-bluetooth,          \
                        bluez5,                       \
                        bluez5"
PACKAGECONFIG[openvpn] = "--enable-openvpn,           \
                          --disable-openvpn,          \
                          ,                           \
                          openvpn"
  \end{minted}
  \end{block}
\end{frame}

\subsection{Conditional features}

\begin{frame}[fragile]
  \frametitle{Conditional features}
  \begin{itemize}
    \item Some values can be set dynamically, thanks to a set of
      functions:
    \item \code{base_contains(variable, checkval, trueval,
      falseval, d)}: if \code{checkval} is found in
      \code{variable}, \code{trueval} is returned; otherwise
      \code{falseval} is used.
    \item Example:
      \begin{block}{}
      \begin{minted}[fontsize=\tiny]{sh}
PACKAGECONFIG ??= "                                                       \
    ${@base_contains('DISTRO_FEATURES', 'wifi','wifi', '', d)}            \
    ${@base_contains('DISTRO_FEATURES', 'bluetooth','bluetooth', '', d)}  \
    ${@base_contains('DISTRO_FEATURES', '3g','3g', '', d)}"
      \end{minted}
      \end{block}
  \end{itemize}
\end{frame}

\subsection{Root filesystem creation}

\begin{frame}[fragile]
  \frametitle{Files and directories selection}
  \begin{itemize}
    \item The \code{FILES} variable controls the list of files and
      directories to be placed into packages.
    \item It must be package specific (e.g. with \code{_${PN}}).
    \item In Poky, defaults to:
  \end{itemize}
  \begin{block}{}
    \begin{minted}[fontsize=\scriptsize]{sh}
FILES_${PN} = \
           "${bindir}/* ${sbindir}/* ${libexecdir}/* ${libdir}/lib*${SOLIBS} \
            ${sysconfdir} ${sharedstatedir} ${localstatedir} \
            ${base_bindir}/* ${base_sbindir}/* \
            ${base_libdir}/*${SOLIBS} \
            ${base_prefix}/lib/udev/rules.d ${prefix}/lib/udev/rules.d \
            ${datadir}/${BPN} ${libdir}/${BPN}/* \
            ${datadir}/pixmaps ${datadir}/applications \
            ${datadir}/idl ${datadir}/omf ${datadir}/sounds \
            ${libdir}/bonobo/servers"
    \end{minted}
  \end{block}
  \begin{itemize}
    \item To prevent configuration files to be overwritten during the
      Package Management System update process, use \code{CONFFILES}.
  \end{itemize}
\end{frame}

\begin{frame}
  \frametitle{Root filesystem generation}
  \begin{itemize}
    \item Image generation overview:
      \begin{enumerate}
        \item The rootfs is created using packages.
        \item One or more images files are created, depending on the
          \code{IMAGE_FSTYPES} value.
      \end{enumerate}
    \item The rootfs creation is specific to the \code{IMAGE_PKGTYPE}
      value. It should be defined in the image recipe, otherwise the
      first valid package type defined in \code{PACKAGE_CLASSES} is
      used.
    \item All the magic is done in
      \code{meta/classes/rootfs_${IMAGE_PKGTYPE}.bbclass}
  \end{itemize}
\end{frame}

\begin{frame}[fragile]
  \frametitle{Example: rootfs creation with .deb packages}
  \begin{block}{}
    \begin{minted}{sh}
rootfs_deb_do_rootfs () {
    [...]

    export INSTALL_ROOTFS_DEB="${IMAGE_ROOTFS}"

    [...]

    apt-get update
    apt-get ${APT_ARGS} install ${package_to_install}  \
        --force-yes --allow-unauthenticated

    [...]
}
    \end{minted}
  \end{block}
\end{frame}
