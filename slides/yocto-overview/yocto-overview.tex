\section{Yocto Project and Poky reference system overview}

\subsection{The Yocto Project overview}

\begin{frame}
  \frametitle{About}
  \begin{itemize}
  \item The Yocto Project is a set of templates, tools and methods
        that allow to build custom embedded Linux-based systems.
  \item It is an open source project initiated by the Linux Foundation
        in 2010 and is still managed by one of its fellows: Richard
        Purdie.
  \end{itemize}
\end{frame}

\begin{frame}
  \frametitle{The Yocto Project lexicon}
  \begin{itemize}
  \item The core components of the Yocto Project are:
    \begin{itemize}
      \item BitBake, the {\em build engine}. It is a task scheduler,
        like \code{make}. It interprets configuration files and
        recipes (also called {\em metadata}) to perform a set of
        tasks, to download, configure and build specified applications
        and filesystem images.
      \item OpenEmbedded-Core, a set of base {\em layers}. It is a set
        of recipes, layers and classes which are shared between all
        OpenEmbedded based systems.
      \item Poky, the {\em reference system}. It is a collection of
        projects and tools, used to bootstrap a new distribution based
        on the Yocto Project.
    \end{itemize}
  \end{itemize}
\end{frame}

\begin{frame}{The Yocto Project lexicon}
  \begin{center}
    \includegraphics[width=\textwidth]{slides/yocto-overview/yocto-project-overview.pdf}
  \end{center}
\end{frame}

\begin{frame}
  \frametitle{The Yocto Project lexicon}
  \begin{itemize}
    \item Organization of OpenEmbedded-Core:
    \begin{itemize}
      \item {\em Recipes} describe how to fetch, configure, compile
        and applications and images. They have a specific syntax.
      \item {\em Layers} are sets of recipes, matching a common
        purpose. For Texas Instruments board support, the {\em
        meta-ti} layer is used.
      \item Multiple layers are used within a same distribution,
        depending on the requirements.
      \item It supports the ARM, MIPS (32 and 64 bits), PowerPC and
        x86 (32 and 64 bits) architectures.
      \item It supports QEMU emulated machines for these architectures.
    \end{itemize}
  \end{itemize}
\end{frame}

\begin{frame}
  \frametitle{The Yocto Project lexicon}
  \begin{itemize}
    \item The Yocto Project is \textbf{not used as} a finite set of
          layers and tools.
    \item Instead, it provides a \textbf{common base} of tools and
          layers on top of which custom and specific layers are added,
          depending on your target.
    \item The main required element is \textbf{Poky}, the reference
          system which includes OpenEmbedded-Core. Other available
          tools are optional, but may be useful in some cases.
  \end{itemize}
\end{frame}

\begin{frame}
  \frametitle{Example of a Yocto Project based BSP}
  \begin{itemize}
    \item To build images for a BeagleBone Black, we need:
    \begin{itemize}
      \item The Poky reference system, containing all common recipes
            and tools.
      \item The {\em meta-ti} layer, a set of Texas Instruments
            specific recipes.
    \end{itemize}
    \item All modifications are made in the {\em meta-ti} layer.
      Editing Poky is a \textbf{no-go}!
    \item We will set up this environment in the lab.
  \end{itemize}
\end{frame}

\subsection{The Poky reference system overview}

\begin{frame}
  \frametitle{Download the Poky reference system}
  \begin{itemize}
    \item All official projects part of the Yocto Project are
          available at \url{http://git.yoctoproject.org/cgit/}
    \item To download the Poky reference system: \\
          {\small
          \code{git clone -b pyro git://git.yoctoproject.org/poky.git}
          }
  \end{itemize}
\end{frame}

\begin{frame}{Poky}
  \begin{center}
    \includegraphics[width=\textwidth]{slides/yocto-overview/yocto-overview-poky.pdf}
  \end{center}
\end{frame}

\begin{frame}
  \frametitle{Poky source tree 1/2}
  \begin{description}[style=nextline]
  \item[bitbake/] Holds all scripts used by the BitBake command.
    Usually matches the stable release of the BitBake project.
  \item[documentation/] All documentation sources for the Yocto
    Project documentation. Can be used to generate nice PDFs.
  \item[meta/] Contains the OpenEmbedded-Core metadata.
  \item[meta-skeleton/] Contains template recipes for BSP and
    kernel development.
  \end{description}
\end{frame}

\begin{frame}
  \frametitle{Poky source tree 2/2}
  \begin{description}[style=nextline]
  \item[meta-poky/] Holds the configuration for the Poky
    reference distribution.
  \item[meta-yocto-bsp/] Configuration for the Yocto Project
    reference hardware board support package.
  \item[LICENSE] The license under which Poky is distributed (a mix of
    GPLv2 and MIT).
  \item[oe-init-build-env] Script to set up the OpenEmbedded build
    environment. It will create the build directory. It takes an optional
    parameter which is the build directory name. By default, this is
    \code{build}. This script has to be sourced because it changes
    environment variables.
  \item[scripts] Contains scripts used to set up the environment,
    development tools, and tools to flash the generated images on the
    target.
  \end{description}
\end{frame}

\begin{frame}
  \frametitle{Documentation}
  \begin{itemize}
    \item Documentation for the current sources, compiled as a "mega
      manual", is available at:
      \url{http://www.yoctoproject.org/docs/current/mega-manual/mega-manual.html}
    \item Variables in particular are described in the variable
      glossary:
      \url{http://www.yoctoproject.org/docs/current/ref-manual/ref-manual.html\#ref-variables-glossary}
  \end{itemize}
\end{frame}
