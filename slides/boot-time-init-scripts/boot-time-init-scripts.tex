\section{Optimizing init scripts and system startup}
\begin{frame}
\frametitle{Methodology}
There are multiple ways to reduce the time spent in init scripts before
starting the application:
\begin{itemize}
	\item Start the application as soon as possible after only the
              strictly necessary dependencies.
	\item Simplify shell scripts
	\item Even starting the application before \code{init}
\end{itemize}
\end{frame}

\begin{frame}
\frametitle{Measuring - bootchart}
If you want to have a more detailed look at the userland boot sequence
than with \code{grabserial}, you can use \code{bootchart}.
\begin{figure}[h!]
	\centering
	\includegraphics[height=0.6\textheight]{slides/boot-time-init-scripts/bootlog.png}
\end{figure}
\end{frame}

\begin{frame}[fragile]
\frametitle{Measuring - bootchart}
\begin{itemize}
	\item You can use \code{bootchartd} from \code{busybox}
	      (\code{CONFIG_BOOTCHARTD=y})
	\item Boot your board passing \code{init=/sbin/bootchartd} on your
	      kernel command line
	\item Copy \code{/var/log/bootlog.tgz} from your target to your host
	\item Generate the timechart:
\begin{block}{}
\begin{verbatim}
cd bootchart-<version>
java -jar bootchart.jar bootlog.tgz
\end{verbatim}
\end{block}
\end{itemize}
\code{bootchart} is available at \url{http://www.bootchart.org}
\end{frame}

\begin{frame}[fragile]
\frametitle{Measuring - systemd}
If you are using \code{systemd} as your \code{init} program, you can use
\code{systemd-analyze}. See
\url{http://www.freedesktop.org/software/systemd/man/systemd-analyze.html}.\\
\begin{block}{}
\tiny
\begin{verbatim}
$ systemd-analyze blame
  6207ms udev-settle.service
   735ms NetworkManager.service
   642ms avahi-daemon.service
   600ms abrtd.service
   517ms rtkit-daemon.service
   396ms dbus.service
   390ms rpcidmapd.service
   346ms systemd-tmpfiles-setup.service
   316ms cups.service
   310ms console-kit-log-system-start.service
   309ms libvirtd.service
   303ms rpcbind.service
   298ms ksmtuned.service
   281ms rpcgssd.service
   277ms sshd.service
...
\end{verbatim}
\end{block}
\end{frame}

\begin{frame}
\frametitle{Init}
Starting as soon as possible after all the dependencies are started:
\begin{itemize}
	\item Depends on your \code{init} program. Here we are assuming sysV
	      \code{init} scripts.
	\item \code{init} scripts run in alphanumeric order and start with
	      a letter (K for stop ({\bf k}ill) and S for {\bf s}tart).
	\item You want to use the lowest number you can for your application.
	\item You can even replace \code{init} with your application!
\end{itemize}
How fast would we be if we could be the first started application?
\end{frame}

\begin{frame}
\frametitle{Optimizing init scripts}
\begin{itemize}
	\item Start all your services directly from a single startup
	      script (e.g. \code{/etc/init.d/rcS}). This eliminates multiple
	      calls to \code{/bin/sh}.
	\item Replace \code{udev} with \code{mdev}. \code{mdev} is part of
	      BusyBox. It is not running as a daemon and you can either run it
	      only once or have it handling hotplug events.
	\item Remove \code{udev} (or \code{mdev}) if you just need it
	      to create device files.  Use \code{devtmpfs}
	      (\code{CONFIG_DEVTMPFS}) instead, automatically managed by the
	      kernel, and cheaper.
\end{itemize}
\end{frame}

\begin{frame}[fragile]
\frametitle{Reduce forking (1)}
\begin{itemize}
\item \code{fork}/\code{exec} system calls are very expensive.
      Because of this, calls to executables from shells are slow.
\item Even an \code{echo} in a BusyBox shell results in a \code{fork}
syscall!
\item Select \code{Shells -> Standalone shell} in BusyBox
      configuration to make the shell call applets whenever possible.
\item Pipes and back-quotes are also implemented by
      \code{fork}/\code{exec}.  You can reduce their usage in
      scripts. Example:
      \begin{block}{}
      \begin{verbatim}
cat /proc/cpuinfo | grep model
      \end{verbatim}
      \end{block}
Replace it with:
      \begin{block}{}
      \begin{verbatim}
grep model /proc/cpuinfo
      \end{verbatim}
      \end{block}
\end{itemize}
See \url{http://elinux.org/Optimize_RC_Scripts}
\end{frame}

\begin{frame}[fragile]
\frametitle{Reduce forking (2)}
Replaced:
\begin{block}{}
\begin{verbatim}
if [ $(expr match "$(cat /proc/cmdline)" '.* debug.*')\
       -ne 0 -o -f /root/debug ]; then
DEBUG=1
\end{verbatim}
\end{block}
By a much cheaper command running only one process:
\begin{block}{}
\begin{verbatim}
res=`grep " debug" /proc/cmdline`
if [ "$res" -o -f /root/debug ]; then
DEBUG=1
\end{verbatim}
\end{block}
This only optimization allowed to save 87 ms on an ARM AT91SAM9263
system (200 MHz)!
\end{frame}


\begin{frame}
\frametitle{Reduce size (1)}
\begin{itemize}
	\item Strip your executables and libraries, removing ELF sections
		only needed for development and debugging. The \code{strip}
		command is provided by your cross-compiling toolchain.
		\code{BR2_STRIP_strip} in Buildroot.
	\item \code{superstrip}:
		\url{http://muppetlabs.com/~breadbox/software/elfkickers.html}.
		Goes beyond \code{strip} and can strip out a few more bits
		that are not used by Linux to start an executable.
		\code{BR2_STRIP_sstrip} in Buildroot.
\end{itemize}
\end{frame}

\begin{frame}
\frametitle{Reduce size (2)}
\begin{itemize}
	\item use \code{mklibs}, available at
		\url{http://packages.debian.org/sid/mklibs}.
		\begin{itemize}
		\item \code{mklibs} produces cut-down shared libraries that contain
		only the routines required by a particular set of executables.
		Really useful with big libraries like OpenGL and QT. It even
		works without having the source code.
		\item Available in Yocto, but not in Buildroot (at least
                in 2013.11).
                \item Limitation: \code{mklibs} could remove \code{dlopen}ed libraries
                (loaded "manually" by applications) because it doesn't see them.
		\end{itemize}
\end{itemize}
\end{frame}

\setuplabframe
{Reducing time in init-scripts}
{
\begin{itemize}
\item Regenerate the root filesystem with Buildroot
\item Use bootchart to measure boot time
\end{itemize}
}


