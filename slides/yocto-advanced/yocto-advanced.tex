\section{Using Yocto Project - advanced usage}

\begin{frame}
  \frametitle{Advanced build usage and configuration}
  \begin{itemize}
    \item Select package variants.
    \item Manually add packages to the generated image.
    \item Run specific tasks with BitBake.
  \end{itemize}
\end{frame}

\begin{frame}
  \frametitle{A little reminder}
  \begin{itemize}
    \item {\em Recipes} describe how to fetch, configure, compile and
      install packages.
    \item These tasks can be run independently (if their dependencies
      are met).
    \item All available packages in Poky are not selected by default
      in the images.
    \item Some packages may provide the same functionality, e.g.
      OpenSSH and Dropbear.
  \end{itemize}
\end{frame}

\subsection{Advanced configuration}

\begin{frame}
  \frametitle{Overview}
  \begin{itemize}
    \item The OpenEmbedded build system uses configuration variables
      to hold information.
    \item Configuration settings are in upper-case by convention, e.g.
      \code{CONF_VERSION}
    \item To make configuration easier, it is possible to prepend,
      append or define these variables in a conditional way.
    \item All variables can be overridden or modified in
      \code{build/conf/local.conf}
  \end{itemize}
\end{frame}

\begin{frame}
  \frametitle{Methods and conditions 1/3}
  \begin{itemize}
    \item Append the keyword \code{_append} to a configuration
      variable to add values {\bf after} the ones previously defined
      (without space).
      \begin{itemize}
        \item \code{IMAGE_INSTALL_append = " dropbear"} adds
          \code{dropbear} to the packages installed on the image.
      \end{itemize}
    \item Append the keyword \code{_prepend} to add values {\bf
      before} the ones previously defined (without space).
      \begin{itemize}
        \item \code{FILESEXTRAPATHS_prepend := "${THISDIR}/${PN}:"}
          %stopzone
          adds the folder to the set of paths where files are located
          (in a recipe).
      \end{itemize}
    \item Append the machine name to only define a configuration
      variable for a given machine. It tries to match with values from
      \code{MACHINEOVERRIDES} which include \code{MACHINE} and
      \code{SOC_FAMILY}.
      \begin{itemize}
        \item \code{KERNEL_DEVICETREE_beaglebone = "am335x-bone.dtb"}
          tells to use the kernel device tree \code{am335x-bone.dtb}
          only when the machine is \code{beaglebone}.
      \end{itemize}
  \end{itemize}
\end{frame}

\begin{frame}
  \frametitle{Methods and conditions 2/3}
  \begin{itemize}
    \item The previous methods can be combined.
    \item If we define:
      \begin{itemize}
        \item \code{IMAGE_INSTALL = "busybox mtd-utils"}
        \item \code{IMAGE_INSTALL_append = " dropbear"}
        \item \code{IMAGE_INSTALL_append_beaglebone = " i2c-tools"}
      \end{itemize}
    \item The resulting configuration variable will be:
      \begin{itemize}
        \item \code{IMAGE_INSTALL = "busybox mtd-utils dropbear
          i2c-tools"} if the machine being built is
          \code{beaglebone}.
        \item \code{IMAGE_INSTALL = "busybox mtd-utils dropbear"}
          otherwise.
      \end{itemize}
  \end{itemize}
\end{frame}

\begin{frame}[fragile]
  \frametitle{Methods and conditions 3/3}
  \begin{itemize}
    \item The most specific variable takes precedence.
    \item Example:
      \begin{minted}[fontsize=\footnotesize]{console}
IMAGE_INSTALL_beaglebone = "busybox mtd-utils i2c-tools"
IMAGE_INSTALL = "busybox mtd-utils"
      \end{minted}
    \item If the machine is \code{beaglebone}:
      \begin{itemize}
        \item \code{IMAGE_INSTALL = "busybox mtd-utils i2c-tools"}
      \end{itemize}
    \item Otherwise:
      \begin{itemize}
        \item \code{IMAGE_INSTALL = "busybox mtd-utils"}
      \end{itemize}
  \end{itemize}
\end{frame}

\begin{frame}
  \frametitle{Operators 1/2}
  \begin{itemize}
    \item Various operators can be used to assign values to
    configuration variables:
      \begin{description}
        \item[=] expand the value when using the variable
        \item[:=] immediately expand the value
        \item[+=] append (with space)
        \item[=+] prepend (with space)
        \item[.=] append (without space)
        \item[=.] prepend (without space)
        \item[?=] assign if no other value was previously assigned
        \item[??=] same as previous, with a lower precedence
      \end{description}
  \end{itemize}
\end{frame}

\begin{frame}
  \frametitle{Operators 2/2}
  \begin{itemize}
    \item Avoid using \code{+=}, \code{=+}, \code{.=} and \code{=.}
      in \code{build/conf/local.conf} due to ordering issues.
      \begin{itemize}
        \item If \code{+=} is parsed before \code{?=}, the latter will
          be discarded.
        \item Using \code{_append} unconditionally appends the value.
      \end{itemize}
  \end{itemize}
\end{frame}

\subsection{Packages variants}

\begin{frame}
  \frametitle{Introduction to package variants}
  \begin{itemize}
    \item Some packages have the same purpose, and only one can be
      used at a time.
    \item The build system uses {\bf virtual packages} to reflect
      this.  A virtual package describes functionalities and several
      packages may provide it.
    \item Only one of the packages that provide the functionality will
    be compiled and integrated into the resulting image.
  \end{itemize}
\end{frame}

\begin{frame}
  \frametitle{Variant examples}
  \begin{itemize}
    \item The virtual packages are often in the form
      \code{virtual/<name>}
    \item Example of available virtual packages with some of their
      variants:
      \begin{itemize}
        \item \code{virtual/bootloader}: u-boot,
          u-boot-ti-staging\dots
        \item \code{virtual/kernel}: linux-yocto, linux-yocto-tiny,
          linux-yocto-rt, linux-ti-staging\dots
        \item \code{virtual/libc}: eglibc, uclibc
        \item \code{virtual/xserver}: xserver-xorg
      \end{itemize}
  \end{itemize}
\end{frame}

\begin{frame}
  \frametitle{Package selection}
  \begin{itemize}
    \item Variants are selected thanks to the
      \code{PREFERRED_PROVIDER} configuration variable.
    \item The package names {\bf have to} suffix this variable.
    \item Examples:
    \begin{itemize}
      \item \code{PREFERRED_PROVIDER_virtual/kernel ?=
        "linux-ti-staging"}
      \item \code{PREFERRED_PROVIDER_virtual/libgl = "mesa"}
    \end{itemize}
  \end{itemize}
\end{frame}

\begin{frame}
  \frametitle{Version selection}
  \begin{itemize}
    \item By default, Bitbake will try to build the provider with the
      highest version number, unless the recipe defines
      \code{DEFAULT_PREFERENCE = "-1"}
    \item When multiple package versions are available, it is also
      possible to explicitly pick a given version with
      \code{PREFERRED_VERSION}.
    \item The package names {\bf have to} suffix this variable.
    \item {\bf \%} can be used as a wildcard.
    \item Example:
    \begin{itemize}
      \item \catcode`\%=11\code{PREFERRED_VERSION_linux-yocto =
        "3.10\%"}\catcode`\@=14
      \item \code{PREFERRED_VERSION_python = "2.7.3"}
    \end{itemize}
  \end{itemize}
\end{frame}

\subsection{Packages}

\begin{frame}
  \frametitle{Selection}
  \begin{itemize}
    \item The set of packages installed into the image is defined by
      the target you choose (e.g. \code{core-image-minimal}).
    \item It is possible to have a custom set by defining our own
      target, and we will see this later.
    \item When developing or debugging, adding packages can be useful,
      without modifying the recipes.
    \item Packages are controlled by the \code{IMAGE_INSTALL}
      configuration variable.
  \end{itemize}
\end{frame}

\begin{frame}
  \frametitle{Exclusion}
  \begin{itemize}
    \item The list of packages to install is also filtered using the
      \code{PACKAGE_EXCLUDE} variable.
    \item However, if a package needs installing to satisfy a
      dependency, it will still be selected.
  \end{itemize}
\end{frame}

\subsection{The power of BitBake}

\begin{frame}
  \frametitle{Common BitBake options}
  \begin{itemize}
    \item BitBake can be used to run a full build for a given target
      with \code{bitbake [target]}.
    \item But it can be more precise, with optional options:
    \begin{description}
      \item[\code{-c <task>}] execute the given task
      \item[\code{-s}] list all locally available packages and their
        versions
      \item[\code{-f}] force the given task to be run by removing its
        stamp file
      \item[\code{world}] keyword for all packages
      \item[\code{-b <recipe>}] execute tasks from the given
        \code{recipe} (without resolving dependencies).
    \end{description}
  \end{itemize}
\end{frame}

\begin{frame}
  \frametitle{BitBake examples}
  \begin{itemize}
    \item \code{bitbake -c listtasks virtual/kernel}
    \begin{itemize}
      \item Gives a list of the available tasks for the package
      \code{virtual/kernel}. Tasks are prefixed with \code{do_}.
    \end{itemize}
    \item \code{bitbake -c menuconfig virtual/kernel}
    \begin{itemize}
      \item Execute the task \code{menuconfig} on the kernel package.
    \end{itemize}
    \item \code{bitbake -f dropbear}
    \begin{itemize}
      \item Force the \code{dropbear} package to be rebuilt from
        scratch.
    \end{itemize}
    \item \code{bitbake -c fetchall world}
    \begin{itemize}
      \item Download all package sources and their dependencies.
    \end{itemize}
    \item For a full description: \code{bitbake --help}
  \end{itemize}
\end{frame}

\begin{frame}[fragile]
  \frametitle{shared state cache}
  \begin{itemize}
    \item BitBake stores the output of each task in a directory, the
      shared state cache. Its location is controlled by the
      \code{SSTATE_DIR} variable.
    \item This cache is use to speed up compilation.
    \item Over time, as you compile more recipes, it can grow quite
      big. It is possible to clean old data with:
      \begin{block}{}
        \begin{minted}[fontsize=\footnotesize]{console}
$ ./scripts/sstate-cache-management.sh --remove-duplicated -d \
  --cache-dir=<SSTATE_DIR>
        \end{minted}
      \end{block}
  \end{itemize}
\end{frame}

\subsection{Network usage}

\begin{frame}
  \frametitle{Source fetching}
  \begin{itemize}
    \item BitBake will look for files to retrieve at the following
      locations, in order:
      \begin{enumerate}
        \item \code{DL_DIR} (the local download directory).
        \item The \code{PREMIRRORS} locations.
        \item The upstream source, as defined in \code{SRC_URI}.
        \item The \code{MIRRORS} locations.
      \end{enumerate}
    \item If all the mirrors fail, the build will fail.
  \end{itemize}
\end{frame}

\begin{frame}[fragile]
  \frametitle{Mirror configuration in Poky}
  \begin{block}{}
    \begin{minted}[fontsize=\scriptsize]{sh}
PREMIRRORS ??= "\
bzr://.*/.*   http://downloads.yoctoproject.org/mirror/sources/ \n \
cvs://.*/.*   http://downloads.yoctoproject.org/mirror/sources/ \n \
git://.*/.*   http://downloads.yoctoproject.org/mirror/sources/ \n \
hg://.*/.*    http://downloads.yoctoproject.org/mirror/sources/ \n \
osc://.*/.*   http://downloads.yoctoproject.org/mirror/sources/ \n \
p4://.*/.*    http://downloads.yoctoproject.org/mirror/sources/ \n \
svk://.*/.*   http://downloads.yoctoproject.org/mirror/sources/ \n \
svn://.*/.*   http://downloads.yoctoproject.org/mirror/sources/ \n"

MIRRORS =+ "\
ftp://.*/.*   http://downloads.yoctoproject.org/mirror/sources/ \n \
http://.*/.*  http://downloads.yoctoproject.org/mirror/sources/ \n \
https://.*/.* http://downloads.yoctoproject.org/mirror/sources/ \n"
    \end{minted}
  \end{block}
\end{frame}

\begin{frame}[fragile]
  \frametitle{Configuring the mirrors}
  \begin{itemize}
    \item It's possible to prepend custom mirrors, using the
      \code{PREMIRRORS} variable:
  \end{itemize}
  \begin{block}{}
    \begin{minted}{sh}
PREMIRRORS_prepend = "\
 git://.*/.* http://www.yoctoproject.org/sources/   \n \
 ftp://.*/.* http://www.yoctoproject.org/sources/   \n \
 http://.*/.* http://www.yoctoproject.org/sources/  \n \
 https://.*/.* http://www.yoctoproject.org/sources/ \n"
    \end{minted}
  \end{block}
  \begin{itemize}
    \item Another solution is to use the \code{own-mirror} class:
  \end{itemize}
  \begin{block}{}
    \begin{minted}{sh}
INHERIT += "own-mirrors"
SOURCE_MIRROR_URL = "http://example.com/my-source-mirror"
    \end{minted}
  \end{block}
\end{frame}

\begin{frame}[fragile]
  \frametitle{Forbidding network access}
  \begin{itemize}
    \item You can use \code{BB_GENERATE_MIRROR_TARBALLS = "1"} to
      generate tarballs of the git repositories in \code{DL_DIR}
    \item You can also completely disable network access using
      \code{BB_NO_NETWORK = "1"}
    \item Or restrict BitBake to only download files from the
      \code{PREMIRRORS}, using \code{BB_FETCH_PREMIRRORONLY = "1"}
  \end{itemize}
\end{frame}
