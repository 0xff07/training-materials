\subsection{Embedded Linux distribution projects}
\begin{frame}
  \frametitle{Build system definition}
  \begin{itemize}
  \item Purposes of a build system:
    \begin{itemize}
      \item Compiling or cross-compiling applications.
      \item Packaging applications.
      \item Testing output binaries and ecosystem compatibility.
      \item Deploying generated images.
    \end{itemize}
  \end{itemize}
\end{frame}

\begin{frame}
  \frametitle{Available system building tools} Large choice of tools
  \small
  \begin{itemize}
  \item {\bf Buildroot}, developed by the community\\
    \url{http://www.buildroot.org}
  \item {\bf PTXdist}, developed by Pengutronix\\
    \url{http://pengutronix.de/software/ptxdist/}
  \item {\bf OpenWRT}, originally a fork of Buildroot for wireless
    routers, now a more generic project\\
    \url{http://www.openwrt.org}
  \item {\bf OpenEmbedded} based build systems\\
    \url{http://www.openembedded.org}:
        \begin{itemize}
          \item Poky (from the Yocto Project)
          \item Arago Project
          \item Ångström
        \end{itemize}
  \item Vendor specific tools (silicon vendor or embedded Linux
    vendor)
  \end{itemize}
\end{frame}

\begin{frame}
  \frametitle{Comparison of distribution projects}
  \begin{itemize}
  \item Buildroot
    \begin{itemize}
      \item Simple to use.
      \item Adapted for small embedded devices.
      \item Not perfect if you need advanced functionalities and
            multiple machines support.
      \item \url{http://buildroot.org/}
    \end{itemize}
  \end{itemize}
\end{frame}

\begin{frame}
  \frametitle{Comparison of distribution projects}
  \begin{itemize}
  \item OpenWRT
    \begin{itemize}
      \item Based on Buildroot.
      \item Primarily used for embedded network devices like routers.
      \item \url{http://openwrt.org/}
    \end{itemize}
  \end{itemize}
\end{frame}

\begin{frame}
  \frametitle{Comparison of distribution projects}
  \begin{itemize}
  \item Poky
    \begin{itemize}
      \item Part of the Yocto Project.
      \item Using OpenEmbedded.
      \item Suitable for more complex embedded systems.
      \item Allows lots of customization.
      \item Can be used for multiple targets at the same time.
      \item \url{http://yoctoproject.org/}
    \end{itemize}
  \end{itemize}
\end{frame}

\subsection{Build system benefits}

\begin{frame}
  \frametitle{Working without a build system}
  \begin{itemize}
    \item Each application has to be built manually, or using custom
          and non stable scripts.
    \item The root file system has to be created from scratch.
    \item The applications configurations have to be done by hand.
    \item Each dependency has to be matched manually.
    \item Integrating software programs from different teams is painful.
  \end{itemize}
\end{frame}

\begin{frame}
  \frametitle{Benefits}
  \begin{itemize}
    \item Build systems automate the process of building a target
      system, including the kernel, and sometimes the toolchain.
    \item They automatically download, configure, compile and install
      all the components in the right order, sometimes after applying
      patches to fix cross-compiling issues.
    \item They make sure all the application dependencies are matched.
    \item They already contain a large number of packages, that should
      fit your main requirements, and are easily extensible.
    \item The build becomes reproducible, which allows to easily
      change the configuration of some components, upgrade them, fix
      bugs, etc.
    \item Several configurations can be handled in the same project.
      It is possible to generate the same root file system for
      different hardware targets or to have a debug image based on the
      production one, with some more flags or debugging applications.
  \end{itemize}
\end{frame}

\begin{frame}
  \frametitle{Workflow}
  \begin{itemize}
    \item Development of each application is done
          \textbf{out of} the build system!
      \begin{itemize}
        \item Development is done on an external repository.
        \item The build system downloads sources from this repository
              and starts the build following the instructions.
      \end{itemize}
    \item The build system is used to build the full system and 
          to provide a working image to the customer.
  \end{itemize}
\end{frame}
