\begin{frame}
\frametitle{Initramfs for boot time reduction}
Create the smallest initramfs possible, just enough to start the critical
application and then switch to the final root filesystem with
\code{switch_root}:
\begin{itemize}
\item Use a light C library reduced to the minimum,
      {\em uClibc} if you are not yet using it for your root filesystem
\item Reduce BusyBox to the strict minimum.
      You could even do without it and implement \code{/init} in C.
\item Use statically linked applications (less CPU overhead, less
      libraries to load, smaller initramfs if no libraries at all).
      \code{BR2_STATIC_LIBS} in Buildroot.
\end{itemize}
\end{frame}

\begin{frame}
\frametitle{Statically linked executables: licensing constraints}
\begin{itemize}
\item Statically linked executables are very useful to reduce size
      (especially in small initramfs), and require less work to start.
\item However, the LGPL license in the uClibc and glibc libraries
      require to leave the user the ability to relink the executable
      with a modified version of the library.
\item Solution to keep static binaries:
      \begin{itemize}
      \item Either provide the executable source code (even
            proprietary), allowing to recompile it with a modified
            version of the library. That's what you do when
            you ship a static BusyBox.
      \item Or also provide a dynamically linked version of the
	    executable (in a separate way), allowing to use another
            library version.
      \item Easiest solution: build your static executables with the
	    \code{musl} library (MIT license: no trouble)
      \end{itemize}
\item References: \\
      {\footnotesize
      \url{https://gnu.org/licenses/gpl-faq.html\#LGPLStaticVsDynamic} \\
      \url{https://gnu.org/copyleft/lesser.html\#section4}
      }
\end{itemize}
\end{frame}

\begin{frame}
\frametitle{Do not compress your initramfs}
\begin{itemize}
\item If you ship your initramfs inside a compressed kernel image, don't compress
      it \\
      (enable \code{CONFIG_INITRAMFS_COMPRESSION_NONE}).
\item Otherwise, by default, your initramfs data will be compressed twice, and
      the kernel will be bigger and will take a little more time to load
      and uncompress.
\item Example on Linux 5.1 with a 1.60 MB initramfs (tar archive size)
      on Beagle Bone Black: this allowed to reduce the kernel size from 4.94
      MB to 4.74 MB (-200 KB) and save about 170 ms of boot time.
\end{itemize}
\end{frame}
