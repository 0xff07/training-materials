\section{Buildroot source and build trees}

\subsection{Source tree}

\begin{frame}{Source tree (1)}
  \begin{itemize}
  \item \code{Makefile}
    \begin{itemize}
    \item top-level \code{Makefile}, handles the configuration and
      general orchestration of the build
    \end{itemize}
  \item \code{Config.in}
    \begin{itemize}
    \item top-level \code{Config.in}, main/general options. Includes
      many other \code{Config.in} files
    \end{itemize}
  \item \code{arch/}
    \begin{itemize}
    \item \code{Config.in.*} files defining the architecture
      variants (processor type, ABI, floating point, etc.)
    \item \code{Config.in}, \code{Config.in.arm},
      \code{Config.in.x86}, \code{Config.in.microblaze}, etc.
    \end{itemize}
  \end{itemize}
\end{frame}

\begin{frame}{Source tree (2)}
  \begin{itemize}
  \item \code{toolchain/}
    \begin{itemize}
    \item packages for generating or using toolchains
    \item \code{toolchain/} virtual package that depends on either
      \code{toolchain-buildroot} or \code{toolchain-external}
    \item \code{toolchain-buildroot/} virtual package to build the
      internal toolchain
    \item \code{toolchain-external/} package to handle external
      toolchains
    \end{itemize}
  \item \code{system/}
    \begin{itemize}
    \item \code{skeleton/} the rootfs skeleton
    \item \code{Config.in}, options for system-wide features like
      init system, \code{/dev} handling, etc.
    \end{itemize}
  \item \code{linux/}
    \begin{itemize}
    \item \code{linux.mk}, the Linux kernel package
    \end{itemize}
  \end{itemize}
\end{frame}

\begin{frame}{Source tree (3)}
  \begin{itemize}
  \item \code{package/}
    \begin{itemize}
    \item all the userspace packages (1600+)
    \item \code{busybox/}, \code{gcc/}, \code{qt5/}, etc.
    \item \code{pkg-generic.mk}, core package infrastructure
    \item \code{pkg-cmake.mk}, \code{pkg-autotools.mk},
      \code{pkg-perl.mk}, etc. Specialized package infrastructures
    \end{itemize}
  \item \code{fs/}
    \begin{itemize}
    \item logic to generate filesystem images in various formats
    \item \code{common.mk}, common logic
    \item \code{cpio/}, \code{ext2/}, \code{squashfs/}, \code{tar/},
      \code{ubifs/}, etc.
    \end{itemize}
  \item \code{boot/}
    \begin{itemize}
    \item bootloader packages
    \item \code{at91bootstrap3/}, \code{barebox/}, \code{grub/},
      \code{syslinux/}, \code{uboot/}, etc.
    \end{itemize}
  \end{itemize}
\end{frame}

\begin{frame}{Source tree (4)}
  \begin{itemize}
  \item \code{configs/}
    \begin{itemize}
    \item default configuration files for various platforms
    \item similar to kernel defconfigs
    \item \code{atmel_xplained_defconfig},
      \code{beaglebone_defconfig}, \code{raspberrypi_defconfig}, etc.
    \end{itemize}
  \item \code{board/}
    \begin{itemize}
    \item board-specific files (kernel configuration files, kernel
      patches, image flashing scripts, etc.)
    \item typically go together with a {\em defconfig} in
      \code{configs/}
    \end{itemize}
  \item \code{support/}
    \begin{itemize}
    \item misc utilities (kconfig code, libtool patches, download
      helpers, and more.)
    \end{itemize}
  \end{itemize}
\end{frame}

\begin{frame}{Source tree (5)}
  \begin{itemize}
  \item \code{docs/}
    \begin{itemize}
    \item Buildroot documentation
    \item Written in AsciiDoc, can generate HTML, PDF, TXT versions:
      \code{make manual}
    \item ~90 pages PDF document
    \item Also available pre-generated online.
    \end{itemize}
  \end{itemize}
\end{frame}

\subsection{Build tree}

\begin{frame}{Build tree: {\tt \$(O)}}
  \begin{itemize}
  \item \code{output/}
  \item Global output directory
  \item Can be customized for out-of-tree build by passing \code{O=<dir>}
  \item Variable: \code{O} (as passed on the command line)
  \item Variable: \code{BASE_DIR} (as an absolute path)
  \end{itemize}
\end{frame}

\begin{frame}{Build tree: {\tt \$(O)/build}}
  \begin{itemize}
  \item \code{output/}
    \begin{itemize}
    \item \code{build/}
      \begin{itemize}
        \tiny
      \item \code{buildroot-config/}
      \item \code{busybox-1.22.1/}
      \item \code{host-pkgconf-0.8.9/}
      \item \code{kmod-1.18/}
      \item \code{build-time.log}
      \end{itemize}
    \item Where all source tarballs are extracted
    \item Where the build of each package takes place
    \item In addition to the package sources and object files, {\em
        stamp} files are created by Buildroot
    \item Variable: \code{BUILD_DIR}
    \end{itemize}
  \end{itemize}
\end{frame}

\begin{frame}{Build tree: {\tt \$(O)/host}}
  \begin{itemize}
  \item \code{output/}
    \begin{itemize}
    \item \code{host/}
      \begin{itemize}
        \tiny
      \item \code{usr/lib}
      \item \code{usr/bin}
      \item \code{usr/sbin}
        \vspace{0.2cm}
      \item \code{usr/<tuple>/sysroot/bin}
      \item \code{usr/<tuple>/sysroot/lib}
      \item \code{usr/<tuple>/sysroot/usr/lib}
      \item \code{usr/<tuple>/sysroot/usr/bin}
      \end{itemize}
    \item Contains both the tools built for the host
      (cross-compiler, etc.) and the {\em sysroot} of the toolchain
    \item Variable: \code{HOST_DIR}
    \item Host tools are directly in \code{host/usr}
    \item The {\em sysroot} is in \code{host/<tuple>/sysroot/usr}
    \item Variable for the {\em sysroot}: \code{STAGING_DIR}
    \end{itemize}
  \end{itemize}
\end{frame}

\begin{frame}{Build tree: {\tt \$(O)/staging}}
  \begin{itemize}
  \item \code{output/}
    \begin{itemize}
    \item \code{staging/}
    \item Just a symbolic link to the {\em sysroot}, i.e to
      \code{host/<tuple>/sysroot/}.
    \item Available for convenience
    \end{itemize}
  \end{itemize}
\end{frame}

\begin{frame}{Build tree: {\tt \$(O)/target}}
  \begin{itemize}
  \item \code{output/}
    \begin{itemize}
    \item \code{target/}
      \begin{itemize}
        \tiny
      \item \code{bin/}
      \item \code{etc/}
      \item \code{lib/}
      \item \code{usr/bin/}
      \item \code{usr/lib/}
      \item \code{usr/share/}
      \item \code{usr/sbin/}
      \item \code{THIS_IS_NOT_YOUR_ROOT_FILESYSTEM}
      \item ...
      \end{itemize}
    \item The target root filesystem
    \item Usual Linux hierarchy
    \item Not completely ready for the target: permissions, device
      files, etc.
    \item Buildroot does not run as root: all files are owned by the
      user running Buildroot, not {\em setuid}, etc.
    \item Used to generate the final root filesystem images in
      \code{images/}
    \item Variable: \code{TARGET_DIR}
    \end{itemize}
  \end{itemize}
\end{frame}

\begin{frame}{Build tree: {\tt \$(O)/images}}
  \begin{itemize}
  \item \code{output/}
    \begin{itemize}
    \item \code{images/}
      \begin{itemize}
        \scriptsize
      \item \code{zImage}
      \item \code{armada-370-mirabox.dtb}
      \item \code{rootfs.tar}
      \item \code{rootfs.ubi}
      \end{itemize}
    \item Contains the final images: kernel image, bootloader image,
      root filesystem image(s)
    \item Variable: \code{BINARIES_DIR}
    \end{itemize}
  \end{itemize}
\end{frame}

\begin{frame}{Build tree: {\tt \$(O)/graphs}}
  \begin{itemize}
  \item \code{output/}
    \begin{itemize}
    \item \code{graphs/}
    \item Visualization of Buildroot operation: dependencies between
      packages, time to build the different packages
    \item \code{make graph-depends}
    \item \code{make graph-build}
    \item Variable: \code{GRAPHS_DIR}
    \end{itemize}
  \end{itemize}
\end{frame}

\begin{frame}{Build tree: {\tt \$(O)/legal-info}}
  \begin{itemize}
  \item \code{output/}
    \begin{itemize}
    \item \code{legal-info/}
      \begin{itemize}
        \scriptsize
      \item \code{manifest.csv}
      \item \code{host-manifest.csv}
      \item \code{licenses.txt}
      \item \code{licenses/}
      \item \code{sources/}
      \item ...
      \end{itemize}
    \item Legal information: license of all packages, and their
      source code, plus a licensing manifest
    \item Useful for license compliance
    \item \code{make legal-info}
    \item Variable: \code{LEGAL_INFO_DIR}
    \end{itemize}
  \end{itemize}
\end{frame}
