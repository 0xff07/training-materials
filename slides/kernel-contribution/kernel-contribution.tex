\subsection{Contributing to the Linux kernel}

\begin{frame}
  \frametitle{Getting help and reporting bugs}
  \begin{itemize}
  \item If you are using a custom kernel from a hardware vendor, contact
        that company. The community will have less interest supporting
        a custom kernel.
  \item Otherwise, or if this doesn't work, try to reproduce the
        issue on the latest version of the kernel. 
  \item Make sure you investigate the issue as much as you can: see
    \kerneldoctext{BUG-HUNTING}
  \item Check for previous bugs reports. Use web search engines,
    accessing public mailing list archives.
  \item If you're the first to face the issue, it's very useful for
    others to report it, even if you cannot investigate it further.
  \item If the subsystem you report a bug on has a mailing list, use
    it. Otherwise, contact the official maintainer (see the
    \kfile{MAINTAINERS} file). Always give as many useful details as
    possible.
  \end{itemize}
\end{frame}

\begin{frame}
  \frametitle{How to Become a Kernel Developer?}
  Recommended resources
  \begin{itemize}
  \item See \kerneldoctext{SubmittingPatches} for guidelines
    and \url{http://kernelnewbies.org/UpstreamMerge} for very
    helpful advice to have your changes merged upstream (by Rik van
    Riel).
  \item Watch the \emph{Write and Submit your first Linux kernel
      Patch} talk by Greg. K.H:
    \url{http://www.youtube.com/watch?v=LLBrBBImJt4}
  \item How to Participate in the Linux Community (by Jonathan
    Corbet). A guide to the kernel development process
    \url{http://j.mp/tX2Ld6}
  \end{itemize}
\end{frame}

\begin{frame}
  \frametitle{Contribute to the Linux Kernel (1)}
  \begin{itemize}
  \item Clone Linus Torvalds' tree:
    \begin{itemize}
    \item
      \code{git clone git://git.kernel.org/pub/scm/linux/kernel/git/torvalds/linux.git}
    \end{itemize}
  \item Keep your tree up to date
    \begin{itemize}
    \item \code{git pull}
    \end{itemize}
  \item Look at the master branch and check whether your issue /
    change hasn't been solved / implemented yet. Also check the
    maintainer's git tree and mailing list (see the \kfile{MAINTAINERS}
    file).You may miss submissions that are not in mainline yet.
  \item If the maintainer has its own git tree, create a remote branch
    tracking this tree. This is much better than creating another
    clone (doesn't duplicate common stuff):
    \begin{itemize}
    \item
      \code{git remote add linux-omap git://git.kernel.org/pub/scm/linux/kernel/git/tmlind/linux-omap.git}
    \item \code{git fetch linux-omap}
    \end{itemize}
  \end{itemize}
\end{frame}

\begin{frame}
  \frametitle{Contribute to the Linux Kernel (2)}
  \begin{itemize}
  \item Either create a new branch starting from the current commit in
    the master branch:
    \begin{itemize}
    \item \code{git checkout -b feature}
    \end{itemize}
  \item Or, if more appropriate, create a new branch starting from the
    maintainer's master branch:
    \begin{itemize}
    \item \code{git checkout -b feature linux-omap/master} (remote
      tree / remote branch)
    \end{itemize}
  \item In your new branch, implement your changes.
  \item Test your changes (must at least compile them).
  \item Run \code{git add} to add any new files to the index.
  \end{itemize}
\end{frame}

\begin{frame}
  \frametitle{Configure git send-email}
  \begin{itemize}
  \item Make sure you already have configured your name and e-mail
    address (should be done before the first commit).
    \begin{itemize}
    \item \code{git config --global user.name 'My Name'}
    \item \code{git config --global user.email me@mydomain.net}
    \end{itemize}
  \item Configure your SMTP settings. Example for a Google Mail
    account:
    \begin{itemize}
    \item \code{git config --global sendemail.smtpserver smtp.googlemail.com}
    \item \code{git config --global sendemail.smtpserverport 587}
    \item \code{git config --global sendemail.smtpencryption tls}
    \item \code{git config --global sendemail.smtpuser jdoe@gmail.com}
    \item \code{git config --global sendemail.smtppass xxx}
    \end{itemize}
  \end{itemize}
\end{frame}

\begin{frame}[fragile]
  \frametitle{Contribute to the Linux Kernel (3)}
  \begin{itemize}
  \item Group your changes by sets of logical changes, corresponding
    to the set of patches that you wish to submit.
  \item Commit and sign these groups of changes (signing required by
    Linux developers).
    \begin{itemize}
    \item \code{git commit -s}
    \item Make sure your first description line is a useful summary
      and starts with the name of the modified subsystem. This first
      description line will appear in your e-mails
    \end{itemize}
  \item The easiest way is to look at previous commit summaries on the
    main file you modify
      \begin{itemize}
      \item \code{git log --pretty=oneline <path-to-file>}
      \end{itemize}
  \item Examples subject lines (\code{[PATCH]} omitted):
\begin{verbatim}
Documentation: prctl/seccomp_filter
PCI: release busn when removing bus
ARM: add support for xz kernel decompression
\end{verbatim}
  \end{itemize}
\end{frame}

\begin{frame}
  \frametitle{Contribute to the Linux Kernel (4)}
  \begin{itemize}
  \item Remove previously generated patches
    \begin{itemize}
    \item \code{rm 00*.patch}
    \end{itemize}
  \item Have git generate patches corresponding to your branch
    (assuming it is the current branch)
    \begin{itemize}
    \item If your branch is based on mainline
      \begin{itemize}
      \item \code{git format-patch master}
      \end{itemize}
    \item If your branch is based on a remote branch
      \begin{itemize}
      \item \code{git format-patch <remote>/<branch>}
      \end{itemize}
    \end{itemize}
  \item Make sure your patches pass \code{checkpatch.pl} checks:
    \begin{itemize}
    \item \code{scripts/checkpatch.pl --strict 00*.patch}
    \end{itemize}
  \item Now, send your patches to yourself
    \begin{itemize}
    \item \code{git send-email --compose --to me@mydomain.com 00*.patch}
    \end{itemize}
  \item If you have just one patch, or a trivial patch, you can remove
    the empty line after \code{In-Reply-To:}. This way, you won't add
    a summary e-mail introducing your changes (recommended otherwise).
  \end{itemize}
\end{frame}

\begin{frame}[fragile]
  \frametitle{Contribute to the Linux Kernel (5)}
  \begin{itemize}
  \item Check that you received your e-mail properly, and that it
    looks good.
  \item Now, find the maintainers for your patches
{\scriptsize
\begin{verbatim}
scripts/get_maintainer.pl ~/patches/00*.patch
Russell King <linux@arm.linux.org.uk> (maintainer:ARM PORT)
Nicolas Pitre <nicolas.pitre@linaro.org>
(commit_signer:1/1=100%)
linux-arm-kernel@lists.infradead.org (open list:ARM PORT)
linux-kernel@vger.kernel.org (open list)
\end{verbatim}
}
  \item Now, send your patches to each of these people and lists
    \begin{itemize}
    \item \code{git send-email --compose --to linux@arm.linux.org.uk --to nicolas.pitre@linaro.org --to linux-arm-kernel@lists.infradead.org --to linux-kernel@vger.kernel.org 00*.patch}
    \end{itemize}
  \item Wait for replies about your changes, take the comments into
    account, and resubmit if needed, until your changes are eventually
    accepted.
  \end{itemize}
\end{frame}

\begin{frame}
  \frametitle{Contribute to the Linux Kernel (6)}
  \begin{itemize}
  \item If you use \code{git format-patch} to produce your patches,
    you will need to update your branch and may need to group your
    changes in a different way (one patch per commit).
  \item Here's what we recommend
    \begin{itemize}
    \item Update your master branch
      \begin{itemize}
      \item \code{git checkout master; git pull}
      \end{itemize}
    \item Back to your branch, implement the changes taking community
      feedback into account. Commit these changes.
    \item Still in your branch: reorganize your commits and commit messages
      \begin{itemize}
      \item \code{git rebase --interactive origin/master}
      \item \code{git rebase} allows to rebase (replay) your changes
        starting from the latest commits in master. In interactive
        mode, it also allows you to merge, edit and even reorder
        commits, in an interactive way.
      \end{itemize}
    \item Third, generate the new patches with \code{git format-patch}.
    \end{itemize}
  \end{itemize}
\end{frame}

