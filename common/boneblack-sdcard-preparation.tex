\section{Prepare the SD card}

Our SD card needs to be split in two partitions:

\begin{itemize}

\item A first partition for the bootloader. It needs to comply with
  the requirements of the AM335x SoC so that it can find the bootloader in
  this partition. It should be a FAT32 partition. We will store the
  bootloader (\code{MLO} and \code{u-boot.img}), the kernel image
  (\code{zImage}), the Device Tree (\code{am335x-boneblack.dtb}) and a
  special U-Boot script for the boot.

\item A second partition for the root filesystem. It can use
  whichever filesystem type you want, but for our system, we'll use
  {\em ext4}.

\end{itemize}

First, let's identify under what name your SD card is identified in
your system: look at the output of \code{cat /proc/partitions} and
find your SD card. In general, if you use the internal SD card reader
of a laptop, it will be \code{mmcblk0}, while if you use an external
USB SD card reader, it will be \code{sdX} (i.e\code{sdb}, \code{sdc},
etc.). {\bf Be careful: \code{/dev/sda} is generally the hard drive of
  your machine!}

If your SD card is \code{/dev/mmcblk0}, then the partitions inside the
SD card are named \code{/dev/mmcblk0p1}, \code{/dev/mmcblk0p2}, etc. If your SD
card is \code{/dev/sdc}, then the partitions inside are named
\code{/dev/sdc1}, \code{/dev/sdc2}, etc.

To format your SD card, do the following steps:

\begin{enumerate}

\item Unmount all partitions of your SD card (they are generally
  automatically mounted by Ubuntu)

\item Erase the beginning of the SD card to ensure that the existing
  partitions are not going to be mistakenly detected:\\
  \code{sudo dd if=/dev/zero of=/dev/mmcblk0 bs=1M count=16}. Use
  \code{sdc} or \code{sdb} instead of \code{mmcblk0} if needed.

\item Create the two partitions.

  \begin{itemize}

  \item Start the \code{cfdisk} tool for that:\\
    \code{sudo cfdisk /dev/mmcblk0}.

  \item Chose the {\em dos} partition table type

  \item Create a first small partition (128 MB), primary, with type
    \code{e} ({\em W95 FAT16}) and mark it bootable

  \item Create a second partition, also primary, with the rest of the
    available space, with type \code{83} ({\em Linux}).

  \item Exit \code{cfdisk}

  \end{itemize}

\item Format the first partition as a {\em FAT32} filesystem:\\
  \code{sudo mkfs.vfat -F 32 -n boot /dev/mmcblk0p1}. Use \code{sdc1}
  or \code{sdb1} instead of \code{mmcblk0p1} if needed.

\item Format the second partition as an {\em ext4} filesystem:\\
  \code{sudo mkfs.ext4 -L rootfs -E nodiscard /dev/mmcblk0p2}. Use
  \code{sdc2} or \code{sdb2} instead of \code{mmcblk0p2} if needed.

\begin{itemize}
\item \code{-L} assigns a volume name to the partition
\item \code{-E nodiscard} disables bad block discarding. While this
      should be a useful option for cards with bad blocks, skipping
      this step saves long minutes in SD cards.
\end{itemize}
\end{enumerate}

Remove the SD card and insert it again, the two partitions should be
mounted automatically, in \code{/media/<user>/boot} and
\code{/media/<user>/rootfs}.
