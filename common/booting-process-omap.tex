\begin{frame}
\frametitle{Booting on ARM TI OMAP2+ / AM33xx}
  \begin{columns}
    \column{0.3\textwidth}
    \includegraphics[height=0.8\textheight]{slides/sysdev-bootloaders-sequence/omap-boot.pdf}
    \column{0.7\textwidth}
    \footnotesize
    \begin{itemize}
    \item {\bf ROM Code}: tries to find a valid bootstrap image from
      various storage sources, and load it into SRAM or RAM (RAM can
      be initialized by ROM code through a configuration header). Size
      limited to \textless 64 KB. No user interaction possible.
    \item {\bf X-Loader} or {\bf U-Boot SPL}: runs from SRAM. Initializes
      the DRAM, the NAND or MMC controller, and loads the secondary
      bootloader into RAM and starts it. No user interaction
      possible. File called \code{MLO}.
    \item {\bf U-Boot}: runs from RAM. Initializes some other hardware
      devices (network, USB, etc.).  Loads the kernel image from
      storage or network to RAM and starts it. Shell with commands
      provided. File called \code{u-boot.bin} or \code{u-boot.img}.
    \item {\bf Linux Kernel}: runs from RAM. Takes over the system
      completely (bootloaders no longer exists).
    \end{itemize}
  \end{columns}
\end{frame}

