\documentclass[a4paper,12pt,obeyspaces,spaces,hyphens]{article}

\usepackage{agenda}
\usepackage{colortbl}
\usepackage{xcolor}
\usepackage{calc}

\hypersetup{pdftitle={Formation - Affichage et rendu graphique sous Linux},
  pdfauthor={Bootlin}}

\renewcommand{\arraystretch}{2.0}

\begin{document}

\thispagestyle{fancy}

\setlength{\arrayrulewidth}{0.8pt}

\begin{center}
\LARGE
Formation - Affichage et rendu graphique sous Linux\\
\large
Séminaire de formation en ligne
\end{center}
\vspace{1cm}
\small
\newcolumntype{g}{>{\columncolor{fedarkblue}}m{4cm}}
\newcolumntype{h}{>{\columncolor{felightblue}}X}

\arrayrulecolor{lightgray} {
  \setlist[1]{itemsep=-5pt}
  \begin{tabularx}{\textwidth}{|g|h|}
    {\bf Title} & {\bf Formation - Affichage et rendu graphique sous Linux} \\
    \hline

    {\bf Overview} &
    Réprésentation des images et des couleurs \par
    Dessin de pixels \par
    Opérations sur les pixels \par
    Vue d'ensemble des composants du pipeline et généralités \par
    Spécificités du matériel d'affichage \par
    Spécificités du matériel de rendu \par
    Intégration système, mémoire et performance \par
    Vue d'ensemble du flux d'affichage \par
    Aspects TTY dans le noyau \par
    Périphériques framebuffer dans le noyau \par
    Aspects DRM dans le noyau \par
    Aspects DRM en espace utilisateur \par
    Aspects X Window en espace utilisateur \par
    Wayland en espace utilisateur \par
    Mesa 3D en espace utilisateur\\
    \hline
    {\bf Supports} &
    Vérifiez que le contenu de la formation correspond à vos besoins :
    \newline \url{https://bootlin.com/doc/training/graphics}. \\
    \hline

    {\bf Durée} & {\bf Quatre} demi-journées - 16 h (4 h par demi-journée)
    \newline 80\% de présentations et 20\% de démonstrations. \\
    \hline

    {\bf Formateur} & Un des ingénieurs mentionnés sur :
    \newline \url{https://bootlin.com/training/trainers/}\\
    \hline

    {\bf Langue} & Présentations : Français
    \newline Supports : Anglais\\
    \hline

    {\bf Public visé} & Développeurs de systèmes multimedia utilisant le
    noyau Linux\\
    \hline

    {\bf Pré-requis} &
    {\bf Connaissance du langage C, connaissances de base en mécanismes
    d'interaction avec le matériel (registres, interruptions...),
    en gestion du système au niveau du noyau (comme les mappins en mémoire
    virtuelle) et en appels systèmes depuis l'espace utilisateur (ioctl,
    mmap...).
    Connaissances de bases sur les interfaces matérielles (bus,
    horloges...)}\\
    \hline

  \end{tabularx}

  \begin{tabularx}{\textwidth}{|g|h|}
    {\bf Équipement nécessaire} &
    \begin{itemize}
    \item Ordinateur avec le système d'exploitation de votre choix, équipé du
          navigateur Google Chrome ou Chromium pour la conférence vidéo.
    \item Une webcam et un micro (de préférence un casque avec micro)
    \item Une connexion à Internet à haut débit
    \end{itemize}\\
    \hline

    {\bf Supports} & Version électronique des présentations.\\
    \hline

\end{tabularx}}
\normalsize

\section{Half day 1}

\feagendatwocolumn
{Présentation - Représentation des images et des couleurs}
{
  \begin{itemize}
  \item Lumière, pixels et images
  \item Échantillonage, domaine de fréquence, aliasing
  \item Quantification et représentation des couleurs
  \item Espaces colorimétriques et canaux, canal alpha
  \item Sous-échantillonage YUV et chroma
  \item Plans de données de pixels, ordre d'analyse
  \item Formats de pixels, codes FourCC codes, modificateurs
  \end{itemize}
  \vspace{0.5em}
  {\em Introduction aux notions de base utilisées pour représenter les images en couleur.}
}
{Présentation - Dessin des pixels}
{
  \begin{itemize}
  \item Accès aux données de pixels et intération
  \item Concepts autour de la pixellisation
  \item Dessin de rectangles
  \item Dessin de gradients linéaires
  \item Dessin de disques
  \item Dessin de graduits circulaires
  \item Dessin de lignes
  \item Aliasing de lignes et de formes, dessin sub-pixel
  \item Cercles et coordonnées polaires
  \item Courbes paramétriques
  \end{itemize}
  \vspace{0.5em}
  {\em Comment accéder aux données de pixels en mémoire et dessiner des formes simples.}
}
\\
\feagendatwocolumn
{Présentation - Opérations sur les pixels}
{
  \begin{itemize}
  \item Copie de région
  \item Alpha blending
  \item Keying de couleur
  \item Mise à l'échelle et interpolation
  \item Filtrage linéaire et convolution
  \item Filtres de floutage
  \item Dithering
  \end{itemize}
  \vspace{0.5em}
  {\em Notions de base autour du filtrage, avec des examples d'utilisation très courants.}
}
{Démo - Dessin et opérations}
{
  \begin{itemize}
  \item Exemples de dessin de divers types de formes et de régions
  \item Exemples d'opérations de base sur les pixels
  \end{itemize}
  \vspace{0.5em}
  {\em Illustration des concepts présentés au fur et à mesure.}
}

\section{Half day 2}

\feagendatwocolumn
{Lecture - Pipeline Components Overview and Generalities}
{
  \begin{itemize}
  \item Types of graphics hardware implementations
  \item Graphics memory and buffers
  \item Graphics pipelines
  \item Display, render and video hardware overview
  \end{itemize}
  \vspace{0.5em}
  {\em Presenting the hardware involved in graphics pipelines.}
}
{Lecture - Display hardware}
{
  \begin{itemize}
  \item Visual display technologies: CRT, plasma, LCD, OLED, EPD
  \item Display timings, modes and EDID
  \item DIsplay interfaces: VGA, DVI, HDMI, DP, LVDS, DSI, DP
  \item Bridges and transcoders
  \end{itemize}
  \vspace{0.5em}
  {\em Presenting the inner workings of display hardware.}
}
\\

\feagendatwocolumn
{Lecture - Rendering Hardware Specifics}
{
  \begin{itemize}
  \item Digital Signal Processors (DSPs)
  \item Dedicated hardware accelerators
  \item Graphics Processing Unit (GPUs)
  \end{itemize}
  \vspace{0.5em}
  {\em Describing the architecture of processing and rendering hardware.}
}
{Lecture - System Integration, Memory and Performance}
{
  \begin{itemize}
  \item Graphics integration and memory
  \item Shared graphics memory access
  \item Graphics memory constraints and performance
  \item Offloading graphics to hardware
  \item Graphics performance tips
  \end{itemize}
  \vspace{0.5em}
  {\em Topics related to graphics integration, memory management and performance aspects.}
}

\section{Half day 3}

\feagendatwocolumn
{Lecture - Display Stack Overview}
{
  \begin{itemize}
  \item System-agnostic overview: kernel, display and render userspace
  \item Linux kernel overview
  \item Linux-compatible low-level userspace overview
  \item X Window and Wayland overview
  \item High-level graphics libraries and desktop environments overview
  \end{itemize}
  \vspace{0.5em}
  {\em Presenting what software components are required for modern computer graphics and how they are divided between kernel and userspace.}
}
{Lecture - TTY Kernel Aspects, Framebuffer Device Kernel Aspects}
{
  \begin{itemize}
  \item Linux TTY subsystem introduction
  \item Virtual terminals and graphics
  \item Virtual terminals switching and graphics
  \end{itemize}
  \vspace{0.5em}
  \begin{itemize}
  \item Fbdev overview
  \item Fbdev basic operations
  \item Fbdev limitations
  \end{itemize}
  \vspace{0.5em}
  {\em How TTYs interact with graphics in Linux along with a short presentation of fbdev and why it's deprecated.}
}
\\

\feagendatwocolumn
{Lecture - DRM Kernel Aspects}
{
  \begin{itemize}
  \item DRM devices
  \item DRM driver identification and capabilities
  \item DRM master, magic and authentication
  \item DRM memory management
  \item DRM KMS dumb buffer API
  \item DRM FourCCs and modifiers
  \item DRM KMS resources probing
  \item DRM KMS modes
  \item DRM KMS framebuffer management
  \item DRM KMS legacy configuration and page flipping
  \item DRM event notification
  \item DRM KMS object properties
  \item DRM KMS atomic
  \item DRM render
  \item DRM Prime zero-copy memory sharing (dma-buf)
  \item DRM sync object fencing
  \item DRM debug and documentation
  \end{itemize}
  \vspace{0.5em}
  {\em An exaustive presentation of the DRM interface.}
}
{Demo - Kernel Aspects}
{
  \begin{itemize}
  \item Linux TTY and virtual terminals
  \item DRM KMS mode-setting
  \item DRM KMS driver walkthrough
  \item DRM render driver walkthrough
  \end{itemize}
  \vspace{0.5em}
  {\em Illustrating how kernel aspects work.}
}

\section{Half day 4}

\feagendatwocolumn
{Lecture - X Window Userspace Aspects}
{
  \begin{itemize}
  \item X11 protocol and architecture
  \item X11 protocol extensions
  \item Xorg architecture and acceleration
  \item Xorg drivers overview
  \item X11 and OpenGL acceleration: GLX and DRI2
  \item Xorg usage, integration and configuration
  \item Major issues with X11
  \item Xorg debug and documentation
  \end{itemize}
  \vspace{0.5em}
  {\em Presenting all things related to X11 and Xorg.}
}
{Lecture - Wayland Userspace Aspects}
{
  \begin{itemize}
  \item Wayland overview and paradigm
  \item Wayland protocol and architecture
  \item Wayland core protocol detail
  \item Wayland extra protocols
  \item Wayland asynchronous interface
  \item Wayland OpenGL integration
  \item Wayland status and adoption
  \item Wayland debug and documentation
  \end{itemize}
  \vspace{0.5em}
  {\em An in-depth presentation of Wayland.}
}\\

\feagendatwocolumn
{Lecture - Mesa 3D Userspace Aspects}
{
  \begin{itemize}
  \item Standardized 3D rendering APIs: OpenGL, OpenGL ES, EGL and Vulkan
  \item Mesa 3D overview
  \item Mesa 3D implementation highlights
  \item Mesa 3D internals: Gallium 3D
  \item Mesa 3D internals: intermediate representations
  \item Mesa 3D Generic Buffer Management (GBM)
  \item Mesa 3D hardware support status
  \item Mesa 3D versus proprietary implementations
  \item Mesa 3D hardware support: debug and documentation
  \end{itemize}
  \vspace{0.5em}
  {\em Presenting 3D APIs and the Mesa 3D implementation.}
}
{Demo - Userspace Aspects}
{
  \begin{itemize}
  \item Xorg code walkthrough
  \item Wayland compositor core walkthrough
  \item Wayland client examples
  \item Mesa code walk-through
  \item OpenGL and EGL examples
  \end{itemize}
  \vspace{0.5em}
  {\em Illustrating userspace aspects, client and server implementations.}
}

\feagendaonecolumn
{Questions and Answers}
{
  \begin{itemize}
  \item Questions and answers with the audience about the course topics
  \item Extra presentations if time is left, according what most
        participants are interested in.
  \end{itemize}
}

\end{document}

