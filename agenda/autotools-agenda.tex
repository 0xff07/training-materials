\documentclass[a4paper,12pt,obeyspaces,spaces,hyphens]{article}

\usepackage{agenda}
\usepackage{colortbl}
\usepackage{xcolor}
\usepackage{calc}

\hypersetup{pdftitle={Autotools training},
  pdfauthor={Bootlin}}

\renewcommand{\arraystretch}{2.0}

\begin{document}

\thispagestyle{fancy}

\setlength{\arrayrulewidth}{0.8pt}

\begin{center}
\LARGE
{\em Autotools} training\\
\large
1-day session
\end{center}
\vspace{1cm}

\small
\newcolumntype{g}{>{\columncolor{fedarkblue}}m{4cm}}
\newcolumntype{h}{>{\columncolor{felightblue}}X}

\arrayrulecolor{lightgray} {
  \setlist[1]{itemsep=-5pt}
  \begin{tabularx}{\textwidth}{|g|h|}
    {\bf Title} & {\bf Autotools training} \\
    \hline

    {\bf Overview} &
    Understand the role of the {\em autotools} \par
    Usage of the {\em autotools} \par
    Basics of {\em autoconf} and {\em automake} \par
    Advanced {\em autoconf} topics: configuration header, checking for
    functions, headers and libraries, writing custom tests, handling
    external software and optional features, pkg-config, etc. \par
    Advanced {\em automake} topics: subdirectories, conditionals,
    shared libraries with {\em libtool}, etc. \par
    References to books and tutorials about {\em autotools}\\
    \hline

    {\bf Duration} & {\bf One} day - 8 hours
    \newline 40\% of lectures, 60\% of practical labs. \\
    \hline

    {\bf Trainer} & {\bf Thomas Petazzoni}. Thomas is a major
    Buildroot developer since 2009, an activity through which he has
    gained a good knowledge of {\em autoconf}, {\em automake} and {\em
      libtool}.\\
    \hline

    {\bf Language} & Oral lectures: English, French.
    \newline Materials: English.\\
    \hline

    {\bf Audience} & Companies already using or interested in using
    {\em autotools} to build their software components.\\
    \hline

    {\bf Prerequisites} & {\bf Knowledge of embedded Linux} as covered
    in our embedded Linux course:
    \newline \url{https://bootlin.com/training/embedded-linux/} \vspace{1em}
    \newline {\bf Knowledge and practice of UNIX or GNU/Linux commands}
    \newline People lacking experience on this topic should get
    trained by themselves, for example with our freely available on-line slides:
    \newline \url{https://bootlin.com/blog/command-line/} \\
    \hline
  \end{tabularx}

  \begin{tabularx}{\textwidth}{|g|h|}
    {\bf Required equipment} &
    {\bf For on-site sessions only.}
    \newline Everything is supplied by Bootlin in public
    sessions.
    \begin{itemize}
    \item Video projector
    \item PC computers with at least 8 GB of RAM, and Ubuntu Linux
    installed in a {\bf free partition of at least 20 GB. Using Linux
      in a virtual machine is not supported}, because of issues
    connecting to real hardware.
    \item We need Ubuntu Desktop 18.04 (Xubuntu and other variants
    are fine). We don't support other distributions, because we can't
    test all possible package versions.
    \item {\bf Connection to the Internet} (direct or through the
    company proxy).
    \item {\bf PC computers with valuable data must be backed up}
    before being used in our sessions.  Some people have already made
    mistakes during our sessions and damaged work data.
    \end{itemize}\\
    \hline

    {\bf Materials} & Electronic copies of presentations and
    labs.
    \newline Electronic copy of lab files.\\
    \hline

\end{tabularx}}

\newpage

\normalsize

\section{Day 1 - Morning}

\feagendatwocolumn
{Lecture - Overview and usage of {\em autotools}}
{
  \begin{itemize}
  \item What the {\em autotools} are, what the alternatives are, and
    what they are useful for.
  \item Usage of an existing software component using the {\em
      autotools}: configuring and building the software component.
  \item Standard Makefile targets, filesystem hierarchy, configuration variables
  \item System types: build, host, target
  \item Cross-compilation
  \item Out of tree build
  \item Diverted installation
  \item Cache variables
  \item Using {\em autoreconf}
  \end{itemize}
}
{Lab - Usage of an existing software component using the {\em autotools}}
{
  \begin{itemize}
  \item First build of an {\em autotools} package
  \item Out-of-tree build and cross-compilation
  \item Overriding cache variables
  \item Using {\em autoreconf}
  \end{itemize}
}

\feagendaonecolumn
{Lecture - autoconf/automake: the basics}
{
  \begin{itemize}
  \item \code{configure.ac} language and basic macros
  \item \code{AC_CONFIG_FILES} and {\em output variables}
  \item Minimal \code{Makefile.am}
  \end{itemize}
}

\section{Day 1 - Afternoon}

\feagendaonecolumn
{Lab - autoconf/automake: the basics}
{
  \begin{itemize}
  \item Your first \code{configure.ac}
  \item Adding and building a program
  \item Going further: \code{autoscan} and \code{make dist}
  \end{itemize}
}

\feagendatwocolumn
{Lecture - Autoconf advanced}
{
  \begin{itemize}
  \item Configuration header
  \item Checking for functions, headers, libraries
  \item Custom tests
  \item Handling external software and optional features
  \item \code{pkg-config}
  \end{itemize}
}
{Lecture - Automake advanced}
{
  \begin{itemize}
  \item Subdirectories
  \item Conditionals
  \item Shared libraries
  \item Misc: variables, macro and auxiliarly directories, silent
    rules, etc.
  \end{itemize}
}

\feagendaonecolumn
{Lab - Implement more advanced options}
{
  \begin{itemize}
  \item Use \code{AC_ARG_ENABLE} and \code{config.h}
  \item Implement a shared library
  \item Switch to multiple directories
  \item Make the compilation of programs conditional
  \item Use \code{pkg-config}
  \end{itemize}
}

\end{document}
