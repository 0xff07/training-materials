\documentclass[a4paper,12pt,obeyspaces,spaces,hyphens]{article}

\usepackage{agenda}
\usepackage{colortbl}
\usepackage{xcolor}
\usepackage{palatino}
\usepackage{calc}

\hypersetup{pdftitle={Boot time optimization},
  pdfauthor={Free Electrons}}

\renewcommand{\arraystretch}{2.0}

\begin{document}

\thispagestyle{fancy}

\setlength{\arrayrulewidth}{0.8pt}

\begin{center}
\LARGE
Boot Time Optimization Training\\
\large
1-day session
\end{center}
\vspace{1cm}

\small
\newcolumntype{g}{>{\columncolor{fedarkblue}}m{4cm}}
\newcolumntype{h}{>{\columncolor{felightblue}}X}

\arrayrulecolor{lightgray} {
  \setlist[1]{itemsep=-5pt}
  \begin{tabularx}{\textwidth}{|g|h|}
    {\bf Title} & {\bf Boot Time Optimization Training}\\
    \hline

    {\bf Overview} &
    Measuring boot time \par
    Reducing user space boot time \par
    Reducing kernel boot time \par
    Bootloader optimizations \par
    Advanced techniques and alternatives \par
    Practical labs with ARM boards (SAMA5D3x evaluation kits from Atmel).\\
    \hline
    {\bf Materials} &
    Check that the course contents correspond to your needs:
    \newline \url{http://free-electrons.com/doc/training/boot-time}. \\
    \hline

    {\bf Duration} & {\bf One} day - 8 hours.
    \newline 50\% of lectures, 50\% of practical labs. \\
    \hline

    {\bf Trainer} & One of the engineers listed on
    \newline \url{http://free-electrons.com/training/trainers/}\\
    \hline

    {\bf Language} & Oral lectures: English or French.
    \newline Materials: English.\\
    \hline

    {\bf Audience} & People developing embedded Linux systems.
    \newline People supporting embedded Linux system developers. \\
    \hline

    {\bf Prerequisites} & {\bf Knowledge and practice of Unix or
      GNU/Linux commands}
    \newline People lacking experience on this topic should get
    trained by themselves, for example with our freely available
    on-line slides:
    \newline \url{http://free-electrons.com/docs/command-line/} \vspace{1em}
    \newline {\bf Knowledge and practice of embedded Linux system
    development} \\
    \hline
  \end{tabularx}

  \begin{tabularx}{\textwidth}{|g|h|}
    {\bf Required equipment} &
    {\bf For on-site sessions only.}
    \newline Everything is supplied by Free Electrons in public sessions.
    \begin{itemize}
    \item Video projector
    \item PC computers with at least 2 GB of RAM, and Ubuntu Linux
    installed in a {\bf free partition of at least 20 GB. Using Linux
      in a virtual machine is not supported}, because of issues
    connecting to real hardware.
    \item We need Ubuntu Desktop 12.04 (32 or 64 bit, Xubuntu and
    Kubuntu variants are fine). We don't support other
    distributions, because we can't test all possible package versions.
    \item {\bf Connection to the Internet} (direct or through the
    company proxy).
    \item {\bf PC computers with valuable data must be backed up}
    before being used in our sessions.  Some people have already made
    mistakes during our sessions and damaged work data.
    \end{itemize}\\
    \hline

    {\bf Materials} & Print and electronic copies of presentations and
    labs.
    \newline Electronic copy of lab files.\\
    \hline

\end{tabularx}}
\normalsize


\section{Morning}

\feagendatwocolumn
{Lecture - Principles}
{
  \begin{itemize}
  \item How to measure boot time
  \item Main ideas
  \end{itemize}
}
{Lab - Measuring boot time}
{
 \begin{itemize}
 \item Flashing the board and accessing its serial line
 \item Measure the initial boot time
 \end{itemize}
}

\feagendatwocolumn
{Lecture - Userland optimizations}
{
  \begin{itemize}
  \item Methodology
  \item Using bootchart
  \item Optimize init scripts
  \item Reduce size (C library, compiler optimizations...)
  \item Use an initramfs
  \item Toolchains
  \end{itemize}
}
{Lab - Reducing boot time in user space}
{
 \begin{itemize}
 \item Regenerate the root filesystem with Buildroot
 \item Use bootchart to measure boot time
 \item Simplify user space scripts
 \end{itemize}
}

\section{Afternoon}
\feagendatwocolumn
{Lecture - Kernel optimizations}
{
  \begin{itemize}
  \item Reducing kernel size
  \item Choosing the right compression method
  \item Reducing kernel initialization time
  \end{itemize}
}
{Lab - Reducing kernel boot time}
{
 \begin{itemize}
 \item Recompile the kernel, switching to an initramfs
 \item Use \code{initcall_debug} to find the biggest
       time consumers
 \item Reduce the number of modules
 \item Tune kernel command line parameters
 \end{itemize}
}

\feagendatwocolumn
{Lecture - Bootloader optimizations}
{
  \begin{itemize}
  \item Reducing kernel image loading size
  \item Removing bootloader features
  \item Skipping the bootloader
  \end{itemize}
}
{Lab - Reducing bootloader time}
{
 \begin{itemize}
 \item Reduce boot time by using the Barebox bootloader
 \item Optimize Barebox
 \end{itemize}
}

\feagendaonecolumn
{Lecture - Advanced techniques}
{
  \begin{itemize}
  \item More kernel tweaks
  \item More user space techniques
  \item Using strace
  \item Using oprofile
  \item Alternatives: suspend to RAM, hibernating, checkpointing
  \end{itemize}
}

\end{document}

