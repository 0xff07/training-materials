\subchapter{Locking}{Objective: practice with basic locking primitives}

During this lab, you will:

\begin{itemize}
\item Practice with locking primitives to implement exclusive access
  to the device.
\end{itemize}

\section{Setup}

Stay in the \code{$HOME/felabs/linux/character} directory.

You need to have completed the previous two labs to perform this one.

Boot your board with the same NFS environment as before, and load your
serial module.

\section{Adding appropriate locking}

We have two shared resources in our driver:
\begin{itemize}

\item The buffer that allows to transfer the read data from the
  interrupt handler to the \code{read()} operation.

\item The device itself. It might not be a good idea to mess with the
  device registers at the same time and in two different contexts.

\end{itemize}

Therefore, your job is to add a spinlock to the driver, and use it in
the appropriate locations to prevent concurrent accesses to the shared
buffer and to the device.

Please note that you don't have to prevent two processes from writing
at the same time: this can happen and is a valid behavior. However, if
two processes write data at the same time to the serial port, the
serial controller should not get confused.
