\subchapter{Downloading bootloader, kernel and Buildroot source
code}{Save time and start fetching the source code that you will need
during these labs}

\section{Installing git packages}

We are going to access bootloader, kernel and Buildroot sources through
their \code{git} repositories, which will allow us to track any changes
that we make to the source code of these projects.

Depending on how fast your network connection is (and how many users
share it), fetching such sources is likely to take a significant amount
of time. That's why we're starting such downloads now.

So, let's install the below packages:

\begin{verbatim}
sudo apt install git gitk git-email
\end{verbatim}

\section{Git configuration}

After installing \code{git} on a new machine, the first thing to do is
to let \code{git} know about your name and e-mail address:

\begin{verbatim}
git config --global user.name ’My Name’
git config --global user.email me@mydomain.net
\end{verbatim}

Such information will be stored in commits. It is important
to configure it properly in case we need to generate and
send patches.

\section{Cloning the Buildroot source tree}

Go to the \code{$HOME/boot-time-labs/rootfs} directory.

\begin{verbatim}
git clone git://git.buildroot.net/buildroot
\end{verbatim}

or if you are behind a proxy blocking \code{git}, here's a slower
alternative:

\begin{verbatim}
git clone https://git.buildroot.net/buildroot
\end{verbatim}

If the connection to the Internet turns out to be not fast enough,
your instructor can give you a USB flash drive with a
\code{tar} archive of a recently cloned tree:

\begin{verbatim}
tar xf buildroot-git.tar.xz
cd buildroot
git checkout master
git pull
\end{verbatim}

We will select a particular release tag later. Let's move on to the next
source repository.

\section{Cloning the U-Boot source tree}

Go to the \code{$HOME/boot-time-labs/bootloader} directory.

\begin{verbatim}
git clone git://git.denx.de/u-boot.git
\end{verbatim}

or

\begin{verbatim}
git clone https://git.denx.de/u-boot.git
\end{verbatim}

Similarly, your instructor can give you a pre-downloaded archive if
needed.

\section{Cloning the mainline Linux tree}

Go to the \code{$HOME/boot-time-labs/kernel} directory.

This represents the biggest amout of sources to download, actually more
than 1 GB of data! Again, you can use a pre-downloaded archive if that
turns out to be to much for your actual connection.

{\small
\begin{verbatim}
git clone git://git.kernel.org/pub/scm/linux/kernel/git/torvalds/linux.git
\end{verbatim}
}

or

{\small
\begin{verbatim}
git clone https://git.kernel.org/pub/scm/linux/kernel/git/torvalds/linux.git
\end{verbatim}
}

\section{Accessing stable Linux releases}

Stay in the Linux source directory.

Having the Linux kernel development sources is great, but when you are
creating products, you prefer to avoid working with a target that moves
every day.

That's why we need to use the {\em stable} releases of the Linux
kernel.

Fortunately, with \code{git}, you won't have to clone an entire source
tree again. All you need to do is add a reference to a {\em remote}
tree, and fetch only the commits which are specific to that remote tree.

{\footnotesize
\begin{verbatim}
git remote add stable git://git.kernel.org/pub/scm/linux/kernel/git/stable/linux-stable.git
git fetch stable
\end{verbatim}
}

As this still represents many git objects to download (253 MiB when 4.19 was
the latest version), if you are using an already downloaded git tree,
your instructor will probably have fetched the {\em stable} branch ahead
of time for you too. You can check by running:

\begin{verbatim}
git branch -a
\end{verbatim}

We will choose a particular stable version in the next labs.

Now, let's continue the lectures. This will leave time for the commands
that you typed to complete their execution (if needed).
