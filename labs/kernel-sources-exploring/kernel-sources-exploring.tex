\subchapter{Kernel source code}{Objective: Get familiar with the
  kernel source code}

After this lab, you will be able to:

\begin{itemize}

\item Create a branch based on a remote tree to explore a particular
      stable kernel version (from the \code{stable} kernel tree).
\item Explore the sources in search for files, function headers or
  other kinds of information...
\item Browse the kernel sources with tools like \code{cscope} and LXR.
\end{itemize}

\section{Choose a particular stable version}

Let's work with a particular stable version of the Linux kernel.
It would have been more logical to do this in the previous lab, but we
wanted to get back to lectures while the \code{fetch} command was
running.

First, let's get the list of branches on our \code{stable} remote tree:

\begin{verbatim}
cd ~/felabs/linux/src/linux
git branch -a
\end{verbatim} 

As we want to work with the Linux 3.11 stable branch, the remote branch
we are interested in is \code{remotes/stable/linux-3.11.y}.

First, open the \code{Makefile} file just to check the Linux kernel
version that you currently have.

Now, let's create a local branch starting from that remote branch:
\begin{verbatim}
git checkout -b 3.11.y stable/linux-3.11.y
\end{verbatim} 

Open \code{Makefile} again and make sure you now have a 3.11.y version. 

\section{Exploring the sources manually}

As a Linux kernel user, you will very often need to find which file
implements a given function. So, it is useful to be familiar with
exploring the kernel sources.

\begin{enumerate}
\item Find the Linux logo image in the sources
\item Find who the maintainer of the 3C505 network driver is.
\item Find the declaration of the \code{platform_device_register()} function.
\end{enumerate}

Tip: if you need the \code{grep} command, we advise you to use \code{git
grep}. This command is similar, but much faster, doing the search only
on the files managed by git (ignoring git internal files and generated
files). 

\section{Use a kernel source indexing tool}

Now that you know how to do things in a manual way, let's use more
automated tools.

Try LXR (Linux Cross Reference) at \url{http://lxr.free-electrons.com}
and choose the Linux version closest to yours.

If you don't have Internet access, you can use \code{cscope} instead.

As in the previous section, use this tool to find where
the \code{platform_device_register()} is declared, implemented and
even used.
