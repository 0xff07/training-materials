\subchapter{Kernel source code}{Objective: Get familiar with the
  kernel source code}

After this lab, you will be able to:

\begin{itemize}

\item Explore the sources in search for files, function headers or
  other kinds of information...
\item Browse the kernel sources with tools like \code{cscope} and LXR.
\end{itemize}

\section{Setup}

Stay in the \code{$HOME/felabs/linux/modules} directory.

\section{Exploring the sources manually}

As a Linux kernel user, you will very often need to find which file
implements a given function. So, it is useful to be familiar with
exploring the kernel sources.

\begin{enumerate}
\item Find the Linux logo image in the sources
\item Find who the maintainer of the 3C505 network driver is.
\item Find the declaration of the \code{platform_device_register()} function.
\end{enumerate}

Tip: if you need the \code{grep} command, we advise you to use \code{git
grep}. This command is similar, but much faster, doing the search only
on the files managed by git (ignoring git internal files and generated
files). 

\section{Use a kernel source indexing tool}

Now that you know how to do things in a manual way, let's use more
automated tools.

Try LXR (Linux Cross Reference) at \url{http://lxr.free-electrons.com}
and choose the Linux version closest to yours.

If you don't have Internet access, you can use \code{cscope} instead.

As in the previous section, use this tool to find where
the \code{platform_device_register()} is declared, implemented and
even used.
