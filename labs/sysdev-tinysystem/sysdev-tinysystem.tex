\subchapter{Tiny embedded system with BusyBox}{Objective: making a tiny yet full featured embedded system}

After this lab, you will:
\begin{itemize}
\item be able to configure and build a Linux kernel that boots on a
  directory on your workstation, shared through the network by NFS.
\item be able to create and configure a minimalistic root filesystem
  from scratch (ex nihilo, out of nothing, entirely hand made...) for
  the IGEP board
\item understand how small and simple an embedded Linux system can be.
\item be able to install BusyBox on this filesystem.
\item be able to create a simple startup script based on /sbin/init.
\item be able to set up a simple web interface for the target.
\item have an idea of how much RAM a Linux kernel smaller than 1 MB needs.
\end{itemize}

\section{Lab implementation}

While (s)he develops a root filesystem for a device, a developer needs
to make frequent changes to the filesystem contents, like modifying
scripts or adding newly compiled programs.

It isn't practical at all to reflash the root filesystem on the target
every time a change is made. Fortunately, it is possible to set up
networking between the development workstation and the target. Then,
workstation files can be accessed by the target through the network,
using NFS.

Unless you test a boot sequence, you no longer need to reboot the
target to test the impact of script or application updates.

\begin{center}
\includegraphics[width=\textwidth]{labs/sysdev-tinysystem/host-vs-target.pdf}
\end{center}

\section{Setup}

Go to the \code{$HOME/felabs/sysdev/tinysystem/} directory.

\section{Kernel configuration}

We will re-use the kernel sources from our previous lab, in
\code{$HOME/felabs/sysdev/kernel/}.

In the kernel configuration built in the previous lab, verify that you
have all options needed for booting the system using a root filesystem
mounted over NFS, and if necessary, enable them and rebuild your
kernel.

\section{Setting up the NFS server}

Create a \code{nfsroot} directory in the current lab directory. This
\code{nfsroot} directory will be used to store the contents of our new
root filesystem.

Install the NFS server by installing the \code{nfs-kernel-server}
package if you don't have it yet. Once installed, edit the
\code{/etc/exports} file as root to add the following line, assuming that the
IP address of your board will be \code{192.168.0.100}:

\footnotesize
\begin{verbatim}
/home/<user>/felabs/sysdev/tinysystem/nfsroot 192.168.0.100(rw,no_root_squash,no_subtree_check)
\end{verbatim}
\normalsize

Make sure that the path and the options are on the same line.
Also make sure that there is no space between the IP address and the NFS
options, otherwise default options will be used for this IP address,
causing your root filesystem to be read-only.

Then, restart the NFS server:

\begin{verbatim}
sudo /etc/init.d/nfs-kernel-server restart
\end{verbatim}

\section{Booting the system}

First, boot the board to the U-Boot prompt. Before booting the kernel,
we need to tell it that the root filesystem should be mounted over
NFS, by setting some kernel parameters.

Use the following U-Boot command to do so, in just 1 line
(Caution: in \code{ttyO2} below, it's the capital letter \code{O}, like in
{\bf O}MAP and not the number zero):

\begin{verbatim}
setenv bootargs console=ttyO2,115200 root=/dev/nfs ip=192.168.0.100 \
   nfsroot=192.168.0.1:/home/<user>/felabs/sysdev/tinysystem/nfsroot
\end{verbatim}

Of course, you need to adapt the IP addresses to your exact network
setup. Save the environment variables (with \code{saveenv}).

Now, boot your system. The kernel should be able to mount the root
filesystem over NFS.

If the kernel fails to mount the NFS filesystem, look carefully at the
error messages in the console. If this doesn't give any clue, you can
also have a look at the NFS server logs in \code{/var/log/syslog}.

However, the kernel will complain that it can't find an init
application:

\footnotesize
\begin{verbatim}
Kernel panic - not syncing: No init found.  Try passing init= option to kernel.
  See Linux Documentation/init.txt for guidance.
\end{verbatim}
\normalsize

Obviously, our root filesystem being empty, there isn't such an
application. Follow on to add Busybox to our root filesystem and
finally make it useful.

\section{Root filesystem with Busybox}

Download the latest BusyBox 1.21.x release.

To configure BusyBox, we won't be able to use \code{make xconfig},
which is currently broken in Ubuntu 12.04, because of Qt library
dependencies.

We are going to use \code{make gconfig} this time. Before doing this,
install the required packages:

\begin{verbatim}
sudo apt-get install libglade2-dev
\end{verbatim}

Now, configure BusyBox with the configuration file provided in the
\code{data/} directory (remember that the Busybox configuration file
is \code{.config} in the Busybox sources).

At least, make sure you build BusyBox statically! Compiling Busybox
statically in the first place makes it easy to set up the system,
because there are no dependencies on libraries. Later on, we will set
up shared libraries and recompile Busybox.

Build BusyBox using the toolchain that you used to build the kernel.

Going back to the BusyBox configuration interface
specify the installation directory for BusyBox
\footnote{You will find this setting in
\code{Install Options -> BusyBox installation prefix}.}.
It should be the path to your \code{nfsroot} directory.

Now run \code{make install} to install BusyBox in this directory.

Try to boot your new system on the board. If everything goes right,
the kernel should again confirm that it managed to mount the NFS root
filesystem. Then, you should get errors about missing \code{/dev/ttyX}
files. This is because the kernel didn't manage to mount the
\code{devtmpfs} filesystem because the \code{/dev} directory doesn't
exist yet. Create the \code{dev} directory and reboot. The errors should
go away. 

At the end, you will access a console and will be able to issue
commands through the default shell.

\section{Virtual filesystems}

Run the \code{ps} command. You can see that it complains that the
\code{/proc} directory does not exist. The ps command and other
process-related commands use the \code{proc} virtual filesystem to get
their information from the kernel.

From the Linux command line in the target, create the \code{proc}, \code{sys} and
\code{etc} directories in your root filesystem.

Now mount the \code{proc} virtual filesystem. Now that \code{/proc} is
available, test again the \code{ps} command.

Note that you can also halt your target in a clean way with the \code{halt}
command, thanks to \code{proc} being mounted.

\section{System configuration and startup}

The first userspace program that gets executed by the kernel is
\code{/sbin/init} and its configuration file is \code{/etc/inittab}.

In the BusyBox sources, read details about \code{/etc/inittab} in the
\code{examples/inittab} file.

Then, create a \code{/etc/inittab} file and a \code{/etc/init.d/rcS}
startup script declared in \code{/etc/inittab}. In this startup
script, mount the \code{/proc} and \code{/sys} filesystems.

Any issue after doing this?

\section{Switching to shared libraries}

Take the \code{hello.c} program supplied in the lab \code{data}
directory. Cross-compile it for ARM, dynamically-linked with the
libraries, and run it on the target.

You will first encounter a not found error caused by the absence of
the \code{ld-uClibc.so.0} executable, which is the dynamic linker
required to execute any program compiled with shared libraries. Using
the find command (see examples in your command memento sheet), look
for this file in the toolchain install directory, and copy it to the
\code{lib/} directory on the target.

Then, running the executable again and see that the loader executes
and finds out which shared libraries are missing. Similarly, find
these libraries in the toolchain and copy them to \code{lib/} on the
target.

Once the small test program works, we are going to recompile Busybox
without the static compilation option, so that Busybox takes advantages of the
shared libraries that are now present on the target.

Before doing that, measure the size of the \code{busybox} executable.

Then, build Busybox with shared libraries, and install it again on the
target filesystem. Make sure that the system still boots and see
how much smaller the \code{busybox} executable got.


\section{Implement a web interface for your device}

Replicate \code{data/www/} to the \code{/www} directory in your target root filesystem.

Now, run the BusyBox http server from the target command line:

\begin{verbatim}
/usr/sbin/httpd -h /www/
\end{verbatim}

It will automatically background itself.

If you use a proxy, configure your host browser so that it doesn't go
through the proxy to connect to the target IP address, or simply
disable proxy usage.  Now, test that your web interface works well by
opening \code{http://192.168.0.100} on the host.

See how the dynamic pages are implemented. Very simple, isn't it?

\section{How much RAM does your system need?}

Check the \code{/proc/meminfo} file and see how much RAM is used by your
system.

You can try to boot your system with less memory, and see whether it
still works properly or not. For example, to test whether 20 MB are
enough, boot the kernel with the \code{mem=20M} parameter. Linux will then
use just 20 MB of RAM, and ignore the rest.

Try to use even less RAM, and see what happens.
