\subchapter{Bootloader optimizations}{Reduce bootloader execution time}

In this lab, we will run the final stage of boot time reduction:
\begin{itemize}
\item Improving the efficiency of the bootloader by optimizing its
usage
\item Recompiling the bootloader with the minimum set of options,
and try to completely skip the second stage of the bootloader.
\end{itemize}

\section{Optimizing U-Boot usage}

By following the indications given in the lectures, start by optimizing
the way U-Boot is used.

At last, you can start by eliminating the infamous 2-second boot delay, something
you've surely been longing to do.

\section{Recompiling the bootloader}

It's now time to eliminate useless features in U-Boot. Go to
\code{~/boot-time-labs/bootloader/u-boot/} and run \code{make
menuconfig} to unselect features that we don't need in our system.

For the moment, don't touch the \code{SPL / TPL} options, as we will try
to use U-Boot's Falcon mode at the end.

In the same way you did when you reduced the kernel configuration,
do the changes {\bf progressively}, and even make backup copies of your
intermediate configurations (\code{.config} file). You will be glad you
did when you break U-Boot.

Once you have reached the minimum set of features, please measure boot
time and fill the below table:

\begin{tabular}{| l | l | r |}
  \hline
  Step & Duration & Description \\
  \hline
  \hline
  U-Boot SPL & & Between \code{U-Boot SPL 2019.01} and \code{U-Boot 2019.01} \\
  \hline
  U-Boot & & Between \code{U-Boot 2019.01} and \code{Starting kernel} \\
  \hline
  Kernel + Init scripts & & Between \code{Starting kernel} and \code{Starting ffmpeg} \\
  \hline
  Application & & Between \code{Starting ffmpeg} and \code{First frame decoded} \\
  \hline
  \hline
  Total & & \\
  \hline
\end{tabular}

\section{Using faster storage}

A last minute surprise: your instructor will give you new SD cards with
faster read performance, at least as fast as the Beagle Bone Black seems
to be able to go.

Why on earth didn't we use such SD cards right from the start of our
labs?

It's because slower storage acts as a magnifying glass (or as a slow
motion device) making it easier to observe elapsed time and the benefits
of our optimizations. If the storage was lightning fast, it would be
harder to appreciate speedups due to a small initramfs, for example.

So, edit the partition table of your new SD card, and create the
first partition in the same way as when you prepared your original SD
card. Then, copy the files over.

You can now go ahead and make tests again, and fill the table with your
latest results:

\begin{tabular}{| l | l | r |}
  \hline
  Step & Duration & Description \\
  \hline
  \hline
  U-Boot SPL & & Between \code{U-Boot SPL 2019.01} and \code{U-Boot 2019.01} \\
  \hline
  U-Boot & & Between \code{U-Boot 2019.01} and \code{Starting kernel} \\
  \hline
  Kernel + Init scripts & & Between \code{Starting kernel} and \code{Starting ffmpeg} \\
  \hline
  Application & & Between \code{Starting ffmpeg} and \code{First frame decoded} \\
  \hline
  \hline
  Total & & \\
  \hline
\end{tabular}

\section{Using U-Boot's Falcon mode}

It's now time to try U-Boot's capability to directly load the
Linux kernel from its first stage (SPL), instead of loading U-Boot.

Your instructor will give you separate guidelines.

When it works, update your table again:

\begin{tabular}{| l | l | r |}
  \hline
  Step & Duration & Description \\
  \hline
  \hline
  U-Boot SPL & & Between \code{U-Boot SPL 2019.01} and \code{Starting kernel} \\
  \hline
  Kernel + Init scripts & & Between \code{Starting kernel} and \code{Starting ffmpeg} \\
  \hline
  Application & & Between \code{Starting ffmpeg} and \code{First frame decoded} \\
  \hline
  \hline
  Total & & \\
  \hline
\end{tabular}

\section{Going further}

There are several things we can do to try to further optimize things:

\begin{itemize}
\item As our storage is now faster, it can be interesting to explore the
various kernel compression schemes again. The optimum solution may be a
different one.
\item Look for a solution to eliminate the delay detecting the USB
webcam.
\item If you don't manage to get rid of this delay, at least take
advantage of this spare time to show signs of life on the screen, by
implementing a splashscreen. You can even implement an animation.
One thing you can do is use BusyBox's \code{fbsplash} tool, to first
show an image on the framebuffer, and then even show a progress bar
(knowing how much time you have to wait for the camera to be ready).
\end{itemize}

