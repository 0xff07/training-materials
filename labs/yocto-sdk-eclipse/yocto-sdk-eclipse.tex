\subchapter{Use the Yocto Project SDK through Eclipse}{Build and use
the Yocto Project SDK with Eclipse}

During this lab, you will:
\begin{itemize}
  \item Integrate the Eclipse Yocto Project plugin
  \item Configure the plugin to work with the previously used Yocto project
  \item Develop and modify Poky from Eclipse
\end{itemize}

\section{Set up the environment}

First we need to set up the environment in order to be able to develop our
applications. We need Poky to build support for Eclipse (an IDE). Run:

\begin{verbatim}
bitbake meta-ide-support
\end{verbatim}

\section{Set the default shell on Ubuntu}

Ubuntu uses Dash as its default shell. However the Yocto Eclipse
plugin needs to use Bash. To change the default shell, execute the
following command as root:
\begin{verbatim}
dpkg-reconfigure dash
\end{verbatim}

\section{Download Eclipse}

Download the Luna version of Eclipse on the official website:
\url{http://www.eclipse.org/downloads/download.php?file=/technology/epp/downloads/release/luna/SR2/eclipse-cpp-luna-SR2-linux-gtk-x86_64.tar.gz}.
Then uncompress the tarball and launch Eclipse. Ubuntu has a known bug, and
you need to run the following command from the extracted Eclipse directory to be
able to use it properly:
\begin{verbatim}
UBUNTU_MENUPROXY=0 ./eclipse
\end{verbatim}

\section{Install the Eclipse plugin}

First, you need to download a few packages to fulfill the Yocto
Eclipse plugin requirements. Open \code{Install New Software} in the
\code{Help} menu. Select the \code{Luna - http://download.eclipse.org/releases/luna}
 repository and install:
\begin{itemize}
  \item Linux Tools LTTng Tracer Control
  \item Linux Tools LTTng Userspace Analysis
  \item LTTng Kernel Analysis
  \item C/C++ Remote (over TCF/TE) Run/Debug Launcher
  \item Remote System Explorer End-user Runtime
  \item Remote System Explorer User Actions
  \item Target Management Terminal (Core SDK)
  \item TCF Remote System Explorer add-in
  \item TCF Target Explorer
\end{itemize}

If the selections do not appear in the menu, it means it is already installed.

You can then either choose to download the already built Eclipse Yocto Project
plugin or to build your own by first downloading its source repository. We will
here download the latest plugin available on the Yocto Project website directly
from Eclipse.

First add the Yocto Project Eclipse Update site to the available software sites:
\url{http://downloads.yoctoproject.org/releases/eclipse-plugin/2.1/luna/}.
Then download:
\begin{itemize}
  \item Yocto Project ADT Plug-in
  \item Yocto Project BitBake Commander Plug-in
  \item Yocto Project Documentation plug-in
\end{itemize}

Finally, you need to set up the plugin itself. Open \code{Preferences} from the
"Window" menu. Then click on \code{Yocto Project ADT} in the preference dialog.
Choose \code{Build System Derived Toolchain}, set the \code{Toolchain Root
Location} to your build directory and the \code{Sysroot Location} to the
\code{tmp/sysroot/beaglebone} directory of your build directory. Then select the
\code{Target Architecture}. Apply and close, the plugin is set up!

%\section{Build a Poky image from Eclipse} % [Mylene] titre modifié
\section{Create a project under Eclipse}

The Eclipse Yocto Project plugin allows, in addition to compiling an application
to the right target architecture, to build a Poky image. We will here configure
Eclipse to build our exact previous BeagleBone image, but it is also possible to
create a build environment from scratch.

Select \code{New} in the \code{File} menu. Then double click on \code{New Yocto
Project} in the \code{Yocto Project Bitbake Commander} category. In the new
window, the directory used by Eclipse will be \code{Location/Project
name}, you need to fill the form with this in mind: use the path of the
directory containing your previously used \code{poky} root directory and put
the name of this directory as the project name. Then click on \code{Finish}.
The new project is now visible in the \code{Project Explorer} panel.

% [Mylene] Est-ce qu'on continue à montrer le build en utilisant toaster ?
% J'ajoute comment installer Toaster mais en commentaires seulement

%In order to build a target image, you need to launch Toaster, the web
%interface for BitBake.
%Toaster is using a web serveur that need to be executed. To do that, you must
%installed the toaster's dependencies:
%\begin{itemize}
%  \item Install \code{pip} tool:\\
%    \code{$ sudo apt install python-pip}
%  \item Install Toaster's dependencies:\\
%    \code{$ sudo pip install -r bitbake/toaster-requirements.txt}
%  \item Go under your build folder:\\
%    \code{$ source oe-init-build-env}
%  \item Start toaster:\\
%    \code{$ source toaster start}
%\end{itemize}
%Once Toaster has been launched, you will be able to use the web interface.
%On a web browser, go to toaster: \url{http://localhost:8000}

%On Eclipse, you can find this tool under the \code{Project} menu, \code{Launch Toaster}.
%It will ask you the server URL and use the above localhost's one.
%The web browser should open the Toaster main page.

%[Mylene] TODO: Ajouter comment configurer toaster pour construire une image.
% Il y a quelques étapes à suivre donc je ne suis pas sûr que ça soit bien de
% garder le build dans Eclipse. Peut-être seulement faire la création du projet
% + création d'une recette ?
% Si on veut garder le build via Eclipse (et donc Toaster), il faudrait prévoir
% une section spéciale qui indique les pas à suivre, sachant qu'il faut tout
% reconfigurer l'environement de build en utilisant l'interface Toaster.
% Personnelement, je ne trouve pas ça très intéressant pour l'utilisateur.

\section{Create a recipe}

The Yocto Project BitBake Commander Plug-in allows to fully manage the Yocto Project
from Eclipse, including creating a new recipe with a user friendly wizard. To
demonstrate this ability, we will create a new recipe for the wonderful Steam
Locomotive command (\code{sl}). The home page is located at:
\code{https://github.com/mtoyoda/sl}.

Select the \code{File → New → Other} menu. Double click on the \code{BitBake
Recipe} under the \code{Yocto Project BitBake Commander} category. Now you can
create the recipe by filling the form according to the previous labs we
followed.
