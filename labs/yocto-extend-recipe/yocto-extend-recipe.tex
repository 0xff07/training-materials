\subchapter{Extend a recipe}{Add your features to an existing recipe}

During this lab, you will:
\begin{itemize}
  \item Apply patches to an existing recipe
  \item Use a custom configuration file for an existing recipe
  \item Extend a recipe to fit your needs
\end{itemize}

\section{Create a basic appended recipe}

To avoid rewriting recipes when a modification is needed on an already existing
package, BitBake allows to extend recipes and to overwrite, append or prepend
configuration variables values through the so-called BitBake append files.

We will first create a basic BitBake append file, without any change made
to the original recipe, to see how it is integrated into the build. We will then
extend some configuration variables of the original recipe.

Try to create an appended recipe with the help of the online Yocto
Project development documentation. You can find it at
\url{http://www.yoctoproject.org/docs/1.4.2/dev-manual/dev-manual.html}. We here
aim to extend the \code{linux-ti-staging} kernel recipe.

Common errors are related to:
\begin{itemize}
  \item Wrong recipe path
  \item Wrong recipe revision (\code{PR}). BitBake append files use a special
  variable to increment the version dynamically, if explicitly used.
\end{itemize}

You can see available \code{bbappend} files and the recipe they apply to by
using the \code{bitbake-layers} tool (again!):
\begin{verbatim}
bitbake-layers show-appends
\end{verbatim}

If the BitBake append file you just created is recognized by your Yocto
environment, you should see:
\begin{verbatim}
linux-ti-staging_3.12.bb:
  $POKY/meta-felabs/recipes-kernel/linux/linux-ti-staging_3.12.bbappend
\end{verbatim}

\section{Add patches to apply in the recipe}

We want our extended \code{linux-ti-staging} kernel to support the Nunchuk as
a joystick input. We can add this by applying a patch during the
\code{do_configure} task. The needed patches are provided with this lab. You can
find them under \code{~/yocto-labs/nunchuck/linux}. For more details about
how to write the driver handling the Nunchuk, have a look on our embedded Linux
kernel and driver development training course at
\url{http://free-electrons.com/training/kernel/}.

Applying a patch is a common task in the daily Yocto process. Many recipes,
appended or not, apply a specific patch on top of a mainline project. It's why
patches do not have to be explicitly applied, if the recipe inherits from the
patch class (directly or not), but only have to be present in the source files
list.

Try adding the patches included in this lab to your BitBake append file.

You can now clean the working directory of the \code{linux-ti-staging} package
and start a fresh build to take the new patches into account:
\begin{verbatim}
bitbake -c cleanall virtual/kernel
bitbake virtual/kernel
\end{verbatim}

This method is the common one when surcharging recipes.

\section{Test the Nunchuk}

Copy the newly generated kernel and device tree images into the first SD card
partition. Then boot the board and wait until you have access to the
\code{busybox} command line.

You can then make sure that the Nunchuk is recognized and is working by
checking the presence of the \code{js0} device file:
\begin{verbatim}
ls /dev/input/js0
\end{verbatim}

Now display the raw events generated by the Nunchuk:
\begin{verbatim}
cat /dev/input/js0
\end{verbatim}

You should see random characters appearing while playing with the Nunchuk. Be
aware that the driver we integrated also handles accelerometer events. Therefore,
moving the device will produce many events!

\section{Play nInvaders!}

The nInvaders game uses keyboard events for its controls. We first need to apply
a patch introducing joystick support. The patch is located at
\code{~/yocto-labs/nunchuck/ninvaders/}.

Add the patch to the nInvaders \code{SRC_URI} and do not forget you need to
specify where it is located.

Then build a full \code{core-image-minimal} and update the NFS root directory
it. After booting the board you should be able to play nInvaders with the
keyboard \dots and the Nunchuk! The \code{C} button is used to confirm and to
fire, and \code{Z} to pause the game.

\section{Going further: use the Yocto Kernel metadata}

When dealing with some kernel recipes, it can be possible to use a more advanced
way to add patches and to manage the configuration. This can be achieved with
\code{linux-yocto} packages. These packages can be found under
\code{meta/recipes-kernel/linux/} and provide the official generic Yocto kernel.
They can be useful as a starting point to create a custom kernel without doing
everything from scratch. You can also create a very modular recipe by doing so.

Read the Yocto Project's kernel documentation at
\url{http://www.yoctoproject.org/docs/current/kernel-dev/kernel-dev.html} and
find how to append a custom feature to an existing kernel recipe for an
explicit architecture. Then write a BitBake append file extending the
\code{linux-yocto_3.10} recipe.
