\subchapter{Create a custom machine configuration}{Let Poky know about your
  hardware!}

During this lab, you will:
\begin{itemize}
  \item Create a custom machine configuration
  \item Understand how the target architecture is dynamically chosen
\end{itemize}

\section{Create a custom machine}

The machine file configures various hardware related settings. As early as in
lab1, we chose the \code{beaglebone} one. While it is not necessary to
make our custom machine image here, we'll create a new one to demonstrate the
process.

Add a new \code{felabs} machine to the previously created layer. In order to
properly boot on the BeagleBone Black board. Since the
\code{meta-felabs} layer is already created, do not use the
\code{yocto-bsp} tool.

This machine describes a board using the \code{cortexa8thf-neon} tune
and is a part of the \code{ti33x} SoC family. Add the following lines
to your machine configuration file:

\begin{verbatim}
SOC_FAMILY = "ti33x"
require conf/machine/include/soc-family.inc

DEFAULTTUNE ?= "cortexa8thf-neon"
require conf/machine/include/tune-cortexa8.inc

UBOOT_ARCH = "arm"
UBOOT_MACHINE = "am335x_boneblack_config"
UBOOT_ENTRYPOINT = "0x80008000"
UBOOT_LOADADDRESS = "0x80008000"
\end{verbatim}

\section{Populate the machine configuration}

This \code{felabs} machine needs:

\begin{itemize}
  \item To select \code{linux-ti-staging} as the preferred provider
    for the kernel.
  \item To use \code{am335x-boneblack.dtb} device tree.
  \item To select \code{u-boot-ti-staging} as the preferred provider
    for the bootloader.
  \item To use a \code{zImage} kernel image type.
  \item To configure the serial console to \code{115200 ttyO0}
  \item And to support some features:
    \begin{itemize}
      \item kernel26
      \item apm
      \item usbgadget
      \item usbhost
      \item vfat
      \item ext2
      \item ethernet
    \end{itemize}
\end{itemize}

\section{Build an image with the new machine}

You can now update the \code{MACHINE} variable value in the local configuration
and start a fresh build.

\section{Have a look on the generated files}

Once the generated images supporting the new \code{felabs} machine are
generated, you can check all the needed images were generated
correctly.

Have a look in the output directory, in
\code{$BUILDDIR/tmp/deploy/images/felabs/}. Is there something
missing?

\section{Update the rootfs}

You can now update your root filesystem, to use the newly
generated image supporting our \code{felabs} machine!
