\subchapter{Create a custom machine configuration}{Let Poky know about your
  hardware!}

During this lab, you will:
\begin{itemize}
  \item Create a custom machine configuration
  \item Understand how the target architecture is dynamically chosen
\end{itemize}

\section{Create a custom machine}

The machine file configures various hardware related settings. As early as in
lab1, we chose the \code{beaglebone} one. While it is not necessary to
make our custom machine image here, we'll create a new one to demonstrate the
process.

Add a new \code{felabs} machine to the previously created layer. In order to
properly boot on the BeagleBone Black board, this machine configuration has to
use the configuration defined in the \code{ti33x} file. Also open it to see how
that the \code{PREFERRED_PROVIDERS} are defined here.

Then add the machine's features you need to be supported, configure the
serial parameters and the file system type to have a \code{tar.gz}
archive.

You can now update the \code{MACHINE} variable value in the local configuration
and start a fresh build.

\section{}
