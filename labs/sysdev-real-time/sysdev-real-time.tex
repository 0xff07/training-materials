\subchapter{Real-time - Timers and scheduling latency}{Objective:
  Learn how to handle real-time processes and practice with the
  different real-time modes. Measure scheduling latency.}

After this lab, you will:
\begin{itemize}
\item Be able to check clock accuracy.
\item Be able to start processes with real-time priority.
\item Be able to build a real-time application against the standard
POSIX real-time API, and against Xenomai's POSIX skin.
\item Have compared scheduling latency on your system, between a standard kernel and a kernel with Xenomai.
\end{itemize}

\section{Setup}

Go to the \code{/home/<user>/felabs/realtime/rttest} directory.

Install the \code{netcat} package.

\section{Root filesystem}

Create an \code{nfsroot} directory.

To compare real-time latency between standard Linux and Xenomai, we
are going to need a root filesystem and a build environment that
supports Xenomai.

Let's build this with Buildroot.

Download and extract the Buildroot 2012.05 sources.

Configure Buildroot with the following settings, using the \code{/}
command in \code{make menuconfig} to find parameters by their name:

\begin{itemize}
\item \code{Target architecture}: \code{ARM (little endian)} 
\item \code{Target Architecture Variant}: \code{generic_arm}
\item In \code{Build options}:
   \begin{itemize}
   \item \code{Number of jobs to run simultaneously}: \code{4}
   \end{itemize}
\item In \code{Toolchain}:
   \begin{itemize}
   \item \code{Toolchain type}: \code{External toolchain}
   \item \code{Toolchain}: \code{Sourcery CodeBench ARM 2011.09}
   \end{itemize}
\item In \code{System configuration}: 
   \begin{itemize}
   \item \code{/dev management}: \code{Dynamic using devtmpfs only}
   \item \code{Port to run a getty (login prompt) on}: \code{ttyO2}
   \end{itemize}
\item In \code{Package Selection for the target}:
   \begin{itemize}
   \item Enable \code{Show packages that are also provided by busybox}.
         We need this to build the standard \code{netcat} command, not
         provided in the default BusyBox configuration.
   \item In \code{Debugging, profiling and benchmark}, enable
         \code{rt-tests}. This will be a few applications to test
         real-time latency.
   \item In \code{Networking applications}, enable \code{netcat}
   \item In \code{Real-Time}, enable \code{Xenomai Userspace}:
         \begin{itemize}
         \item \code{Custom Xenomai version}: \code{2.6.0}
         \item Enable \code{Install testsuite}
	 \item Make sure that \code{POSIX skin library} is enabled.
  	 \end{itemize}
   \end{itemize}
\end{itemize}

Now, build your root filesystem.

At the end of the build job, extract the
\code{output/images/rootfs.tar} archive in the \code{nfsroot}
directory.

The last thing to do is to add a few files that we will need 
in our tests:

\begin{verbatim}
cp data/* nfsroot/root
\end{verbatim}

\section{Compile a standard Linux kernel}

Download the exact Linux 2.6.38.8 version. That's the most recent
ARM Linux version that Xenomai 2.6.0 supports. You will have trouble
applying Xenomai kernel patches otherwise.

Configure your kernel with the default configuration for the IGEPv2
board.

In the kernel configuration interface:
\begin{itemize}
\item Enable \code{CONFIG_DEVTMPFS} and \code{CONFIG_DEVTMPFS_MOUNT}
      The root filesystem that we use has an empty \code{/dev}
      directory, and we let the kernel populate it with the devices
      present on the system.
\item For the moment, remove \code{CONFIG_HIGH_RES_TIMERS},
      to start by testing the kernel without high-resolution timers.
\end{itemize}

Boot the IGEP board by mounting the root filesystem that you built.
As usual, login as \code{root}, there is no password.

\section{Compiling with the POSIX RT library}

The root filesystem was built with the GNU C library, because it has
better support for the POSIX RT API.

In our case, when we created this lab, uClibc
didn't support the \code{clock_nanosleep} function used in our
\code{rttest.c} program. {\em uClibc} also does not support priority
inheritance on mutexes.

Therefore, we will need to compile our test application with the
toolchain that Buildroot used.

Let's configure our \code{PATH} to use this toolchain:

\begin{verbatim}
export
PATH=$HOME/felabs/realtime/rttest/buildroot-2012.05/output/host/usr/bin:$PATH
\end{verbatim}

Have a look at the \code{rttest.c} source file available in \code{root/} in
the \code{nfsroot/} directory. See how it shows the resolution of the
\code{CLOCK_MONOTONIC} clock.

Now compile this program:
\begin{verbatim}
arm-none-linux-gnueabi-gcc -o rttest rttest.c -lrt
\end{verbatim}

Execute the program on the board. Is the clock resolution good or bad?
Compare it to the timer tick of your system, as defined by \code{CONFIG_HZ}.

Obviously, this resolution will not provide accurate sleep times, and
this is because our kernel doesn't use high-resolution timers. So
let's enable the following options in the kernel configuration:
\code{CONFIG_HIGH_RES_TIMERS}.

Recompile your kernel, boot your IGEP board with the new version, and
check the new resolution. Better, isn't it?

\section{Testing the non-preemptible kernel}

Now, do the following tests:
\begin{itemize}
\item Test the program with nothing special and write down the
  results.
\item Test your program and at the same time, add some workload to the
  board, by running \code{/root/doload 300 > /dev/null 2>&1 &} on the board,
  and using \code{netcat 192.168.0.100 5566} on your workstation when
  you see the message \code{Listening on any address 5566} in order to
  flood the network interface of the IGEP board (where 192.168.0.100
  is the IP address of the IGEP board).
\item Test your program again with the workload, but by running the
  program in the \code{SCHED_FIFO} scheduling class at priority 99,
  using the \code{chrt} command.
\end{itemize}

\section{Testing the preemptible kernel}

Recompile your kernel with \code{CONFIG_PREEMPT} enabled, which
enables kernel preemption (except for critical sections protected by
spinlocks).

Run the simple tests again with this new preemptible kernel and compare
the results.

\section{Testing Xenomai scheduling latency}

Get the latest Xenomai 2.6.x release from its download area at
\url{http://download.gna.org/xenomai/stable/} and untar Xenomai.

Prepare the kernel for Xenomai compilation:
\begin{verbatim}
./scripts/prepare-kernel.sh --arch=arm --linux=/path/to/linux-2.6.38.8
\end{verbatim}

Now, run the kernel configuration interface, and make sure that
the below options are enabled, taking your time to read their
description:

\begin{itemize}
\item \code{CONFIG_XENOMAI}
\item \code{CONFIG_XENO_DRIVERS_TIMERBENCH}
\item \code{CONFIG_XENO_HW_UNLOCKED_SWITCH}
\end{itemize}

Now, you also have to disable \code{CONFIG_ARCH_OMAP2},
and \code{CONFIG_ARCH_OMAP4}. Xenomai doesn't seem to support
the OMAP2 and OMAP4 boards, and the kernel won't build
if you don't disable the above options. 

Compile your kernel, and in the meantime,
compile \code{rttest} for the Xenomai POSIX skin:

\begin{verbatim}
cd $HOME/felabs/realtime/rttest/nfsroot/root
export DESTDIR=$HOME/felabs/realtime/rttest/buildroot-2012.05/output/staging
CFL=$($DESTDIR/usr/bin/xeno-config --skin posix --cflags)
LDF=$($DESTDIR/usr/bin/xeno-config --skin posix --ldflags)
arm-none-linux-gnueabi-gcc $CFL -o rttest rttest.c $LDF
\end{verbatim}

Now boot the board with the new kernel.

Run the following commands on the board:

\begin{verbatim}
echo 0 > /proc/xenomai/latency
\end{verbatim}

In the current version, you will get the following error:

\begin{verbatim}
sh: write error: Bad address
\end{verbatim}

You can check that \code{/proc/xenomai/latency} contains \code{0}. It
should be safe to ignore this error.

This will disable the timer compensation feature of Xenomai. This
feature allows Xenomai to adjust the timer programming to take into
account the time the system needs to schedule a task after being woken
up by a timer. However, this feature needs to be calibrated
specifically for each system. By disabling this feature, we will have
raw Xenomai results, that could be further improved by doing proper
calibration of this compensation mechanism.

Run the tests again, compare the results.

\section{Testing Xenomai interrupt latency}

Measure the interrupt latency with and without load, running the
following command:

\begin{verbatim}
latency -t 2
\end{verbatim}
