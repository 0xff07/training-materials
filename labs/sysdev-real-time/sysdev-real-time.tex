\subchapter{Real-time - Timers and scheduling latency}{Objective:
  Learn how to handle real-time processes and practice with the
  different real-time modes. Measure scheduling latency.}

After this lab, you will:
\begin{itemize}
\item Be able to check clock accuracy.
\item Be able to start processes with real-time priority.
\item Be able to build a real-time application against the standard
POSIX real-time API, and against Xenomai's POSIX skin.
\item Have compared scheduling latency on your system, between a
  standard kernel, a kernel with \code{PREEMPT_RT} and a kernel
  with Xenomai.
\end{itemize}

\section{Setup}

Go to the \code{$HOME/__SESSION_NAME__-labs/realtime/} directory.

Install the \code{netcat} package.

\section{Root filesystem}

Create an \code{nfsroot} directory.

To compare real-time latency between standard Linux and Xenomai, we
are going to need a root filesystem and a build environment that
supports Xenomai.

Let's build this with Buildroot.

Download and extract the Buildroot 2016.02 sources. As the latest
version of Xenomai doesn't seem to work on the Xplained board (yet),
we need an older version of Buildroot that will build Xenomai 2.6.

Configure Buildroot with the following settings,
using the \code{/} command in \code{make
menuconfig} to find parameters by their name:

\begin{itemize}
\item In \code{Target}:
   \begin{itemize}
   \item \code{Target architecture}: \code{ARM (little endian)}
   \item \code{Target Architecture Variant}: \code{cortex-A5}
   \end{itemize}
\item In \code{Toolchain}:
   \begin{itemize}
   \item \code{Toolchain type}: \code{External toolchain}
   \item \code{Toolchain}: \code{Sourcery CodeBench ARM 2014.05}
   \end{itemize}
\item In \code{System configuration}:
   \begin{itemize}
   \item in \code{Run a getty (login prompt) after boot},  \code{TTY port}: \code{ttyS0}
   \end{itemize}
\item In \code{Target packages}:
   \begin{itemize}
   \item Enable \code{Show packages that are also provided by busybox}.
         We need this to build the standard \code{netcat} command, not
         provided in the default BusyBox configuration.
   \item In \code{Debugging, profiling and benchmark}, enable
         \code{rt-tests}. This will be a few applications to test
         real-time latency.
   \item In \code{Networking applications}, enable \code{netcat}
   \item In \code{Real-Time}, enable \code{Xenomai Userspace}:
         \begin{itemize}
         \item Enable \code{Install testsuite}
	 \item Make sure that \code{POSIX skin library} and
	       \code{Native skin library}\footnote{Needed by
	       the Xenomai testsuite.} are enabled.
	 \end{itemize}
   \end{itemize}
\end{itemize}

Now, build your root filesystem.

As you are using a 64 bit distribution, Buildroot should also ask
you to install 32 bit compatibility packages to be able to execute the
Sourcery CodeBench external toolchain:

\begin{verbatim}
sudo apt install libc6-i386 lib32stdc++6 lib32z1
\end{verbatim}

At the end of the build job, extract the
\code{output/images/rootfs.tar} archive in the \code{nfsroot}
directory.

The last thing to do is to add a few files that we will need in our
tests:

\begin{verbatim}
cp data/* nfsroot/root
\end{verbatim}

\section{Downloading sources and patches}

We will use a recent kernel version that is supported by the PREEMPT\_RT
patchset.

So, go to
\url{https://kernel.org/pub/linux/kernel/projects/rt/5.6/},
download the latest patch available in a single file.

Then go to \url{http://kernel.org} and download the exact version
corresponding to the patch you downloaded. At the time of this writing,
this version was 5.6.4.

\section{Compile a standard Linux kernel}

Extract the sources of your 5.6.x kernel but don't apply the
PREEMPT\_RT patches yet.

Configure your kernel for your Xplained board, and then make sure
that the below settings are disabled:
\code{CONFIG_PROVE_LOCKING}, \code{CONFIG_DEBUG_LOCK_ALLOC},
\code{CONFIG_DEBUG_MUTEXES} and \code{CONFIG_DEBUG_SPINLOCK}.

Also, for the moment, disable the \code{CONFIG_HIGH_RES_TIMERS}
option which impact we want to measure.

Boot the Xplained board by mounting the root filesystem that you
built. As usual, login as \code{root}, there is no password.

\section{Compiling with the POSIX RT library}

The root filesystem was built with the GNU C library, because it has
better support for the POSIX RT API.

In our case, when we created this lab, uClibc didn't support the
\code{clock_nanosleep} function used in our \code{rttest.c}
program. {\em uClibc} also does not support priority inheritance on
mutexes.

Therefore, we will need to compile our test application with the
toolchain that Buildroot used.

Let's configure our \code{PATH} to use this toolchain:

\scriptsize
\begin{verbatim}
export PATH=$HOME/__SESSION_NAME__-labs/realtime/buildroot-YYYY.MM/output/host/usr/bin:$PATH
\end{verbatim}
\normalsize

Have a look at the \code{rttest.c} source file available in
\code{root/} in the \code{nfsroot/} directory. See how it shows the
resolution of the \code{CLOCK_MONOTONIC} clock.

Now compile this program:
\begin{verbatim}
arm-none-linux-gnueabi-gcc -o rttest rttest.c -lrt
\end{verbatim}

Execute the program on the board. Is the clock resolution good or bad?
Compare it to the timer tick of your system, as defined by
\code{CONFIG_HZ}.

Copy the results in a file, in order to be able to compare them
with further results.

Obviously, this resolution will not provide accurate sleep times, and
this is because our kernel doesn't use high-resolution timers. So
let's add back the \code{CONFIG_HIGH_RES_TIMERS} option in the kernel
configuration.

Recompile your kernel, boot your Xplained with the new version, and
check the new resolution. Better, isn't it?

\section{Testing the non-preemptible kernel}

Now, do the following tests:
\begin{itemize}
\item Test the program with nothing special and write down the
  results.
\item Test your program and at the same time, add some workload to the
  board, by running \code{/root/doload 300 > /dev/null 2>&1 &} on the
  board, and using \code{netcat 192.168.0.100 5566} on your
  workstation in order to flood the network interface of the Xplained
  board (where \code{192.168.0.100} is the IP address of the Xplained
  board).
\item Test your program again with the workload, but by running the
  program in the \code{SCHED_FIFO} scheduling class at priority
  \code{99}, using the \code{chrt} command.
\end{itemize}

\section{Testing the preemptible kernel}

Recompile your kernel with \code{CONFIG_PREEMPT} enabled, which
enables kernel preemption (except for critical sections protected by
spinlocks).

Run the simple tests again with this new preemptible kernel and compare
the results.

\section{Compiling and testing the PREEMPT\_RT kernel}

Download the latest \code{PREEMPT_RT} kernel patch and apply it to
your kernel sources.

Configure your kernel with \code{CONFIG_PREEMPT_RT_FULL} and boot it.

Repeat the tests and compare the results again. You should see a massive
improvement in the maximum latency.

\section{Testing Xenomai scheduling latency}

Stay in \code{$HOME/__SESSION_NAME__-labs/realtime}.

Download the 2.6.4 release of Xenomai (that's what our version of
Buildroot supports by default), and extract it.

As you can see in the \code{ksrc/arch/arm/patches} directory,
the most recent Linux kernel version supported by Xenomai for ARM is
3.14.17.

Then, download the 3.14.17 Linux sources ({\bf not the latest 3.14.x
sources} because the Xenomai patches only apply to this exact version),
and extract them.
 
Now, prepare our kernel for Xenomai compilation:
\begin{verbatim}
cd $HOME/__SESSION_NAME__-labs/realtime
./xenomai-2.6.4/scripts/prepare-kernel.sh --arch=arm \
--linux=linux-3.14.17 \
--adeos=xenomai-2.6.4/ksrc/arch/arm/patches/ipipe-core-3.14.17-arm-4.patch
\end{verbatim}

Now, configure your kernel for SAMA5 boards, then start the kernel
configuration interface, and make sure that the below options are
enabled, taking your time to read their description:

\begin{itemize}
\item \code{CONFIG_XENOMAI}
\item \code{CONFIG_XENO_DRIVERS_TIMERBENCH}
\item \code{CONFIG_XENO_HW_UNLOCKED_SWITCH}
\end{itemize}

Compile your kernel, using the same Sourcery CodeBench compiler
as earlier in the lab\footnote{Your own toolchain is too recent
for the 3.14 kernel, which doesn't support compiling with gcc5 yet. The
Sourcery CodeBench gcc version is 4.8.x.}. 

While the kernel compiles, we can start to build our application against
the Xenomai libraries. We will need {\em pkg-config} built by Buildroot.
So go in your Buildroot source directory, and force Buildroot to build
the host variant of {\em pkg-config}:

\begin{verbatim}
cd $HOME/__SESSION_NAME__-labs/realtime/buildroot-YYYY.MM/
make host-pkgconf
\end{verbatim}

We can now compile \code{rttest} for the Xenomai POSIX skin:

\scriptsize
\begin{verbatim}
cd $HOME/__SESSION_NAME__-labs/realtime/nfsroot/root
export PATH=$HOME/__SESSION_NAME__-labs/realtime/buildroot-YYYY.MM/output/host/usr/bin:$PATH
arm-none-linux-gnueabi-gcc -o rttest rttest.c \
  $(pkg-config --libs --cflags libxenomai_posix)
\end{verbatim}
\normalsize

Now boot the board with the new kernel.

Run the following commands on the board:

\begin{verbatim}
echo 0 > /proc/xenomai/latency
\end{verbatim}

This will disable the timer compensation feature of Xenomai. This
feature allows Xenomai to adjust the timer programming to take into
account the time the system needs to schedule a task after being woken
up by a timer. However, this feature needs to be calibrated
specifically for each system. By disabling this feature, we will have
raw Xenomai results, that could be further improved by doing proper
calibration of this compensation mechanism.

Run the tests again, compare the results.

\section{Testing Xenomai interrupt latency}

Measure the interrupt latency with and without load, running the
following command:

\begin{verbatim}
latency -t 2
\end{verbatim}
