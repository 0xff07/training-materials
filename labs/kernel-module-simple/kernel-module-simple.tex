\subchapter{Writing modules}{Objective: create a simple kernel module}

After this lab, you will be able to:

\begin{itemize}
\item Compile and test standalone kernel modules, which code is outside of the main Linux sources.
\item Write a kernel module with several capabilities, including module parameters.
\item Access kernel internals from your module.
\item Set up the environment to compile it.
\item Create a kernel patch.
\end{itemize}

\section{Setup}

Go to the \code{~/linux-kernel-labs/modules/nfsroot/root/hello} directory.
Boot your board if needed.

\section{Writing a module}

Look at the contents of the current directory. All the files you generate
there will also be visible from the target. That's great to load
modules!

Add C code to the \code{hello_version.c} file, to implement a module which
displays this kind of message when loaded:

\begin{verbatim}
Hello Master. You are currently using Linux <version>.
\end{verbatim}

... and displays a goodbye message when unloaded.

Suggestion: you can look for files in kernel sources which
contain \code{version} in their name, and see what they do.

You may just start with a module that displays a hello message, and
add version information later.

Caution: you must use a kernel variable or function to get version
information, and not just the value of a C macro. Otherwise, you will
only get the version of the kernel you used to build the
module.

\section{Building your module}

The current directory contains a \code{Makefile} file, which lets you
build modules outside a kernel source tree.  Compile your module.

\section{Testing your module}

Load your new module file on the target. Check that it works as
expected. Until this, unload it, modify its code, compile and load it
again as many times as needed.

Run a command to check that your module is on the list of loaded
modules. Now, try to get the list of loaded modules with only the
\code{cat} command.

\section{Adding a parameter to your module}

Add a \code{who} parameter to your module. Your module will say
\code{Hello <who>} instead of \code{Hello Master}.

Compile and test your module by checking that it takes the \code{who}
parameter into account when you load it.

\section{Adding time information}

Improve your module, so that when you unload it, it tells you how many
seconds elapsed since you loaded it.  You can use the
\code{do_gettimeofday()} function to achieve this.

You may search for other drivers in the kernel sources using the
\code{do_gettimeofday()} function. Looking for other examples always helps!

\section{Following Linux coding standards}

Your code should adhere to strict coding standards, if you want to
have it one day merged in the mainline sources. One of the main
reasons is code readability. If anyone used one's own style, given the
number of contributors, reading kernel code would be very unpleasant.

Fortunately, the Linux kernel community provides you with a utility to
find coding standards violations.

Run the \code{scripts/checkpatch.pl -h} command in the kernel sources,
to find which options are available.  Now, run:

\begin{verbatim}
~/linux-kernel-labs/src/linux/scripts/checkpatch.pl --file --no-tree hello_version.c
\end{verbatim}

See how many violations are reported on your code, and fix your code
until there are no errors left. If there are many indentation related
errors, make sure you use a properly configured source code editor,
according to the kernel coding style rules in
\code{Documentation/CodingStyle}.

\section{Adding the hello\_version module to the kernel sources}

As we are going to make changes to the kernel sources, first create a
special branch for such changes:

\begin{verbatim}
git checkout 4.19.y
git checkout -b hello
\end{verbatim}

Add your module sources to the \code{drivers/misc/} directory in your
kernel sources. Of course, also modify kernel configuration and
building files accordingly, so that you can select your module in
\code{make xconfig} and have it compiled by the \code{make} command.

Run one of the kernel configuration interfaces and check that it
shows your new driver lets you configure it as a module.

Run the \code{make} command and make sure that the code of your new
driver is getting compiled.

Then, commit your changes in the current branch (try to choose an
appropriate commit message):

\begin{verbatim}
cd ~/linux-kernel-labs/src/linux
git add <files>
git commit -as
\end{verbatim}

\begin{itemize}
\item \code{git add} adds files to the next commit. It is mandatory to
  use for new files that should be added under version control.
\item \code{git commit -a} creates a commit with all modified files
  that already under version control
\item \code{git commit -s} adds a \code{Signed-off-by:} line to the
      commit message. All contributions to the Linux kernel must have
      such a line.
\end{itemize}

\section{Create a kernel patch}

You can be proud of your new module! To be able to share it with
others, create a patch which adds your new files to the mainline
kernel.

Creating a patch with \code{git} is extremely easy! You just generate it
from the commits between your branch and another branch, usually the
one you started from:

\begin{verbatim}
git format-patch 4.16.y
\end{verbatim}

Have a look at the generated file. You can see that its name reused
the commit message.

If you want to change the last commit message at this stage, you
can run:

\begin{verbatim}
git commit --amend
\end{verbatim}

And run \code{git format-patch} again.
