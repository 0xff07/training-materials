\subchapter{Writing modules}{Objective: create a simple kernel module}

After this lab, you will be able to:

\begin{itemize}
\item Compile and test standalone kernel modules, which code is outside of the main Linux sources.
\item Write a kernel module with several capabilities, including module parameters.
\item Access kernel internals from your module.
\item Setup the environment to compile it
\item Create a kernel patch
\end{itemize}

\section{Setup}

Stay inside the \code{$HOME/felabs/linux/modules} directory.
Boot your board again, as you did in the previous lab.

\section{Writing a module}

Go to the \code{nfsroot/root} directory. All the files you generate
there will also be visible from the target. That's great to load
modules!

Create a \code{hello_version.c} file implementing a module which
displays this kind of message when loaded:

\begin{verbatim}
Hello Master. You are currently using Linux <version>.
\end{verbatim}

... and displays a goodbye message when unloaded.

You may just start with a module that displays a hello message, and
add version information later.

Caution: you must use a kernel variable or function to get version
information, and not just the value of a C macro. Otherwise, you will
only get the version of the kernel you used to build the
module. Suggestion: you can look for files in kernel sources which
contain \code{version} in their name, and see what they do.

\section{Building your module}

The current directory contains a \code{Makefile} file, which lets you
build modules outside a kernel source tree.  Compile your module.

\section{Testing your module}

Load your new module file. Check that it works as
expected. Until this, unload it, modify its code, compile and load it
again as many times as needed.

Run a command to check that your module is on the list of loaded
modules. Now, try to get the list of loaded modules with only the cat
command.

\section{Adding a parameter to your module}

Add a who parameter to your module. Your module will say \code{Hello
<who>} instead of \code{Hello Master}.

 Compile and test your module by checking that it takes the who
parameter into account when you load it.

\section{Adding time information}

Improve your module, so that when you unload it, it tells you how many
seconds elapsed since you loaded it.  You can use the
\code{do_gettimeofday()} function to achieve this.

You may search for other drivers in the kernel sources using the
\code{do_gettimeofday()} function. Looking for other examples always helps!

\section{Following Linux coding standards}

Your code should adhere to strict coding standards, if you want to
have it one day merged in the mainline sources. One of the main
reasons is code readability. If anyone used one's own style, given the
number of contributors, reading kernel code would be very unpleasant.

Fortunately, the Linux kernel community provides you with a utility to
find coding standards violations.

Run the \code{scripts/checkpatch.pl -h} command in the kernel sources,
to find which options are available.  Now, run:

\begin{verbatim}
scripts/checkpatch.pl --file --no-tree <path>/hello_version.c
\end{verbatim}

See how many violations are reported on your code. If there are
indenting errors, you can first run your code through the \code{indent}
command:

\begin{verbatim}
sudo apt-get install indent
indent -linux hello_version.c
\end{verbatim}

You can now compare the indented file with the original:

\begin{verbatim}
sudo apt-get install meld
meld hello_version.c~ hello_version.c
\end{verbatim}

Now, get back to \code{checkpatch.pl} and fix your code until there are
no errors left.

\section{Adding the {\tt hello\_version} module to the kernel sources}

Add your module sources to the \code{drivers/misc/} directory in your
kernel sources. Of course, also modify kernel configuration and
building files accordingly, so that you can select your module in
\code{make xconfig} and have it compiled by the \code{make} command.

Configure your kernel with the config file corresponding to your
running kernel. Now check that the configuration interface shows your
new driver and lets you configure it as a module.

Run the \code{make} command and make sure that the code of your new
driver is getting compiled. Then, install your kernel module using
\code{make modules_install}. Beware, the modules should be installed
in the root filesystem of the target, not in the root filesystem of
your development workstation!

\section{Create a kernel patch}

You can be proud of your new module! To be able to share it with
others, create a patch which adds your new files to the mainline
kernel.

Test that your patch file is compatible with the patch command by
applying it to unmodified kernel sources.
