\subchapter{Kernel - Cross-compiling}{Objective: Learn how to
cross-compile a kernel for an ARM target platform.}

After this lab, you will be able to:
\begin{itemize}
\item Set up a cross-compiling environment
\item Configure the kernel Makefile accordingly
\item Cross compile the kernel for the Microchip SAMA5D3 Xplained ARM board
\item Use U-Boot to download the kernel
\item Check that the kernel you compiled starts the system
\end{itemize}

\section{Setup}

Go to the \code{$HOME/embedded-linux-labs/kernel} directory.

\section{Target system}

We are going to cross-compile and boot a Linux kernel for the Microchip
SAMA5D3 Xplained board.

\section{Kernel sources}

We will re-use the kernel sources downloaded and patched in the
previous lab.

\section{Cross-compiling environment setup}

To cross-compile Linux, you need to have a cross-compiling
toolchain. We will use the cross-compiling toolchain that we
previously produced, so we just need to make it available in the PATH:

\begin{verbatim}
export PATH=$HOME/x-tools/arm-cortexa5-linux-uclibcgnueabihf/bin:$PATH
\end{verbatim}

Also, don't forget to either:

\begin{itemize}
\item Define the value of the \code{ARCH} and \code{CROSS_COMPILE}
  variables in your environment (using \code{export})
\item {\bf Or} specify them on the command line at every invocation of
  \code{make}, i.e: \code{make ARCH=... CROSS_COMPILE=... <target>}
\end{itemize}

\section{Linux kernel configuration}

By running \code{make help}, find the proper Makefile target to
configure the kernel for the Xplained board (hint: the default
configuration is not named after the board, but after the SoC
name). Once found, use this target to configure the kernel with the
ready-made configuration.

Don't hesitate to visualize the new settings by running
\code{make xconfig} afterwards!

In the kernel configuration, as an experiment, change the kernel
compression from Gzip to XZ. This compression algorithm is far more
efficient than Gzip, in terms of compression ratio, at the expense of
a higher decompression time.

\section{Cross compiling}

At this stage, you need to install the \code{libssl-dev}
package to compiling the kernel.

You're now ready to cross-compile your kernel. Simply run:

\begin{verbatim}
make
\end{verbatim}

and wait a while for the kernel to compile. Don't forget to use
\code{make -j<n>} if you have multiple cores on your machine!

Look at the end of the kernel build output to see which file contains
the kernel image. You can also see the Device Tree \code{.dtb} files
which got compiled. Find which \code{.dtb} file corresponds to your
board.

Copy the linux kernel image and DTB files to the TFTP server 
home directory.

\section{Load and boot the kernel using U-Boot}

We will use TFTP to load the kernel image on the Xplained board:

\begin{itemize}

\item On your workstation, copy the \code{zImage} and DTB files to the
  directory exposed by the TFTP server.

\item On the target (in the U-Boot prompt), load \code{zImage} from
  TFTP into RAM at address 0x21000000:\\
  \code{tftp 0x21000000 zImage}

\item Now, also load the DTB file into RAM at address 0x22000000:\\
  \code{tftp 0x22000000 at91-sama5d3_xplained.dtb}

\item Boot the kernel with its device tree:\\
  \code{bootz 0x21000000 - 0x22000000}

\end{itemize}

You should see Linux boot and finally panicking. This is expected: we
haven't provided a working root filesystem for our device yet.

You can now automate all this every time the board is booted or
reset. Reset the board, and specify a different \code{bootcmd}:

{\scriptsize
\begin{verbatim}
setenv bootcmd 'tftp 0x21000000 zImage; tftp 0x22000000 at91-sama5d3_xplained.dtb; bootz 0x21000000 - 0x22000000'
saveenv
\end{verbatim}
}

\section{Flashing the kernel and DTB in NAND flash}

In order to let the kernel boot on the board autonomously, we can
flash the kernel image and DTB in the NAND flash available on the
Xplained board.

After storing the first stage bootloader, U-boot and its environment
variables, we will keep special areas in NAND flash for the DTB
and Linux kernel images:

\begin{center}
  \includegraphics[width=\textwidth]{labs/sysdev-kernel-cross-compiling/flash-map.pdf}
\end{center}

So, let's start by erasing the corresponding 128 KiB of NAND flash
for the DTB:

\begin{verbatim}
nand erase 0x140000 0x20000
        (NAND offset) (size)
\end{verbatim}

Then, let's erase the 5 MiB of NAND flash for the kernel image:

\begin{verbatim}
nand erase 0x160000 0x500000
\end{verbatim}

Then, copy the DTB and kernel binaries from TFTP into memory, using the
same addresses as before.

Then, flash the DTB and kernel binaries:

\begin{verbatim}
nand write 0x22000000 0x140000 0x20000
           (RAM addr) (NAND offset) (size)
nand write 0x21000000 0x160000 0x500000
\end{verbatim}

Power your board off and on, to clear RAM contents. We should now be
able to load the DTB and kernel image from NAND and boot with:

\begin{verbatim}
nand read 0x22000000 0x140000 0x20000
          (RAM addr) (offset) (size)
nand read 0x21000000 0x160000 0x500000
bootz 0x21000000 - 0x22000000
\end{verbatim}

Write a U-Boot script that automates the DTB + kernel download
and flashing procedure.

You are now ready to modify \code{bootcmd} to boot the board
from flash. But first, save the settings for booting from \code{tftp}:

\begin{verbatim}
setenv bootcmdtftp ${bootcmd}
\end{verbatim}

This will be useful to switch back to \code{tftp} booting mode
later in the labs.

Finally, using \code{editenv bootcmd}, 
adjust \code{bootcmd} so that the Xplained board starts
using the kernel in flash.

Now, reset the board to check that it boots
in the same way from NAND flash. Check that this is really your own version of
the kernel that's running\footnote{Look at the kernel log. You will find
the kernel version number as well as the date when it was compiled.
That's very useful to check that you're not loading an older version
of the kernel instead of the one that you've just compiled.}
