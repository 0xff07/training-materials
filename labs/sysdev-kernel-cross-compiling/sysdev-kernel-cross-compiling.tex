\subchapter{Kernel - Cross-compiling}{Objective: Learn how to cross-compile a kernel for an OMAP target platform.}

After this lab, you will be able to:
\begin{itemize}
\item Set up a cross-compiling environment
\item Configure the kernel Makefile accordingly
\item Cross compile the kernel for the IGEPv2 arm board
\item Use U-Boot to download the kernel
\item Check that the kernel you compiled starts the system
\end{itemize}

\section{Setup}

Go to the \code{$HOME/felabs-sysdev/kernel} directory.

Install the following packages: \code{libqt4-dev} and
\code{u-boot-tools}. \code{libqt4-dev} is needed for the
\code{xconfig} kernel configuration interface, and \code{u-boot-tools}
is needed to build the \code{uImage} kernel image file for U-Boot.

\section{Target system}

We are going to cross-compile and boot a Linux kernel for the IGEPv2
board.

\section{Kernel sources}

We will re-use the kernel sources downloaded and patched in the
previous lab.

\section{Cross-compiling environment setup}

To cross-compile Linux, you need to have a cross-compiling
toolchain. We will use the cross-compiling toolchain that we
previously produced, so we just need to make it available in the PATH:

\begin{verbatim}
export PATH=/usr/local/xtools/arm-unknown-linux-uclibcgnueabi/bin:$PATH
\end{verbatim}

Also, don't forget to either:

\begin{itemize}
\item Define the value of the \code{ARCH} and \code{CROSS_COMPILE}
  variables in your environment (using \code{export})
\item {\bf Or} specify them on the command line at every invocation of
  \code{make}, i.e: \code{make ARCH=... CROSS_COMPILE=... <target>}
\end{itemize}

\section{Linux kernel configuration}

By running \code{make help}, find the proper Makefile target to
configure the kernel for the IGEPv2 board (hint: the default
configuration is not named after the board, but after the CPU
name). Once found, use this target to configure the kernel with the
ready-made configuration.

Don't hesitate to visualize the new settings by running
\code{make xconfig} afterwards!

In the kernel configuration:

\begin{itemize}

\item Disable support for the IGEPv2 board compiled into the kernel
  (\code{CONFIG_MACH_IGEP0020}). We will boot our kernel with a device
  tree for our board, and won't compile the board description file
  (\code{arch/arm/mach-omap2/board-igep0020.c} in the kernel
  sources.). Technically speaking, you can leave this option enabled,
  and still boot using a {\em Device Tree}, but disabling it makes
  sure that your board will not fall back to legacy booting if you do
  a mistake! You will have to review dependencies to be able to
  disable this kernel configuration setting.
\item As an experiment, let's change the kernel compression from Gzip
  to XZ. This compression algorithm is far more efficient than Gzip,
  in terms of compression ratio, at the expense of a higher
  decompression time.
\end{itemize}

\section{Cross compiling}

You're now ready to cross-compile your kernel. Simply run:

\begin{verbatim}
make
\end{verbatim}

and wait a while for the kernel to compile. Don't forget to use
\code{make -j<n>} if you have multiple cores on your machine!

Look at the end of the kernel build output to see which file contains
the kernel image. You can also see the Device Tree \code{.dtb} files
which got compiled. Find which \code{.dtb} file corresponds to your
board.

However, the default image produced by the kernel build process is not
suitable to be booted from U-Boot. A post-processing operation must be
performed using the \code{mkimage} tool provided by U-Boot developers. This
tool has already been installed in your system as part of the
\code{u-boot-tools} package. To run the post-processing operation on
the kernel image, simply run:

\begin{verbatim}
make LOADADDR=0x80008000 uImage
\end{verbatim}

The \code{LOADADDR} indicates to U-Boot where the kernel image should
be loaded.

\section{Setting up serial communication with the board}

Plug the IGEP board on your computer. Start Picocom on
\code{/dev/ttyS0}, or on \code{/dev/ttyUSB0} if you are using a serial
to USB adapter.

You should now see the U-Boot prompt:

\begin{verbatim}
U-Boot #
\end{verbatim}

Make sure that the bootargs U-Boot environment variable is not set (it
could remain from a previous training session, and this could disturb
the next lab):

\begin{verbatim}
setenv bootargs
saveenv
\end{verbatim}

\section{Load and boot the kernel using U-Boot}

We will use TFTP to load the kernel image to the IGEP board:

\begin{itemize}

\item On your workstation, copy the \code{uImage} and DTB files to the
  directory exposed by the TFTP server.

\item On the target, load \code{uImage} from TFTP into RAM at address
  0x80000000:\\
  \code{tftp 0x80000000 uImage}

\item Now, also load the DTB file into RAM at address 0x81000000:\\
  \code{tftp 0x81000000 omap3-igep0020.dtb}

\item Boot the kernel with its device tree:\\
  \code{bootm 0x80000000 - 0x81000000}

\end{itemize}

You should see Linux boot and finally hang with the following message:

\begin{verbatim}
Waiting for root device /dev/mmcblk0p2...
\end{verbatim}

This is expected: we haven't provided a working root filesystem for
our device yet.

You can now automate all this every time the board is booted or
reset. Reset the board, and specify a different \code{bootcmd}:

{\scriptsize
\begin{verbatim}
setenv bootcmd 'tftp 80000000 uImage; tftp 81000000 omap3-igep0020.dtb; bootm 80000000 - 81000000'
saveenv
\end{verbatim}
}

\section{Flashing the kernel and DTB in NAND flash}

In order to let the kernel boot on the board autonomously, we can
flash the kernel image and DTB in the NAND flash available on the IGEP
board. See the bootloader lab for details about U-boot's \code{nand}
command.

After storing the first stage bootloader, U-boot and its environment
variables, we will keep special areas in NAND flash for the DTB
and Linux kernel images:

\begin{center}
  \includegraphics[width=\textwidth]{labs/sysdev-kernel-cross-compiling/flash-map.pdf}
\end{center}

So, let's start by erasing the corresponding 128 KiB of NAND flash
for the DTB:

\begin{verbatim}
nand erase 0x2e0000 0x20000
        (NAND offset) (size)
\end{verbatim}

Then, let's erase the 5 MiB of NAND flash for the kernel image:

\begin{verbatim}
nand erase 0x300000 0x500000
\end{verbatim}

Then, copy the DTB and kernel binaries from TFTP into memory, using the
same addresses as before.

Then, flash the DTB and kernel binaries:

\begin{verbatim}
nand write 0x81000000 0x2e0000 0x20000
           (RAM addr) (NAND offset) (size)
nand write 0x80000000 0x300000 0x500000
\end{verbatim}

Power your board off and on, to clear RAM contents. We should now be
able to load the DTB and kernel image from NAND and boot with:

\begin{verbatim}
nand read 0x81000000 0x2e0000 0x20000
nboot 0x80000000 0        0x300000
      (RAM addr) (device) (NAND offset)
bootm 0x80000000 - 0x81000000
\end{verbatim}

\code{nboot} copies the kernel to RAM, using the \code{uImage} headers
to know how many bytes to copy. To load the kernel
to RAM, image, you could have used \code{nand read
0x80000000 0x300000 0x500000}, but you would have copied more bytes than
the actual size of your kernel. \footnote{\code{nboot} can save a lot 
of boot time, as it avoids having to copy a pessimistic number of
bytes from flash to RAM. Note that U-boot is not always configured
with \code{nboot} support.}.

Write a U-Boot script that automates the DTB + kernel download
and flashing procedure. Finally, adjust \code{bootcmd} so that
the IGEP board boots using the kernel in Flash.

Now, power off the board and power it on again to check that it boots
fine from NAND flash. Check that this is really your own version of
the kernel that's running.
