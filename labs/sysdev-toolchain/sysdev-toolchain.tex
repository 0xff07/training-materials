\subchapter{Building a cross-compiling toolchain}{Objective: Learn how
  to compile your own cross-compiling toolchain for the uClibc C
  library}

After this lab, you will be able to:

\begin{itemize}
\item Configure the {\em crosstool-ng} tool
\item Execute {\em crosstool-ng} and build up your own cross-compiling toolchain
\end{itemize}

\section{Setup}

Go to the \code{$HOME/felabs/sysdev/toolchain} directory.

\section{Install needed packages}

Install the packages needed for this lab:

\begin{verbatim}
sudo apt-get install autoconf automake libtool libexpat1-dev \
     libncurses5-dev bison flex patch curl cvs texinfo \
     build-essential subversion gawk python-dev gperf
\end{verbatim}

\section{Getting Crosstool-ng}

Get the latest 1.19.x release of Crosstool-ng at
\url{http://crosstool-ng.org}. Expand the archive right in the current
directory, and enter the Crosstool-ng source directory.

\section{Installing Crosstool-ng}

We can either install Crosstool-ng globally on the system, or keep it
locally in its download directory. We'll choose the latter
solution. As documented in
\code{docs/2\ -\ Installing\ crosstool-NG.txt}, do:

\begin{verbatim}
./configure --enable-local
make
make install
\end{verbatim}

Then you can get Crosstool-ng help by running

\begin{verbatim}
./ct-ng help
\end{verbatim}

\section{Configure the toolchain to produce}

A single installation of Crosstool-ng allows to produce as many
toolchains as you want, for different architectures, with different C
libraries and different versions of the various components.

Crosstool-ng comes with a set of ready-made configuration files for
various typical setups: Crosstool-ng calls them {\em samples}. They can be
listed by using \code{./ct-ng list-samples}.

We will use the arm-unknown-linux-uclibcgnueabi sample. It can be loaded by issuing:

\begin{verbatim}
./ct-ng arm-unknown-linux-uclibcgnueabi
\end{verbatim}

Then, to refine the configuration, let's run the \code{menuconfig} interface:

\begin{verbatim}
./ct-ng menuconfig
\end{verbatim}

In \code{Path and misc options}:

\begin{itemize}
\item Change \code{Prefix directory} to
  \code{/usr/local/xtools/${CT_TARGET}}. This is the place where
  the toolchain will be installed.
\item Change \code{Maximum log level to see} to \code{DEBUG} so that we can have more
  details on what happened during the build in case something went wrong.
\end{itemize}

In \code{Toolchain options}:
\begin{itemize}
\item Set \code{Tuple's alias} to \code{arm-linux}. This way, we will
  be able to use the compiler as \code{arm-linux-gcc} instead of
  \code{arm-unknown-linux-uclibcgnueabi-gcc}, which is much longer to
  type.
\end{itemize}

In \code{Debug facilities}:
\begin{itemize}
\item Enable \code{gdb}, \code{strace} and \code{ltrace}.
\item Remove the other options (\code{dmalloc} and \code{duma}).
\item In \code{gdb} options:
  \begin{itemize}
  \item Make sure that the \code{Cross-gdb} and \code{Build a static gdbserver}
	options are enabled; the other options are not needed.
  \item Set \code{gdb version} to \code{7.4.1}.
  \end{itemize}
\end{itemize}

Explore the different other available options by traveling through the
menus and looking at the help for some of the options. Don't hesitate
to ask your trainer for details on the available options. However,
remember that we tested the labs with the configuration described
above. You might waste time with unexpected issues if you customize the
toolchain configuration.

\section{Produce the toolchain}

First, create the directory \code{/usr/local/xtools/} and change its
owner to your user, so that Crosstool-ng can write to it.

Then, create the directory \code{$HOME/src} in which Crosstool-NG
will save the tarballs it will download.

Nothing is simpler:

\begin{verbatim}
./ct-ng build
\end{verbatim}

And wait!

\subsection{Known issues}

\subsubsection{Source archives not found on the Internet}

It is frequent that Crosstool-ng aborts because it can't find a
source archive on the Internet, when such an archive has moved or has
been replaced by more recent versions. New Crosstool-ng versions ship
with updated URLs, but in the meantime, you need work-arounds.

If this happens to you, what you can do is look for the source archive by
yourself on the Internet, and copy such an archive to the \code{src}
directory in your home directory. Note that even source archives
compressed in a different way (for example, ending with \code{.gz}
instead of \code{.bz2}) will be fine too. Then, all you have to do is run
\code{./ct-ng build} again, and it will use the source archive that you
downloaded.

\subsubsection{ppl-0.10.2 compiling error with gcc 4.7.1}

If you are using gcc 4.7.1, for example in Ubuntu 12.10 (not officially
supported in these labs), compilation will fail in \code{ppl-0.10.2} with
the below error:

\begin{verbatim}
error: 'f_info' was not declared in this scope
\end{verbatim}

One solution is to add the \code{-fpermissive} flag to the
\code{CT_EXTRA_FLAGS_FOR_HOST} setting (in \code{Path and misc options
->  Extra host compiler flags}). 

\section{Testing the toolchain}

You can now test your toolchain by adding
\code{/usr/local/xtools/arm-unknown-linux-uclibcgnueabi/bin/} to your
\code{PATH} environment variable and compiling the simple
\code{hello.c} program in your main lab directory with
\code{arm-linux-gcc}.

You can use the \code{file} command on your binary to make sure it has
correctly been compiled for the ARM architecture.

\section{Cleaning up}

To save about 3 GB of storage space, do a \code{./ct-ng clean} in the
Crosstool-NG source directory. This will remove the source code of the
different toolchain components, as well as all the generated files
that are now useless since the toolchain has been installed in
\code{/usr/local/xtools}.
