\subchapter{Building a cross-compiling toolchain}{Objective: Learn how
  to compile your own cross-compiling toolchain for the uClibc C
  library}

After this lab, you will be able to:

\begin{itemize}
\item Configure the {\em crosstool-ng} tool
\item Execute {\em crosstool-ng} and build up your own cross-compiling toolchain
\end{itemize}

\section{Setup}

Go to the \code{/home/<user>/felabs/sysdev/toolchain} directory.

Make sure you have at least 2 GB of free disk space.

\section{Install needed packages}

Install the packages needed for this lab:

\begin{verbatim}
sudo apt-get install autoconf automake libtool libexpat1-dev \
     libncurses5-dev bison flex patch curl cvs texinfo \
     build-essential subversion gawk python-dev gperf
sudo apt-get clean
\end{verbatim}

\section{Getting Crosstool-ng}

Get the latest 1.15.x release of Crosstool-ng at
\url{http://crosstool-ng.org}. Expand the archive right in the current
directory, and enter the Crosstool-ng source directory.

\section{Installing Crosstool-ng}

We can either install Crosstool-ng globally on the system, or keep it
locally in its download directory. We'll choose the latter
solution. As documented in
\code{docs/2\ -\ Installing\ crosstool-NG.txt}, do:

\begin{verbatim}
./configure --enable-local
make
make install
\end{verbatim}

Then you can get Crosstool-ng help by running

\begin{verbatim}
./ct-ng help
\end{verbatim}

\section{Configure the toolchain to produce}

A single installation of Crosstool-ng allows to produce as many
toolchains as you want, for different architectures, with different C
libraries and different versions of the various components.

Crosstool-ng comes with a set of ready-made configuration files for
various typical setups: Crosstool-ng calls them {\em samples}. They can be
listed by using \code{./ct-ng list-samples}.

We will use the arm-unknown-linux-uclibcgnueabi sample. It can be loaded by issuing:

\begin{verbatim}
./ct-ng arm-unknown-linux-uclibcgnueabi
\end{verbatim}

Then, to refine the configuration, let's run the menuconfig interface:

\begin{verbatim}
./ct-ng menuconfig
\end{verbatim}

In \code{Path and misc options}:

\begin{itemize}
\item Change \code{Prefix directory} to
  \code{/usr/local/xtools/${CT_TARGET}}. This is the place where
  the toolchain will be installed.
\item Set \code{Number of parallel jobs} to 2 times the number of CPU cores
  in your workstation\footnote{You can find this number by running \code{cat /proc/cpuinfo}}. Building your toolchain will be faster.
\item Change \code{Maximum log level to see} to \code{DEBUG} so that we can have more
  details on what happened during the build in case something went wrong.
\end{itemize}

In \code{Toolchain options}:
\begin{itemize}
\item Set \code{Tuple's alias} to \code{arm-linux}. This way, we will be able
  to use the compiler as \code{arm-linux-gcc} instead of
  \code{arm-unknown-linux-uclibcgnueabi-gcc}, which is much longer.
\end{itemize}

In \code{Debug facilities}:
\begin{itemize}
\item Enable \code{gdb}, \code{strace} and \code{ltrace}. Remove the
  other options (\code{dmalloc} and \code{duma}). In \code{gdb} options, make
  sure that the \code{Cross-gdb} and \code{Build a static gdbserver} options are
  enabled; the other options are not needed.
\end{itemize}

Explore the different other available options by traveling through the
menus and looking at the help for some of the options. Don't hesitate
to ask your trainer for details on the available options. However,
remember that we tested the labs with the configuration described
above.

\section{Produce the toolchain}

First, create the directory \code{/usr/local/xtools/} and change its
owner to your user, so that Crosstool-ng can write to it.

Then, create the directory \code{$HOME/src} in which Crosstool-NG
will save the tarballs it will download.

Nothing is simpler:

\begin{verbatim}
./ct-ng build
\end{verbatim}

And wait!

\section{Testing the toolchain}

You can now test your toolchain by adding
\code{/usr/local/xtools/arm-unknown-linux-uclibcgnueabi/bin/} to your
\code{PATH} environment variable and compiling the simple
\code{hello.c} program in your main lab directory with
\code{arm-linux-gcc}.

You can use the \code{file} command on your binary to make sure it has
correctly been compiled for the ARM architecture.

\section{Cleaning up}

To save almost 2 GB of storage space, remove the \code{.build/}
subdirectory in the Crosstool-ng directory.
