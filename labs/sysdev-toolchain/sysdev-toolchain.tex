\subchapter{Building a cross-compiling toolchain}{Objective: Learn how
  to compile your own cross-compiling toolchain for the uClibc C
  library}

After this lab, you will be able to:

\begin{itemize}
\item Configure the {\em crosstool-ng} tool
\item Execute {\em crosstool-ng} and build up your own cross-compiling toolchain
\end{itemize}

\section{Setup}

Go to the \code{$HOME/embedded-linux-labs/toolchain} directory.

\section{Install needed packages}

Install the packages needed for this lab:

\begin{verbatim}
sudo apt install build-essential git autoconf bison flex \
     texinfo help2man gawk libtool-bin libncurses5-dev
\end{verbatim}

\section{Getting Crosstool-ng}

Let's download the sources of Crosstool-ng, through its git
source repository, and switch to a commit that we have tested: 

\begin{verbatim}
git clone https://github.com/crosstool-ng/crosstool-ng.git
cd crosstool-ng/
git checkout eb65ba65 
\end{verbatim}

\section{Installing Crosstool-ng}

We can either install Crosstool-ng globally on the system, or keep it
locally in its download directory. We'll choose the latter
solution. As documented in
\code{docs/2 - Installing crosstool-NG.txt}, do:

\begin{verbatim}
./bootstrap
./configure --enable-local
make
\end{verbatim}

Then you can get Crosstool-ng help by running

\begin{verbatim}
./ct-ng help
\end{verbatim}

\section{Configure the toolchain to produce}

A single installation of Crosstool-ng allows to produce as many
toolchains as you want, for different architectures, with different C
libraries and different versions of the various components.

Crosstool-ng comes with a set of ready-made configuration files for
various typical setups: Crosstool-ng calls them {\em samples}. They can be
listed by using \code{./ct-ng list-samples}.

We will start with the \code{arm-cortexa5-linux-uclibcgnueabihf} sample. It
can be loaded by issuing:

\begin{verbatim}
./ct-ng arm-cortexa5-linux-uclibcgnueabihf
\end{verbatim}

Then, to refine the configuration, let's run the \code{menuconfig} interface:

\begin{verbatim}
./ct-ng menuconfig
\end{verbatim}

In \code{Path and misc options}:
\begin{itemize}
\item Change \code{Maximum log level to see} to \code{DEBUG} so that we can have more
  details on what happened during the build in case something went wrong.
\end{itemize}

\ifdefstring{\labboard}{discovery}{
In \code{Target options}:
\begin{itemize}
\item Set \code{Emit assembly for CPU} to \code{cortex-a7}.
\item Set \code{Use specific FPU} to \code{vfpv4}.
\end{itemize}
}{}

In \code{Toolchain options}:
\begin{itemize}
\item Set \code{Tuple's vendor string} to \code{training}.
\item Set \code{Tuple's alias} to \code{arm-linux}. This way, we will
  be able to use the compiler as \code{arm-linux-gcc} instead of
  \code{arm-training-linux-uclibcgnueabihf-gcc}, which is much longer
  to type.
\end{itemize}

In \code{C-library}:
\begin{itemize}
  \item Enable \code{IPv6} support. That's because of Buildroot (which
  we will use later, which doesn't accept to use toolchains without IPv6
  support.
\end{itemize}

In \code{Debug facilities}, disable every option. Some of these
options will be useful in a real toolchain, but in our labs, we will
do debugging work with another toolchain anyway.  Hence, not compiling
debugging features here will reduce toolchain building time.

Explore the different other available options by traveling through the
menus and looking at the help for some of the options. Don't hesitate
to ask your trainer for details on the available options. However,
remember that we tested the labs with the configuration described
above. You might waste time with unexpected issues if you customize the
toolchain configuration.

\section{Produce the toolchain}

Nothing is simpler:

\begin{verbatim}
./ct-ng build
\end{verbatim}

The toolchain will be installed by default in \code{$HOME/x-tools/}.
That's something you could have changed in Crosstool-ng's configuration.

And wait!

\subsection{Known issues}

\subsubsection{Source archives not found on the Internet}

It is frequent that Crosstool-ng aborts because it can't find a
source archive on the Internet, when such an archive has moved or has
been replaced by more recent versions. New Crosstool-ng versions ship
with updated URLs, but in the meantime, you need work-arounds.

If this happens to you, what you can do is look for the source archive by
yourself on the Internet, and copy such an archive to the \code{src}
directory in your home directory. Note that even source archives
compressed in a different way (for example, ending with \code{.gz}
instead of \code{.bz2}) will be fine too. Then, all you have to do is run
\code{./ct-ng build} again, and it will use the source archive that you
downloaded.

\section{Testing the toolchain}

You can now test your toolchain by adding
\code{$HOME/x-tools/arm-training-linux-uclibcgnueabihf/bin/} to your
\code{PATH} environment variable and compiling the simple
\code{hello.c} program in your main lab directory with
\code{arm-linux-gcc}.

You can use the \code{file} command on your binary to make sure it has
correctly been compiled for the ARM architecture.

\section{Cleaning up}

{\em Do this only if you have limited storage space. In case you made a
mistake in the toolchain configuration, you may need to run Crosstool-ng
again. Keeping generated files would save a significant amount of time.}

To save about 7 GB of storage space, do a \code{./ct-ng clean} in the
Crosstool-NG source directory. This will remove the source code of the
different toolchain components, as well as all the generated files
that are now useless since the toolchain has been installed in
\code{$HOME/x-tools}.
