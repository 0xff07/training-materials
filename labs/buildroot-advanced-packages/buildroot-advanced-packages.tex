\subchapter
{Advanced package recipe tricks}
{Objectives:
  \begin{itemize}
  \item Package an application with a mandatory dependency and an
    optional dependency
  \item Package a library, hosted on Github
  \item Use {\em hooks} to tweak packages
  \item Add a patch to a package
  \end{itemize}
}

\section{Start packaging application {\tt bar}}

For the purpose of this training, we have created a completely stupid
and useless application called \code{bar}. Its home page is
\url{http://free-electrons.com/~thomas/bar/}, for where you can
download an archive of application's source code.

Create an initial package for \code{bar} in \code{package/bar}, with
the necessary code in \code{package/bar/bar.mk} and
\code{package/bar/Config.in}. Don't forget
\code{package/bar/bar.hash}. At this point, your \code{bar.mk} should
only define the \code{<pkg>_VERSION}, \code{<pkg>_SOURCE} and
\code{<pkg>_SITE} variables.

Enable the \code{bar} package in your Buildroot configuration, and
start the build. It should download \code{bar}, extract it, and start
the configure script. And then it should fail with an error related to
\code{libfoo}. And indeed, as the \code{README} file available in
\code{bar}'s source code says, it has a mandatory dependency on
\code{libfoo}. So let's move on to the next section, and we'll start
packaging \code{libfoo}.

\section{Packaging {\tt libfoo}: initial packaging}

According to \code{bar}'s \code{README} file, \code{libfoo} is only
available on {\em Github} at
\url{https://github.com/tpetazzoni/libfoo}.

Create an initial package for \code{libfoo} in \code{package/libfoo},
with the relevant minimal variables to get \code{libfoo} downloaded
properly. Since it's hosted on {\em Github}, remember to use the
\code{github} function to define \code{<pkg>_SITE}. Also, notice that
there is a tagged version \code{v0.1}, you should probably use it.

Enable the \code{libfoo} package and start the build. You should get
an error due to the \code{configure} script being missing. What can
you do about it? Hint: there is one Buildroot variable for {\em
autotools} packages to solve this problem.

Once this problem is solved, start the build again, you should reach a
second problem related to a non-existing \code{m4} directory. So, add
a {\em post-extract hook} to create the \code{m4} directory in
\code{libfoo} source code.

After this second problem is solved, \code{libfoo} should build
fine. Look in \code{output/target/usr/lib}, the dynamic version of the
library should be installed. However, if you look in
\code{output/staging/}, you will see no sign of \code{libfoo}, neither
the library in \code{output/staging/usr/lib} or the header file in
\code{output/staging/usr/include}. This is an issue because the
compiler will only look in \code{output/staging} for libraries and
headers, so we must change our package so that it also installs to the
{\em staging directory}. Adjust your \code{libfoo.mk} file to achieve
this, restart the build of \code{libfoo}, and make sure that you see
\code{foo.h} in \code{output/staging/usr/include} and \code{libfoo.*}
in \code{output/staging/usr/lib}.

Now everything looks good, but there are some more improvements we can
do.

\section{Improvements to {\tt libfoo} packaging}

If you look in \code{output/target/usr/bin}, you can see a program
called \code{libfoo-example1}. This is just an example program for
\code{libfoo}, it is typically not very useful in a real target
system. So we would like this example program to not be installed. To
achieve this, add a {\em post-install target hook} that removes
\code{libfoo-example1}. Rebuild the \code{libfoo} package and verify
that \code{libfoo-example1} has been properly removed.

Now, if you go in \code{output/build/libfoo-v0.1}, and run
\code{./configure --help} to see the available options, you should see
an option named \code{--enable-debug-output}, which enables a
debugging feature of \code{libfoo}. Add a sub-option in
\code{package/libfoo/Config.in} to enable the debugging feature, and
the corresponding code in \code{libfoo.mk} to pass
\code{--enable-debug-output} or \code{--disable-debug-output} when
appropriate.

Enable this new option in \code{menuconfig}, and restart the build of
the package. Verify in the build output that
\code{--enable-debug-output} was properly passed as argument to the
\code{configure} script.

Now, the packaging of \code{libfoo} seems to be alright, so let's get
back to our \code{bar} application.

\section{Finalize the packaging of {\tt bar}}

So, \code{bar} was failing to configure because \code{libfoo} was
missing. Now that \code{libfoo} is available, modify \code{bar} to add
\code{libfoo} as a dependency. Remember that this needs to be done in
two places: \code{Config.in} file and \code{bar.mk} file.

Restart the build, and it should succeed! Now you can run the
\code{bar} application on your target, and discover how absolutely
useless it is, except for allowing you to learn about Buildroot
packaging!

\section{{\tt bar} packaging: {\em libconfig} dependency}

But there's some more things we can do to improve \code{bar}'s
packaging. If you go to \code{output/build/bar-1.0} and run
\code{./configure --help}, you will see that it supports a
\code{--with-libconfig} option. And indeed, \code{bar}'s \code{README}
file also mentions \code{libconfig} as an optional dependency.

So, change \code{bar.mk} to add {\em libconfig} as an optional
dependency. No need to add a new \code{Config.in} option for that:
just make sure that when {\em libconfig} is enabled in the Buildroot
configuration, \code{--with-libconfig} is passed to \code{bar}'s {\em
configure} script, and that {\em libconfig} is built before
\code{bar}. Also, pass \code{--without-libconfig} when {\em libconfig}
is not enabled.

Enable \code{libconfig} in your Buildroot configuration, and restart
the build of \code{bar}. What happens?

It fails to build with messages like \code{error: unknown type name
‘config_t’}. Seems like the author of \code{bar} messed up and forgot
to include the appropriate header file. Let's try to fix this: go to
\code{bar}'s source code in \code{output/build/bar-1.0} and edit
\code{src/main.c}. Right after the \code{#if defined(USE_LIBCONFIG)},
add a \code{#include <libconfig.h>}. Save, and restart the build of
\code{bar}. Now it builds fine!

However, try to rebuild \code{bar} from scratch by doing \code{make
bar-dirclean all}. The build problem happens again. This is because
doing a change directly in \code{output/build/} might be good for
doing a quick test, but not for a permanent solution: everything in
\code{output/} deleted when doing a \code{make clean}. So instead of
manually changing the package source code, we need to generate a
proper patch for it.

There are multiple ways to create patches, but we'll simply use Git to
do so. As the \code{bar} project home page indicates, a Git repository
is available on Github at \code{https://github.com/tpetazzoni/bar}.

Start by cloning the Git repository:

\begin{verbatim}
git clone git@github.com:tpetazzoni/bar.git
\end{verbatim}

Once the cloning is done, go inside the \code{bar} directory, and
create a new branch named \code{buildroot}, which starts the
\code{v1.0} tag (which matches the \code{bar-1.0.tar.xz} tarball we're
using):

\begin{verbatim}
git branch buildroot v1.0
\end{verbatim}

Move to this newly created branch\footnote{Yes, we can use \code{git
checkout -b} to create the branch and move to it in one command}:

\begin{verbatim}
git checkout buildroot
\end{verbatim}

Do the \code{#include <libconfig.h>} change to \code{src/main.c}, and
commit the result:

\begin{verbatim}
git commit -a -m "Fix missing <libconfig.h> include"
\end{verbatim}

Generate the patch for the last commit (i.e the one you just created):

\begin{verbatim}
git format-patch HEAD^
\end{verbatim}

and copy the generated \code{0001-*.patch} file to \code{package/bar/}
in the Buildroot sources.

Now, restart the build with \code{make bar-dirclean all}, it should
built fully successfully!

You can even check that \code{bar} is linked against
\code{libconfig.so} by doing:

\begin{verbatim}
./output/host/usr/bin/arm-linux-gnueabihf-readelf -d output/target/usr/bin/bar
\end{verbatim}

On the target, test \code{bar}. Then, create a file called
\code{bar.cfg} in the current directory, with the following contents:

\begin{verbatim}
verbose = "yes"
\end{verbatim}

And run \code{bar} again, and see what difference it makes.

Congratulations, you've finished package the most useless application
in the world!

\section{Preparing for the next lab}

In preparation for the next lab, we need to do a clean full rebuild,
so simply issue:

\begin{verbatim}
make clean all 2>&1 | tee build.log
\end{verbatim}
