\subchapter{Add a native application to the build}{Learn how to begin with
the Android build system}

After this lab, you will be able to:
\begin{itemize}
  \item Add an external binary to a system
  \item Express dependencies on other components of the build system
\end{itemize}

\section{Add the binary to the compiled image}
Copy the \code{mlbin.c} file from the
\code{$HOME/felabs/android/native-app} directory and put it
into the \code{external/ml} folder.

Just as for \code{libusb}, you now need to make an \code{Android.mk} file
giving all the details needed by the build system to compile. But
unlike \code{libusb}, this binary is an executable and depends on another
piece of software.

Make it compile and be integrated in the generated images. Once you
have the images, boot the board, plug a missile launcher and test the
application. You should see the launcher move.

\section{Fix the Problems}

The permission model of Android is heavily based on user
permissions. By default, your USB device won't be accessible to
applications. We didn't see this behaviour because we ran the mlbin
program as root. To test if the program runs properly, test it as a
random user on the system.

To do so, you can use the command \code{su}, and you can test with the
user \code{radio} for example.

When you start your tests, you will find that \code{libusb} cannot
open the usb devices because of restricted permissions. This can be
fixed through \code{ueventd.rc} files. Add a device-specific
\code{ueventd} configuration file to your build to make the files
under \code{/dev/bus/usb/} world readable.

\paragraph{Warning} The ueventd parser is buggy in Gingerbread and
doesn't read the \code{androidboot.hardware} parameter. You will have
to name the file \code{ueventd.omap3.rc}.

After completing this lab, you can ask your instructor to
give you a URL where Free Electrons' solution is available, and compare
it with what you implemented.
