\subchapter{Add a native application to the build}{Learn how to begin with
the Android build system}

After this lab, you will be able to:
\begin{itemize}
  \item Add an external binary to a system
  \item Express dependencies on other components of the build system
\end{itemize}

\section{Develop the application}

Copy the \code{mlbin.c} file from the
\code{/home/<user>/felabs/android/native-app} directory and put it
into the \code{external/ml} folder.

As you can see, the application is almost empty, so you need to fill
it with the appropriate code. You will need to implement an
initialization function, basic functions to control the launcher and
finally a freeing function.

You will need the \code{libusb_init}, \code{libusb_get_device_list},
\code{libusb_get_device_descriptor}, \code{libusb_open},
\code{libusb_detach_kernel_driver} and \code{libusb_claim_interface}
functions to initialize our program, the matching functions to exit
properly are \code{libusb_exit}, \code{libusb_close}
and \code{libusb_reclaim_interface}.

You can send an instruction by using the code
\begin{lstlisting}
unsigned char[] data = {0x5f, ACTION, 0xe0, 0xff, 0xfe};
libusb_control_transfer(devh, 0x21, 0x09, 0, 0, data, 5, 300};
\end{lstlisting}

\section{Add the binary to the compiled image}

Just as for \code{libusb}, you now need to make an \code{Android.mk} file
giving all the details needed by the build system to compile. But
unlike \code{libusb}, this binary is an executable and depends on another
piece of software.

Make it compile and be integrated in the generated images. Once you
have the images, boot the board, plug a missile launcher and test the
application. You should see the launcher move.

However, when you start your tests, you will find that \code{libusb} cannot open
the usb devices because of restricted permissions. This can be fixed
through \code{ueventd.rc} files. Add a device-specific uevend
configuration file to your build to make the files under
\code{/dev/usb/} world readable.
