\subchapter{Output-only misc driver}{Objective: implement the
  write part of a {\em misc} driver}

After this lab, you will be able to:

\begin{itemize}
\item Write a simple misc driver, allowing to write data to the
  serial ports of your Beaglebone.
\item Write simple \code{file_operations} functions for a device,
  including \code{ioctl} controls.
\item Copy data from user memory space to kernel memory space and
  eventually to the device.
\item You will practice kernel standard error codes a little bit too.
\end{itemize}

You must have completed the previous lab to work on this one. 

\section{Misc driver registration}

In the same way we added an input interface to our Nunchuk driver, it is
now time to give an interface to our serial driver. As our needs are
simple, we won't use the {\em Serial framework} provided by the Linux
kernel, but will use the {\em Misc framework} to implement a simple
character driver.

Let's start by adding the infrastructure to register a {\em misc}
driver.

The first thing to do is to create:

\begin{itemize}
\item An \code{feserial_write()} write file operation stub.
      See the slides or the code for the prototype to use.
      Just place a \code{return -EINVAL;} statement in the function
      body so far, to signal that there is something wrong with this
      function so far.
\item Similarly, an \code{feserial_read()} read file operation stub.
\item A \code{file_operations} structure declaring these file
      operations.
\end{itemize}

The next step is to create a \code{miscdevice} structure and initialize
it. However, we are facing the same usual constraint to handle multiple
devices. Like in the Nunchuk driver, we have to add such a structure
to our device specific private data structure:

\begin{verbatim}
struct feserial_dev {
        struct miscdevice miscdev;
        void __iomem *regs;
};
\end{verbatim}

To be able to access our private data structure in other parts of the
driver, you need to attach it to the \code{pdev} structure using the
\code{platform_set_drvdata()} function. Look for examples in the
source code to find out how to do it.

Now, at the end of the \code{probe()} routine, when the device is fully ready
to work, you can now initialize the \code{miscdevice} structure
for each found device:

\begin{itemize}
\item To get an automatically assigned minor number.
\item To specify a name for the device file in {\em devtmpfs}. We
      propose to use
      \code{kasprintf(GFP_KERNEL, "feserial-\%x", res->start)}.
      \code{kasprintf()} allocates a buffer and runs \code{ksprintf()}
      to fill its contents.
      Don't forget to call \code{kfree()} on this buffer in
      the \code{remove()} function!
\item To pass the file operations structure that you defined.
\end{itemize}

See the lectures for details if needed!

The last things to do (at least to have a {\em misc} driver, even if 
its file operations are not ready yet), are to add the registration and
deregistration routines. That's typically the time when you will need
to access the \code{feserial_dev} structure for each device from the
\code{pdev} structure passed to the \code{remove()} routine.

Make sure that your driver compiles and loads well, and that you
now see two new device files in \code{/dev}.

At this stage, make sure you can load and unload the driver multiple
times. This should reveal registration and deregistration issues if
there are any.

\section{Apply a kernel patch}

The next step is to implement the \code{write()} routine. However, we
will need to access our \code{feserial_dev} structure from inside that
routine. 

At the moment, it is necessary to implement an \code{open} file
operation to attach a private structure to an open device file. This is
a bit ugly as we would have nothing special to do in such a function.

Let's apply a patch that addresses this issue:
\footnote{This patch has been submitted but hasn't been accepted yet in
the mainline kernel, because it breaks the \code{FUSE} code which makes 
weird assumptions about the {\em misc} framework.}

\begin{itemize}
\item Go back to the Linux source directory. Make sure you are still in
      the \code{uart} branch (type \code{git branch}).
\item Run \code{git status} to check whether you have uncommitted
      changes. Commit these if they correspond to useful changes (these 
      should be your Device Tree edits).
\item Apply the new patch using the following command:
      \code{git am ~/felabs/linux/src/patches/0001-char-misc*.patch}
\item Rebuild and update your kernel image and reboot.
\end{itemize}

\section{Implement the write() routine}

Now, add code to your write function, to copy user data to the serial
port, writing characters one by one.

The first thing to do is to retrieve the \code{feserial_dev} structure
from the \code{miscdevice} structure itself, accessible through the
\code{private_data} field of the open file structure (\code{file}).

At the time we registered our {\em misc} device, we didn't keep any
pointeur to the \code{feserial_dev} structure. However, the
\code{miscdevice} structure is accessible, and being a member of the
\code{feserial_dev} structure, we can use a magic macro to compute
the address of the parent structure:

\begin{verbatim}
struct feserial_dev *dev =
	container_of(file->private_data, struct feserial_dev, miscdev);
\end{verbatim}

See \url{http://linuxwell.com/2012/11/10/magical-container_of-macro/}
for interesting implementation details about this macro.

Now, add code that copies (in a secure way) each character from the
user space buffer to the UART device. 

Once done, compile and load your module. Test that your \code{write} function
works properly by using (example for UART2):

\begin{verbatim}
echo "test" > /dev/feserial-48024000
\end{verbatim}

The \code{test} string should appear on the remote side (i.e in
the \code{picocom} process connected to \code{/dev/ttyUSB1}).

If it works, you can triumph and do a victory dance in front of the
whole class!

Make sure that both UART devices work on the same way.

You'll quickly discover that newlines do not work properly. To fix
this, when the user space application sends \verb+"\n"+, you must send
\verb+"\n\r"+ to the serial port.

\section{Going further: ioctl operation}

Do it only if you finish ahead of the crowd!

We would like to maintain a count of the number of characters
written through the serial port. So we need to implement two
\code{unlocked_ioctl()} operations:
\begin{itemize}

 \item \code{SERIAL_RESET_COUNTER}, which as its name says, will
   reset the counter to zero

 \item \code{SERIAL_GET_COUNTER}, which will return the current
   value of the counter in a variable passed by address.

\end{itemize}

Two test applications (in source format) are already available in the
\code{root/serial/} NFS shared directory.  
They assume that \code{SERIAL_RESET_COUNTER} is ioctl operation \code{0}
and that \code{SERIAL_GET_COUNTER} is ioctl operation \code{1}.

Modify their source code according to the exact name of the device file
you wish to use, and compile them on your host:

\begin{verbatim}
arm-linux-gnueabi-gcc -static -o serial-get-counter serial-get-counter.c
\end{verbatim}

The new executables are then ready to run on your target.
