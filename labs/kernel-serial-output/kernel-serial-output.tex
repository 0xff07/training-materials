\subchapter{Output-only misc driver}{Objective: implement the
  write part of a {\em misc} driver}

After this lab, you will be able to:

\begin{itemize}
\item Write a simple misc driver, allowing to write data to the
  serial ports of your Beaglebone.
\item Write simple \code{file_operations} functions for a device,
  including \code{ioctl} controls.
\item Copy data from user memory space to kernel memory space and
  eventually to the device.
\item You will practice kernel standard error codes a little bit too.
\end{itemize}

You must have completed the previous lab to work on this one.

\section{Misc driver registration}

In the same way we added an input interface to our Nunchuk driver, it is
now time to give an interface to our serial driver. As our needs are
simple, we won't use the {\em Serial framework} provided by the Linux
kernel, but will use the {\em Misc framework} to implement a simple
character driver.

Let's start by adding the infrastructure to register a {\em misc}
driver.

The first thing to do is to create:

\begin{itemize}
\item A \code{serial_write()} write file operation stub.
      See the slides or the code for the prototype to use.
      Just place a \code{return -EINVAL;} statement in the function
      body so far, to signal that there is something wrong with this
      function so far.
\item Similarly, a \code{serial_read()} read file operation stub.
\item A \ksym{file_operations} structure declaring these file
      operations.
\end{itemize}

The next step is to create a \ksym{miscdevice} structure and initialize
it. However, we are facing the same usual constraint to handle multiple
devices. Like in the Nunchuk driver, we have to add such a structure
to our device specific private data structure:

\begin{verbatim}
struct serial_dev {
        struct miscdevice miscdev;
        void __iomem *regs;
};
\end{verbatim}

To be able to access our private data structure in other parts of the
driver, you need to attach it to the \code{pdev} structure using the
\kfunc{platform_set_drvdata} function. Look for examples in the
source code to find out how to do it.

Now, at the end of the \code{probe()} routine, when the device is fully ready
to work, you can now initialize the \ksym{miscdevice} structure
for each found device:

\begin{itemize}
\item To get an automatically assigned minor number.
\item To specify a name for the device file in {\em devtmpfs}. We
  propose to use {\tt devm\_kasprintf(\&pdev->dev, GFP\_KERNEL,
  "serial-\%x", res->start)}. \kfunc{devm_kasprintf} allocates a
  buffer and runs \kfunc{kasprintf} to fill its contents.
\item To pass the \ksym{file_operations} structure that you defined.
\item To set the \code{parent} pointer to the appropriate value.
\end{itemize}

See the lectures for details if needed!

The last things to do (at least to have a {\em misc} driver, even if
its file operations are not ready yet), are to add the registration and
deregistration routines. That's typically the time when you will need
to access the \code{serial_dev} structure for each device from the
\code{pdev} structure passed to the \code{remove()} routine.

Make sure that your driver compiles and loads well, and that you
now see two new device files in \code{/dev}.

At this stage, make sure you can load and unload the driver multiple
times. This should reveal registration and deregistration issues if
there are any.

Check in \code{/sys/class/misc} for an entry corresponding to the
registered devices, and within those directories, check what the
\code{device} symbolic link is pointed to. Check the contents of the
\code{dev} file as well, and compare it with the major/minor number of
the device nodes created in \code{/dev} for your devices.

\section{Implement the write() routine}

Now, add code to your write function, to copy user data to the serial
port, writing characters one by one.

The first thing to do is to retrieve the \code{serial_dev} structure
from the \code{miscdevice} structure itself, accessible through the
\code{private_data} field of the open file structure (\code{file}).

At the time we registered our {\em misc} device, we didn't keep any
pointer to the \code{serial_dev} structure. However, as the
\kstruct{miscdevice} structure is accessible through
\code{file->private_data}, and is a member of the
\code{serial_dev} structure, we can use a magic macro to compute
the address of the parent structure:

\begin{verbatim}
struct serial_dev *dev =
	container_of(file->private_data, struct serial_dev, miscdev);
\end{verbatim}

See \url{https://radek.io/2012/11/10/magical-container_of-macro/}
for interesting implementation details about this macro.

This wouldn't have been possible if the \kstruct{miscdevice} structure
was allocated separately and was just referred to by a pointer in
\code{serial_dev}, instead of being a member of it.

Another possibility, but more complicated, would have been to access the
\code{parent} device pointer in \kstruct{miscdevice}, which then through
the \kfunc{platform_get_drvdata} function would
have given us access to the \code{serial_dev} structure containing the
virtual address of the device. There are always multiple possibilities
in kernel programming!

Now, add code that copies (in a secure way) each character from the
user space buffer to the UART device.

Once done, compile and load your module. Test that your \code{write} function
works properly by using (example for UART2):

\begin{verbatim}
echo "test" > /dev/serial-48024000
\end{verbatim}

The \code{test} string should appear on the remote side (i.e in
the \code{picocom} process connected to \code{/dev/ttyUSB1}).

If it works, you can triumph and do a victory dance in front of the
whole class!

Make sure that both UART devices work on the same way.

You'll quickly discover that newlines do not work properly. To fix
this, when the user space application sends \verb+"\n"+, you must send
\verb+"\n\r"+ to the serial port.

\section{Module reference counting}

Start an application in the background that writes nothing to the
UART:

\begin{verbatim}
cat > /dev/serial-48024000 &
\end{verbatim}

Now, with \code{lsmod}, look at the reference count of your module: it
is still 0, even though an application has your device opened. This
means that you can \code{rmmod} your module even when an application
is using it, which is not correct and can quickly cause a kernel
crash.

To fix this, we need to tell the kernel that our module is in charge
of this character device. This is done by setting the \code{.owner}
field of \code{struct file_operations} to the special value
\code{THIS_MODULE}.

After changing this, make sure the reference counter of your module,
visible through \code{lsmod}, gets incremented when you run an
application that uses the UART.

\section{Going further: ioctl operation}

Do it only if you finish ahead of the crowd!

We would like to maintain a count of the number of characters
written through the serial port. So we need to implement two
\code{unlocked_ioctl()} operations:
\begin{itemize}

 \item \code{SERIAL_RESET_COUNTER}, which as its name says, will
   reset the counter to zero

 \item \code{SERIAL_GET_COUNTER}, which will return the current
   value of the counter in a variable passed by address.

\end{itemize}

Two test applications (in source format) are already available in the
\code{root/serial/} NFS shared directory.
They assume that \code{SERIAL_RESET_COUNTER} is ioctl operation \code{0}
and that \code{SERIAL_GET_COUNTER} is ioctl operation \code{1}.

Compile them:

\begin{verbatim}
arm-linux-gnueabi-gcc -static -o serial-get-counter serial-get-counter.c
arm-linux-gnueabi-gcc -static -o serial-reset-counter serial-reset-counter.c
\end{verbatim}

The new executables are then ready to run on your target. They take as
argument the path to the device file corresponding to your UART.
