\subchapter{Kernel compiling and booting}{Objective: compile and boot
a kernel for your board, booting on a directory on your workstation
shared by NFS.}

After this lab, you will be able to:
\begin{itemize}

\item Cross-compile a kernel for the ARM platform

\item Boot this kernel on an NFS root filesystem, which is somewhere
on your development workstation\footnote{NFS root filesystems are
particularly useful to compile modules on your host, and make them
directly visible on the target. You no longer have to update the root
filesystem by hand and transfer it to the target (requiring a shutdown
and reboot).}

\end{itemize}

\section{Lab implementation}

While developing a kernel module, the developer wants to change the
source code, compile and test the new kernel module very
frequently. While writing and compiling the kernel module is done the
development workstation, the test of the kernel module usually has to
be done on the target, since it might interact with hardware specific
to the target.

However, flashing the root filesystem on the target for every test is
time-consuming and would use the flash chip needlessly.

Fortunately, it is possible to set up networking between the
development workstation and the target. Then, workstation files can be
accessed through the network by the target, using NFS.

\begin{center}
\includegraphics[width=\textwidth]{labs/kernel-module-environment/host-vs-target.pdf}
\end{center}

\section{Setup}

Go to the \code{$HOME/felabs/linux/src/linux} directory.

Install packages needed for configuring, compiling and booting
a kernel for your board:

\begin{verbatim}
sudo apt-get install libqt4-dev g++ u-boot-tools
\end{verbatim}

\code{libqt4-dev} and \code{g++} are needed for \code{make xconfig}.
\code{u-boot-tools} is needed to build the \code{uImage} file for
U-boot (\code{mkimage} utility).

\section{Apply a kernel patch for BeagleBone Black}

At the time of this writing, the mainline Linux 3.11 kernel is missing a 
specific Device Tree Source (DTS) file for the BeagleBone Black.
Without it, you could use the DTS for the original Beagle Bone, but it
would blow up the HDMI transceiver after a dozen boots \footnote{See
\url{http://article.gmane.org/gmane.linux.kernel.stable/63648} for
details.}. 

We are providing a patch to add such a DTS file. Let's create 
a branch, starting from our \code{3.11.y} branch, and including this
patch.

First, make sure you are in the \code{3.11.y} branch:
\begin{verbatim}
git branch
\end{verbatim}

Then, let's create the new branch and apply the patch for BeagleBone
Black:
\begin{verbatim}
git checkout -b 3.11.y-bbb
git am ../patches/*.patch
\end{verbatim}
 
\section{Cross-compiling toolchain setup}

We are going to install a cross-compiling toolchain from Linaro, a
very popular source for ARM toolchains (amongst other useful resources
for Linux on ARM).

\begin{verbatim}
sudo apt-get install gcc-arm-linux-gnueabi
\end{verbatim}

Now find out the path and name of the cross-compiler executable by looking at the contents of the package:

\begin{verbatim}
dpkg -L gcc-arm-linux-gnueabi
\end{verbatim}

\section{Kernel configuration}

Set the \code{ARCH} and \code{CROSS_COMPILE} definitions for the \code{arm}
platform and to use your cross-compiler.

To reuse these settings in the future and in different terminals,
we advise you to create an \code{env.sh} file in the parent directory,
with the below contents:

\begin{verbatim}
export ARCH=arm
export CROSS_COMPILE=<cross-compiler-prefix>
\end{verbatim}

You can the load these definitions at any time by sourcing this file:

\begin{verbatim}
source ../env.sh
\end{verbatim}
 
Configure this kernel with the ready-made configuration for boards in
the OMAP2 and later family which the AM335x found in the BeagleBone
belongs to.

Make sure that this configuration has \code{CONFIG_ROOT_NFS=y} (support
booting on an NFS exported root directory).

Compile your kernel and generate the \code{uImage} kernel image that
U-boot needs (the U-boot bootloader needs the kernel \code{zImage}
file to be encapsulated in a special container and the kernel
\code{Makefile} can generate this container for you by running the
\code{mkimage} tool found in the \code{uboot-mkimage} package).

This file will contain, among other things, the load address of the
kernel. Nowadays, the kernel can boot on several platforms, each with
different load addresses, that makes the use of \code{uImage} not very
convenient. So, if the default load address doesn't work for you, you'll
need to specify it during the generation of \code{uImage} using the
\code{LOADADDR} environment variable.

\begin{verbatim}
make LOADADDR=0x80008000 uImage
\end{verbatim}

Now also generate the Device Tree Binaries (DTBs):

\begin{verbatim}
make dtbs
\end{verbatim}

Now, copy the \code{uImage} and \code{am335x-boneblack.dtb} files
to the TFTP server home directory (\code{/var/lib/tftpboot}).

\section{Setting up the NFS server}

Install the NFS server by installing the \code{nfs-kernel-server}
package. Once installed, edit the \code{/etc/exports} file as
\code{root} to add the following lines, assuming that the IP address
of your board will be \code{192.168.0.100}:

\scriptsize
\begin{verbatim}
/home/<user>/felabs/linux/modules/nfsroot 192.168.0.100(rw,no_root_squash,no_subtree_check)
/home/<user>/felabs/linux/character/nfsroot 192.168.0.100(rw,no_root_squash,no_subtree_check)
/home/<user>/felabs/linux/debugging/nfsroot 192.168.0.100(rw,no_root_squash,no_subtree_check)
\end{verbatim}
\normalsize

Then, restart the NFS server:

\begin{verbatim}
sudo /etc/init.d/nfs-kernel-server restart
\end{verbatim}

If there is any error message, this usually means that there was a
syntax error in the \code{/etc/exports} file. Don't proceed until these
errors disappear.

\section{Boot the system}

First, boot the board to the U-Boot prompt. Before booting the kernel,
we need to tell it which console to use and that the root filesystem
should be mounted over NFS, by setting some kernel parameters.

Do this by setting U-boot's \code{bootargs} environment variable:
Modify the \code{uEnv.txt} file by adding the below line (in just 1 line):

\begin{verbatim}
setenv bootargs root=/dev/nfs ip=192.168.0.100 console=ttyO0
  nfsroot=192.168.0.1:/home/<user>/felabs/linux/modules/nfsroot
\end{verbatim}

Of course, you need to adapt the IP addresses to your exact network
setup. Now, download the kernel image through \code{tftp}:

\begin{verbatim}
tftp 0x81000000 uImage
\end{verbatim}

You'll also need to download the device tree blob:

\begin{verbatim}
tftp 0x82000000 am335x-boneblack.dtb
\end{verbatim}

Now, boot your kernel:

\begin{verbatim}
bootm 0x81000000 - 0x82000000
\end{verbatim}

If everything goes right, you should reach a shell prompt. Otherwise,
check your setup or ask your instructor for details.

If the kernel fails to mount the NFS filesystem, look carefully at the
error messages in the console. If this doesn't give any clue, you can
also have a look at the NFS server logs in \code{/var/log/syslog}.

\section{Automate the boot process}

To avoid typing the same U-boot commands over and over again each time
you power on or reset your board, here's a way to automatically call the
above commands at boot time:

Modify your \code{uEnv.txt} file and add the below lines:
{\scriptsize
\begin{verbatim}
bootargs=root=/dev/nfs ip=192.168.0.100 console=ttyO0 nfsroot=192.168.0.1:/home/<user>/felabs/linux/modules/nfsroot
uenvcmd=tftp 0x81000000 uImage; tftp 0x82000000 am335x-boneblack.dtb; bootm 0x81000000 - 0x82000000
\end{verbatim}
}

\code{uenvcmd} will automatically be run after the U-boot timeout
expires and the \code{uEnv.txt} settings are loaded.
Don't hesitate to change it according to your exact needs.

We could also copy the \code{uImage} file to the eMMC flash and avoid
downloading it over and over again. However, handling flash is outside
of the scope of this course. See our
\href{http://free-electrons.com/training/embedded-linux/}{Embedded
Linux system development course} and its on-line materials for
details.

