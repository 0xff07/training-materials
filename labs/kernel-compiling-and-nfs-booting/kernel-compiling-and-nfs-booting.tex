\subchapter{Kernel compiling and booting}{Objective: compile and boot
a kernel for your board, booting on a directory on your workstation
shared by NFS.}

After this lab, you will be able to:
\begin{itemize}

\item Cross-compile the Linux kernel for the ARM platform.

\item Boot this kernel on an NFS root filesystem, which is somewhere
on your development workstation\footnote{NFS root filesystems are
particularly useful to compile modules on your host, and make them
directly visible on the target. You no longer have to update the root
filesystem by hand and transfer it to the target (requiring a shutdown
and reboot).}.

\end{itemize}

\section{Lab implementation}

While developing a kernel module, the developer wants to change the
source code, compile and test the new kernel module very
frequently. While writing and compiling the kernel module is done on the
development workstation, the test of the kernel module usually has to
be done on the target, since it might interact with hardware specific
to the target.

However, flashing the root filesystem on the target for every test is
time-consuming and would use the flash chip needlessly.

Fortunately, it is possible to set up networking between the
development workstation and the target. Then, workstation files can be
accessed through the network by the target, using NFS.

\begin{center}
\includegraphics[width=\textwidth]{labs/kernel-compiling-and-nfs-booting/host-vs-target.pdf}
\end{center}

\section{Setup}

Go to the \code{$HOME/linux-kernel-labs/src/linux} directory.

Install packages needed for configuring, compiling and booting
the kernel for your board:

\begin{verbatim}
sudo apt install qt5-default g++ pkg-config libssl-dev
\end{verbatim}

\code{qt5-default} and \code{g++} are needed for \code{make xconfig}.

\section{Cross-compiling toolchain setup}

We are going to install a cross-compiling toolchain from Linaro, a
very popular source for ARM toolchains (amongst other useful resources
for Linux on ARM).

\begin{verbatim}
sudo apt install gcc-arm-linux-gnueabi
\end{verbatim}

Now find out the path and name of the cross-compiler executable by looking at the contents of the package:

\begin{verbatim}
dpkg -L gcc-arm-linux-gnueabi
\end{verbatim}

\section{Kernel configuration}

Configure your kernel sources with the ready-made configuration for boards in
the OMAP2 and later family which the AM335x found in the BeagleBone
belongs to. Don't forget to set the \code{ARCH} and
\code{CROSS_COMPILE} definitions for the \code{arm} platform and to
use your cross-compiler.

For the BeagleBone Black Wireless board, add the below options:
\begin{itemize}
  \item \code{CONFIG_USB_GADGET=y}
  \item \code{CONFIG_USB_MUSB_HDRC} {\em Driver for the USB OTG
        controller}
  \item \code{CONFIG_USB_MUSB_GADGET} {\em Use the USB OTG controller
	in device (gadget) mode} 
  \item \code{CONFIG_MUSB_DSPS}=y
  \item Check the dependencies of \code{CONFIG_AM335X_PHY_USB}
        and find the way to set \code{CONFIG_AM335X_PHY_USB=y}
  \item Find the "USB Gadget precomposed configurations" menu
        and set it to {\em static} instead of {\em module}
	so that \code{CONFIG_USB_ETH=y}
\end{itemize}

Make sure that this configuration has \code{CONFIG_ROOT_NFS=y} (support
booting on an NFS exported root directory).

\section{Kernel compiling}

Compile your kernel and generate the Device Tree Binaries (DTBs)
(running 4 compile jobs in parallel):

\begin{verbatim}
make -j 4 
\end{verbatim}

Now, copy the \code{zImage} and \code{am335x-boneblack.dtb}
or \code{am335x-boneblack-wireless.dtb} files
to the TFTP server home directory (\code{/var/lib/tftpboot}).

\section{Setting up the NFS server}

Install the NFS server by installing the \code{nfs-kernel-server}
package. Once installed, edit the \code{/etc/exports} file as
\code{root} to add the following lines, assuming that the IP address
of your board will be \code{192.168.0.100}:

\scriptsize
\begin{verbatim}
/home/<user>/linux-kernel-labs/modules/nfsroot 192.168.0.100(rw,no_root_squash,no_subtree_check)
\end{verbatim}
\normalsize

Make sure that the path and the options are on the same line.
Also make sure that there is no space between the IP address and the NFS
options, otherwise default options will be used for this IP address,
causing your root filesystem to be read-only.

Then, restart the NFS server:

\begin{verbatim}
sudo /etc/init.d/nfs-kernel-server restart
\end{verbatim}

If there is any error message, this usually means that there was a
syntax error in the \code{/etc/exports} file. Don't proceed until these
errors disappear.

\section{Boot the system}

First, boot the board to the U-Boot prompt. Before booting the kernel,
we need to tell it which console to use and that the root filesystem
should be mounted over NFS, by setting some kernel parameters.

Do this by setting U-boot's \code{bootargs} environment variable (all in
just one line, pay attention to the \code{O} character, like "OMAP", in
\code{ttyO0}):

For the BeagleBone Back board:
\begin{verbatim}
setenv bootargs root=/dev/nfs rw ip=192.168.0.100 console=ttyO0
  nfsroot=192.168.0.1:/home/<user>/linux-kernel-labs/modules/nfsroot
\end{verbatim}

For the BeagleBone Back Wireless board:
\begin{verbatim}
setenv bootargs root=/dev/nfs rw ip=192.168.0.100:::::usb0
  g_ether.dev_addr=11:22:33:44:55:02 g_ether.host_addr=11:22:33:44:55:01
  console=ttyO0,115200n8
  nfsroot=192.168.0.1:/home/<user>/linux-kernel-labs/modules/nfsroot
\end{verbatim}

Now save this definition:
\begin{verbatim}
saveenv
\end{verbatim}

If you later want to make changes to this setting, you can type the
below command in U-boot:

\begin{verbatim}
editenv bootargs
\end{verbatim}

Now, download the kernel image through \code{tftp}:

\begin{verbatim}
tftp 0x81000000 zImage
\end{verbatim}

You'll also need to download the device tree blob:

\begin{verbatim}
tftp 0x82000000 <board>.dtb
\end{verbatim}

Now, boot your kernel:

\begin{verbatim}
bootz 0x81000000 - 0x82000000
\end{verbatim}

If everything goes right, you should reach a login prompt (user:
\code{root}, no password). Otherwise, check your setup and
ask your instructor for support if you are stuck.

If the kernel fails to mount the NFS filesystem, look carefully at the
error messages in the console. If this doesn't give any clue, you can
also have a look at the NFS server logs in \code{/var/log/syslog}.

\section{Checking the kernel version}

It's often a good idea to make sure you booted the right kernel.
By mistake, you could have booted a kernel previously stored in flash
(typically through a default boot command in U-Boot), or forgotten to 
update the kernel image in \code{/var/lib/tftpboot}.

This could explain some unexpected behaviour.

There are two ways of checking your kernel version:
\begin{itemize}
\item By looking at the first kernel messages
\item By running the \code{uname -a} command after booting Linux. 
\end{itemize}

In both cases, you will not only know the kernel version, but also
the date when the kernel was compiled and the name of the user who
did it.

Similarly, you can also check the command line actually received by
the kernel, either by looking at the first boot messages, or once you
have reached a command line shell, by running \code{cat /proc/cmdline}.

\section{Automate the boot process}

To avoid typing the same U-boot commands over and over again each time
you power on or reset your board, you can use U-Boot's \code{bootcmd}
environment variable:

{\scriptsize
\begin{verbatim}
setenv bootcmd 'tftp 0x81000000 zImage; tftp 0x82000000 <board>.dtb; bootz 0x81000000 - 0x82000000'
saveenv
\end{verbatim}
}

Don't hesitate to change it according to your exact needs.

We could also copy the \code{zImage} file to the eMMC flash and avoid
downloading it over and over again. However, detailed bootloader
usage is outside of the scope of this course. See our
\href{http://bootlin.com/training/embedded-linux/}{Embedded
Linux system development course} and its on-line materials for
details.
