\subchapter{Third party libraries and applications}{Objective: Learn
  how to leverage existing libraries and applications: how to
  configure, compile and install them}

To illustrate how to use existing libraries and applications, we will
extend the small root filesystem built in the {\em A tiny embedded
  system} lab to add the {\em ALSA} libraries and tools and an audio
playback application using these libraries.

We'll see that manually re-using existing libraries is quite tedious,
so that more automated procedures are necessary to make it
easier. However, learning how to perform these operations manually
will significantly help you when you'll face issues with more
automated tools.

\section{Audio support in the Kernel}

Recompile your kernel with audio support. The options we want are:
\code{CONFIG_SOUND}, \code{CONFIG_SND_USB} and
\code{CONFIG_SND_USB_AUDIO}.

Boot your board with this new kernel.

\section{Figuring out library dependencies}

We're going to integrate the \code{alsa-utils} and \code{ogg123}. As
most software components, they in turn depend on other libraries, and
these dependencies are different depending on the configuration chosen
for them. In our case, the dependency chain for the \code{alsa-utils}
is quite simple, it only depends on the \code{alsa-lib} library.

The dependencies are a bit more complex for \code{ogg123}. It is part
of the \code{vorbis-tools}, that depend on \code{libao} and
\code{libvorbis}. \code{libao} in turn depend on \code{alsa-lib}, and
\code{libvorbis} on \code{libogg}.

\code{libao}, the \code{alsa-utils} and the \code{alsa-lib} are here
to abstract the use of \code{ALSA}, one of the Audio Subsystems
supported in Linux. The \code{vorbis-tools}, \code{libvorbis} and
\code{libogg} are used to handle the audio files encoded using the Ogg
codec, which is quite common.

So, we end up with the following dependency tree:

\begin{center}
\includegraphics[width=\textwidth]{labs/sysdev-thirdparty/alsa-dependencies.pdf}
\end{center}

Of course, all these libraries rely on the C library, which is not
mentioned here, because it is already part of the root filesystem
built in the {\em A tiny embedded system} lab. You might wonder how to
figure out this dependency tree by yourself. Basically, there are
several ways, that can be combined:

\begin{itemize}
\item Read the library documentation, which often mentions the
  dependencies;
\item Read the help message of the configure script (by running
  \code{./configure --help}).
\item By running the configure script, compiling and looking at the
  errors.
\end{itemize}

To configure, compile and install all the components of our system,
we're going to start from the bottom of the tree with {\em alsa-lib},
then continue with the {\em alsa-utils}, \emph{libao}, {\em libogg},
and \emph{libvorbis}, to finally compile the \emph{vorbis-tools}.

\section{Preparation}

For our cross-compilation work, we will need to separate spaces:
\begin{itemize}
\item A \emph{staging} space in which we will directly install all the
  packages: non-stripped versions of the libraries, headers,
  documentation and other files needed for the compilation. This
  \emph{staging} space can be quite big, but will not be used on our
  target, only for compiling libraries or applications;
\item A \emph{target} space, in which we will copy only the required
  files from the \emph{staging} space: binaries and libraries, after
  stripping, configuration files needed at runtime, etc. This target
  space will take a lot less space than the \emph{staging} space, and
  it will contain only the files that are really needed to make the
  system work on the target.
\end{itemize}

To sum up, the {\em staging} space will contain everything that's
needed for compilation, while the {\em target} space will contain only
what's needed for execution.

So, in \code{$HOME/felabs-sysdev/thirdparty}, create two
directories: \code{staging} and \code{target}.

For the target, we need a basic system with BusyBox, device nodes and
initialization scripts. We will re-use the system built in the {\em A
  tiny embedded system} lab, so copy this system in the target
directory:

\begin{verbatim}
sudo cp -a $HOME/felabs-sysdev/tinysystem/nfsroot/* target/
\end{verbatim}

The copy must be done as \code{root}, because the root filesystem of
the {\em A tiny embedded system} lab contains a few device nodes.

\section{Testing}

Make sure the \code{target/} directory is exported by your NFS server
to your board by modifying \code{/etc/exports} and restarting your NFS
server.

Make sure your board boots as it used to and that you can get a
prompt.

\section{alsa-lib}

\code{alsa-lib} is a library supposed to handle the interaction with
the ALSA subsystem. It is available at
\url{http://alsa-project.org}. Download version 1.0.28, and extract it
in \code{$HOME/felabs-sysdev/thirdparty/}.

By looking at the \code{configure} script, we see that this configure
script has been generated by \code{autoconf} (the header contain a
sentence like {\em Generated by GNU Autoconf 2.62}). Most of the time,
\code{autoconf} comes with \code{automake}, that generates Makefiles
from Makefile.am files. So the alsa-lib uses a rather common build
system. Let's try to configure and build alsa-lib:

\begin{verbatim}
./configure
make
\end{verbatim}

You can see that the files are getting compiled with gcc, which
generates code for x86 and not for the target platform. This is
obviously not what we want, so we clean-up the object and tell the
configure script to use the ARM cross-compiler:

\begin{verbatim}
make clean
CC=arm-linux-gcc ./configure
\end{verbatim}

Of course, the \code{arm-linux-gcc} cross-compiler must be in your
PATH prior to running the configure script. The CC environment
variable is the classical name for specifying the compiler to
use.

Quickly, you should get an error saying:

\footnotesize
\begin{verbatim}
checking whether we are cross compiling... configure: error: in `.../thirdparty/alsa-lib-1.0.28':
configure: error: cannot run C compiled programs.
If you meant to cross compile, use `--host'.
See `config.log' for more details
\end{verbatim}
\normalsize

If you look at the config.log file, you can see that the configure
script compiles a binary with the cross-compiler and then tries to run
it on the development workstation. This is a rather usual thing to do
for a configure script, and that's why it tests so early that it's
actually doable, and bails out if not.

Obviously, it cannot work in our case, and the scripts exits. The job
of the configure script is to test the configuration of the system. To
do so, it tries to compile and run a few sample applications to test
if this library is available, if this compiler option is supported,
etc. But in our case, running the test examples is definitely not
possible.

We need to tell the configure script that we are cross-compiling, and
this can be done using the \code{--build} and \code{--host} options,
as described in the help of the configure script:

\begin{verbatim}
System types:
  --build=BUILD	configure for building on BUILD [guessed]
  --host=HOST	cross-compile to build programs to run on HOST [BUILD]
\end{verbatim}

The \code{--build} option allows to specify on which system the
package is built, while the \code{--host} option allows to specify on
which system the package will run. By default, the value of the
\code{--build} option is guessed and the value of \code{--host} is the
same as the value of the \code{--build} option. The value is guessed
using the \code{./config.guess} script, which on your system should
return \code{i686-pc-linux-gnu}. See
\url{http://www.gnu.org/software/autoconf/manual/html_node/Specifying-Names.html}
for more details on these options.

So, let's override the value of the \code{--host} option:

\begin{verbatim}
CC=arm-linux-gcc ./configure --host=arm-linux
\end{verbatim}

The configure script should end properly now, and create a
Makefile. Run the \code{make} command, and you should see that at some
point it is failing because of a missing python header.

You can get rid of this error by disabling the python support in the
configure script by using the \code{--disable-python} option. Start
make again, and this time you should have the compilation ending fine.

Look at the result of compiling in src/.libs: a set of object files
and a set of \code{libasound.so*} files.

The \code{libasound.so*} files are a dynamic version of the
library. The shared library itself is \code{libasound.so.2.0.0}, it has
been generated by the following command line:

\begin{verbatim}
arm-linux-gcc -shared conf.o confmisc.o input.o output.o \
              async.o error.o dlmisc.o socket.o shmarea.o \
              userfile.o names.o -lm -ldl -lpthread -lrt  \
              -Wl,-soname -Wl,libasound.so.2 -o libasound.so.2.0.0
\end{verbatim}

And creates the symbolic link \code{libasound.so} and
\code{libasound.so.2}.

\begin{verbatim}
ln -s libasound.so.2.0.0 libasound.so.2
ln -s libasound.so.2.0.0 libasound.so
\end{verbatim}

These symlinks are needed for two different reasons:

\begin{itemize}
\item \code{libasound.so} is used at compile time when you want to
  compile an application that is dynamically linked against the
  library. To do so, you pass the \code{-lLIBNAME} option to the
  compiler, which will look for a file named
  \code{lib<LIBNAME>.so}. In our case, the compilation option is
  \code{-lasound} and the name of the library file is
  \code{libasound.so}. So, the \code{libasound.so} symlink is needed
  at compile time;
\item \code{libasound.so.2} is needed because it is the {\em SONAME}
  of the library. {\em SONAME} stands for {\em Shared Object Name}. It
  is the name of the library as it will be stored in applications
  linked against this library. It means that at runtime, the dynamic
  loader will look for exactly this name when looking for the shared
  library. So this symbolic link is needed at runtime.
\end{itemize}

To know what's the {\em SONAME} of a library, you can use:
\begin{verbatim}
arm-linux-readelf -d libasound.so.2.0.0
\end{verbatim}

and look at the \code{(SONAME)} line. You'll also see that this
library needs the C library, because of the \code{(NEEDED)} line on
\code{libc.so.0}.

The mechanism of \code{SONAME} allows to change the library without
recompiling the applications linked with this library. Let's say that
a security problem is found in the alsa-lib release that provides
libasound 2.0.0, and fixed in the next alsa-lib release, which will
now provide libasound 2.0.1.

You can just recompile the library, install it on your target system,
change the link \code{libasound.so.2} so that it points to
\code{libasound.so.2.0.1} and restart your applications. And it will
work, because your applications don't look specifically for
\code{libasound.so.2.0.0} but for the {\em SONAME}
\code{libasound.so.2}.

However, it also means that as a library developer, if you break the
ABI of the library, you must change the {\em SONAME}: change from
\code{libasound.so.2} to \code{libasound.so.3}.

Finally, the last step is to tell the configure script where the
library is going to be installed. Most configure scripts consider that
the installation prefix is \code{/usr/local/} (so that the library is
installed in \code{/usr/local/lib}, the headers in
\code{/usr/local/include}, etc.). But in our system, we simply want
the libraries to be installed in the \code{/usr} prefix, so let's tell
the configure script about this:

\begin{verbatim}
CC=arm-linux-gcc ./configure --host=arm-linux --disable-python --prefix=/usr
make
\end{verbatim}

For this library, this option may not change anything to the resulting
binaries, but for safety, it is always recommended to make sure that
the prefix matches where your library will be running on the target
system.

Do not confuse the {\em prefix} (where the application or library will
be running on the target system) from the location where the
application or library will be installed on your host while building
the root filesystem.

For example, libasound will be installed in
\code{$HOME/felabs-sysdev/thirdparty/target/usr/lib/} because this is
the directory where we are building the root filesystem, but once our
target system will be running, it will see libasound in
\code{/usr/lib}.

The prefix corresponds to the path in the target system and {\bf
  never} on the host. So, one should {\bf never} pass a prefix like
\code{$HOME/felabs-sysdev/thirdparty/target/usr}, otherwise at
runtime, the application or library may look for files inside this
directory on the target system, which obviously doesn't exist! By
default, most build systems will install the application or library in
the given prefix (\code{/usr} or \code{/usr/local}), but with most
build systems (including {\em autotools}), the installation prefix can
be overriden, and be different from the configuration prefix.

We now only have the installation process left to do.

First, let's make the installation in the {\em staging} space:
\begin{verbatim}
make DESTDIR=$HOME/felabs-sysdev/thirdparty/staging install
\end{verbatim}

Now look at what has been installed by alsa-lib:
\begin{itemize}
\item Some configuration files in \code{/usr/share/alsa}
\item The headers in \code{/usr/include}
\item The shared library and its libtool file in \code{/usr/lib}
\item A pkgconfig file in \code{/usr/lib/pkgconfig}. We'll come back
  to these later
\end{itemize}

Finally, let's install the library in the {\em target} space:

\begin{enumerate}
\item Create the \code{target/usr/lib} directory, it will contain the
  stripped version of the library
\item Copy the dynamic version of the library. Only
  \code{libasound.so.2} and \code{libasound.so.2.0.0} are needed,
  since \code{libasound.so.2} is the {\em SONAME} of the library and
  \code{libasound.so.2.0.0} is the real binary:
  \begin{itemize}
  \item \code{cp -a src/.libs/libasound.so.2* ../target/usr/lib}
  \end{itemize}
\item Strip the library:
  \begin{itemize}
  \item \code{arm-linux-strip ../target/usr/lib/libasound.so.2.0.0}
  \end{itemize}
\end{enumerate}

And we're done with alsa-lib!

\section{Alsa-utils}

Download alsa-utils from the ALSA offical webpage. We tested the lab
with version 1.0.28.

Once uncompressed, we quickly discover that the libpng build system is
based on the {\em autotools}, so we will work once again with a
regular configure script.

As we've seen previously, we will have to provide the prefix and host
options and the CC variable:

\begin{verbatim}
CC=arm-linux-gcc ./configure --host=arm-linux --prefix=/usr
\end{verbatim}

Now, we should quiclky get an error in the execution of the configure
script:

\begin{verbatim}
checking for libasound headers version >= 1.0.27... not present.
configure: error: Sufficiently new version of libasound not found.
\end{verbatim}

Again, we can check in config.log what the configure script is trying
to do:

\footnotesize
\begin{verbatim}
configure:7130: checking for libasound headers version >= 1.0.27
configure:7192: arm-linux-gnueabihf-gcc -c -g -O2  conftest.c >&5
conftest.c:15:28: fatal error: alsa/asoundlib.h: No such file or directory
\end{verbatim}
\normalsize

Of course, since {\em alsa-utils} uses the {\em alsa-lib}, it includes
its header file! So we need to tell the C compiler where the headers
can be found: there are not in the default directory
\code{/usr/include/}, but in the \code{/usr/include} directory of our
{\em staging} space. The help text of the configure script says:

\begin{verbatim}
  CPPFLAGS              C/C++/Objective C preprocessor flags,
                        e.g. -I<include dir> if you have headers
                        in a nonstandard directory <include dir>
\end{verbatim}

Let's use it:

\begin{verbatim}
CPPFLAGS=-I$HOME/felabs-sysdev/thirdparty/staging/usr/include \
CC=arm-linux-gcc \
./configure --host=arm-linux --prefix=/usr
\end{verbatim}

Now, it should stop a bit later, this time with the error:
\begin{verbatim}
checking for libasound headers version >= 1.0.27... found.
checking for snd_ctl_open in -lasound... no
configure: error: No linkable libasound was found.
\end{verbatim}

The configure script tries to compile an application against {\em
  libasound} (as can be seen from the \code{-lasound} option): {\em
  alsa-utils} uses the {\em alsa-lib}, so the \code{configure} script
wants to make sure this library is already installed. Unfortunately,
the \code{ld} linker doesn't find this library. So, let's tell the
linker where to look for libraries using the \code{-L} option followed
by the directory where our libraries are (in
\code{staging/usr/lib}). This \code{-L} option can be passed to the
linker by using the \code{LDFLAGS} at configure time, as told by the
help text of the configure script:

\begin{verbatim}
  LDFLAGS       linker flags, e.g. -L<lib dir> if you have
                libraries in a nonstandard directory <lib dir>
\end{verbatim}

Let's use this \code{LDFLAGS} variable:

\begin{verbatim}
LDFLAGS=-L$HOME/felabs-sysdev/thirdparty/staging/usr/lib \
CPPFLAGS=-I$HOME/felabs-sysdev/thirdparty/staging/usr/include \
CC=arm-linux-gcc \
./configure --host=arm-linux
\end{verbatim}

Let's also specify the prefix, so that the library is compiled to be
installed in \code{/usr} and not \code{/usr/local}:

\begin{verbatim}
LDFLAGS=-L$HOME/felabs-sysdev/thirdparty/staging/usr/lib \
CPPFLAGS=-I$HOME/felabs-sysdev/thirdparty/staging/usr/include \
CC=arm-linux-gcc \
./configure --host=arm-linux --prefix=/usr
\end{verbatim}

Once again, it should fail a bit further down the tests, this time
complaining about the panelw library missing. This library is part of
\code{ncurses}, a graphical framework to design UIs in the
terminal. This is only used for one of the tools provided by
alsa-utils, alsamixer, that we are not going to use. Hence, we can
just disable the build of alsamixer.

Of course, if we wanted it, we would have had to build ncurses first,
just like we built alsa-lib. We will also need to disable the xmlto
support that generates the documentation.

\begin{verbatim}
LDFLAGS=-L$HOME/felabs-sysdev/thirdparty/staging/usr/lib \
CPPFLAGS=-I$HOME/felabs-sysdev/thirdparty/staging/usr/include \
CC=arm-linux-gcc \
./configure --host=arm-linux --prefix=/usr \
--disable-alsamixer --disable-xmlto
\end{verbatim}

Then, run the compilation with make. Hopefully, it works!

Let's now begin the installation process.  Before really installing in
the staging directory, let's install in a dummy directory, to see
what's going to be installed (this dummy directory will not be used
afterwards, it is only to verify what will be installed before
polluting the staging space):

\begin{verbatim}
make DESTDIR=/tmp/alsa-utils/ install
\end{verbatim}

The \code{DESTDIR} variable can be used with all Makefiles based on
automake. It allows to override the installation directory: instead of
being installed in the configuration-prefix, the files will be
installed in \code{DESTDIR/configuration-prefix}.

Now, let's see what has been installed in \code{/tmp/alsa-utils/}:

\begin{verbatim}
./lib/udev/rules.d/90-alsa-restore.rules
./usr/bin/aseqnet
./usr/bin/aseqdump
./usr/bin/arecordmidi
./usr/bin/aplaymidi
./usr/bin/aconnect
./usr/bin/alsaloop
./usr/bin/speaker-test
./usr/bin/iecset
./usr/bin/aplay
./usr/bin/amidi
./usr/bin/amixer
./usr/bin/alsaucm
./usr/sbin/alsaconf
./usr/sbin/alsactl
./usr/share/sounds/alsa/Side_Left.wav
./usr/share/sounds/alsa/Rear_Left.wav
./usr/share/sounds/alsa/Noise.wav
./usr/share/sounds/alsa/Front_Right.wav
./usr/share/sounds/alsa/Front_Center.wav
./usr/share/sounds/alsa/Side_Right.wav
./usr/share/sounds/alsa/Rear_Right.wav
./usr/share/sounds/alsa/Rear_Center.wav
./usr/share/sounds/alsa/Front_Left.wav
./usr/share/locale/ru/LC_MESSAGES/alsaconf.mo
./usr/share/locale/ja/LC_MESSAGES/alsaconf.mo
./usr/share/locale/ja/LC_MESSAGES/alsa-utils.mo
./usr/share/locale/fr/LC_MESSAGES/alsa-utils.mo
./usr/share/locale/de/LC_MESSAGES/alsa-utils.mo
./usr/share/man/fr/man8/alsaconf.8
./usr/share/man/man8/alsaconf.8
./usr/share/man/man1/aseqnet.1
./usr/share/man/man1/aseqdump.1
./usr/share/man/man1/arecordmidi.1
./usr/share/man/man1/aplaymidi.1
./usr/share/man/man1/aconnect.1
./usr/share/man/man1/alsaloop.1
./usr/share/man/man1/speaker-test.1
./usr/share/man/man1/iecset.1
./usr/share/man/man1/aplay.1
./usr/share/man/man1/amidi.1
./usr/share/man/man1/amixer.1
./usr/share/man/man1/alsactl.1
./usr/share/alsa/speaker-test/sample_map.csv
./usr/share/alsa/init/ca0106
./usr/share/alsa/init/hda
./usr/share/alsa/init/test
./usr/share/alsa/init/info
./usr/share/alsa/init/help
./usr/share/alsa/init/default
./usr/share/alsa/init/00main
\end{verbatim}

So, we have:
\begin{itemize}
\item The udev rules in \code{lib/udev}
\item The alsa-utils binaries in \code{/usr/bin} and \code{/usr/sbin}
\item Some sound samples in \code{/usr/share/sounds}
\item The various translations in \code{/usr/share/locale}
\item The manual pages in \code{/usr/share/man/}, explaining how to
  use the various tools
\item Some configuration samples in \code{/usr/share/alsa}.
\end{itemize}

Now, let's make the installation in the {\em staging} space:

\begin{verbatim}
make DESTDIR=$HOME/felabs-sysdev/thirdparty/staging/ install
\end{verbatim}

Then, let's install only the necessary files in the {\em target}
space, manually:

\begin{verbatim}
cp -a staging/usr/bin/a* staging/usr/bin/speaker-test target/usr/bin/
cp -a staging/usr/sbin/alsa* target/usr/sbin
arm-linux-strip target/usr/bin/a*
arm-linux-strip target/usr/bin/speaker-test
arm-linux-strip target/usr/sbin/alsactl

mkdir -p target/usr/share/alsa/pcm
cp -a staging/usr/share/alsa/alsa.conf* target/usr/share/alsa
cp -a staging/usr/share/alsa/cards target/usr/share/alsa
cp -a staging/usr/share/alsa/pcm/default.conf target/usr/share/alsa/pcm
\end{verbatim}

And we're finally done with the alsa-utils!

Now test that all is working fine by running the speaker-test utils on
your board, with the headset provided by your instructor plugged
in. You will need to add the missing libraries from the toolchain
install directory.

\section{libogg}

Now, let's work on {\em libogg}. Download the version 1.3.2 from
\url{http://xiph.org} and extract it.

Configuring {\em libogg} is very similar to the configuration of the
previous libraries:

\begin{verbatim}
CC=arm-linux-gcc ./configure --host=arm-linux --prefix=/usr
\end{verbatim}

Of course, compile the library:

\begin{verbatim}
make
\end{verbatim}

Installation to the {\em staging} space can be done using the
classical \code{DESTDIR} mechanism:

\begin{verbatim}
make DESTDIR=$HOME/felabs-sysdev/thirdparty/staging/ install
\end{verbatim}

And finally, install manually the only needed files at runtime in the
{\em target} space:

\begin{verbatim}
cd ..
cp -a staging/usr/lib/libogg.so.0* target/usr/lib/
arm-linux-strip target/usr/lib/libogg.so.0.8.2
\end{verbatim}

Done with libogg!

\section{libvorbis}

{\em Libvorbis} is the next step. Grab the version 1.3.4 from
\url{http://xiph.org} and uncompress it.

Once again, the libvorbis build system is a nice example of what can
be done with a good usage of the autotools. Cross-compiling libvorbis
is very easy, and almost identical to what we've seen with the
alsa-utils. First, the configure step:

\begin{verbatim}
CC=arm-linux-gcc \
./configure --host=arm-linux --prefix=/usr
\end{verbatim}

It will fail with:

\begin{verbatim}
configure: error: Ogg >= 1.0 required !
\end{verbatim}

By running \code{./configure --help}, you will find the
\code{--with-ogg-libraries} and \code{--with-ogg-includes} options.
Use those:

\begin{verbatim}
CC=arm-linux-gcc ./configure --host=arm-linux --prefix=/usr \
  --with-ogg-includes=$HOME/felabs-sysdev/thirdparty/staging/usr/include \
  --with-ogg-libraries=$HOME/felabs-sysdev/thirdparty/staging/usr/lib
\end{verbatim}

Then, compile the library:

\begin{verbatim}
make
\end{verbatim}

Install it in the {\em staging} space:

\begin{verbatim}
make DESTDIR=$HOME/felabs-sysdev/thirdparty/staging/ install
\end{verbatim}

And install only the required files in the {\em target} space:

\begin{verbatim}
cd ..
cp -a staging/usr/lib/libvorbis.so.0* target/usr/lib/
arm-linux-strip target/usr/lib/libvorbis.so.0.4.7
cp -a staging/usr/lib/libvorbisfile.so.3* target/usr/lib/
arm-linux-strip target/usr/lib/libvorbisfile.so.3.3.6
\end{verbatim}

And we're done with libvorbis!

\section{libao}

Now, let's work on {\em libao}. Download the version 1.2.0 from
\url{http://xiph.org} and extract it.

Configuring {\em libao} is once again fairly easy, and similar to
every sane autotools based build system:

\begin{verbatim}
LDFLAGS=-L$HOME/felabs-sysdev/thirdparty/staging/usr/lib \
CPPFLAGS=-I$HOME/felabs-sysdev/thirdparty/staging/usr/include \
CC=arm-linux-gcc ./configure --host=arm-linux \
                             --prefix=/usr
\end{verbatim}

Of course, compile the library:

\begin{verbatim}
make
\end{verbatim}

Installation to the {\em staging} space can be done using the
classical \code{DESTDIR} mechanism:

\begin{verbatim}
make DESTDIR=$HOME/felabs-sysdev/thirdparty/staging/ install
\end{verbatim}

And finally, install manually the only needed files at runtime in the
{\em target} space:

\begin{verbatim}
cd ..
cp -a staging/usr/lib/libao.so.4* target/usr/lib/
arm-linux-strip target/usr/lib/libao.so.4.1.0
\end{verbatim}

We will also need the alsa plugins that is loaded dynamically by the
libao at startup:
\begin{verbatim}
mkdir -p target/usr/lib/ao/plugins-4/
cp -a staging/usr/lib/ao/plugins-4/libalsa.so target/usr/lib/ao/plugins-4/
\end{verbatim}

Done with libao!

\section{vorbis-tools}

Finally, thanks to all the libraries we compiled previously, all the
dependencies are ready. We can now build the vorbis-tools themselves.
Download the version 1.4.0 from the official website, at
\url{http://xiph.org/}. As usual, extract the tarball.

Before starting the configuration, let's have a look at the available
options by running \code{./configure --help}. A lot of options are
available. We see that we can, in addition to the usual autotools
configuration options:

\begin{itemize}
\item Enable/Disable the various tools that are going to be built:
  ogg123, oggdec, oggenc, etc.
\item Enable or disable support for various other codecs: FLAC, Speex,
  etc.
\item Enable or disable the use of various libraries that can
  optionally be used by the vorbis tools
\end{itemize}

So, let's begin with our usual configure line:

\begin{verbatim}
LDFLAGS=-L$HOME/felabs-sysdev/thirdparty/staging/usr/lib \
CPPFLAGS=-I$HOME/felabs-sysdev/thirdparty/staging/usr/include \
CC=arm-linux-gcc \
./configure --host=arm-linux --prefix=/usr
\end{verbatim}

At the end, you should see the following warning:

\begin{verbatim}
configure: WARNING: Prerequisites for ogg123 not met, ogg123 will be skipped.
Please ensure that you have POSIX threads, libao, and (optionally) libcurl
libraries and headers present if you would like to build ogg123.
\end{verbatim}

Which is unfortunate, since we precisely want ogg123.

If you look back at the script output, you should see at some point
that it tests for the libao and fails to find it:

\begin{verbatim}
checking for AO... no
configure: WARNING: libao too old; >= 1.0.0 required
\end{verbatim}

Ok, now at the end of the configure, we get:

If you look into the \code{config.log} file now, you should find
something like:

\begin{verbatim}
configure:22343: checking for AO
configure:22351: $PKG_CONFIG --exists --print-errors "ao >= 1.0.0"
Package ao was not found in the pkg-config search path.
Perhaps you should add the directory containing `ao.pc'
to the PKG_CONFIG_PATH environment variable
No package 'ao' found
\end{verbatim}

In this case, the configure script uses the {\em pkg-config} system to
get the configuration parameters to link the library against libao. By
default, {\em pkg-config} looks in \code{/usr/lib/pkgconfig/} for
\code{.pc} files, and because the \code{libao-dev} package is probably
not installed in your system the configure script will not find the
libao we just compiled.

It would have been worse if we had the package installed, because it
would have detected and used our host package to compile libao, which,
since we're cross-compiling, is a pretty bad thing to do.

This is one of the biggest issue with cross-compilation: mixing host
and target libraries, because build systems have a tendency to look
for libraries in the default paths.

So, now, we must tell {\em pkg-config} to look inside the
\code{/usr/lib/pkgconfig/} directory of our {\em staging} space. This
is done through the \code{PKG_CONFIG_PATH} environment variable, as
explained in the manual page of \code{pkg-config}.

Moreover, the \code{.pc} files contain references to paths. For
example, in
\code{$HOME/felabs-sysdev/thirdparty/staging/usr/lib/pkgconfig/ao.pc},
we can see:

\begin{verbatim}
prefix=/usr
exec_prefix=${prefix}
libdir=${exec_prefix}/lib
includedir=${prefix}/include
[...]
Libs: -L${libdir} -lao
Cflags: -I${includedir}
\end{verbatim}

So we must tell \code{pkg-config} that these paths are not absolute,
but relative to our {\em staging} space. This can be done using the
\code{PKG_CONFIG_SYSROOT_DIR} environment variable.

Then, let's run the configuration of the vorbis-tools again, passing
the \code{PKG_CONFIG_PATH} and \code{PKG_CONFIG_SYSROOT_DIR}
environment variables:

\small
\begin{verbatim}
LDFLAGS=-L$HOME/felabs-sysdev/thirdparty/staging/usr/lib \
CPPFLAGS=-I$HOME/felabs-sysdev/thirdparty/staging/usr/include \
PKG_CONFIG_PATH=$HOME/felabs-sysdev/thirdparty/staging/usr/lib/pkgconfig \
PKG_CONFIG_SYSROOT_DIR=$HOME/felabs-sysdev/thirdparty/staging \
CC=arm-linux-gcc \
./configure --host=arm-linux --prefix=/usr
\end{verbatim}
\normalsize

Now, the configure script should end properly, we can now start the
compilation:
\begin{verbatim}
make
\end{verbatim}

It should fail rather quickly, complaining that the curl headers are
missing. This is because the configure script, in curl's case, didn't
actually test whether it was available or not, but just assumed it was.

It may also fail with the following cryptic message:
\small
\begin{verbatim}
if arm-linux-gcc -DSYSCONFDIR=\"/usr/etc\" -DLOCALEDIR=\"/usr/share/locale\"
 -DHAVE_CONFIG_H -I. -I. -I.. -I/usr/include -I../include -I../intl
 -I/home/ubuntu/felabs-sysdev/thirdparty/staging/usr/include  -O2 -Wall
 -ffast-math -fsigned-char -g -O2 -MT audio.o -MD -MP -MF ".deps/audio.Tpo"
 -c -o audio.o audio.c; \
	then mv -f ".deps/audio.Tpo" ".deps/audio.Po"; else rm -f ".deps/audio.Tpo"; exit 1; fi
In file included from /usr/include/stdio.h:28:0,
                 from audio.c:22:
/usr/include/features.h:324:26: fatal error: bits/predefs.h: No such file or directory
 #include <bits/predefs.h>
                          ^
compilation terminated.
make[2]: *** [audio.o] Error 1
make[2]: Leaving directory `/home/ubuntu/felabs-sysdev/thirdparty/vorbis-tools-1.4.0/ogg123'
make[1]: *** [all-recursive] Error 1
make[1]: Leaving directory `/home/ubuntu/felabs-sysdev/thirdparty/vorbis-tools-1.4.0'
make: *** [all] Error 2
\end{verbatim}
\normalsize

You can notice that \code{/usr/include} is added to the include paths.
Again, this is not what we want because it contains includes for the
host, not the target. It is coming from the autodetected value for
\code{CURL_CFLAGS}.

Add the \code{--without-curl} option to the configure invocation,
restart the compilation.

The compilation may then fail with an error related to libm. While
the code uses the function from this library, the generated Makefile
doesn't give the right command line argument in order to link against
the libm.

If you look at the configure help, you can see
\begin{verbatim}
LIBS        libraries to pass to the linker, e.g. -l<library>
\end{verbatim}

And this is exactly what we are supposed to use to add new linker
flags.

Add this to the configure command line to get
\begin{verbatim}
LDFLAGS=-L$HOME/felabs-sysdev/thirdparty/staging/usr/lib \
CPPFLAGS=-I$HOME/felabs-sysdev/thirdparty/staging/usr/include \
PKG_CONFIG_PATH=$HOME/felabs-sysdev/thirdparty/staging/usr/lib/pkgconfig \
PKG_CONFIG_SYSROOT_DIR=$HOME/felabs-sysdev/thirdparty/staging \
LIBS=-lm \
CC=arm-linux-gcc \
./configure --host=arm-linux --prefix=/usr --without-curl
\end{verbatim}

Finally, it builds!

Now, install the vorbis-tools to the {\em staging} space using:

\begin{verbatim}
make DESTDIR=$HOME/felabs-sysdev/thirdparty/staging/ install
\end{verbatim}

And then install them in the {\em target} space:

\begin{verbatim}
cd ..
cp -a staging/usr/bin/ogg* target/usr/bin
arm-linux-strip target/usr/bin/ogg*
\end{verbatim}

You can now test that everything works! Run \code{ogg123} on the
sample file found in \code{thirdparty/data}, and everything should
work fine!
