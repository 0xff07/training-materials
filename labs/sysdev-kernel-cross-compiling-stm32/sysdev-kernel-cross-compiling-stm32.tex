\subchapter{Kernel - Cross-compiling}{Objective: Learn how to
cross-compile a kernel for an ARM target platform.}

After this lab, you will be able to:
\begin{itemize}
\item Set up a cross-compiling environment
\item Cross compile the kernel for the STM32MP157A-DK1 Discovery kit
\item Use U-Boot to download the kernel
\item Check that the kernel you compiled starts the system
\end{itemize}

\section{Setup}

Go to the \code{$HOME/__SESSION_NAME__-labs/kernel} directory.

\section{Target system}

We are going to cross-compile and boot a Linux kernel for the
STM32MP157A-DK1 Discovery kit.

\section{Kernel sources}

We will re-use the kernel sources downloaded and patched in the
previous lab.

\section{Cross-compiling environment setup}

To cross-compile Linux, you need to have a cross-compiling
toolchain. We will use the cross-compiling toolchain that we
previously produced, so we just need to make it available in the PATH:

\begin{verbatim}
export PATH=$HOME/x-tools/arm-training-linux-uclibcgnueabihf/bin:$PATH
\end{verbatim}

Also, don't forget to either:

\begin{itemize}
\item Define the value of the \code{ARCH} and \code{CROSS_COMPILE}
  variables in your environment (using \code{export})
\item {\bf Or} specify them on the command line at every invocation of
  \code{make}, i.e: \code{make ARCH=... CROSS_COMPILE=... <target>}
\end{itemize}

\section{Linux kernel configuration}

We could use the \code{multi_v7_defconfig} default configuration to
build a working kernel but to save time and compile less code, we
provide in
\code{$HOME/__SESSION_NAME__-labs/linux-stm32/} a minimal configuration
file, named \code{stm32mp157_defconfig}. Copy it to \code{.config} and
then run either \code{make olddefconfig} or \code{make oldconfig} if
you want to get asked for all the possible choices.

Don't hesitate to visualize the new settings by running
\code{make xconfig} afterwards!

In the kernel configuration, as an experiment, change the kernel
compression from Gzip to XZ. This compression algorithm is far more
efficient than Gzip, in terms of compression ratio, at the expense of
a higher decompression time.

\section{Cross compiling}

At this stage, you need to install the \code{libssl-dev}
package to compile the kernel.

You're now ready to cross-compile your kernel. Simply run:

\begin{verbatim}
make
\end{verbatim}

and wait a while for the kernel to compile. Don't forget to use
\code{make -j<n>} if you have multiple cores on your machine!

Look at the end of the kernel build output to see which file contains
the kernel image. You can also see the Device Tree \code{.dtb} files
which got compiled. Find which \code{.dtb} file corresponds to your
board.

Copy the linux kernel image and DTB files to the TFTP server 
home directory.

\section{Load and boot the kernel using U-Boot}

As we are going to boot the Linux kernel from U-Boot,
we need to set the \code{bootargs} environment corresponding
to the Linux kernel command line:

\code{setenv bootargs console=ttyS0}

We will use TFTP to load the kernel image on the Discovery kit:

\begin{itemize}

\item On your workstation, copy the \code{zImage} and DTB files to the
  directory exposed by the TFTP server.

\item On the target (in the U-Boot prompt), load \code{zImage} from
  TFTP into RAM at address 0xc0000000:\\
  \code{tftp 0xc0000000 zImage}

\item Now, also load the DTB file into RAM at address 0xc4000000:\\
  \code{tftp 0xc4000000 stm32mp157a-dk1.dtb}

\item Boot the kernel with its device tree:\\
  \code{bootz 0xc0000000 - 0xc4000000}

\end{itemize}

You should see Linux boot and finally panicking. This is expected: we
haven't provided a working root filesystem for our device yet.

You can now automate all this every time the board is booted or
reset. Reset the board, and specify a different \code{bootcmd}:

{\scriptsize
\begin{verbatim}
setenv bootcmd 'tftp 0xc0000000 zImage; tftp 0xc4000000 stm32mp157a-dk1.dtb; bootz 0xc0000000 - 0xc4000000'
saveenv
\end{verbatim}
}

\section{Writing the kernel and DTB on the SD card}

In order to let the kernel boot on the board autonomously, we can
copy the kernel image and DTB in the boot partition we created
previously.

Insert the SD card in your PC, it will get auto-mounted. Copy the
kernel and device tree:
\begin{verbatim}
sudo cp arch/arm/boot/dts/stm32mp157a-dk1.dtb arch/arm/boot/zImage /media/training/boot/
sudo umount /dev/mmcblk0p4
\end{verbatim}

Insert the SD card back in the board and reset it. You should now be
able to load the DTB and kernel image from the SD card and boot with:

\begin{verbatim}
ext2load mmc 0:4 0xc0000000 zImage
ext2load mmc 0:4 0xc4000000 stm32mp157a-dk1.dtb
bootz 0xc0000000 - 0xc4000000
\end{verbatim}

You are now ready to modify \code{bootcmd} to boot the board
from SD card. But first, save the settings for booting from
\code{tftp}:

\begin{verbatim}
setenv bootcmdtftp ${bootcmd}
\end{verbatim}

This will be useful to switch back to \code{tftp} booting mode
later in the labs.

Finally, using \code{editenv bootcmd}, adjust \code{bootcmd} so that
the Discovery board starts using the kernel from the SD card.

Now, reset the board to check that it boots in the same way from the
SD card. Check that this is really your own version of the kernel
that's running\footnote{Look at the kernel log. You will find the
kernel version number as well as the date when it was compiled.
That's very useful to check that you're not loading an older version
of the kernel instead of the one that you've just compiled.}
