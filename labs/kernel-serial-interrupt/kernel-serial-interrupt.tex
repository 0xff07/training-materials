\subchapter{Sleeping and handling interrupts}{Objective: learn how to register and implement a simple interrupt handler, and how to put a process to sleep and wake it up at a later point}

During this lab, you will:

\begin{itemize}
\item Register an interrupt handler for the serial controller of the
  Calao board
\item See how Linux handles shared interrupt lines
\item Implement the read() operation of the serial port driver to put
  the process to sleep when no data are available
\item Implement the interrupt handler to wake-up the sleeping process
  waiting for received characters
\item Handle communication between the interrupt handler and the
  read() operation.
\end{itemize}

\section{Setup}

This lab is a continuation of the {\em Output-only character driver
  lab}, so we'll re-use the code in
\code{$HOME/felabs/linux/character}. Your Calao board should boot over
NFS and mount \code{/home/<user>/felabs/linux/character/nfsroot/} as the root
filesystem.

For demonstration purposes, we need to ensure that the tick system will
use the shared IRQ. So configure your kernel and {\em disable} TC
Block Clocksource (\code{CONFIG_ATMEL_TCB_CLKSRC}). Then compile
and boot your new kernel.

\section{Register the handler}

First, declare an interrupt handler function. Then, in the module
initialization function, register this handler to IRQ number
\code{NR_IRQS_LEGACY + AT91_ID_SYS}. Note that this IRQ is shared,
so the appropriate flags must be passed at registration time.

Then, in the interrupt handler, just print a message and
return \code{IRQ_NONE} (to tell the kernel that we haven't handled
the interrupt).

Compile and load your module. Look at the kernel logs: they are full
of our message indicating that interrupts are occurring, even if we
are not receiving from the serial port! It shows you that interrupt
handlers on shared IRQ lines are all called every time an interrupt
occurs.

\section{Enable and filter the interrupts}

In fact, at the moment, reception and interrupts are not enabled at
the level of the serial port controller. So in the initialization
function of the module:
\begin{itemize}
\item Write \code{ATMEL_US_RSTSTA | ATMEL_US_RSTRX} to the
  \code{ATMEL_US_CR} register
\item Write \code{ATMEL_US_TXEN | ATMEL_US_RXEN} to the
  \code{ATMEL_US_CR} register
\item Write \code{ATMEL_US_RXRDY} to the \code{ATMEL_US_IER}
  register (IER stands for Interrupt Enable Register).
\end{itemize}

Now, in our interrupt handler we want to filter out the interrupts
that come from the serial controller. To do so, read the value of the
\code{ATMEL_US_CSR} register and the value of the
\code{ATMEL_US_IMR} register. If the result of a {\em binary and}
operation between these two values is different from zero, then it
means that the interrupt is coming from our serial controller.

If the interrupt comes from our serial port controller, print a
message and return \code{IRQ_HANDLED}. If the interrupt doesn't come from
our serial port controller, just return \code{IRQ_NONE} without printing a
message.

Compile and load your driver. Have a look at the kernel messages. You
should no longer be flooded with interrupt messages.

Start \code{picocom} on \code{/dev/ttyUSB0}. Press one character (nothing will
appear since the target system is not echoing back what we're
typing). Then, in the kernel log, you should see the message of our
interrupt handler. If not, check your code once again and ask your
instructor for clarification!

\section{Read the received characters}

You can read the received characters by reading the \code{ATMEL_US_RHR}
register using \code{readl()}. It must be done in code that loops until the
\code{ATMEL_US_RXRDY} bit of the \code{ATMEL_US_CSR} register goes back to
zero. This method of operation allows to read several characters in a
single interrupt.

Note that our hardware doesn't give us any special register to
acknowledge interrupts. What happens is that interrupts are
acknowledged (allowing more interrupts to be sent in the future), when
the driver accesses the \code{ATMEL_US_RHR} register to read each
character.

For each received character, print a message containing the character.
Compile and load your driver.

From \code{picocom} on \code{/dev/ttyUSB0} on the host, send characters
to the target. The kernel messages on the target should properly tell
you which characters are being received.

\section{Sleeping, waking up and communication}

Now, we would like to implement the \code{read()} operation of our
driver so that a userspace application reading from our device can
receive the characters from the serial port.

First, we need a communication mechanism between the interrupt handler
and the \code{read()} operation. We will implement a very simple
circular buffer. So let's declare a global buffer in our driver:

\begin{verbatim}
#define SERIAL_BUFSIZE 16
static char serial_buf[SERIAL_BUFSIZE];
\end{verbatim}

Two integers that will contain the next location in the circular
buffer that we can write to, and the next location we can read from:

\begin{verbatim}
static int serial_buf_rd, serial_buf_wr;
\end{verbatim}

In the interrupt handler, store the received character at location
\code{serial_buf_wr} in the circular buffer, and increment the value
of \code{serial_buf_wr}. If this value reaches \code{SERIAL_BUFSIZE},
reset it to zero.

In the \code{read()} operation, if the \code{serial_buf_rd} value is
different from the \code{serial_buf_wr} value, it means that one
character can be read from the circular buffer. So, read this
character, store it in the userspace buffer, update the
\code{serial_buf_rd} variable, and return to userspace (we will only
read one character at a time, even if the userspace application
requested more than one).

Now, what happens in our \code{read()} function if no character is available
for reading (i.e, if \code{serial_buf_wr} is equal to \code{serial_buf_rd})? We
should put the process to sleep!

To do so, declare a global wait queue in our driver, named for example
\code{serial_wait}. In the \code{read()} function, use \code{wait_event_interruptible()}
to wait until \code{serial_buf_wr} is different from \code{serial_buf_rd}. And
in the interrupt handler, after storing the received characters in the
circular buffer, use \code{wake_up()} to wake up all processes waiting on
the wait queue.

Compile and load your driver. Run \code{cat /dev/serial} on the target,
and then in Picocom on the development workstation side, type some
characters. They should appear on the remote side if everything works
correctly!

Don't be surprised if the keys you type in Picocom don't appear on the
screen. This happens because they are not echoed back by the target.
