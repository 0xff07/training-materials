\subchapter{Measuring boot time}{Measuring the various
components of boot time}

Our first goal is to measure boot time in a coarse way. In a second
step, we will measure the time spent in the various components of boot
time.

\section{Setting your goals}

As explained in the lectures, the first thing to do is to choose what do
measure. In the demo running on this particular board, a natural choice is to
measure the time taken to start the demo video.

Ideally, we would measure the time elapsed since power-on or since a
reset event. However, we have no oscilloscope here, and therefore we can only
rely on messages sent on the serial port.

\begin{itemize}
\item The earliest message we can have on the serial port is the
      \code{RomBOOT} one. We will take its time of arrival as the origin of
      time.
\item We will implement a way to send a message to the serial line when
      the video starts playing.
\end{itemize}

\section{Measuring overall boot time}

We will use \code{grabserial} to measure the time spent while booting.
It prints the timing information of each message received on the
serial line\footnote{Beware that grabserial displays the arrival time
of the beginning of a line on a serial port. If a message was sent
without a trailing line feed, you could get the impression that
the following characters have been received much earlier. This is
particularly true when only a whitespace was first sent to the
serial line.}, and display the time elapsed since a given event (like
receiving a special string), used as the origin of time.

You will find \code{grabserial} on \url{http://elinux.org/Grabserial}.
You can also get it from your USB flash drive:

Download it and put it in your \code{PATH}:

\begin{verbatim}
cd
mkdir bin && cd bin
cp /media/$USER/BootTime/downloads/grabserial .
chmod a+x grabserial
\end{verbatim}

The command we will be using is:

\begin{verbatim}
~/bin/grabserial -d /dev/ttyACM0 -m RomBOOT -t -e 30
\end{verbatim}

Here are details about this command:

\begin{itemize}
\item \code{-m "RomBOOT"} specifies the string marking the
origin of time.
\item \code{-t} makes \code{grabserial} show the time when
{\bf the first character of each line is received}.
\item \code{-e 30} tells \code{grabserial} to stop measuring after 30
seconds.
\end{itemize}

First, you will have to exit \code{picocom} using \code{C-a C-x} to run
\code{grabserial}.

Start \code{grabserial} and reset the board using \code{PB1}. You will
see that it takes around 13.5 seconds to reach the Buildroot prompt and
show the demo video:

\begin{verbatim}
$ ~/bin/grabserial -d /dev/ttyACM0 -m RomBOOT  -t -e 30
[0.000002 0.000002] RomBOOT
[0.054440 0.054440]
[0.054594 0.000154]
[0.054719 0.000125] AT91Bootstrap 3.5.2 (Wed Jan 30 18:42:28 CET 2013)
[0.059009 0.004290]
[0.059185 0.000176] 1-Wire: Loading 1-Wire information ...
[0.062645 0.003460] 1-Wire: ROM Searching ... Done, 0x3 1-Wire chips
found
[0.110066 0.047421]
...
[12.158893 0.064820] Starting portmap: done
[12.284482 0.125589] Initializing random number generator... done.
[12.393179 0.108697] ALSA: Restoring mixer settings...
[12.447447 0.054268] Starting network...
[13.077587 0.630140] Starting sshd: OK
[13.467746 0.390159]
[13.470225 0.002479] Welcome to Buildroot
[13.472141 0.001916] buildroot login:
$
\end{verbatim}

Note that there's no information on exactly when the first user space
code gets executed. We will fix that in the {\em Add instrumentation}
section.

You can make a visual check that the \code{buildroot
login:} message is issued at about the same time as the video
starts playing. Of course, the accuracy is about 1 or 2 seconds,
and it's time to measure the video starting time in a more accurate way.

\section{Trick to estimate the video start time}

For this kind of need, finding out when a device starts functioning,
the ideal solution would be to monitor the LCD screen pins with an
oscilloscope, and based on the recorded activity, find out when the
video actually starts playing.

When only software tracing is available, a standard solution is to
add instrumentation to the application (here the video player) to write a message to
the serial line.

In our case, however, our video player is based on GStreamer, a very
complex piece of software. We can write a message on the serial line
before starting the player application, but the time to load this
application and actually start playing the video is far from being
negligible. This time is one of the components of boot time that we want to
measure.

A trick here is to replace the video by the shortest video possible
(with the same codecs), containing only one frame. Assuming that
the time to play the video itself is negligible, we can have a pretty
accurate measurement of the time to load the video player and start
playing the video.

\begin{center}
\includegraphics[width=\textwidth]{labs/boottime-measuring/videoplayer.pdf}
\end{center}

\section{Reconstitute the build environment for the root filesystem}

In every boot time reduction project, you have to rebuild the root
filesystem, either by using the original build system, or by compiling
the software stack with a new build system which could make it
easier to optimize and simplify the root filesystem.

Another option would have been to make direct changes to the root
filesystem. However, what you do is difficult to reproduce, and would
make the system more difficult to maintain. In a real production
project, it's much better to make changes to the configuration of
the build system used to generate the root filesystem.

This way, your changes to reduce boot time are always applied,
whatever the other changes you make on the system.

This is also true when you need to add instrumentation. If you do
this by hand, you will lose all your instrumentation the next
time you have to update or optimize a component.

\subsection{Install development software}

To update and recompile the root filesystem and
components like the Linux kernel, you will need to install several
software development packages:

\begin{verbatim}
sudo apt-get install git libncurses5-dev u-boot-tools g++ bison
sudo apt-get install flex gettext texinfo libqt4-dev lzop
\end{verbatim}

Also install the \code{meld} package. It is a very
convenient utility to compare different versions of a file, which
is often useful in an optimization project.

\begin{verbatim}
sudo apt-get install meld
\end{verbatim}

If you are using the official Ubuntu version required for this course
(Ubuntu Desktop 12.04), we advise you to use the precompiled
Buildroot environment available in the USB flash drive provided by your
instructor.

If you don't mind spending one to several hours
downloading source archives and compiling your root filesystem from
scratch, you can do it by follow the instructions available on
\url{http://free-electrons.com/labs/boottime/README.txt}.

Here's how to use the pre-compiled Buildroot environment (make sure you
use the exact same directories, otherwise the pre-compiled environment
won't work):

\begin{verbatim}
cd /opt/boot-time-labs/buildroot/
tar xf /media/$USER/BootTime/precompiled/buildroot-at91-`arch`.tar.xz
\end{verbatim}

You can see that the above command automatically picks up the right
pre-compiled environment, depending on whether you have a 32 bit or
a 64 bit version of Ubuntu.

\section{Study how user space boots}

It's important to understand how the various user space programs
are started in the system. They all originate from the \code{init}
program, which is the first and only program executed by the Linux
kernel \footnote{There is one exception: the kernel can also run
helpers to handle hotplug events.}, when it's done booting the device.

The programs that are executed by the \code{init} program can
found by reading the \code{/etc/inittab} file, which is a configuration
file for \code{init}.

Start \code{picocom} if needed and study this file. You will see that
a \code{/etc/init.d/rcS} script is also executed, and executes more
scripts at his turn. Take some time to read these scripts, until you
understand how the video player is started.

\section{Add instrumentation}

Make sure you are in the \code{/opt/boot-time-labs/buildroot/buildroot-at91} directory.

Copy the \code{../data/1frame.avi} file
\footnote {Here's the command that was used to generate a video
containing only the first frame of the original video: \\
\code{avconv -i MPEG2_480_272.avi -vcodec copy -acodec copy -frames 1
1frame.avi}}
to \code{board/atmel/sama5d3ek_demo/}, where the original video is
stored.

Now, modify the \code{board/atmel/sama5d3ek_demo/post-build.sh} file
\footnote{This script is called after compiling the root filesystem
components, and is used to add custom scripts and files to the system.}

\begin{itemize}
\item To copy the \code{1frame.avi} video to \code{/root} on the target
      file system. Do this by looking for the instruction that copies the
      original video, and by using a similar command.
\item To modify the contents of the \code{/root/go.sh} script
      which starts the video player:
      \begin{itemize}
      \item To add an \code{echo -e "\nStarting video player"} command
	    \footnote{We are using a newline (\code{\n}) at the
beginning of the string because \code{grabserial} only pays attention to
the arrival time of the first character in each line. Without this, we
would get the wrong time if the current line already had some
characters sent earlier. If we didn't use \code{echo -e} (\code{e}:
suppport for escape characters), \code{\n} wouldn't have been considered
as a newline.}
            right before invoking \code{gst-launch}, the video player that the demo
	    uses.
      \item To play the \code{1frame.avi} video instead of the original
  	    one.
      \item To add an \code{echo -e "\nDone playing video"} right after
 	    playing the video.
      \end{itemize}
\end{itemize}

The next thing we need is to know when the \code{init} program starts to
execute.

Let's have a look at home this program is built. Run the below command
to check the configuration of \code{BusyBox}, which implements
\code{init}:

\begin{verbatim}
make busybox-menuconfig
\end{verbatim}

Go to the \code{Init utilities} menu, and disable \code{Be _extra_ quiet
on boot}. Exit and save your new configuration. This way, we will have
a message on the serial line when \code{init} starts to execute.

It's now time to rebuild your root filesystem:
\begin{verbatim}
make
\end{verbatim}

The above command should only take a few seconds to run.

Make sure that the \code{output/target/root/go.sh} file
contains the right modifications and that there are no syntax errors.
Update the \code{board/atmel/sama5d3ek_demo/post-build.sh} file until
you get this file right.

If you can't manage to get it right, have a look at the
\code{../solutions/post-build1.sh} file.

\section{Re-flash the root filesystem}

We will use \code{SAM-BA} again to update the root filesystem.

Let's copy our new root filesystem to it to our \code{SAM-BA} directory:

\begin{verbatim}
cp output/images/rootfs.ubi \
/opt/boot-time-labs/flashing/sama5d3xek-demo/
\end{verbatim}

Put your board in RomBOOT mode (follow the procedure in the previous
chapter), and assuming that the SAM-BA device is \code{/dev/ttyACM1},
reflash your board, using the same instructions as before:

\begin{verbatim}
cd /opt/boot-time-labs/flashing/sama5d3xek-demo/
sam-ba /dev/ttyACM1 AT91SAMa5d3x-EK sama5d3xek_demo_linux_nandflash.tcl
\end{verbatim}

Press the reset button and make sure that your board boots as
expected. You can also check that the new instrumentation messages are
there. Don't pay attention to timing this time, as SSH keys had to be
generated from scratch again.

To stop the video player from running in a quick and endless way,
you can run the below command in the serial console:

\begin{verbatim}
killall go.sh
\end{verbatim}

We will remove the endless loop a little bit later.

\section{Measure boot time components}

You can now exit \code{picocom} and run \code{grabserial} again.

We can now time all the various steps of system booting in a pretty
accurate way. Do it by filling writing down the time stamps in the below
table.

Then, in each row, compute the elapsed time, which is the difference
between the time stamp for the next step and the time stamp for the current step.
For this purpose, you can use the \code{bc -l} command line calculator.

\begin{tabular}{| l | c | c | r |}
  \hline
  Boot phase & Start string & Time stamp & Elapsed \\
  \hline
  \hline
  RomBOOT & \code{RomBOOT} & & \\
  \hline
  Bootstrap & \code{AT91Bootstrap 3.5.2} & & \\
  \hline
  Bootloader & \code{U-Boot 2012.10} & & \\
  \hline
  Kernel & \code{Booting Linux on physical CPU 0} & & \\
  \hline
  Init scripts & \code{init started: BusyBox v1.21.0} & & \\
  \hline
  Critical application & \code{Starting video player} & & \\
  \hline
  Critical application ready & \code{Done playing video} & & \\
  \hline
  \hline
  \multicolumn{3}{| c |}{Total} & \\
  \hline
\end{tabular}

In the next labs, we will work on reducing boot time for each of the boot phases.
