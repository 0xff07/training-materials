\subchapter{Build the bootloader}{Objective: compile and install the
bootloader.}

After this lab, you will be able to compile U-Boot for your target
platform and run it from a micro SD card provided by your instructor.

\section{Setup}

Go to the \code{~/boot-time-labs/bootloader/u-boot/} directory.

Let's use the 2019.04 version:
\begin{verbatim}
git checkout v2019.04
\end{verbatim}

\section{Compiling environment}

If the previous Buildroot lab is already over, we can use the
toolchain it built to compile our bootloader and kernel:

\begin{verbatim}
export PATH=/home/<user>/boot-time-labs/rootfs/buildroot/output/host/bin:$PATH
export CROSS_COMPILE=arm-buildroot-linux-uclibcgnueabihf-
\end{verbatim}

Otherwise, let's take a cross-compiling toolchain provided by Ubuntu:

\begin{verbatim}
sudo apt install gcc-arm-linux-gnueabihf
export CROSS_COMPILE=arm-linux-gnueabihf-
\end{verbatim}

You will need the same settings when you compile the kernel too, and
when you recompiling U-Boot and the kernel to optimize them. Let's make
such settings permanent by adding the below lines at the end of your
\code{~/.bashrc} file:

\begin{verbatim}
export PATH=/home/<user>/boot-time-labs/rootfs/buildroot/output/host/bin:$PATH
export CROSS_COMPILE=arm-buildroot-linux-uclibcgnueabihf-
\end{verbatim}

\section{Configuring U-Boot}

Let's use the ready-made U-Boot configuration for the Beaglebone Black
board (you can find such configuration files in the \code{configs/}
directory:

\begin{verbatim}
make am335x_boneblack_vboot_defconfig
\end{verbatim}

\section{Compiling U-Boot}

Just run:
\begin{verbatim}
make
\end{verbatim}

or, to compile faster:

\begin{verbatim}
make -j 8
\end{verbatim}

This runs 8 compiler jobs in parallel (for example if you have 4 CPU
cores on your workstation... using more jobs than cores guarantees that
the CPUs and I/Os are always fully loaded, for optimum performance.

At the end, you have \code{MLO} and \code{u-boot.img} files that we will
put on a micro SD card for booting.

\section{Prepare the SD card}

Our SD card needs to be split in two partitions:

\begin{itemize}

\item A first partition for the bootloader. It needs to comply with
  the requirements of the AM335x SoC so that it can find the bootloader in
  this partition. It should be a FAT32 partition. We will store the
  bootloader (\code{MLO} and \code{u-boot.img}), the kernel image
  (\code{zImage}), the Device Tree (\code{am335x-boneblack.dtb}) and a
  special U-Boot script for the boot.

\item A second partition for the root filesystem. It can use
  whichever filesystem type you want, but for our system, we'll use
  {\em ext4}.

\end{itemize}

First, let's identify under what name your SD card is identified in
your system: look at the output of \code{cat /proc/partitions} and
find your SD card. In general, if you use the internal SD card reader
of a laptop, it will be \code{mmcblk0}, while if you use an external
USB SD card reader, it will be \code{sdX} (i.e\code{sdb}, \code{sdc},
etc.). {\bf Be careful: \code{/dev/sda} is generally the hard drive of
  your machine!}

If your SD card is \code{/dev/mmcblk0}, then the partitions inside the
SD card are named \code{/dev/mmcblk0p1}, \code{/dev/mmcblk0p2}, etc. If your SD
card is \code{/dev/sdc}, then the partitions inside are named
\code{/dev/sdc1}, \code{/dev/sdc2}, etc.

To format your SD card, do the following steps:

\begin{enumerate}

\item Unmount all partitions of your SD card (they are generally
  automatically mounted by Ubuntu)

\item Erase the beginning of the SD card to ensure that the existing
  partitions are not going to be mistakenly detected:\\
  \code{sudo dd if=/dev/zero of=/dev/mmcblk0 bs=1M count=16}. Use
  \code{sdc} or \code{sdb} instead of \code{mmcblk0} if needed.

\item Create the two partitions.

  \begin{itemize}

  \item Start the \code{cfdisk} tool for that:\\
    \code{sudo cfdisk /dev/mmcblk0}

  \item Chose the {\em dos} partition table type

  \item Create a first small partition (128 MB), primary, with type
    \code{e} ({\em W95 FAT16}) and mark it bootable

  \item Create a second partition, also primary, with the rest of the
    available space, with type \code{83} ({\em Linux}).

  \item Exit \code{cfdisk}

  \end{itemize}

\item Format the first partition as a {\em FAT32} filesystem:\\
  \code{sudo mkfs.vfat -F 32 -n boot /dev/mmcblk0p1}. Use \code{sdc1}
  or \code{sdb1} instead of \code{mmcblk0p1} if needed.

\item Format the second partition as an {\em ext4} filesystem:\\
  \code{sudo mkfs.ext4 -L rootfs -E nodiscard /dev/mmcblk0p2}. Use
  \code{sdc2} or \code{sdb2} instead of \code{mmcblk0p2} if needed.

\begin{itemize}
\item \code{-L} assigns a volume name to the partition
\item \code{-E nodiscard} disables bad block discarding. While this
      should be a useful option for cards with bad blocks, skipping
      this step saves long minutes in SD cards.
\end{itemize}
\end{enumerate}

Remove the SD card and insert it again, the two partitions should be
mounted automatically, in \code{/media/$USER/boot} and
\code{/media/$USER/rootfs}.


\section{Booting your new bootloader}

On a board in a normal state, there should be a bootloader on the on-board MMC
(eMMC) storage, and this will prevent you from using a bootloader on an
external SD card (unless you hold the \code{USER} button while powering
up your board, which is just suitable for exceptional needs).

Therefore, to override this behaviour and use the external SD card,
instead, let's wipe out the \code{MLO} file on the eMMC.

Power up or reset your board, and in the U-Boot prompt, run:

\begin{verbatim}
fatls mmc 1
\end{verbatim}

You should see the \code{MLO} file in the list of files. Let's remove it
by issuing the below command:

\begin{verbatim}
fatwrite mmc 1 81000000 100000
\end{verbatim}

If your board doesn't boot at all, please fix it first using our instructions
on
\url{https://raw.githubusercontent.com/bootlin/training-materials/master/lab-data/common/bootloader/beaglebone-black/README.txt}.

Now, copy your newly compiled \code{MLO} and \code{u-boot.img} files to
the SD card's \code{boot} partition, and after cleanly umounting this
partition, insert the SD card into the board and reset it.

You should now see your board booting with your own \code{MLO} and U-Boot
binaries (the versions and compiled dates are shown in the console). 
