\subchapter{Kernel sources}{Objective: Learn how to get the kernel
  sources and patch them.}

After this lab, you will be able to:
\begin{itemize}
\item Get the kernel sources from the official location
\item Apply kernel patches
\end{itemize}

\section{Setup}

Create the \code{$HOME/__SESSION_NAME__-labs/kernel} directory and go into it.

\section{Get the sources}

Go to the Linux kernel web site (\url{https://kernel.org/}) and
identify the latest stable version.

Just to make sure you know how to do it, check the version of the
Linux kernel running on your machine.
%uname -r

\ifdefstring{\labboard}{discovery}{
We will use \code{linux-5.10.x}, which this lab was tested with.

To practice with the \code{patch} command later, download the full 5.9
sources. Unpack the archive, which creates a \code{linux-5.9}
directory.
}{
We will use \code{linux-5.9.x}, which this lab was tested with.

To practice with the \code{patch} command later, download the full 5.8
sources. Unpack the archive, which creates a \code{linux-5.8}
directory.
}

Remember that you can use \code{wget <URL>} on the command
line to download files.

\section{Apply patches}

\ifdefstring{\labboard}{discovery}{
Download the patch files corresponding to the latest 5.10 stable
release: a first patch to move from 5.9 to 5.10 and if one exists,
a second patch to move from 5.9 to 5.10.x.

}{
Download the patch files corresponding to the latest 5.9 stable
release: a first patch to move from 5.8 to 5.9 and if one exists,
a second patch to move from 5.9 to 5.9.x.
}

Without uncompressing them to a separate file, apply the patches to the Linux
source directory.

%xzcat ../patchfile.xz | patch -p1 (--dry-run)
%patch -p1 (--dry-run) < ../diff_file

\ifdefstring{\labboard}{discovery}{
Rename the \code{linux-5.9} directory to \code{linux-5.10.<x>}.
}{
Rename the \code{linux-5.8} directory to \code{linux-5.9.<x>}.
}
