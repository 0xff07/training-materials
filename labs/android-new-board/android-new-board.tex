\subchapter{Supporting a new board}{Learn how to make Android on new
hardware}

After this lab, you will be able to:
\begin{itemize}
  \item Boot Android on a real hardware
  \item Troubleshoot simple problems on Android
  \item Generate a working build
\end{itemize}

\section{Setup}

Go to the \code{$HOME/felabs/android/linaro} directory.

Install the \code{bzr}, \code{python-argparse}, \code{python-parted}
and \code{python-yaml} packages.

\section{Download the source code}

Linaro has been working for some time now on a optimized Android
distribution, available on at least one chip of most of the major ARM
vendors. At each release, Linaro also maintains an Android flavored
Linux kernel corresponding to the latest upstream Linux release.

One of the platforms supported by the Linaro's Android is the
BeagleBoard, which is very similar to the DevKit8000 we use. In
particular, both boards have the same OMAP3530 System-on-Chip.
Moreover, the exact same kernel can be used to boot both boards.
For these reasons, we will use this distribution as a
starting point for the rest of the labs.

First, we need to download the source code. We will use the Linaro
Android 11.11 release, which is based on Android 2.3 Gingerbread and
Linux 3.1.1
\footnote{At the time of this writing, Linaro doesn't support Android
  4.0 Ice Cream Sandwich or later on the Beagle board, and the port
  from \href{http://code.google.com/p/rowboat} {Rowboat project}
  doesn't seem to be fully stable either. We decided to stick with the
  most stable version for our hardware.}.

\begin{verbatim}
mkdir src
cd src
repo init \
 -u git://git.free-electrons.com/android/linaro/platform/manifest.git \
 -b linaro-android-11.11-release
\end{verbatim}

Note that we're using a Free Electrons mirror of Linaro Android. This
was needed as some git repositories that the Linaro Android 11.11 manifest
specified disappeared or were moved. 

Take a look at the \code{.repo/manifest.xml} file and compare it with
the one in \code{$HOME/felabs/android/aosp/android/}. You can see that
Linaro took full advantage of the capabilities of the manifest file,
using multiple git servers instead of a unique one, and replacing some
of the components (like Dalvik and Bionic, for example) by its own
optimized versions.

Now, let's run the big download job:
\begin{verbatim}
repo sync -j4
\end{verbatim}

Once again, even with a fast Internet connection, the \code{repo sync}
command can take more than one hour. Your instructor will keep you
busy with other things during these downloads.

In the mean time, you should also download and extract the associated
Linaro toolchain: \footnote{The original URL is
  \url{http://android-build.linaro.org/builds/~linaro-android/toolchain-4.6-2011.11/4/android-toolchain-eabi-linaro-4.6-2011.11-4-2011-11-15_12-22-49-linux-x86.tar.bz2}. We
  wanted to give you something much shorter to type.}
\begin{verbatim}
cd ..
wget --trust-server-names http://goo.gl/Wn4dM
tar jxf android-toolchain-*
\end{verbatim}

In case this toolchain is removed from the Linaro website, you can also
fetch it from \url{http://free-electrons.com/labs/tools/}.

\section{Build Android for the BeagleBoard}

Get back to the \code{src} directory.

As we said earlier, the DevKit8000 is very similar to the BeagleBoard,
so we will first compile a build for the BeagleBoard. The command to
run is:

\code{make TARGET_PRODUCT=beagleboard TARGET_TOOLS_PREFIX=~/felabs/android/linaro/android-toolchain-eabi/bin/arm-eabi- boottarball systemtarball userdatatarball -j8}

Once again, you can expect this build job to take at least one hour to
run, even on a recent and rather fast laptop.

This job will build three tarballs in
\code{out/target/product/beagleboard}: \code{boot.tar.bz2},
\code{system.tar.bz2}, \code{userdata.tar.bz2}, plus the images of
Xloader, U-Boot, Linux kernel and its initramfs (which are also
contained in \code{boot.tar.bz2}).

We then need to put these images on an SD card so that we can boot on
this system.

First, take the SD card provided by your instructor, and insert it
into an SD card reader slot in your workstation, or into a USB card
reader provided by your instructor too. Then, using the \code{dmesg}
command, find which device your workstation uses for your SD card.
Let's assume that this device is \code{/dev/sdc}.

Now we can use a \code{linaro-android-media-create} script that Linaro
developed, which takes the three tarballs and aggregates them to
produce a ready-to-boot SD card.

You can install this script with the following commands:

\begin{verbatim}
bzr branch lp:linaro-image-tools
cd linaro-image-tools
bzr revert -r2011.11
cd ..
sudo ./linaro-image-tools/linaro-android-media-create --mmc /dev/sdc \
     --dev beagle --system out/target/product/beagleboard/system.tar.bz2 \
     --userdata out/target/product/beagleboard/userdata.tar.bz2 \
     --boot out/target/product/beagleboard/boot.tar.bz2
\end{verbatim}

Once this command is over, you can remove the SD card and insert it
into the corresponding slot on the DevKit8000 board.

\section{Setting up serial communication with the board}

To see the board boot, we need to read the first boot messages issued
on the board's serial port.

Your instructor will provide you with a special serial cable
\footnote{This should be the same cable as for the Beagle and IGEPv2
  boards} for the DevKit8000 board, together with a USB-to-serial
adaptor for the connection with your laptop.

When you plug in this adaptor, a serial port should appear on your
workstation: \code{/dev/ttyUSB0}.

You can also see this device appear by looking at the output of the
\code{dmesg} command.

To communicate with the board through the serial port, install a
serial communication program, such as \code{picocom}:
\footnote{\code{picocom} is one of the simplest utilities to access a
  serial console. \code{minicom} looks more featureful, but is also
  more complex to configure.}

\begin{verbatim}
sudo apt-get install picocom
\end{verbatim}

You also need to make your user belong to the \code{dialout} group to be
allowed to write to the serial console:

\begin{verbatim}
sudo adduser <user> dialout
\end{verbatim}

You need to log out and in again for the group change to be effective.

Run \code{picocom -b 115200 /dev/ttyUSB0}, to start serial
communication on \code{/dev/ttyUSB0}, with a baudrate of 115200. If
you wish to exit picocom, press \code{[Ctrl][a]} followed by
\code{[Ctrl][x]}.

\section{Booting the board}

Once you inserted the SD card, you can boot the board by holding the
\code{boot} key while switching the board on. On the serial port, you
should see Android going through its boot process, until you finally
have a shell on the serial link.

However, as you may have seen, the system boots, but you have no
display at all. We are going to fix this.

\section{Fix the blank screen}

The first problem we see is that the display remains blank the whole
time. This is because of the generated U-Boot being targeted for the
BeagleBoard and not for the DevKit8000. In the Android build system,
all the hardware related configuration is set in the file
\code{BoardConfig.mk}. In the BeagleBoard product definition, find
where the U-boot config file to use is set and change this
configuration to use the default configuration for the DevKit8000
(\code{devkit8000_config}).

To avoid rebuild errors, you should remove the directory where U-boot
was built:

\begin{verbatim}
rm -r out/target/product/beagleboard/obj/u-boot
\end{verbatim}

Now, compile and test your changes.

Don't be surprised if Android takes several minutes to boot. Android
has some work to do the first time it's booted.

You should now see the display working, while it has a major glitch:
it prints only a portion of the screen.

\section{Fix the resolution}

The actual resolution is set to 640x480 on the screen, while its
resolution is only 480x272. This kind of adjustment is mostly done
through the kernel command line. On the SD card's \code{boot}
partition, you will find a file named \code{boot.txt}, which is a
U-Boot script setting all the parameters needed to boot the board
properly, including the kernel command line. Add the
\code{omapdss.def_disp} and change the \code{omapfb.mode} properties
so that it uses the LCD instead of the DVI output and the correct
resolution.

You can find some documentation for these options in the
\code{kernel/Documentation/arm/OMAP/DSS} file, in the
\code{Kernel boot arguments} section.

You will then have to generate from this \code{boot.txt} file a
\code{boot.scr} file using the following command:

\code{mkimage -A arm -O linux -T script -C none -a 0 -e 0 -n 'Execute uImage.bin' -d boot.txt boot.scr}
\footnote{This command puts the \code{boot.txt} file contents into a
  special container, allowing U-boot to know that the \code{boot.scr}
  file is a script, and telling it how to handle this file. All the
  files handled by U-boot should be put in such a container.}
 
Once this is done, test your changes by booting the board. You should
see the display working correctly now.

\section{Fix the touchscreen}

If you test the touchscreen with the previous Android build you made,
you should see that it is almost unusable, while the system receives
some inputs. This mean that the kernel has the driver for the
touchscreen controller, but it returns bogus values. If you pay
attention however, you will find that the touchscreen doesn't have the
same orientation as the display.

Get back to the OMAP DSS documentation to find an option that might
address this problem.

\section{Bring it all together}

These tweaks are not persistent because the \code{boot.scr} file is
overwritten every time by the \code{linaro-android-media-create}
script.

However, the build system can help us here. Use the
\code{BOARD_KERNEL_CMDLINE} variable to set these new value. It will
be processed by the build system and then used when it generates the
boot image. Since this variable is related to the hardware, this
variable should be in \code{BoardConfig.mk}

You can also add \code{no_console_suspend} to the bootargs as Android
by default suspends the shell on the serial line after a minute or so,
making it pretty hard to use it properly.

\section{Add the Buttons}

If you happen to launch an application, you will find that you cannot
get back to the home screen, as no button is mapped to the \code{back}
and \code{home} keys.

Button mapping is done in two steps. First, in the kernel, the board
should define the available buttons. In this case, we will use the
small black buttons right next to the screen on the DevKit8000. These
buttons are handled by the \code{gpio-keys} driver, and defined in the
devkit8000 board file in the kernel, in the \code{arch/arm/mach-omap2}
folder.

If you look into that file, you will see that only one button is
defined, the one corresponding to the button labeled \code{user key}
on the board.

We will map this button to the \code{back} key in Android. In the
\code{gpio_keys_button} structure, there is a \code{code} field. The
value of this field defines the keycode sent to the input subsystem
when you press that button, and will be later dispatched to the
userspace. Replace the keycode used by \code{KEY_EXIT} and look up its
value in the \code{include/linux/input.h} file.

The Android input system loads keymaps and key layouts for each loaded
input driver. To load these properly, it uses the same name as the
input driver, with an extension. In our case, the input driver being
\code{gpio-keys}, we will need to modify the \code{gpio-keys.kl} file.

This file consists of a list of entries, corresponding to what actions
every keycode should trigger in the system. Add a new entry to this
file for the back button, which should be like:
\code{key <keycode> BACK}

Once you're done, rebuild the system, boot it, and you should be able
to use the \code{back} key now!
