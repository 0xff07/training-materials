\subchapter{Filesystems - Block file systems}{Objective: configure and
  boot an embedded Linux system relying on block storage}

After this lab, you will be able to:
\begin{itemize}
\item Manage partitions on block storage.
\item Produce file system images.
\item Configure the kernel to use these file systems
\item Use the tmpfs file system to store temporary files
\end{itemize}

\section{Goals}

After doing the {\em A tiny embedded system} lab, we are going to copy
the filesystem contents to the MMC flash drive. The filesystem will be
split into several partitions, and your sama5d3 X-plained board will
be booted with this MMC card, without using NFS anymore.

\section{Setup}

Throughout this lab, we will continue to use the root filesystem we
have created in the \code{$HOME/embedded-linux-labs/tinysystem/nfsroot}
directory, which we will progressively adapt to use block filesystems.

\section{Filesystem support in the kernel}

Recompile your kernel with support for SquashFS and ext3. Update your
kernel image in NAND flash.

Boot your board with this new kernel and on the NFS filesystem you
used in this previous lab\footnote{If you didn't do or complete the
  tinysystem lab, you can use the \code{data/rootfs} directory
  instead.}.

Now, check the contents of \code{/proc/filesystems}. You should see
 that ext3 and SquashFS are now supported.

\section{Prepare the MMC card}

We're going to use an MMC card for our block device.

Plug the MMC card your instructor gave you on your workstation. Type
the \code{dmesg} command to see which device is used by your
workstation. In case the device is \code{/dev/sdb}, you will see
something like

\begin{verbatim}
sd 3:0:0:0: [sdb] 3842048 512-byte hardware sectors: (1.96 GB/1.83 GiB)
\end{verbatim}

If your PC has an internal MMC/SD card reader, the device may also
been seen as \code{/dev/mmcblk0}, and the first partition as
\code{mmcblk0p1}\footnote{This is not always the case with internal
  MMC/SD card readers. On some PCs, such devices are behind an
  internal USB bus, and thus are visible in the same way external card
  readers are}. You will see that the MMC/SD card is seen in the same
way by the board.

In the following instructions, we will assume that your MMC/SD card
is seen as \code{/dev/sdb} by your PC workstation.

\fbox{
  \begin{minipage}{\textwidth}
    {\bfseries Caution: read this carefully before proceeding. You
      could destroy existing partitions on your PC!

      Do not make the confusion between the device that is used by
      your board to represent your MMC/SD disk (probably
      \code{/dev/sda}), and the device that your workstation uses when
      the card reader is inserted (probably \code{/dev/sdb}).

      So, don't use the \code{/dev/sda} device to reflash your MMC
      disk from your workstation. People have already destroyed their
      Windows partition by making this mistake.}
  \end{minipage}
}

You can also run \code{cat /proc/partitions} to list all block devices
in your system. Again, make sure to distinguish the SD/MMC card from
the hard drive of your development workstation!

Type the \code{mount} command to check your currently mounted
partitions. If MMC/SD partitions are mounted, unmount them:

\begin{verbatim}
$ sudo umount /dev/sdb*
\end{verbatim}

Then, clear possible MMC/SD card contents remaining from previous
training sessions:

\begin{verbatim}
$ sudo dd if=/dev/zero of=/dev/sdb bs=1M count=256
\end{verbatim}

Now, let's use the \code{cfdisk} command to create the partitions that
we are going to use:

\begin{verbatim}
$ sudo cfdisk /dev/sdb
\end{verbatim}

In the \code{cfdisk} interface, delete existing partitions, then
create three primary partitions, starting from the beginning, with the
following properties:

\begin{itemize}

\item One partition, 64MB big, with the \code{FAT16} partition type.

\item One partition, 8 MB big\footnote{For the needs of our system,
  the partition could even be much smaller, and 1 MB would be enough.
  However, with the 8 GB SD cards that we use in our labs, 8 MB will
  be the smallest partition that \code{cfdisk} will allow you to
  create.}, that will be used for the root filesystem. Due to the
  geometry of the device, the partition might be larger than 8 MB,
  but it does not matter. Keep the \code{Linux} type for the
  partition.

\item One partition, that fills the rest of the MMC card, that will be
  used for the data filesystem. Here also, keep the \code{Linux} type
  for the partition.

\end{itemize}

Press \code{Write} when you are done.

Now run \code{cat /proc/partitions} and check whether you see the new
partitions or not. If you don't, you can have the partition table
reloaded by removing the MMC card and insert it again. 

\section{Data partition on the MMC disk}

Using the \code{mkfs.ext3} create a journaled file system on the
third partition of the MMC disk (Caution: one again, make sure
you're not touching hard disk partitions instead of the MMC ones!).
Remember that you can use the \code{-L} option to set a volume name
for the partition. Move the contents of the \code{/www/upload/files}
directory (in your target root filesystem) into this new partition.
The goal is to use the third partition of the MMC card as the storage
for the uploaded images.

Connect the MMC disk to your board. You should see the MMC partitions
in \code{/proc/partitions}.

Mount this third partition on \code{/www/upload/files}.

Once this works, modify the startup scripts in your root filesystem
to do it automatically at boot time.

Reboot your target system and with the mount command, check that
\code{/www/upload/files} is now a mount point for the third MMC disk
partition. Also make sure that you can still upload new images, and
that these images are listed in the web interface.

\section{Adding a tmpfs partition for log files}

For the moment, the upload script was storing its log file in
\code{/www/upload/files/upload.log}. To avoid seeing this log file in
the directory containing uploaded files, let's store it in
\code{/var/log} instead.

Add the \code{/var/log/} directory to your root filesystem and modify
the startup scripts to mount a \code{tmpfs} filesystem on this
directory. You can test your \code{tmpfs} mount command line on the
system before adding it to the startup script, in order to be sure
that it works properly.

Modify the \code{www/cgi-bin/upload.cfg} configuration file to store
the log file in \code{/var/log/upload.log}. You will lose your log
file each time you reboot your system, but that's OK in our
system. That's what \code{tmpfs} is for: temporary data that you don't need
to keep across system reboots.

Reboot your system and check that it works as expected.

\section{Making a SquashFS image}

We are going to store the root filesystem in a SquashFS filesystem in
the second partition of the MMC disk.

In order to create SquashFS images on your host, you need to install
the \code{squashfs-tools} package. Now create a SquashFS image of your
NFS root directory.

Finally, using the \code{dd} command, copy the file system image to
the second partition of the MMC disk.

\section{Booting on the SquashFS partition}

In the U-boot shell, configure the kernel command line to use the
second partition of the MMC disk as the root file system. Also add the
\code{rootwait} boot argument, to wait for the MMC disk to be properly
initialized before trying to mount the root filesystem. Since the MMC
cards are detected asynchronously by the kernel, the kernel might try
to mount the root filesystem too early without \code{rootwait}.

Check that your system still works. Congratulations if it does!

\section{Store the kernel image and DTB on the MMC card}

Finally, copy the \code{zImage} kernel image and DTB to the first
partition of the MMC card and adjust the \code{bootcmd} of U-Boot so
that it loads the kernel and DTB from the MMC card instead of loading
them from the NAND.

You'll first need to format the first partition, using:
\begin{verbatim}
sudo mkfs.vfat -F 16 -n boot /dev/sdb1
\end{verbatim}

It will create a new FAT16 partition, called \code{boot}. Remove and
plug the MMC card. You can now copy the kernel image and Device Tree
to it.

In U-boot, you can load a file from a FAT filesystem using a command
like

\begin{verbatim}
fatload mmc 0:1 0x21000000 filename
\end{verbatim}

Which will load the file named \code{filename} from the first
partition of the first MMC device to the system memory at the address
\code{0x21000000}.
