\subchapter{Filesystems - Block file systems}{Objective: configure and
  boot an embedded Linux system relying on block storage}

After this lab, you will be able to:
\begin{itemize}
\item Manage partitions on block storage.
\item Produce file system images.
\item Configure the kernel to use these file systems
\item Use the tmpfs file system to store temporary files
\end{itemize}

\section{Goals}

After doing the {\em A tiny embedded system} lab, we are going to copy
the filesystem contents to the MMC flash drive. The filesystem will be
split into several partitions, and your IGEP board will be booted with
this MMC card, without using NFS anymore.

\section{Setup}

Stay in the \code{$HOME/felabs/sysdev/tinysystem} directory.

Recompile your kernel with support for SquashFS and ext3.

Boot your board with this new kernel and on the NFS filesystem you
used in this previous lab.\footnote{If you didn't do or complete the
  tinysystem lab, you can use the \code{data/rootfs} directory instead.}

\section{Add partitions to the MMC card}

Using \code{cfdisk}
\footnote{Now that one partition already exists, you don't have to specify
headers and sectors again. Just run \code{cfdisk /dev/sdx}},
add two additional partitions to the MMC card (in
addition to the existing \code{boot} partition created in the
bootloaders lab):

\begin{itemize}

\item One partition, 8 MB big
  \footnote{For the needs of our system, the partition could even be
  much smaller, and 1 MB would be enough. However, with the 8 GB SD
  cards that we use in our labs, 8 MB will be the smallest partition
  that \code{cfdisk} will allow you to create.},
  that will be used for the root
  filesystem. Due to the geometry of the device, the partition might
  be larger than 8 MB, but it does not matter. Keep the \code{Linux}
  type for the partition.

\item One partition, that fills the rest of the MMC card, that will be
  used for the data filesystem. Here also, keep the \code{Linux} type
  for the partition.

\end{itemize}

At the end, you should have three partitions: one for the boot, one
for the root filesystem and one for the data filesystem.

\section{Data partition on the MMC disk}

\fbox{\begin{minipage}{\textwidth}
{\bfseries
Caution: read this carefully before proceeding. You could destroy
existing partitions on your PC!

Do not make the confusion between the device that is used by your
board to represent your MMC disk (\code{/dev/mmcblk0)} or
\code{/dev/sda} if you are connecting the MMC card to the board 
with a USB card reader), and the device that your workstation
uses (probably \code{/dev/sdb}).

So, don't use the \code{/dev/sdaX} device to reflash your MMC disk from your
workstation. People have already destroyed their Windows partition by
making this mistake.}
\end{minipage}}

Using the \code{mkfs.ext3} create a journaled file system on the third
partition of the MMC disk. Remember that you can use the \code{-L}
option to set a volume name for the partition. Move the contents of
the \code{www/upload/files} directory (in your target root filesystem)
into this new partition. The goal is to use the third partition of the
MMC card as the storage for the uploaded images.

Connect the MMC disk to your board. You should see the MMC partitions
in \code{/proc/partitions}.
  
Mount this third partition on \code{/www/upload/files}.

Once this works, modify the startup scripts in your root filesystem
to do it automatically at boot time.

Reboot your target system and with the mount command, check that
\code{/www/upload/files} is now a mount point for the third MMC disk
partition. Also make sure that you can still upload new images, and
that these images are listed in the web interface.

\section{Adding a tmpfs partition for log files}

For the moment, the upload script was storing its log file in
\code{/www/upload/files/upload.log}. To avoid seeing this log file in
the directory containing uploaded files, let's store it in
\code{/var/log} instead.

Add the \code{/var/log/} directory to your root filesystem and modify
the startup scripts to mount a \code{tmpfs} filesystem on this
directory. You can test your \code{tmpfs} mount command line on the
system before adding it to the startup script, in order to be sure
that it works properly.

Modify the \code{www/cgi-bin/upload.cfg} configuration file to store
the log file in \code{/var/log/upload.log}. You will lose your log
file each time you reboot your system, but that's OK in our
system. That's what \code{tmpfs} is for: temporary data that you don't need
to keep across system reboots.

Reboot your system and check that it works as expected.

\section{Making a SquashFS image}

We are going to store the root filesystem in a SquashFS filesystem in
the second partition of the MMC disk.

In order to create SquashFS images on your host, you need to install
the \code{squashfs-tools} package. Now create a SquashFS image of your
NFS root directory.

Finally, using the \code{dd} command, copy the file system image to
the second partition of the MMC disk.

\section{Booting on the SquashFS partition}

In the U-boot shell, configure the kernel command line to use the
second partition of the MMC disk as the root file system. Also add the
\code{rootwait} boot argument, to wait for the MMC disk to be properly
initialized before trying to mount the root filesystem. Since the MMC
cards are detected asynchronously by the kernel, the kernel might try
to mount the root filesystem too early without \code{rootwait}.

Check that your system still works. Congratulations if it does!

\section{Store the kernel image on the MMC card}

Finally, copy the \code{uImage} kernel image to the first partition of
the MMC card (the partition called \code{boot}), and adjust the
\code{bootcmd} of U-Boot so that it loads the kernel from the MMC card
instead of loading the kernel through the network.
