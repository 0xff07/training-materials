\subchapter{Filesystems - Block file systems}{Objective: configure and
  boot an embedded Linux system relying on block storage}

After this lab, you will be able to:
\begin{itemize}
\item Manage partitions on block storage.
\item Produce file system images.
\item Configure the kernel to use these file systems
\item Use the tmpfs file system to store temporary files
\end{itemize}

\section{Goals}

After doing the {\em A tiny embedded system} lab, we are going to copy
the filesystem contents to the SD card. The filesystem will be
split into several partitions, and your sama5d3 X-plained board will
be booted with this SD card, without using NFS anymore.

\section{Setup}

Throughout this lab, we will continue to use the root filesystem we
have created in the \code{$HOME/embedded-linux-labs/tinysystem/nfsroot}
directory, which we will progressively adapt to use block filesystems.

\section{Filesystem support in the kernel}

Recompile your kernel with support for SquashFS and ext4\footnote{Basic
configuration options for these filesystems will be sufficient. No need
for things like extended attributes.}.

Update your kernel image in NAND flash.

Boot your board with this new kernel and on the NFS filesystem you
used in this previous lab.

Now, check the contents of \code{/proc/filesystems}. You should see
 that ext4 and SquashFS are now supported.

\section{Prepare the SD card}

We're going to use an SD card for our block device.

Plug the SD card your instructor gave you on your workstation. Type
the \code{dmesg} command to see which device is used by your
workstation. In case the device is \code{/dev/mmcblk0}, you will see
something like

\begin{verbatim}
[46939.425299] mmc0: new high speed SDHC card at address 0007
[46939.427947] mmcblk0: mmc0:0007 SD16G 14.5 GiB
\end{verbatim}

The device file name may be different (such as \code{/dev/sdb}
if the card reader is connected to a USB bus (either inside your PC
or using a USB card reader).

In the following instructions, we will assume that your SD card
is seen as \code{/dev/mmcblk0} by your PC workstation.

Type the \code{mount} command to check your currently mounted
partitions. If SD partitions are mounted, unmount them:

\begin{verbatim}
$ sudo umount /dev/mmcblk0*
\end{verbatim}

Then, clear possible SD card contents remaining from previous
training sessions (only the first megabytes matter):

\begin{verbatim}
$ sudo dd if=/dev/zero of=/dev/mmcblk0 bs=1M count=256
\end{verbatim}

Now, let's use the \code{cfdisk} command to create the partitions that
we are going to use:

\begin{verbatim}
$ sudo cfdisk /dev/mmcblk0
\end{verbatim}

If \code{cfdisk} asks you to \code{Select a label type}, choose
\code{dos}. This corresponds to traditional partitions tables that DOS/Windows
would understand. \code{gpt} partition tables are needed for disks bigger
than 2 TB.

In the \code{cfdisk} interface, delete existing partitions, then
create three primary partitions, starting from the beginning, with the
following properties:

\begin{itemize}

\item One partition, 64MB big, with the \code{FAT16} partition type.

\item One partition, 8 MB big\footnote{For the needs of our system,
  the partition could even be much smaller, and 1 MB would be enough.
  However, with the 8 GB SD cards that we use in our labs, 8 MB will
  be the smallest partition that \code{cfdisk} will allow you to
  create.}, that will be used for the root filesystem. Due to the
  geometry of the device, the partition might be larger than 8 MB,
  but it does not matter. Keep the \code{Linux} type for the
  partition.

\item One partition, that fills the rest of the SD card, that will be
  used for the data filesystem. Here also, keep the \code{Linux} type
  for the partition.

\end{itemize}

Press \code{Write} when you are done.

To make sure that partition definitions are reloaded on your
workstation, remove the SD card and insert it again.

\section{Data partition on the SD card}

Using the \code{mkfs.ext4} create a journaled file system on the
third partition of the SD card:

\begin{verbatim}
sudo mkfs.ext4 -L data -E nodiscard /dev/mmcblk0p3
\end{verbatim}

\begin{itemize}
\item \code{-L} assigns a volume name to the partition
\item \code{-E nodiscard} disables bad block discarding. While this
      should be a useful option for cards with bad blocks, skipping
      this step saves long minutes in SD cards.
\end{itemize}

Now, mount this new partition and move the contents of the
\code{/www/upload/files} directory (in your target root filesystem) into
it. The goal is to use the third partition of the SD card as the storage
for the uploaded images.

Connect the SD card to your board. You should see the partitions
in \code{/proc/partitions}.

Mount this third partition on \code{/www/upload/files}.

Once this works, modify the startup scripts in your root filesystem
to do it automatically at boot time.

Reboot your target system and with the mount command, check that
\code{/www/upload/files} is now a mount point for the third SD card
partition. Also make sure that you can still upload new images, and
that these images are listed in the web interface.

\section{Adding a tmpfs partition for log files}

For the moment, the upload script was storing its log file in
\code{/www/upload/files/upload.log}. To avoid seeing this log file in
the directory containing uploaded files, let's store it in
\code{/var/log} instead.

Add the \code{/var/log/} directory to your root filesystem and modify
the startup scripts to mount a \code{tmpfs} filesystem on this
directory. You can test your \code{tmpfs} mount command line on the
system before adding it to the startup script, in order to be sure
that it works properly.

Modify the \code{www/cgi-bin/upload.cfg} configuration file to store
the log file in \code{/var/log/upload.log}. You will lose your log
file each time you reboot your system, but that's OK in our
system. That's what \code{tmpfs} is for: temporary data that you don't need
to keep across system reboots.

Reboot your system and check that it works as expected.

\section{Making a SquashFS image}

We are going to store the root filesystem in a SquashFS filesystem in
the second partition of the SD card.

In order to create SquashFS images on your host, you need to install
the \code{squashfs-tools} package. Now create a SquashFS image of your
NFS root directory.

Finally, using the \code{dd} command, copy the file system image to
the second partition of the SD card.

\section{Booting on the SquashFS partition}

In the U-boot shell, configure the kernel command line to use the
second partition of the SD card as the root file system. Also add the
\code{rootwait} boot argument, to wait for the SD card to be properly
initialized before trying to mount the root filesystem. Since the SD
cards are detected asynchronously by the kernel, the kernel might try
to mount the root filesystem too early without \code{rootwait}.

Check that your system still works. Congratulations if it does!

\section{Store the kernel image and DTB on the SD card}

You'll first need to format the first partition, using:
\begin{verbatim}
sudo mkfs.vfat -F 16 -n boot /dev/mmcblk0p1
\end{verbatim}

It will create a new FAT16 partition, called \code{boot}. Remove and
plug the SD card back in. You can now copy the kernel image and
Device Tree to it.

You now need to adjust the \code{bootcmd} of U-Boot so
that it loads the kernel and DTB from the SD card instead of loading
them from the NAND.

In U-boot, you can load a file from a FAT filesystem using a command
like

\begin{verbatim}
fatload mmc 0:1 0x21000000 filename
\end{verbatim}

Which will load the file named \code{filename} from the first
partition of the device handled by the first MMC controller to the
system memory at the address \code{0x21000000}.

\section{Going further}

At this point our board still uses the bootloaders
(\code{at91bootstrap} and \code{U-Boot}) stored in the NAND flash.
Let's try to have everything on our SD card.
The ROM code can load the first stage bootloader from an MMC or SD
card, from a file named \code{boot.bin} located in the first FAT
partition. U-Boot will be stored as \code{u-boot.bin}.

For this you will need to recompile \code{at91bootstrap} (you'll need
to switch to version 3.8.12) to support booting from an SD card.
Then recompile U-Boot after reconfiguring it with its MMC configuration
(we previously used the configuration for running from NAND flash).
