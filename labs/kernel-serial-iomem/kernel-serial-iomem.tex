\subchapter{Accessing I/O memory and ports}{Objective: read /
  write data from / to a hardware device}

Throughout the upcoming labs, we will implement a character driver
allowing to write data to the serial port of the CALAO board, and to
read data from it.

After this lab, you will be able to:

\begin{itemize}

\item Use \code{ssh} to get a remote terminal to an embedded Linux
  system

\item Access a console through the network, typically when a serial or
  graphical console cannot be used.

\item Practice with I/O ports and I/O memory to control the device and
  exchange data with it.

\end{itemize}

\section{Setup}

Go to the \code{/home/<user>/felabs/linux/character} directory.

As in the {\em Module development environment} lab, we will use a the
CALAO board booted from NFS.

As in the previous labs, the target IP address will be \code{192.168.0.100},
and the host address will be \code{192.168.0.1}.

Extract the latest Linux 3.0.x kernel sources in the current
directory, and configure them with the default configuration for
{\em at91sam9263} boards.

In this lab, we will develop our own driver for the board's serial
port. The consequence is that we will have to disable the standard
AT91 serial port driver and will thus lose the serial console.

Instead of running commands through a shell on the serial line, we
will access our target through SSH, a secure shell over the network.

To replace the serial console and still get kernel messages (in
particular kernel fault messages) we will use a netconsole, a
mechanism that allows to get console messages over the network.

So, configure your kernel with:
\begin{itemize}
\item Root over NFS support
\item Loadable module support
\item Netconsole support (\code{CONFIG_NETCONSOLE})
\end{itemize}

You will also have to update the kernel command line so that Linux
loads the root filesystem over NFS from
\code{/home/<user>/felabs/linux/character/nfsroot}.

The Dropbear SSH server is already installed on your target, and is
started automatically.

\section{Boot tests}

We are first going to make sure that your new kernel boots fine and
that the SSH connection works well, before disabling the serial driver
and the serial console. NFS configuration issues are frequent, and
they would be more difficult to fix if we have no console left.

So, boot your new kernel and try to connect to your board with ssh:

\begin{verbatim}
ssh root@192.168.0.100
\end{verbatim}

The \code{root} password is empty, just press \code{Enter}. Good job!

\section{Disabling the serial driver and console}

Now that everything works, rebuild your kernel without the serial port
driver (in \code{Device Drivers} $\rightarrow$ \code{Character Drivers}
$\rightarrow$ \code{Serial drivers} $\rightarrow$ \code{AT91 / AT32
on-chip serial port support}). Update your kernel.

You also need to add the following option to the kernel command line,
to enable the network console:

\begin{verbatim}
netconsole=4444@192.168.0.100/eth0,5555@192.168.0.1/
\end{verbatim}

To get the console output, we will use the \code{netcat} command on the host
to listen for UDP packets on port 5555:

\begin{verbatim}
netcat -u -l 192.168.0.1 5555
\end{verbatim}

Now, boot your new kernel. You should see kernel messages only in the
\code{netcat} output (after a little while).

\section{Driver source code}

Now go to the \code{nfsroot/root} directory, which will contain the
kernel module source code.

You have a ready-made \code{Makefile} to compile your module, and a
template \code{serial.c} module which you will fill with your own
code. Modify the \code{Makefile} so that it points to your kernel sources.

\section{Device initialization}

In the module initialization function, start by reserving the I/O
memory region starting at address (\code{AT91_BASE_SYS + AT91_DBGU}), for a
size of \code{SZ_512} (512 bytes). The \code{AT91_*} constants are already defined
in Linux kernel headers.

Compile your module, load it and make sure that this memory region
appears in \code{/proc/iomem}.

Now, obtain a virtual address corresponding to the start of this
memory area.

Don't forget to undo all the above in the module exit function!

\section{Standalone write routine}

Implement a C routine taking one character as a parameter and writing
it to the serial port, using the following steps:

\begin{enumerate}

 \item Wait until the \code{ATMEL_US_TXRDY} bit gets set in
 the \code{ATMEL_US_CSR} register (\code{ATMEL_US_CSR} is an
 offset in the I/O memory region previously remapped). You can
 busy-wait for this condition to happen. In the busy-wait loop, you
 can call the \code{cpu_relax()} kernel function to relax the CPU
 during the wait.

 \item Write the character to the \code{ATMEL_US_THR} register.

\end{enumerate}

Note that all the I/O registers of the AT91 processor are 32 bits
wide.

Add a call to this routine from your module init function. Recompile
your module and load it on the target. You should see the
corresponding character in Minicom, still showing what was written to
the serial line by the board.

\section{Driver sanity check}

Remove your module and try to load it again. If the second attempt to
load the module fails, it is probably because your driver doesn't
properly free the resources it allocated or register, either at module
exit time, or after a failing during the module init function.  Check
and fix your module init and exit functions if you have such a
problem.
