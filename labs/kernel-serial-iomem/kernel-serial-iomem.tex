\subchapter{Accessing I/O memory and ports}{Objective: read /
  write data from / to a hardware device}

Throughout the upcoming labs, we will implement a character driver
allowing to write data to additional CPU serial ports available on
the BeagleBone, and to read data from them.

After this lab, you will be able to:

\begin{itemize}
\item Add UART devices to the board device tree
\item Access I/O registers to control the device and
      send first characters to it.
\end{itemize}

\section{Setup}

Go to your kernel source directory.

Create a new branch for this new series of labs. Since this new stuff
is independent from the nunchuk changes, it's best to create a separate
branch!

\begin{verbatim}
git checkout 3.11.y-bbb
git checkout  -b uart
\end{verbatim}

\section{Add UART devices}

Before developing a driver for additional UARTS on the board, we
need to add the corresponding descriptions to the board Device Tree.

First, open the board reference manual and find the connectors
and pinmux modes for UART2 and UART4.

Using an new USB-serial cable with male connectors, provided by your
instructor, connect your PC to UART2. The wire colors are the same
as for the cable that you're using for the console:

\begin{itemize}
\item The blue wire should be connected \code{GND} 
\item The red wire (\code{TX}) should be connected to the board's \code{RX} pin
\item The green wire (\code{RX}) should be connected to the board's \code{TX} pin
\end{itemize}

You can (or even should) show your connections to the instructor
to make sure that you haven't swapped the \code{RX} and \code{TX} pins. 

Now, open the \code{arch/arm/boot/dts/am335x-bone-common.dtsi}
file and create declarations for UART2 and UART4 in the pin muxing
section:

\begin{verbatim}
    /* Pins 21 (TX) and 22 (RX) of connector P9 */
    uart2_pins: uart2_pins {
        pinctrl-single,pins = <
            0x154 (PIN_OUTPUT_PULLDOWN | MUX_MODE1) /* spi0_d0.uart2_tx, MODE 1 */
            0x150 (PIN_INPUT_PULLUP | MUX_MODE1) /* spi0_sclk.uart2_rx, MODE 1 */
        >;
    };

    /* Pins 11 (RX) and 13 (TX) of connector P9 */
    uart4_pins: uart4_pins {
        pinctrl-single,pins = <
            0x74 (PIN_OUTPUT_PULLDOWN | MUX_MODE6) /* gpmc_wpn.uart4_tx, MODE 6 */
            0x70 (PIN_INPUT_PULLUP | MUX_MODE6) /* gpmc_wait0.uart4_rx, MODE 6 */
        >;
    };
\end{verbatim}

Then, declare the corresponding devices:

\begin{verbatim}
    uartfe2: feserial@48024000 {
        compatible = "free-electrons,serial";
        /* Tell the OMAP harware power management that the block
           must be enabled, otherwise it's switched off
           Caution: starting counting at 1, not 0 */
        ti,hwmods = "uart3";
        clock-frequency = <48000000>;
        reg = <0x48024000 0x2000>;
        interrupts = <74>;
        status = "okay";
        pinctrl-names = "default";
        pinctrl-0 = <&uart2_pins>;
    };

    uartfe4: feserial@481a8000 {
        compatible = "free-electrons,serial";
        ti,hwmods = "uart5";
        clock-frequency = <48000000>;
        reg = <0x481a8000 0x2000>;
        interrupts = <45>;
        status = "okay";
        pinctrl-names = "default";
        pinctrl-0 = <&uart4_pins>;
    };
\end{verbatim}

Note: we are calling these devices with \code{uartfe} instead of
\code{uart} to avoid conflicts with declarations in
\code{arch/arm/boot/dts/am33xx.dtsi}. The \code{uart} devices are 
meant to be used by the regular serial driver.

We will see how to use the device parameters in the driver code.

Rebuild and update your DTB.

\section{Operate a platform device driver}

Go to the \code{~/felabs/linux/modules/nfsroot/root/serial/} directory.
You will find a \code{feserial.c} file already provides a platform
driver skeleton.

Add the code needed to match the driver with the devices which you have
just declared in the device tree.

Compile your module and load it on your target. Check the kernel log
messages, that should confirm that the \code{probe()} routine was
called \footnote{Don't be surprised if the \code{probe()} routine is
actually called twice! That's because we have declared two devices.
Even if we only connect a serial-to-USB dongle to one of them, both
of them are ready to be used!}.

\section{Get base addresses from the device tree}

We are going to read from memory mapped registers and read from them.
The first thing we need is the base physical address for the each
device.

Such information is precisely available in the Device Tree. You can
extract it with the below code:

\begin{verbatim}
struct resource *res;
res = platform_get_resource(pdev, IORESOURCE_MEM, 0);
\end{verbatim}

Add such code to your \code{probe()} routine, with proper error
handling when \code{res == NULL}, and print the start address
(\code{res->start}) to make sure that the address values that
you get match the ones in the device tree. 

You can remove the printing instruction as soon as the collected
addresses are correct.

\section{Create a device private structure}

The next step is to start allocating and registering resources,
which eventually will have to be freed and unregistered too.

In the same way as in the nunchuk lab, we now need to create a
structure that that will hold device specific information and help
keeping pointers between logical and physical devices.

As the first thing to store will be the base virtual address for
each device (obtained through \code{ioremap()}), let's declare this
structure as follows:

\begin{verbatim}
struct feserial_dev {
        void __iomem *regs;
};
\end{verbatim}

The first thing to do is allocate such a structure at the beginning
of the \code{probe()} routine. Let's do it with a \code{devm_} function.
The advantage of such routines is that each allocation or registration
is attached to a device structure. When a device or a module is removed,
all such allocations or registrations are automatically undone. This
allows to greatly simplify driver code.

So, add the below line to your code:

\begin{verbatim}
struct feserial_dev *dev;
...
dev = devm_kzalloc(&pdev->dev, sizeof(struct feserial_dev), GFP_KERNEL);
\end{verbatim}

You can now get a virtual address for your device's base physical
address, by calling:

\begin{verbatim}
        dev->regs = devm_request_and_ioremap(&pdev->dev, res);

        if (!dev->regs) {
                dev_err(&pdev->dev, "Cannot remap registers\n");
                return -ENOMEM;
        }
\end{verbatim}

What's nice is that you won't ever have to release this resource,
neither in the \code{remove()} routine, nor if there are failures
in subsequent steps of the \code{probe()} routine.

Make sure that your updated driver compiles, loads and unloads well.

\section{Device initialization}

Now that we have a virtual address to access registers, we are ready to
configure a few registers which will allow us to enable the UART
devices. Of course, this will be done in the \code{probe()} routine.

\subsection{Accessing device registers}

As we will have multiple registers to read, create a \code{reg_read()}
routine, returning an \code{unsigned int} value, and  taking a \code{dev}
pointer to an \code{feserial_dev} structure and an \code{offset} integer
offset.

In this function, read from a 32 bits register at the base virtual
address for the device plus the offset multiplied by 4\footnote{You have
to multiply by 4 because we are using a void* pointer so pointer
arithmetic will use the offset as bytes whereas we are using 32 bits
registers.}, and return this value.

Create a similar \code{reg_write()} routine, writing an unsigned integer
value at a given integer offset (don't forget to multiply it by 4) from
the device base virtual address. The following code samples are using
the \code{writel()} convention of passing the value first, then the
offset. Your prototype should look like:
\begin{verbatim}
    static void reg_write(struct feserial_dev *dev, int val, int off);
\end{verbatim}

All the UART register offsets have standardized values, shared between
several types of serial drivers (see
\code{include/uapi/linux/serial_reg.h}). This explains why they are not
completely ready to use and we have to multiply them by 4 for OMAP SoCs.

We are now ready to read and write registers!

\subsection{Power management initialization}

Add the below lines to the probe function:

\begin{verbatim}
    pm_runtime_enable(&pdev->dev);
    pm_runtime_get_sync(&pdev->dev);
\end{verbatim}

And add the below line to the \code{remove()} routine:

\begin{verbatim}
    pm_runtime_disable(&pdev->dev);
\end{verbatim}

\subsection{Line and baud rate configuration}

After these lines, let's add code to initialize the line
and configure the baud rate. This shows how to get a special
property from the device tree, in this case \code{clock-frequency}:

\begin{verbatim}
        /* Configure the baud rate to 115200 */

        of_property_read_u32(pdev->dev.of_node, "clock-frequency",
                             &uartclk);
        baud_divisor = uartclk / 16 / 115200;
        reg_write(dev, 0x07, UART_OMAP_MDR1);
        reg_write(dev, 0x00, UART_LCR);
        reg_write(dev, UART_LCR_DLAB, UART_LCR);
        reg_write(dev, baud_divisor & 0xff, UART_DLL);
        reg_write(dev, (baud_divisor >> 8) & 0xff, UART_DLM);
        reg_write(dev, UART_LCR_WLEN8, UART_LCR);
\end{verbatim}

Declare \code{baud_divisor} and \code{uartclk} as \code{unsigned int}.

\subsection{Soft reset}

The last thing to do is to request a software reset:

\begin{verbatim}
        /* Soft reset */
        reg_write(dev, UART_FCR_CLEAR_RCVR | UART_FCR_CLEAR_XMIT, UART_FCR);
        reg_write(dev, 0x00, UART_OMAP_MDR1);
\end{verbatim}

We are now ready to transmit characters over the serial ports!

If you have a bit of spare time, you can look at section 19 of the
AM335x TRM for details about how to use the UART ports, to understand
better what we are doing here.
 
\section{Standalone write routine}

Implement a C routine taking a pointer to an \code{feserial_dev}
structure and one character as parameters, and writing
this character to the serial port, using the following steps:

\begin{enumerate}
\item Wait until the \code{UART_LSR_THRE} bit gets set in the
  \code{UART_LSR} register.  You can busy-wait for this condition to happen.
  In the busy-wait loop, you can call the \code{cpu_relax()} kernel function
  to ensure the compiler won't optimise away this loop.
\item Write the character to the \code{UART_TX} register.
\end{enumerate}

Add a call to this routine from your module \code{probe()} function,
and recompile your module.

Open a new \code{picocom} instance on your new serial port (not the
serial console):

\begin{verbatim}
picocom -b 115200 /dev/ttyUSB1
\end{verbatim}

Load your module on the target. You should see the
corresponding character in the new \code{picocom} installed,
showing what was written to UART2.

You can also check that you also get the same character on UART4
(just connect to the UART4 pins instead of the UART2 ones).

\section{Driver sanity check}

Remove your module and try to load it again. If the second attempt to
load the module fails, it is probably because your driver doesn't
properly free the resources it allocated or registered, either at module
exit time, or after a failure during the module \code{probe()} function. Check
and fix your module code if you have such problems.
