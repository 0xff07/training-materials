\subchapter{Lab3: Add a custom application}{Add a new recipe to support a required
	custom application}

During this lab, you will:
\begin{itemize}
  \item Write a recipe for a custom application
  \item Integrate this application into the build
\end{itemize}

This is the first step of adding an application to Yocto. The
remaining part is covered in the next lab, "Create a Yocto layer".

\section{Setup and organization}

In this lab we will add a recipe handling the \code{nInvaders}
application. Before starting the recipe itself, find the
\code{recipes-extended} directory in the OpenEmbedded-Core layer and
add a subdirectory for your application.

A recipe for an application is usually divided into a version specific \code{bb}
file and a common one. Try to follow this logic and separate the configuration
variables accordingly.

Tip: it is possible to include a file into a recipe with the keyword
\code{require}.

\section{First hands on nInvaders}

The nInvaders application is a terminal based game following the space invaders
family. In order to deal with the text based user interface, nInvaders uses the
ncurses library.

First try to find the project homepage, download the sources and have a first
look: license, Makefile, requirements\dots

\section{Write the common recipe}

Create an appropriate common file, ending in \code{.inc}

In this file add the common configuration variables: source URI,
description\dots

\section{Write the version specific recipe}

Create a file that respects the Yocto nomenclature: \code{${PN}_${PV}.bb}
%stopzone

Add the required common configuration variables: archive checksum, license file
checksum, package revision\dots

\section{Testing and troubleshooting}

You can check the whole packaging process is working fine by explicitly running
the build task on the \code{nInvaders} recipe:
\begin{verbatim}
bitbake ninvaders
\end{verbatim}

Try to make the recipe on your own. Also eliminate the warnings related to your
recipe: some configuration variables are not mandatory but it is a very good
practice to define them all.

If you hang on a problem, check the following points:
\begin{itemize}
  \item The common recipe is included in the version specific one
  \item The checksum and the URI are valid
  \item The dependencies are explicitly defined
  \item The internal state has changed, clean the working directory: \\
     \code{bitbake -c cleanall ninvaders}
\end{itemize}

Tip: BitBake has command line flags to increase its verbosity and activate debug
outputs. Also, remember that you need to cross-compile nInvaders for ARM ! Maybe,
you will have to configure your recipe to resolve some mistakes done in the
application's Makefile (which is often the case). A bitbake variable permits
to add some Makefile's options, you should look for it.

\section{Update the rootfs and test}

Now that you've compiled the \code{nInvaders} application, generate a new
rootfs image with \code{bitbake core-image-minimal}. Then update the
NFS root directory. You can confirm the \code{nInvaders} program is
present by running:
\begin{verbatim}
find /nfs -iname ninvaders
\end{verbatim}

Access the board command line through SSH. You should be able to
launch the \code{nInvaders} program. Now, it's time to play!
