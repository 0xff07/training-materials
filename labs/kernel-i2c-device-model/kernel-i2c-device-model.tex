\subchapter{Device Model - I2C device}{Objective: declare an I2C device
  and basic driver hooks called when this device is detected}

Throughout the upcoming labs, we will implement a driver for an I2C
device, which offers the functionality of an I2C nunchuks.

After this lab, you will be able to:

\begin{itemize}
\item Add an I2C device to a device tree
\item Implement basic \code{probe()} and \code{remove()} driver
functions and make sure that they are called when such a device is added or removed.
\end{itemize}

\section{setup}

Go to the \code{~/felabs/linux/src/linux} directory. Check out the
\code{3.11.y-bbb} branch if needed. 

Now create a new \code{nunchuk} branch starting from the
\code{3.11.y-bbb} branch,  for your upcoming work on the nunchuk
driver.  

\section{Connecting the nunchuk}

Take the nunchuk device provided by your instructor.

Now let's identify the 4 pins of the nunchuk connector.

\begin{itemize}
\item The \code{ground} pin.
\item The \code{power} pin.
\item The \code{clock} pin.
\item The \code{data} pin.
\end{itemize}

(TODO: give more details when we receive the final connectors)

Open the System Reference Manual that you downloaded earlier,
and look for "connector P9" in the table of contents, and then
follow the link to the corresponding section. Look at the table listing
the pinout of the P9 connector.

Now connect the nunchuk pins:
\begin{itemize}
\item The \code{ground} pin to pins 1 or 2 (\code{GND})
\item The \code{power} pin to pins 3 or 4 (\code{DC_3.3V})
\item The \code{clock} pin to pin 17 (\code{I2C1_SCL})
\item The \code{data} pin to pin 18 (\code{I2C1_SDA})
\end{itemize}

(TODO: add photos)

\section{Update the board device tree}

To let the Linux kernel handle a new device, the first thing is....



