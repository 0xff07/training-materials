\subchapter{Device Model - I2C device}{Objective: declare an I2C device
  and basic driver hooks called when this device is detected}

Throughout the upcoming labs, we will implement a driver for an I2C
device, which offers the functionality of an I2C nunchuk.

After this lab, you will be able to:

\begin{itemize}
\item Add an I2C device to a device tree.
\item Implement basic \code{probe()} and \code{remove()} driver
functions and make sure that they are called when there is a
device/driver match.
\item Find your driver and device in \code{/sys}.
\end{itemize}

\section{Setup}

Go to the \code{~/linux-kernel-labs/src/linux} directory. Check out the
\code{5.5.y} branch.

Now create a new \code{nunchuk} branch starting from the
\code{5.5.y} branch,  for your upcoming work on the nunchuk
driver.

During this lab, we will start to implement a driver for a
Nunchuk I2C device, but at this stage we won't need to connect
it yet, as we will enable I2C on the right board pins in the
next lab.

Download a useful document sharing useful details about the nunchuk
and its connector:\\
\url{https://bootlin.com/labs/doc/nunchuk.pdf}

\section{Create a custom device tree}

To let the Linux kernel handle a new device, we need to add a
description of this device in the board device tree.

As the Beaglebone Black device tree is provided by the kernel community,
and will continue to evolve on its own, we don't want to make changes
directly to the device tree file for this board. 

The easiest way to customize the board DTS is to create a new DTS file
that includes the Beaglebone Black or Black Wireless DTS, and add
its own definitions.

So, create a new \code{am335x-customboneblack.dts} file in which you
just include the regular board DTS file. We will add further definitions
in the next sections.

Now, modify the corresponding \code{Makefile} to make sure the
new DTS is compiled automatically.

\subsection{Enable the second I2C bus}

We are first going to enable and configure the second I2C bus
(\code{i2c1}).

First, as an exercise, find the DTS include file defining (\code{i2c1})
for the SoC in our board.

Then, make a reference to this definition in your custom DTS and
enable this bus. Also configure it to function at 100 KHz. That's
enough so far!

\subsection{Declare the Nunchuk device}

As a child node to the \code{i2c1} bus, now declare the \code{nunchuk}
device, choosing \code{nintendo,nunchuk} for its \code{compatible}
property. You will find the I2C slave address of the nunchuk on the
nunckuk document that we have downloaded earlier\footnote{This I2C slave
addressed is enforced by the device itself. You can't change it.}.

\subsection{Checking the device tree on the running system}

Now, just compile your DTB by asking the kernel Makefile to recompile
only DTBs:

\begin{verbatim}
make dtbs
\end{verbatim}

Now, copy the new DTB to the tftp server home directory, change the DTB
file name in the U-Boot configuration\footnote{Tip: you just need to run
\code{editenv bootcmd} and \code{saveenv}.}, and boot the board.

Through the \code{/sys/firmware/devicetree} directory, it is possible to check
the Device Tree settings that your system has loaded. That's useful when
you are not sure exactly which settings were actually loaded.

For example, you can check the presence of a new \code{nunchuk} node in
your device tree:

{\small
\begin{verbatim}
# find /sys/firmware/devicetree -name "*nunchuk*"
/sys/firmware/devicetree/base/ocp/interconnect@48000000/segment@0/target-module@2a000/i2c@0/nunchuk@52
\end{verbatim}
}

As the base address of the I2C1 controller registers was not explicited
in the DTSI file (at least not in the corresponding node), you can now
compute this address from the above line: it's \code{0x8000000 + 0x2a000
= 0x802a000}.

Also find the same address in the big processor Technical Reference
Manual\footnote{Tip: to do your search, put an underscore character
in the middle of the address, as in \code{FFFF_FFFF}... that's how
addresses are written in this document.}.

Back to \code{/sys/firmware/devicetree/}, you can also check the whole
structure of the loaded Device Tree, using the Device Tree Compiler
(\code{dtc}, which we put in the root filesystem. That's better than
checking the source files and includes in the source directory:

\begin{verbatim}
# dtc -I fs /sys/firmware/devicetree/base/ > /tmp/dts
\end{verbatim}

Look for \code{i2c1} and \code{nunchuk} in the output file,
and see where the nodes are instantiated. Don't hesitate to ask your
instructor for questions!

\section{Implement a basic I2C driver for the nunchuk}

It is now time to start writing the first building blocks of the I2C
driver for our nunchuk.

In a new terminal, go to
\code{~/linux-kernel-labs/modules/nfsroot/root/nunchuk/}.  This directory
contains a Makefile and an almost empty \code{nunchuk.c} file.

Now, you can compile your out-of-tree module by running \code{make}. As
the current directory is part of the NFS root that the board boots on,
the generated \code{.ko} file will immediately be visible on the board
too.

Relying on explanations given during the lectures, fill the
\code{nunchuk.c} file to implement:

\begin{itemize}
\item \code{probe()} and \code{remove()} functions that will
      be called when a nunchuk is found.
      For the moment, just put a call to \kfunc{pr_info} inside
      to confirm that these function are called.
\item Initialize a \ksym{i2c_driver} structure, and register
      the i2c driver using it. Make sure that you use
      a \code{compatible} property that matches the one in the
      Device Tree.
\end{itemize}

You can now compile your module and reboot your board, to
boot with the updated DTB.

\section{Driver tests}

You can now load the \code{/root/nunchuk/nunchuk.ko} file.
You need to check that the \code{probe()} function gets called
then, and that the \code{remove()} function gets called too
when you remove the module.

Once your new Device Tree and module work as expected, commit
your DT changes in your Linux tree:

\begin{verbatim}
git commit -sa
\end{verbatim}

\section{Exploring /sys}

Take a little time to explore \code{/sys}:

\begin{itemize}
\item Find the representation of your driver. That's a way
      of finding the matching devices.
\item Find the representation of your device, containing its name.
      You will find a link to the driver too.
\end{itemize}
