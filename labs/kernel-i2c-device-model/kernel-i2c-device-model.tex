\subchapter{Device Model - I2C device}{Objective: declare an I2C device
  and basic driver hooks called when this device is detected}

Throughout the upcoming labs, we will implement a driver for an I2C
device, which offers the functionality of an I2C nunchuks.

After this lab, you will be able to:

\begin{itemize}
\item Add an I2C device to a device tree
\item Implement basic \code{probe()} and \code{remove()} driver
functions and make sure that they are called when such a device is added or removed.
\end{itemize}

\section{Setup}

Go to the \code{~/felabs/linux/src/linux} directory. Check out the
\code{3.11.y-bbb} branch if needed. 

Now create a new \code{nunchuk} branch starting from the
\code{3.11.y-bbb} branch,  for your upcoming work on the nunchuk
driver.  

\section{Connecting the nunchuk}

Take the nunchuk device provided by your instructor.

We will connect it to the second I2C port of the CPU (\code{i2c1}),
which pins are available on the \code{P9} connector.

Download a useful document sharing useful details about the nunchuk
and its connector: 
\url{http://web.engr.oregonstate.edu/~sullivae/ece375/pdf/nunchuk.pdf}

Now we can identify the 4 pins of the nunchuk connector.

\begin{itemize}
\item The \code{ground} pin.
\item The \code{power} pin.
\item The \code{clock} pin.
\item The \code{data} pin.
\end{itemize}

(TODO: give more details when we receive the final connectors)

Open the System Reference Manual that you downloaded earlier,
and look for "connector P9" in the table of contents, and then
follow the link to the corresponding section. Look at the table listing
the pinout of the P9 connector.

Now connect the nunchuk pins:
\begin{itemize}
\item The \code{ground} pin to pins 1 or 2 (\code{GND})
\item The \code{power} pin to pins 3 or 4 (\code{DC_3.3V})
\item The \code{clock} pin to pin 17 (\code{I2C1_SCL})
\item The \code{data} pin to pin 18 (\code{I2C1_SDA})
\end{itemize}

(TODO: add photos)

\section{Update the board device tree}

To let the Linux kernel handle a new device, the first thing is to add a
description for it in the board device tree.

Do this by editing the \code{arch/arm/boot/dts/am335x-bone-common.dtsi}
file describing all the buses and devices. You will need to follow the
examples given in the lectures.

\begin{enumerate}
\item Add a node declaring a second I2C bus (\code{i2c1}), functioning
      at 400 KHz too. As for \code{i2c0}, you will need to declare
      the base address of its registers. Open the processor datasheet
      and find this address
      \footnote{Tip: you can look-up the \code{i2c0} base address which
      you already know from the existing Device Tree. The base address
      for \code{i2c1} won't be far away.}.
\item As a child node to this second bus, declare the \code{nunchuk}
      device, choosing \code{nintendo,nunchuk} for its \code{compatible}
      property. You find the I2C slave address of the nunchuk on
      the nunckuk document that we have used earlier
      \footnote{This I2C slave addressed is enforced by the device
      itself. You can't change it.}.
\end{enumerate}

Once this is done, recompile your DTB and copy the updated version to
the tftp server home directory.

\section{Implement a basic I2C driver for the nunchuk}

It is now time to start writing the first building blocks of the I2C
driver for our nunchuk.

In a new terminal, go to \code{~/felabs/linux/modules/nfsroot/nunchuk/}.
This directory contains a Makefile and an almost empty \code{nunchuk.c}
file.

Source the same \code{env.sh} file that you using for kernel compiling.

Now, you can compile your out-of-tree module by running \code{make}. As
the current directory is part of the NFS root that the board boots on,
the generated \code{.ko} file will immediately be visible on the board
too.

Relying on explanations given during the lectures, fill the
\code{nunchuk.c} file to implement:

\begin{itemize}
\item \code{probe()} and \code{remove()} functions that will
      be called when a nunchuk is found.
      For the moment, just put a call to \code{pr_info()} inside
      to confirm that these function are called.
\item Initialize an \code{i2c_driver} structure, and register
      the i2c driver using it. Make sure that you use
      a \code{compatible} property that matches the one in the
      Device Tree.
\end{itemize}

You can now compile your module and reboot your board, to 
boot with the updated DTB.

\section{Driver tests}

You can now load the \code{/root/nunchuk/nunchuk.ko} file.
You need to check that the \code{probe()} function gets called
then, and that the \code{remove()} function gets called too
when you remove the module.

Once your new Device Tree and module work as expected, commit
your DT changes in your Linux tree:

\begin{verbatim}
git commit -sa 
\end{verbatim}
