\subchapter{Build the system}{Objective: compile the root filesystem.}

After this lab, you will have a ready to use root filesystem to boot
your system with, including a video player application.

We haven't compiled the bootloader and kernel for our board yet,
but since this part can take a long time (especially compiling the
cross-compiling toolchain), let's start it now, while we are still
fetching kernel sources or going through lectures.

\section{Setup}

As specified in the Buildroot
manual\footnote{\url{https://buildroot.org/downloads/manual/manual.html\#requirement-mandatory}},
Buildroot requires a few packages to be installed on your
machine. Let's install them using Ubuntu's package manager:

\begin{verbatim}
sudo apt install sed make binutils gcc g++ bash patch \
  gzip bzip2 perl tar cpio python unzip rsync wget libncurses-dev
\end{verbatim}

You will later also find that you also need extra packages:

\begin{verbatim}
sudo apt install bison flex
\end{verbatim}

\section{Choosing a Buildroot release}

Buildroot is one of the best tools for building a custom root filesystem
for a dedicated embedded system with a fixed set of features, typically
like the one we're trying to build.

Go to the /code{~/boot-time-labs/rootfs/buildroot/} directory.

We will use the latest revision of the \code{2019.02} release, which is
one of Buildroot's long term releases:

\begin{verbatim}
git tag | grep 2019.02
git checkout 2019.02.2
\end{verbatim}

\section{Configuring Buildroot}

To minimize external dependencies and maximize flexibility, we will ask
Buildroot to generate its own toolchain. This can be better than using
external toolchains, as we have the ability to tweak toolchain settings
in a fine way if needed.

Start the Buildroot configuration utility:

\begin{verbatim}
make menuconfig
\end{verbatim}

\begin{itemize}

\item \code{Target Options} menu
  \begin{itemize}

  \item \code{Target Architecture}: select \code{ARM (little endian)}

  \item \code{Target Architecture Variant}: select \code{cortex-A8}

  \item On ARM two {\em Application Binary Interfaces} are available:
    \code{EABI} and \code{EABIhf}. Unless you have backward
    compatibility concerns with pre-built binaries, \code{EABIhf} is
    more efficient, so make this choice as the \code{Target ABI}
    (which should already be the default anyway).

  \item The other parameters can be left to their default value:
    \code{ELF} is the only available \code{Target Binary Format},
    \code{VFPv3-D16} is a sane default for the {\em Floating Point
      Unit}, and using the \code{ARM} instruction set is also a good
    default (we could use the \code{Thumb-2} instruction set for
    slightly more compact code).
  \end{itemize}

\item As we wish to make Buildroot compile its own toolchain as as we
will compile the kernel and bootloader separately, for the moment, we
can keep the default build, toolchain and system settings.

\item \code{Target packages} menu
  \begin{itemize}
  \item In \code{Audio and video applications}, select \code{ffmpeg} and
  inside the \code{ffmpeg} submenu, add the below options:
     \begin{itemize}
     \item Select \code{Build libswscale}
     \end{itemize}
  \end{itemize}

\item For the moment, all the remaining default settings are fine for us.
\end{itemize}

However, we need to do one thing to customize the root filesystem: we
need to add a script that will automatically start the \code{ffmpeg}
video player.

To do so, we will use Buildroot's {\em Root filesystem overlay}
capability, which allows to drop ready-made files into the final root
filesystem archive\footnote{See
\url{https://buildroot.org/downloads/manual/manual.html\#customize} for
details.}.

To begin with, let's start by creating a specific directory to store our
Buildroot customizations for our project.

\begin{verbatim}
mkdir board/beaglecam
\end{verbatim}

And in this directory, let's create a directory for root filesystem
overlays:

\begin{verbatim}
mkdir board/beaglecam/rootfs-overlay
\end{verbatim}

Now, let's copy a script that we're providing you to \code{etc/init.d}:

\begin{verbatim}
mkdir -p board/beaglecam/rootfs-overlay/etc/init.d/
cp ~/boot-time-labs/rootfs/data/S50playvideo board/beaglecam/rootfs-overlay/etc/init.d/
\end{verbatim}

We can now run \code{make menuconfig} again and in \code{System
configuration}, add \code{board/beaglecam/rootfs-overlay} to
\code{Root filesystem overlay directories}.

\section{Running Buildroot}

We are now ready to execute Buildroot:

\begin{verbatim}
make
\end{verbatim}

Enjoy, and be patient, as building a cross-compiling toolchain takes
time!
