\subchapter{Appendix - Results}{Results obtained in the various steps}

\section{Initial boot time components}

Here are the results we got after making the initial measurements:

\begin{tabular}{| l | c | c | r |}
  \hline
  Boot phase & Start string & Time stamp & Elapsed \\
  \hline                        
  \hline
  RomBOOT & \code{RomBOOT} & 0.000000 & 55 ms \\
  \hline                        
  Bootstrap & \code{AT91Bootstrap 3.5.2} & 0.054626 & 978 ms \\
  \hline                        
  Bootloader & \code{U-Boot 2012.10} & 1.032380 & 7,720 ms \\
  \hline  
  Kernel & \code{Booting Linux on physical CPU 0} & 8.752495 & 2,795 ms \\
  \hline  
  Init scripts & \code{init started: BusyBox v1.21.0} & 11.547943 & 1,890 ms \\
  \hline  
  Critical application & \code{Starting video player} & 13.438071 & 1,244 ms \\
  \hline  
  Critical application ready & \code{Done playing video} & 14.682485 & \\
  \hline  
  \hline  
  \multicolumn{3}{| c |}{Total} & 14,682 ms \\
  \hline  
\end{tabular}

The original boot log is available in
\code{/opt/felabs/results/measuring/boot-instrumented1.log}.

\section{Init script optimizations}

\section{Results}

Here are the results we obtained from our own experiments:

\begin{tabular}{| l | c | c | r |}
  \hline
  Technique & Total time stamp & Difference with previous
experiment \\
  \hline
  \hline
  Original time & 14.682485 & N/A \\
  \hline
  Remove \code{udev} & 13.667242 & -1,015 ms \\
  \hline
  Postpone services & 13.003734 & -666 ms \\
  \hline
  Optimize launcher & 12.920196 & -84 ms\\
  \hline
  Other trick & & \\
  \hline
  Other trick & & \\
  \hline
  \hline
  \multicolumn{2}{| c |}{Total gain} & 1,765 ms\\
  \hline
\end{tabular}

The original boot logs are available in
\code{/opt/felabs/results/init-scripts/}.

\section{Kernel optimizations}

\subsection{Measuring}

Here are the longuest initcalls that we discovered on the boot graph:

\begin{itemize}
\item \code{atmel_hlcdfb_init} (about 800 ms!): obviously the
      framebuffer driver
      for the Atmel's LCD controller. A quick look at the source code
      \code{drivers/video/atmel_hlcdfb.c} doesn't reveal any obvious
      delay loop and anything pathological. There is no module parameter
      either which could be used to shorten probe time.
\item \code{atmel_serial_init} (about 500 ms!): obviously Atmel's serial
      driver. No obvious issue found in the code. Another driver
      that needs optimizing!
\item \code{atmel_nand_init} (about 100 ms). Nothing suspicious found.
      May be scanning for bad blocks though (not immediately found in
      the code).
\item \code{ubi_init} (about 1000 ms!). That's the longuest one.
      This corresponds to the time running the \code{ubi_attach}
      functionality. A {\em UBI fastmap} feature was added in Linux 3.7
      to address this issue, but our kernel version doesn't have it
      yet. A solution would be to switch to a newer stable kernel
      when available. At least, you can't remove this one
      as long as the rootfilesystem (in a UBI partition) depends on
      it.
\item \code{macb_init} (Ethernet driver), \code{ehci_hcd_init} and
      \code{ohci_hcd_mod_init} (USB host controller drivers): the
      corresponding devices are not used in the boot sequence. We can
      build their drivers as modules and load them later.
\item \code{qt1070_driver_init} (about 500 ms)! Reading the code
      reveals that this is a driver for the touch keys on the left
      hand of the LCD (\code{K1} to \code{K2} keys). Another driver
      which would be worth optimizing, but which could be built as a module.
\item \code{atmci_init} (about 100 ms). That's the Atmel Multimedia
      Card (MMC) interface. It can also be build as a module.
\end{itemize}

\subsection{Postponing, compiling some drivers as modules}

Thanks to the boot graph highlighting them, here are the
kernel configuration settings that we modified, to compile
some drivers as modules: \code{CONFIG_MACB, CONFIG_USB_OHCI_HCD,
CONFIG_USB_EHCI_HCD, CONFIG_KEYBOARD_QT1070}.

\subsection{Results}

Here are the results we obtained from our own experiments:

\begin{tabular}{| p{6cm} | c | r |}
  \hline
  Technique & Total time stamp & Difference with previous experiment \\
  \hline
  \hline
  Original time & 13.252827 & N/A \\
  \hline
  Postpone functionality (drivers compiled as modules) & 12.302508 & -950 ms\\
  \hline
  Remove unnecessary functionality (\code{quiet} option) & 11.166088 & -1.136 ms\\
  \hline
  Remove unnecessary functionality (skip delay loop calibration) & Not tested & \\
  \hline

  \hline
  \multicolumn{2}{| c |}{Total gain} & -2,087 ms \\
  \hline
\end{tabular}

