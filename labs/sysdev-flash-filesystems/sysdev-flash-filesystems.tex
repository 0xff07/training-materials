\subchapter{Filesystems - Flash file systems}{Objective: Understand flash file systems usage and their integration on the target}

After this lab, you will be able to:
\begin{itemize}
\item Prepare filesystem images and flash them.
\item Define partitions in embedded flash storage.
\end{itemize}

\section{Setup}

Stay in \code{$HOME/felabs/sysdev/tinysystem}. Install the
\code{mtd-utils} package, which will be useful to create JFFS2
filesystem images.

\section{Goals}

Instead of using an external MMC card as in the previous lab, we will
make our system use its internal flash storage.

The root filesystem will still be in a read-only filesystem, put on an
MTD partition.  Read/write data will be stored in a JFFS2 filesystem
in another MTD partition. The layout of the internal NAND flash will
be:

\begin{center}
  \includegraphics[width=\textwidth]{labs/sysdev-flash-filesystems/flash-map.pdf}
\end{center}

\section{Enabling NAND flash and filesystems}

Apply the \code{0001-ARM-omap-switch-back-to-SW-ECC-on-IGEP.patch}
patch available in the
\code{$HOME/felabs/sysdev/flash-filesystems/data} directory to your
kernel tree. This is a Device Tree patch that fixes a problem
of the 3.11.x/3.12 kernel, which caused the kernel to use a
different ECC scheme than the U-Boot bootloader, making it
impossible to flash from U-Boot, and read from the kernel.

After applying this patch, recompile your kernel with support for
JFFS2 and for support for MTD partitions specified in the kernel
command line (\code{CONFIG_MTD_CMDLINE_PARTS}).

Also enable support for the flash chips on the board
(\code{CONFIG_MTD_NAND_OMAP2}). You also need to enable support for
hardware BCH error correction (\code{CONFIG_NAND_OMAP_BCH}) and select
the \code{8 bits / 512 bytes (recommended)} mode
(\code{MTD_NAND_OMAP_BCH8}).

Last but not least, disable \code{CONFIG_PROVE_LOCKING}. This option is
currently causing problems with the JFFS2 filesystem. This option is
in \code{Kernel Hacking} $\rightarrow$
\code{Lock debugging: prove locking correctness}.

After compiling your kernel, update the DTB file used by your board
(remember that we applied a Device Tree patch). 

You will update your kernel image on flash in the next section.

\section{Filesystem image preparation}

Find the erase block size of the NAND flash device in your board.

Prepare a JFFS2 filesystem image from the \code{/www/upload/files}
directory from the previous lab.

Modify the \code{/etc/init.d/rcS} file to mount a JFFS2 filesystem on
the seventh flash partition (we will declare flash partitions in the
next section), instead of an ext3 filesystem on the
third MMC disk partition.

Create a JFFS2 image for your root filesystem, with the same options
as for the data filesystem.

\section{MTD partitioning and flashing}

Look at the way default flash partitions are defined in the board
Device Tree sources (\code{arch/arm/boot/dts/omap3-igep0020.dts}).

However, they do not match the way we wish to organize our flash
storage. Therefore, we will define our own partitions at boot time,
on the kernel command line.

Enter the U-Boot shell and erase NAND flash, from offset 0x300000,
up to the end of the NAND flash storage. You'll have to compute the
remaining size of the flash, from 0x300000 to the end. Remember that
you can look at U-Boot booting messages to find what the size of
the NAND flash is.

Before flashing JFFS2 images, make sure they will be flashed during
the software ECC scheme, by running the \code{nandecc sw} command in
U-Boot.

Using the \code{tftp} command, download and flash the new kernel
image at the correct location.

Using the \code{tftp} command, download and flash the JFFS2 image
of the root filesystem the correct location.

Using the \code{tftp} command, download and flash the JFFS2 image of the
data filesystem at the correction location.

Don't forget that you can write U-Boot scripts to automate these
procedures. This is very handy to avoid mistakes when typing commands!

Set the \code{bootargs} variable so that:

\begin{itemize}
\item You define the 7 MTD partitions, as detailed previously
\item The root filesystem is mounted from the 6\textsuperscript{th}
  partition, and is mounted read-only (kernel parameter \code{ro}).
  {\bf Important: even if this partition is mounted read-only, the
  MTD partition itself must be declared as read-write. Otherwise,
  Linux won't be able to perform ECC checks on it, which involve
  both reading and writing.}
\end{itemize}

Boot the target, check that MTD partitions are well configured, and
that your system still works as expected. Your root filesystem should
be mounted read-only, while the data filesystem should be mounted
read-write, allowing you to upload data using the web server.
