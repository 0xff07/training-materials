\subchapter{Filesystems - Flash file systems}{Objective: Understand flash file systems usage and their integration on the target}

After this lab, you will be able to:
\begin{itemize}
\item Prepare filesystem images and flash them.
\item Define partitions in embedded flash storage.
\end{itemize}

\section{Setup}

Stay in \code{$HOME/embedded-linux-labs/tinysystem}. Install the
\code{mtd-utils} package, which will be useful to create JFFS2
filesystem images.

\section{Goals}

Instead of using an external MMC card as in the previous lab, we will
make our system use its internal flash storage.

The root filesystem will still be in a read-only filesystem, put on an
MTD partition.  Read/write data will be stored in a JFFS2 filesystem
in another MTD partition. The layout of the internal NAND flash will
be:

\begin{center}
  \includegraphics[width=\textwidth]{labs/sysdev-flash-filesystems/flash-map.pdf}
\end{center}

\section{Enabling NAND flash and filesystems}

First, recompile your kernel with support for JFFS2 and for support
for MTD partitions specified in the kernel command line
(\code{CONFIG_MTD_CMDLINE_PARTS}).

Last but not least, disable \code{CONFIG_PROVE_LOCKING}. This option
is currently causing problems with the JFFS2 filesystem. This option
is in \code{Kernel Hacking} $\rightarrow$ \code{Lock debugging: prove
  locking correctness}.

Recompile your kernel, and we will update your kernel image on flash
in the next section.

\section{Filesystem image preparation}

Find the erase block size of the NAND flash device in your board.

Prepare a JFFS2 filesystem image from the \code{/www/upload/files}
directory from the previous lab.

Now, we will get back to the root filesystem contents in the
\code{nfsroot} directory, and will generate a JFFS2 image from them.

In this directory, modify the \code{/etc/init.d/rcS} file to
mount a JFFS2 filesystem on the eighth flash partition
(we will declare flash partitions in the next section),
instead of an ext3 filesystem on the third MMC disk partition.

Create a JFFS2 image for your root filesystem, with the same options
as for the data filesystem.

\section{MTD partitioning and flashing}

Look at the way default flash partitions are defined in the board
Device Tree sources
(\code{arch/arm/boot/dts/at91-sama5d3_xplained.dts}).

However, they do not match the way we wish to organize our flash
storage. Therefore, we will define our own partitions at boot time,
on the kernel command line.

Enter the U-Boot shell and erase NAND flash, from offset \code{0x160000},
up to the end of the NAND flash storage. You'll have to compute the
remaining size of the flash, from \code{0x160000} to the end. Remember
that you can look at U-Boot booting messages to find what the size of
the NAND flash is.

The DTB (device tree blob) has been already written in the NAND flash
and it didn't change so we keep it.

Using the \code{tftp} command, download and flash the new kernel
image at the correct location.

Using the \code{tftp} command, download and flash the JFFS2 image
of the root filesystem the correct location.

Using the \code{tftp} command, download and flash the JFFS2 image of
the data filesystem at the correction location.

Use the \code{trimffs} version of the command \code{nand write} for the
JFFS2 image. The command \code{nand write.trimffs} skips the blank
sectors instead of writing them. It is needed because the algorithm
used by the hardware ECC for the SAMA5D3 SoC generates a checksum with
bytes different of 0xFF if the page is blank, whereas Linux expects to
have the OOB full of 0xFF if the associated page is blank too.

Don't forget that you can write U-Boot scripts to automate these
procedures. This is very handy to avoid mistakes when typing commands!

Set the \code{bootargs} variable so that:

\begin{itemize}
\item You define the 8 MTD partitions, as detailed previously
\item The root filesystem is mounted from the 7\textsuperscript{th}
  partition, and is mounted read-only (kernel parameter
  \code{ro}).
  \begin{itemize}
  \item \bf{Important: even if this partition is mounted read-only,
      the MTD partition itself must be declared as read-write.
      Otherwise, Linux won't be able to perform ECC checks on it,
      which involve both reading and writing.}
  \end{itemize}
\end{itemize}

Boot the target, check that MTD partitions are well configured, and
that your system still works as expected. Your root filesystem should
be mounted read-only, while the data filesystem should be mounted
read-write, allowing you to upload data using the web server.
