\subchapter{Filesystems - Flash file systems}{Objective: Understand flash file systems usage and their integration on the target}

After this lab, you will be able to:
\begin{itemize}
\item Prepare filesystem images and flash them.
\item Define partitions in embedded flash storage.
\end{itemize}

\section{Setup}

Stay in \code{/home/<user>/felabs/sysdev/fs}. Install the
\code{mtd-utils} package, which will be useful to create JFFS2
filesystem images.

\section{Goals}

Instead of using an external MMC card as in the previous lab, we will
make our system use its internal flash storage.

The root filesystem will still be in a read-only filesystem, put on an
MTD partition.  Read/write data will be stored in a JFFS2 filesystem
in another MTD partition. The layout of the internal NAND flash will
be:

\begin{itemize}
\item From 0 to 0x80000, X-Loader (512 KB)
\item From 0x80000 to 0x200000, U-Boot (1536 KB)
\item From 0x200000 to 0x280000, U-Boot environment (512 KB)
\item From 0x280000 to 0x680000, Linux kernel (4 MB)
\item From 0x680000 to 0x880000, the JFFS2 root filesystem (2 MB) as read-only
\item From 0x880000 to the end, the JFFS2 data filesystem
\end{itemize}

\section{Enabling NAND flash and filesystems}

Recompile your kernel with support for JFFS2 and for support for MTD
partitions specified in the kernel command line
(\code{CONFIG_MTD_CMDLINE_PARTS}).

Also enable support for the flash chips on the board
(\code{CONFIG_MTD_NAND_OMAP2}).

Update your kernel image on flash.

\section{Filesystem image preparation}

Find the erase block size of the NAND flash device in your board.

Prepare a JFFS2 filesystem image from the \code{/www/uploads/files}
directory from the previous lab.

Modify the \code{/etc/init.d/rcS} file to mount a JFFS2 filesystem on
the sixth flash partition, instead of an ext3 filesystem on the third
MMC disk partition.

Create a JFFS2 image for your root filesystem. with the same options
as the data filesystem.

\section{MTD partitioning and flashing}

Memory layout and partitioning can be defined inside kernel sources,
naturally in the \code{arch/<arch>/<march>/board-<name>.c} since it is
board dependent. Nevertheless, during device development, it can be
useful to define partitions at boot time, on the kernel command line.

Enter the U-Boot shell and erase NAND flash, from offset
0x00280000, up to the end of the NAND flash storage (tip: if you don't
specify the number of bytes to erase, \code{onenand erase} will
continue until it reaches the end).

Using the \code{tftp} command, download and flash the kernel image at
the correct location.

Using the \code{tftp} command, download and flash the JFFS2-ro image
the correct location.

Using the \code{tftp} command, download and flash the JFFS2 image at
the correction location.

Don't forget that you can write U-Boot scripts to automate those
procedures. This is very handy to avoid mistakes when typing commands!

Look at the way MTD partitions are defined in the kernel sources
(\code{arch/arm/mach-omap2/board-igep-0020.c})

Set the \code{bootargs} variable so that:

\begin{itemize}
\item you define the 6 MTD partitions, as detailed previously
\item the root filesystem is mounted from the 5\textsuperscript{th}
  partition, and is mounted read-only (kernel parameter \code{ro})
\end{itemize}

Boot the target, check that MTD partitions are well configured, and
that your system still works as expected. Your root filesystem should
be mounted read-only, while the data filesystem should be mounted
read-write, allowing you to upload data using the web server.
