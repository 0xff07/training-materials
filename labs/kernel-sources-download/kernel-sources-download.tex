\subchapter{Downloading kernel source code}{Get your own copy of the
mainline Linux kernel source tree}

\section{Setup}

Go to the \code{$HOME/felabs/linux/src} directory.

\section{Installing git packages}

First, let's install software packages that we will need
throughout the practical labs:

\begin{verbatim}
sudo apt-get install git gitk git-email
\end{verbatim}

\section{Cloning the mainline Linux tree}

To begin working with the Linux kernel sources, we need to clone its
reference git tree, the one managed by Linus Torvalds.

The trouble is you have to download about 1.5 GB of data!

If you are running this command from home, or if you have very fast
access to the Internet at work (and if you are not 256 participants in the
training room), you can do it directly by connecting to
\url{http://git.kernel.org}:

{\small
\begin{verbatim}
git clone git://git.kernel.org/pub/scm/linux/kernel/git/torvalds/linux.git
\end{verbatim}
}

or if the network port for \code{git} is blocked by the corporate
firewall, you can use the \code{http} protocol as a less efficient
fallback:

{\small
\begin{verbatim}
git clone http://git.kernel.org/pub/scm/linux/kernel/git/torvalds/linux.git 
\end{verbatim}
}

If Internet access is not fast enough and if multiple people have to
share it, your instructor will give you a USB flash drive with a
\code{tar.xz} archive of a recently cloned Linux source tree.

You will just have to extract this archive in the current directory,
and then pull the most recent changes over the network:

\begin{verbatim}
tar Jxf linux-git.tar.xz
cd linux
git pull
\end{verbatim}

Of course, if you directly ran \code{git clone}, you won't have run 
\code{git pull} today. You may run \code{git pull} every morning though,
or at least every time you need an update of the upstream source tree.

Now, let's continue the lectures. This will leave time for the commands
that you typed to complete their execution (if needed).
