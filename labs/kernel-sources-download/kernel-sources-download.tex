\subchapter{Downloading kernel source code}{Get your own copy of the
mainline Linux kernel source tree}

\section{Setup}

Create the \code{$HOME/linux-kernel-labs/src} directory.

\section{Installing git packages}

First, let's install software packages that we will need
throughout the practical labs:

\begin{verbatim}
sudo apt install git gitk git-email
\end{verbatim}

\section{Git configuration}

After installing \code{git} on a new machine, the first thing to do is
to let \code{git} know about your name and e-mail address:

\begin{verbatim}
git config --global user.name ’My Name’
git config --global user.email me@mydomain.net
\end{verbatim}

Such information will be stored in commits. It is important
to configure it properly when the time comes to generate and
send patches, in particular.

\section{Cloning the mainline Linux tree}

To begin working with the Linux kernel sources, we need to clone its
reference git tree, the one managed by Linus Torvalds.

However, this requires downloading more than 2.7 GB of data. If you
are running this command from home, or if you have very fast access to
the Internet at work (and if you are not 256 participants in the
training room), you can do it directly by connecting to
\url{https://git.kernel.org}:

{\small
\begin{verbatim}
git clone git://git.kernel.org/pub/scm/linux/kernel/git/torvalds/linux.git
\end{verbatim}
}

or if the network port for \code{git} is blocked by the corporate
firewall, you can use the \code{https} protocol as a less efficient
fallback:

{\small
\begin{verbatim}
git clone https://git.kernel.org/pub/scm/linux/kernel/git/torvalds/linux.git
\end{verbatim}
}

If Internet access is not fast enough and if multiple people have to
share it, your instructor will give you a USB flash drive with a
\code{tar.gz} archive of a recently cloned Linux source tree.

You will just have to extract this archive in the current directory,
and then pull the most recent changes over the network:

\begin{verbatim}
tar xf linux-git.tar.gz
cd linux
git checkout master
git pull
\end{verbatim}

Of course, if you directly ran
\code{git clone}, you won't have to run \code{git pull}, as
\code{git clone} already retrieved the latest changes. You may need to
run \code{git pull} in the future though, if you want to update a
newer Linux version.

\section{Accessing stable releases}

Having the Linux kernel development sources is great, but when you are
creating products, you prefer to avoid working with a target that moves
every day.

That's why we need to use the {\em stable} releases of the Linux
kernel.

Fortunately, with \code{git}, you won't have to clone an entire source
tree again. All you need to do is add a reference to a {\em remote}
tree, and fetch only the commits which are specific to that remote tree.

{\footnotesize
\begin{verbatim}
cd ~/linux-kernel-labs/src/linux/
git remote add stable git://git.kernel.org/pub/scm/linux/kernel/git/stable/linux-stable.git
git fetch stable
\end{verbatim}
}

As this still represents many git objects to download (450 MiB when 5.9 was
the latest version), if you are using an already downloaded git tree,
your instructor will probably have fetched the {\em stable} branch ahead
of time for you too. You can check by running:

\begin{verbatim}
git branch -a
\end{verbatim}

We will choose a particular stable version in the next labs.

Now, let's continue the lectures. This will leave time for the commands
that you typed to complete their execution (if needed).
