%\subchapter{Remote application debugging}{Objective: Use strace and
%  ltrace to diagnose program issues. Use gdbserver and a
\subchapter{Remote application debugging}{Objective: Use strace
  to diagnose program issues. Use gdbserver and a
  cross-debugger to remotely debug an embedded application}

\section{Setup}

Go to the \code{$HOME/felabs-sysdev/debugging} directory.

\section{Debugging setup}

Boot your ARM board over NFS on the filesystem produced in the {\em
  Using a build system, example with Buildroot} lab, with the same kernel.

\section{Setting up gdbserver, strace and ltrace}

\code{gdbserver}, \code{strace} and \code{ltrace} have already been
compiled for your target architecture as part of the cross-compiling
toolchain. Find them in the installation directory of your
toolchain. Copy these binaries to the \code{/usr/bin/} directory in
the root filesystem of your target system.

Unfortunately, the latest \code{ltrace} release is quite old and is
not working on recent kernels. You can go back to Buildroot, apply the
patch named \code{0001-ltrace-use-current-master.patch}, select the
\code{ltrace} package and build. this will build a fairly recent
version of \code{ltrace} that you can now install in your NFS
directory.

\section{Using strace}

\code{strace} allows to trace all the system calls made by a process:
opening, reading and writing files, starting other processes,
accessing time, etc. When something goes wrong in your application,
strace is an invaluable tool to see what it actually does, even when
you don't have the source code.

With your cross-compiling toolchain, compile the
\code{data/vista-emulator.c} program, strip it with \code{arm-linux-strip},
and copy the resulting binary to
the \code{/root} directory of the root filesystem (you might need to create
this directory if it doesn't exist yet).

\begin{verbatim}
arm-linux-gcc -o vista-emulator data/vista-emulator.c
cp vista-emulator path/to/root/filesystem/root
\end{verbatim}

Back to target system, try to run the \code{/root/vista-emulator}
program. It should hang indefinitely!

Interrupt this program by hitting \code{[Ctrl] [C]}.

Now, running this program again through the strace command and
understand why it hangs. You can guess it without reading the source
code!

Now add what the program was waiting for, and now see your program
proceed to another bug, failing with a segmentation fault.

\section{Using ltrace}

Try to run the program through \code{ltrace}. You may see that
another library is required to run this utility. Find this library in
the Buildroot output and add it to your root filesystem.

Now you should see what the program does: it tries to consume as much
system memory as it can!

\section{Using gdbserver}

We are now going to use \code{gdbserver} to understand why the program
segfaults.

Compile \code{vista-emulator.c} again with the \code{-g} option to
include debugging symbols. This time, just keep it on your workstation,
as you already have the version without debugging symbols on your target.

Then, on the target side, run \code{vista-emulator} under
\code{gdbserver}. gdbserver will listen on a TCP port for a connection
from GDB, and will control the execution of vista-emulator according
to the GDB commands:

\begin{verbatim}
gdbserver localhost:2345 vista-emulator
\end{verbatim}

On the host side, run \code{arm-linux-gdb} (also found in your toolchain):
\begin{verbatim}
arm-linux-gdb vista-emulator
\end{verbatim}

You can also start the debugger through the \code{ddd} interface:
\begin{verbatim}
ddd --debugger arm-linux-gdb vista-emulator
\end{verbatim}

GDB starts and loads the debugging information from the
\code{vista-emulator} binary that has been compiled with \code{-g}.

Then, we need to tell where to find our libraries, since they are not
present in the default \code{/lib} and \code{/usr/lib} directories on
your workstation. This is done by setting GDB \code{sysroot} variable
(on one line):

\begin{verbatim}
(gdb) set sysroot /usr/local/xtools/arm-unknown-linux-uclibcgnueabihf/
arm-unknown-linux-uclibcgnueabihf/sysroot/
\end{verbatim}

And tell gdb to connect to the remote system:
\begin{verbatim}
(gdb) target remote <target-ip-address>:2345
\end{verbatim}

Then, use \code{gdb} as usual to set breakpoints, look at the source
code, run the application step by step, etc. Graphical versions of
gdb, such as \code{ddd} can also be used in the same way. In our case,
we'll just start the program and wait for it to hit the segmentation
fault:
\begin{verbatim}
(gdb) continue
\end{verbatim}

You could then ask for a backtrace to see where this happened:
\begin{verbatim}
(gdb) backtrace
\end{verbatim}

This will tell you that the segmentation fault occurred in a function
of the C library, called by our program. This should help you in
finding the bug in our application.

\section{What to remember}

During this lab, we learned that...
\begin{itemize}
\item Compiling an application for the target system is very similar
  to compiling an application for the host, except that the
  cross-compilation introduces a few complexities when libraries are
  used.

\item It's easy to study the behavior of programs and diagnose issues
%  without even having the source code, thanks to strace and ltrace.
  without even having the source code, thanks to strace.

\item You can leave a small \code{gdbserver} program (300 KB) on your target
  that allows to debug target applications, using a standard GDB
  debugger on the development host.

\item It is fine to strip applications and binaries on the target
  machine, as long as the programs and libraries with debugging
  symbols are available on the development host.

\end{itemize}

