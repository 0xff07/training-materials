\subchapter{Remote application debugging}{Objective: Use \code{strace}
  to diagnose program issues. Use \code{gdbserver} and a
  cross-debugger to remotely debug an embedded application}

\section{Setup}

Go to the \code{$HOME/embedded-linux-labs/debugging} directory.
Create an \code{nfsroot} directory.

\section{Debugging setup}

Because of issues in {\em gdb} and {\em ltrace} in the uClibc version
that we are using in our toolchain, we will use a different toolchain
in this lab, based on {\em glibc}. 

As {\em glibc} has more complete features that lighter libraries,
it looks like a a good idea to do your application debugging work
with a {\em glibc} toolchain first, and then switch to lighter libraries
once your application and software stack is production ready. 

Extract the 2017.08.x sources into the current directory.

Then, in the \code{menuconfig} interface, configure the target
architecture as done previously but configure the toolchain and
target packages differently:

\begin{itemize}
\item In \code{Toolchain}:
   \begin{itemize}
   \item \code{Toolchain type}: \code{External toolchain}
   \item \code{Toolchain}: \code{Custom Toolchain}
   \item \code{Toolchain origin}: \code{Toolchain to be downloaded and installed}
   \item \code{Toolchain URL}:
\code{http://toolchains.free-electrons.com/downloads/releases/toolchains/armv7-eabihf/tarballs/armv7-eabihf--glibc--bleeding-edge-2017.08-rc2-5-g5c1b185-1.tar.bz2}\\
	      You can easily choose such a toolchain on
	      \url{http://toolchains.free-electrons.com} by selecting the
	      architecture, the C library and the compiler version you need. 
	      While you can try with other versions, the above toolchain
	      is known to make this lab work.
   \item \code{External toolchain gcc version}: \code{7.x}
   \item \code{External toolchain kernel headers series}: \code{4.9.x}
   \item \code{External toochain C library}: \code{glibc/eglibc}
   \item Select \code{Toolchain has SSP support?}
   \item Select \code{Toolchain has RPC support?}
   \item Select \code{Toolchain has C++ support?}
   \item Select \code{Copy gdb server to the Target}
   \end{itemize}
 \item \code{Target packages}
   \begin{itemize}
   \item \code{Debugging, profiling and benchmark}
     \begin{itemize}
     \item Select \code{ltrace}
     \item Select \code{strace}
     \end{itemize}
   \end{itemize}
\end{itemize}

Now, build your root filesystem.

Go back to the \code{$HOME/embedded-linux-labs/debugging} directory
and extract the \code{buildroot-2017.08.x/output/images/rootfs.tar}
archive in the \code{nfsroot} directory.

Add this directory to the \code{/etc/exports} file and restart
\code{nfs-kernel-server}.

Boot your ARM board over NFS on this new filesystem, using the same
kernel as before.

\section{Using strace}

Now, go to the \code{$HOME/embedded-linux-labs/debugging} directory.

\code{strace} allows to trace all the system calls made by a process:
opening, reading and writing files, starting other processes,
accessing time, etc. When something goes wrong in your application,
strace is an invaluable tool to see what it actually does, even when
you don't have the source code.


Update the PATH:
\footnotesize
\begin{verbatim}
export PATH=$HOME/embedded-linux-labs/debugging/buildroot-2017.08.x/output/host/bin:$PATH
\end{verbatim}
\normalsize

With your cross-compiling toolchain
compile the \code{data/vista-emulator.c} program, strip it with
\code{arm-linux-strip}, and copy the resulting binary to the
\code{/root} directory of the root filesystem.

Back to target system, try to run the \code{/root/vista-emulator}
program. It should hang indefinitely!

Interrupt this program by hitting \code{[Ctrl] [C]}.

Now, running this program again through the \code{strace} command and
understand why it hangs. You can guess it without reading the source
code!

Now add what the program was waiting for, and now see your program
proceed to another bug, failing with a segmentation fault.

\section{Using ltrace}

Now run the program through \code{ltrace}.

Now you should see what the program does: it tries to consume as much
system memory as it can!

Also run the program through \code{ltrace -c}, to see what function call
statistics this utility can provide.

It's also interesting to run the program again with \code{strace}. You
will see that memory allocations translate into \code{mmap()} system 
calls. That's how you can recognize them when you're using
\code{strace}.

\section{Using gdbserver}

We are now going to use \code{gdbserver} to understand why the program
segfaults.

Compile \code{vista-emulator.c} again with the \code{-g} option to
include debugging symbols. This time, just keep it on your workstation,
as you already have the version without debugging symbols on your target.

Then, on the target side, run \code{vista-emulator} under
\code{gdbserver}. \code{gdbserver} will listen on a TCP port for a
connection from \code{gdb}, and will control the execution of
\code{vista-emulator} according to the \code{gdb} commands:

\begin{verbatim}
gdbserver localhost:2345 vista-emulator
\end{verbatim}

On the host side, run \code{arm-linux-gdb} (also found in your toolchain):
\begin{verbatim}
arm-linux-gdb vista-emulator
\end{verbatim}

You can also start the debugger through the \code{ddd} interface:
\begin{verbatim}
ddd --debugger arm-linux-gdb vista-emulator
\end{verbatim}

\code{gdb} starts and loads the debugging information from the
\code{vista-emulator} binary that has been compiled with \code{-g}.

Then, we need to tell where to find our libraries, since they are not
present in the default \code{/lib} and \code{/usr/lib} directories on
your workstation. This is done by setting the \code{gdb} \code{sysroot}
variable (on one line):

\begin{verbatim}
(gdb) set sysroot /home/<user>/embedded-linux-labs/debugging/
buildroot-2017.08.x/output/staging
\end{verbatim}

And tell \code{gdb} to connect to the remote system:
\begin{verbatim}
(gdb) target remote <target-ip-address>:2345
\end{verbatim}

Then, use \code{gdb} as usual to set breakpoints, look at the source
code, run the application step by step, etc. Graphical versions of
\code{gdb}, such as \code{ddd} can also be used in the same way.
In our case, we'll just start the program and wait for it to hit
the segmentation fault:
\begin{verbatim}
(gdb) continue
\end{verbatim}

You could then ask for a backtrace to see where this happened:
\begin{verbatim}
(gdb) backtrace
\end{verbatim}

This will tell you that the segmentation fault occurred in a function
of the C library, called by our program. This should help you in
finding the bug in our application.

\section{What to remember}

During this lab, we learned that...
\begin{itemize}

\item It's easy to study the behavior of programs and diagnose issues
%  without even having the source code, thanks to strace and ltrace.
  without even having the source code, thanks to strace.

\item You can leave a small \code{gdbserver} program (about 300 KB) on your target
  that allows to debug target applications, using a standard \code{gdb}
  debugger on the development host.

\item It is fine to strip applications and binaries on the target
  machine, as long as the programs and libraries with debugging
  symbols are available on the development host.

\end{itemize}
