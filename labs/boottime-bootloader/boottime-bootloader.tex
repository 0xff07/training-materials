\subchapter{Bootloader optimizations}{Reduce boot time by
using and optimizing Barebox}

\section{Switching to Barebox}

We could have applied the same methodology as in the previous labs, and
started to reduce boot time by measuring all the U-boot steps, by making
it simpler and by tuning its options.

However, there's another bootloader available, called Barebox. It
supports an increasing number of hardware platforms, in particular the
Microchip based ones. One of its strengths is that it can copy the kernel
image with the CPU RAM caches enabled, allowing for faster copy to RAM
and execution from it.

Let's download the sources for Barebox:\footnote{Once again, the
bootloader sources can also be found in the USB disk provided by
the instructor. See also \url{https://bootlin.com/labs/boottime/README.txt}
for details about where we got these sources from.}

\begin{verbatim}
cd /opt/boot-time-labs/barebox
cp /opt/BootTime/downloads/barebox-2013.08.tar.xz .
tar xf barebox-2013.08.tar.xz
\end{verbatim}

We also apply some patches providing configuration and environments files
that reduce the Barebox features to a minimum:

\begin{verbatim}
cd barebox-2013.08
cat ../patches/*.patch | patch -p1
\end{verbatim}

We can now configure and compile Barebox:

\begin{verbatim}
source ../../kernel/env.sh
make sama5d3xek_fast_defconfig
make
\end{verbatim}

You can flash the generated image \code{arch/arm/pbl/zbarebox.bin} to
replace \code{u-boot}. This time you will have to use the
\code{zImage} instead of the \code{uImage} for the kernel. Edit the
\code{sama5d3x_demo_linux_nandflash.tcl} script to do that.

\section{Optimizing the kernel}

Try to remove as much as possible from the kernel configuration to be
faster.

Measure your new boot time.
