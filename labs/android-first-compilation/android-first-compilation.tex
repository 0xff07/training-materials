\subchapter{First compilation}{Get used to the build mechanism}

During this lab, you will:
\begin{itemize}
  \item Configure which system to build Android for
  \item Compile your first Android root filesystem
\end{itemize}

\section{Setup}

Stay in the \code{/home/<user>/felabs/android/aosp/android} directory.

\section{Build environment}

Now that \code{repo sync} is over, we will compile an Android system for the
emulator. This will include building the emulator itself.

First, run \code{source build/envsetup.sh}.

This file contains many useful shell functions and aliases, such as \code{croot} to
go to the root directory of the Android source code or \code{lunch}, to select
the build options.

Check your \code{PATH} environment variable:

\begin{verbatim}
echo $PATH
\end{verbatim}

You will see that  \code{build/envsetup.sh} hasn't modified your \code{PATH}.
This will be done during the build job.

The target product for the emulator is {\it generic}, and we want to have an 
engineering build. To do this, run \code{lunch generic-eng}

\section{Compile the root filesystem}

The build system will use the proper setup to build this target. Before running
\code{make}, first check the number of CPUs cores in your workstation, as seen by
the operating system (hyperthreading could make your OS see 4 cores instead of 2,
for example).

\begin{verbatim}
cat /proc/cpuinfo
\end{verbatim}

You can use \code{make -j} (\code{j} stands for {\it jobs} to instruct \code{make}
to run multiple compile jobs in parallel, taking advantage of the multiple
CPU cores, and making sure that I/Os
and CPU cores are always at 100\%. This will significantly reduce build time.

For example, if Linux sees 4 cores, you will get good performance by specifying  
the double number of parallel jobs: 

\begin{verbatim}
make -j 8
\end{verbatim}

Go grab (several cups of) coffee, and you will have the system compiled in three
images in the \code{out/target/product/generic} folder.

\textbf{Known issue}: in rare circumstances, some builds can fail when \code{make}
is run with multiple jobs in parallel. If this happens, you can run \code{make}
with no option after the failure, and this is likely to make the issue disappear.

\section{Test your Android build}

Now, look at the \code{PATH} environment variable again. It should now contain
extra directories with new binaries produced by the build job.

Run the \code{emulator} command. It will run the emulator with the newly generated images.
Wait a few minutes for the emulator to load the system, and then 
check the build version in the \code{Settings} application in Android.
