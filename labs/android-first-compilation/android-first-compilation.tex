\subchapter{First compilation}{Get used to the build mechanism}

After this lab, you will be able to:
\begin{itemize}
  \item Find your way in the Android source code
  \item Use the tools used by Android
  \item Compile a root filesystem
\end{itemize}

\section{Setup}

Go to the \code{/home/<user>/felabs/android/aosp} directory.

\section{Install needed packages}

Install the needed packages to fetch and build Android:

\begin{verbatim}
sudo apt-get install git-core gnupg flex bison gperf build-essential \
     zip curl zlib1g-dev libc6-dev lib32ncurses5-dev ia32-libs \
     x11proto-core-dev libx11-dev lib32readline5-dev lib32z-dev \
     libgl1-mesa-dev g++-multilib mingw32 tofrodos python-markdown \
     libxml2-utils xsltproc openjdk-6-jdk
sudo apt-get clean
\end{verbatim}

If you have a slow connection to the Internet, installing these packages
could take several tens of minutes. Your instructor will then take advantage
of this waiting time to continue with the lectures. 

\section{Fetch the source code}

Android consists of many separate Git repositories, each containing a piece
of software needed for the whole stack. This organization adds a lot of
flexibility, allowing to easily add or remove a particular piece from the source
tree, but also introduces a lot of overhead to manage all these repos.\\

To address this issue, Google created a tool called Repo. As Repo is just a
python script, it has not made its way in the Ubuntu packages, and we need to
download it from Google.\\

\begin{verbatim}
mkdir $HOME/bin
export PATH=$HOME/bin:$PATH
curl https://dl-ssl.google.com/dl/googlesource/git-repo/repo > $HOME/bin/repo
chmod a+x $HOME/bin/repo
\end{verbatim}

We can now fetch the Android source code.\\

\begin{verbatim}
mkdir android
cd android
repo init -u https://android.googlesource.com/platform/manifest \
     -b android-2.3.7_r1
repo sync
\end{verbatim}

And we are now ready to begin !

\section{Compile a filesystem}

We will compile an Android system for the emulator. First, be sure that the
emulator is in your path. You can check by running the command \code{emulator}
in a terminal.\\

Now, run \code{source build/envsetup.sh}.

It contains many useful shell functions and aliases, such as \code{croot} to
go to the root directory of the Android source code or \code{lunch}, to select
the build options.\\

The target product for the emulator is {\it generic}, and we want to have an 
engineering build. To do this, run \code{lunch generic-eng}.\\

The build system will use the proper setup to build this target, so that we
only have \code{make} to run now.\\

Go grab (several cups of) coffee, and you will have the system compiled in three
images in the folder \code{out/target/product/generic}.

Run the emulator with these newly generated images and check the build version
in the Settings application in Android.
