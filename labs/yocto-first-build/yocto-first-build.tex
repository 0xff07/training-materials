\subchapter{First Yocto build}{Your first dive into a Yocto project and its
	build mechanism}

During this lab, you will:
\begin{itemize}
  \item Setup a Yocto environment
  \item Configure the project and choose a target
  \item Build your first Yocto image
\end{itemize}

\section{Setup}

Before starting this lab, make sure your home directory is not encrypted. Yocto
cannot be used on top of an eCryptFS file system due to limitations on file name
lengths.

Go to the \code{\$HOME/felabs/yocto/} directory.

Install the required packages:
\begin{verbatim}
sudo apt-get install chrpath gawk texinfo
\end{verbatim}

\section{Install the TFTP server}

First we need to setup a TFTP server on the training computer.

Install the \code{tftpd-hdpa} server and create the root TFTP directory:
\begin{verbatim}
sudo apt-get install tftp-hdpa
sudo mkdir -m 777 /tftp
\end{verbatim}

Then make sure the \code{tftp-hdpa} server uses the right directory by checking
the \code{TFTP_DIRECTORY} variable in \code{/etc/default/tftpd-hpa}.

Finally restart the service to make sure all modifications are effective:
\begin{verbatim}
sudo service tftpd-hdpa restart
\end{verbatim}

\section{Download Yocto}

Download the latest stable version of the Yocto project and extract it:
\begin{verbatim}
wget http://downloads.yoctoproject.org/releases/yocto/yocto-1.5.1/\
  poky-dora-10.0.1.tar.bz2
tar xf poky-dora-10.0.1.tar.bz2
\end{verbatim}

Go to the Yocto root directory: \code{cd poky-dora-10.0.1}

Then download the Yocto TI layer:
\begin{verbatim}
git clone git://git.yoctoproject.org/meta-ti.git
\end{verbatim}

\section{Setup the build environment}

Export all needed variables and setup the build directory:
\begin{verbatim}
source oe-init-build-env
\end{verbatim}

In order to choose the target and to configure the generic build settings,
edit the local configuration file (\code{\$BUILDDIR/conf/local.conf}). Set
the target machine to \code{beaglebone} and update the parallelization variables
(\code{BB_NUMBER_THREADS} and \code{PARALLEL_MAKE}) according to your computer
capabilities.

Also, if you need to save disk space on your computer you can add \code{INHERIT
+= "rm_work"} in the previous configuration file. This will remove the
package work directory once a package is built.

Finally, don't forget to let the configuration aware of the TI layer. Edit the
layer configuration file (\code{\$BUILDDIR/conf/bblayers.conf}) and append the
full path to the \code{meta-ti} directory to the \code{BBLAYERS} variable.

\section{Build your first image}

Now that you're ready to start the compilation, simply run:
\begin{verbatim}
bitbake core-image-minimal
\end{verbatim}

Once the build finished, all output images can be find under
\code{\$BUILDDIR/tmp/deploy/images/beaglebone}.

\section{Flash the board}

In order to boot the BeagleBone Black we will flash the bootloader and the
kernel image in the internal eMMC here.

First we need to set the correct environment variables from the U-Boot command
line so that the BeagleBone Black can communicate with the TFTP server:
\begin{verbatim}
setenv ipaddr 192.168.0.2
setenv serverip 192.168.0.1
setenv loadaddr 0x80200000
\end{verbatim}

On the training computer copy the bootloader, kernel and rootfs images in the
root TFTP directory:
\begin{verbatim}
cp $BUILDDIR/tmp/deploy/images/beaglebone/{MLO,u-boot.img,zImage} /tftp
\end{verbatim}

Then retrieve these files on the BeagleBone Board memory. We will first download
the first stage bootloader, the \code{MLO}:
\begin{verbatim}
tftp 0x80200000 MLO
\end{verbatim}

Finally flash this file into the internal eMMC:
\begin{verbatim}
TBD
\end{verbatim}
