\subchapter{Android source code}{Download the source code for Android and all its components}

During this labs, you will:
\begin{itemize}
  \item Install all the development packages needed to fetch and compile Android
  \item Download the \code{repo} utility
  \item Use \code{repo} to download the source code for Android
        and for all its components 
\end{itemize}

\section{Setup}

Go to the \code{$HOME/felabs/android/aosp} directory.

\section{Install needed packages}

Install the packages needed to fetch Android's source code:

\begin{verbatim}
sudo apt-get install git curl
\end{verbatim}

\section{Fetch the source code}

Android sources are made of many separate Git repositories, each
containing a piece of software needed for the whole stack. This
organization adds a lot of flexibility, allowing to easily add or
remove a particular piece from the source tree, but also introduces a
lot of overhead to manage all these repos.

To address this issue, Google created a tool called Repo. As Repo is
just a python script, it has not made its way in the Ubuntu packages,
and we need to download it from Google.

\begin{verbatim}
mkdir $HOME/bin
export PATH=$HOME/bin:$PATH
curl https://dl-ssl.google.com/dl/googlesource/git-repo/repo > \
    $HOME/bin/repo
chmod a+x $HOME/bin/repo
\end{verbatim}

We can now fetch the Android source code:

\begin{verbatim}
mkdir android
cd android
repo init -u https://android.googlesource.com/platform/manifest \
     -b android-2.3.7_r1
repo sync -j2
\end{verbatim}

\code{repo sync} will synchronize the files for all the source
repositories listed in the manifest. If you have a slow connection to
the Internet, this could even take a few hours to
complete. Fortunately, your instructor will take advantage of this
waiting time to continue with the lectures.

We are using the \code{-jX} option of \code{repo sync} to have \code{X}
downloads in parallel. Otherwise, if the source repositories are fetched
sequencially, a lot of time and available network bandwidth is wasted
between downloads, processing the files downloaded by git.

In our case, we are using \code{2} because we all share the same
internet connection, but you can definitely increase this number up to
\code{8} when you are alone.

If this really takes too much time, your instructor may distribute a
ready-made archive containing what \code{repo sync} would have
downloaded.

To save time, do not wait for the \code{repo sync} command to
complete. You can already jump to the next section.

\section{Install packages needed at compile time}

While \code{repo sync} runs, download the packages that you will need
to compile Android and its components from source:

\begin{verbatim}
sudo apt-get install xsltproc gnupg flex bison gperf build-essential \
     zlib1g-dev libc6-dev lib32ncurses5-dev ia32-libs \
     x11proto-core-dev libx11-dev lib32readline5-dev lib32z-dev \
     libgl1-mesa-dev g++-multilib mingw32 tofrodos python-markdown \
     libxml2-utils xsltproc openjdk-6-jdk
sudo apt-get clean
\end{verbatim}

Again, if you have a slow connection to the Internet, installing these
packages could take several tens of minutes. Anyway, we are getting
back to the lectures.
