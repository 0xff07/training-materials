\subchapter{Kernel debugging mechanisms and kernel crash
  analysis}{Objective: Use kernel debugging mechanisms and analyze a
  kernel crash}

In this lab, we will continue to work on the code of our serial driver.

\section{pr\_debug() and dynamic debugging}

Add a \code{pr_debug()} call in the \code{write()} operation that shows
each character being written (or its hexadecimal representation) and
add a similar \code{pr_debug()} call in your interrupt handler to show
each character being received.

Check what happens with your module. Do you see the debugging messages
that you added? Your kernel probably does not have
\code{CONFIG_DYNAMIC_DEBUG} set and your driver is not compiled with
\code{DEBUG} defined., so you shouldn't see any message.

Now, recompile your kernel with \code{CONFIG_DYNAMIC_DEBUG} and reboot. The
dynamic debug feature can be configured using \code{debugfs}, so you'll have
to mount the \code{debugfs} filesystem first. Then, after reading the dynamic
debug documentation in the kernel sources, do the following things:

\begin{itemize}

\item List all available debug messages in the kernel

\item Enable all debugging messages of your serial module, and check
  that you indeed see those messages.

\item Enable just one single debug message in your serial module, and
  check that you see just this message and not the other debug
  messages of your module.

\end{itemize}

Now, you have a good mechanism to keep many debug messages in your
drivers and be able to selectively enable only some of them.

\section{debugfs}

Since you have enabled debugfs to control the dynamic debug feature,
we will also use it to add a new \code{debugfs} entry. Modify your driver to
add:

\begin{itemize}

\item A directory called \code{serial} in the \code{debugfs} filesystem

\item And file called \code{counter} inside the \code{serial}
  directory of the \code{debugfs} filesystem. This file should allow to see
  the contents of the \code{counter} variable of your module.

\end{itemize}

Recompile and reload your driver, and check that in
\code{/sys/kernel/debug/serial/counter} you can see the amount of characters
that have been transmitted by your driver.

\section{Kernel crash analysis}

In this part, we will get back to a standard kernel configuration,
accessing the kernel logs on the serial console.

\subsection{Setup}

Go to the \code{~/felabs/linux/modules/nfsroot/root/debugging/} directory.

Make sure your kernel is built with the following options:

\begin{itemize}

\item The \code{CONFIG_DEBUG_INFO} configuration option,
  (\code{Kernel Hacking} section) which makes it possible to see source
  code in the disassembled kernel.

\item The \code{CONFIG_ARM_UNWIND} configuration option
  (\code{Kernel Hacking} section) disabled. This option enables a
  new mechanism to handle stack backtraces, but this new mechanism is
  not yet as functional and reliable as the old mechanism based on frame
  pointers. In our case, with our board, you get a backtrace only if
  this option is disabled.
\end{itemize}

Compile the \code{drvbroken} module provided in \code{nfsroot/root/debugging}.

On your board, load the \code{drvbroken.ko} module. See it crashing
in a nice way.

\section{Analyzing the crash message}

Analyze the crash message carefully. Knowing that on ARM, the \code{PC}
register contains the location of the instruction being executed, find
in which function does the crash happens, and what the function call
stack is.

Using LXR (for example \url{http://lxr.free-electrons.com}) or the
kernel source code, have a look at the definition of this
function. This, with a careful review of the driver source code should
probably be enough to help you understand and fix the issue.

\section{Further analysis of the problem}

If the function source code is not enough, then you can look at the
disassembled version of the function, either using:

\begin{verbatim}
cd ~/felabs/linux/src/linux/
arm-linux-gnueabi-objdump -S vmlinux > vmlinux.disasm
\end{verbatim}

or, using \code{gdb-multiarch}\footnote{gdb-multiarch is a new package
supporting multiple architectures at once. If you have a cross
toolchain including gdb, you can also run arm-linux-gdb directly.}

\begin{verbatim}
sudo apt-get install gdb-multiarch
gdb-multiarch vmlinux
(gdb) set arch arm
(gdb) set gnutarget elf32-littlearm
(gdb) disassemble function_name
\end{verbatim}

Then find at which exact instruction the crash occurs. The offset is
provided by the crash output, as well as a dump of the code around the
crashing instruction.

Of course, analyzing the disassembled version of the function requires
some assembly skills on the architecture you are working on.
