\subchapter
{Usage of existing {\em autotools} projects}
{Objectives:
  \begin{itemize}
  \item First build of an {\em autotools} package
  \item Out of tree build and cross-compilation
  \item Overriding cache variables
  \item Using {\em autoreconf}
  \end{itemize}
}

\section{Download ethtool}

Go to the {\tt \$HOME/\longname-labs/} directory.

To start our exploration of {\em autotools}, we'll use the {\bf
  ethtool} set of programs. Download the 3.18 version from the project
home page at
\url{https://www.kernel.org/pub/software/network/ethtool/} and extract
it.

\section{Exploration of the sources}

Inside the sources, look at the various files available, and pay
special attention to \code{configure.ac} and \code{Makefile.am}. Even
though we haven't yet discussed their syntax and contents, try to get
some idea about what they could be doing.

Compare their length with the length of their corresponding generated
files \code{configure} and \code{Makefile.in}. Also, check if there is
already a \code{Makefile} or not in the project.

Read the output of \code{./configure --help}.

\section{First build}

Run \code{./configure}, and look at the output. Also look at the
contents of the \code{config.log} file.

Then, run the build with \code{make}.

Finally, try to run the installation with \code{make install}. As you
can see, it fails with {\em Permission denied} messages, because it
tries to install to \code{/usr/local}, which is the default {\em
  prefix}.

Since we don't want to install to \code{/usr/local}, let's configure
{\em ethtool} to install to a different place:

\begin{verbatim}
mkdir $HOME/sys
./configure --prefix=$HOME/sys/
make
make install
\end{verbatim}

This time, in \code{$HOME/sys}, you should have the {\em ethtool}
program an man pages installed.

\section{Out of tree build and cross-compiling}

Now, create a separate build directory, say \code{$HOME/ethtool-arm/},
and move to this directory. Then, call the \code{configure} script of
{\em ethtool}:

\begin{verbatim}
../ethtool-3.18/configure
\end{verbatim}

This will abort with an error: it doesn't want to do {\em out of tree}
build if the source tree has been configured (which we did in the
previous section of this lab). So as suggested, clean up the {\em
  ethtool} source tree by running \code{make distclean}.

To cross-compile {\em ethtool}, we'll first have to install a
cross-compiler:

\begin{verbatim}
apt-get install gcc-arm-linux-gnueabihf
\end{verbatim}

Then you can do:

\begin{verbatim}
../ethtool-3.18/configure --host=arm-linux-gnueabihf
\end{verbatim}

Start the build, and verify that the cross-compiler is used.

\section{Overriding cache variables}

If you look at the {\em configure} output, you can see:

\begin{verbatim}
checking for strtol... yes
\end{verbatim}

The {\em configure} script is checking for the availability of the
\code{strtol()} function. Now let's pretend for some reason that the
detected value is not appropriate (it's obviously not the case for
such a simple test, but we need to take an example!). We want to
override this test by passing the appropriate value as an environment
variable to the \code{configure} script.

Bfore doing this change, look at the
\code{config.h} {\em configuration header} and see the value of the
definition related to \code{strtol}.

Read the \code{config.log}, and identify which variable can be used to
override the result of the \code{strtol} test. Pass it in the
environment of \code{./configure}, and check that {\em configure}
outputs:

\begin{verbatim}
checking for strtol... (cached) no
\end{verbatim}

You can also check in \code{config.h} the new value for definition
related to \code{strtol}.

However, it's interesting to notice that in practice the {\em ethtool}
source code assumes \code{strtol} is present, without taking into
account the \code{HAVE_STRTOL} definition. So if the function was
really absent, there would be a build failure.

\section{Autoreconfiguring}

Now, let's look at another piece of software, which isn't distributed
with pre-generated \code{configure} and \code{Makefile.in} files.

Go back to your \code{$HOME} directory (or another directory you
created for this training session), and use {\em Git} to fetch the
source code of a library called {\em libconfuse}:

\begin{verbatim}
git clone https://github.com/martinh/libconfuse
\end{verbatim}

Note: you may need to install {\em Git}, using \code{apt-get install
  git}.

Look at the {\em libconfuse} source code, and see that you only have
\code{configure.ac} but not the generated \code{configure} script, and
only the \code{Makefile.am} files, and not the generated
\code{Makefile.in} files.

Now, run \code{autoreconf -i}, and look at which files where
generated.

If you get issues when running \code{autoreconf}, it is most likely
because your system doesn't have the {\em autotools} installed. In
this case, run:

\begin{verbatim}
apt-get install autoconf automake libtool autopoint gettext
\end{verbatim}

(Note: \code{autopoint} and \code{gettext} are needed because this
package uses internationalization features provided by {\em gettext},
which we won't cover in this training.)

You can now build and install {\em libconfuse} in
\code{$HOME/sys}. For good measure, don't forget to disable building
the examples: look at \code{./configure --help} to know how to do
that.

When building, you should see a message \code{flex: command not
  found}. Indeed, this tool is not present on our
system. Interestingly, if you look back at the output of the
\code{./configure} script, it did check for \code{flex}, concluded it
wasn't available, but did not error out. As you can see not all
\code{configure} scripts are properly written!

Install {\em flex}:

\begin{verbatim}
apt-get install flex
\end{verbatim}

Re-run \code{./configure}, and restart the build.

