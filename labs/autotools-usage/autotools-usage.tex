\subchapter
{Usage of existing {\em autotools} projects}
{Objectives:
  \begin{itemize}
  \item First build of an {\em autotools} package
  \item Out of tree build and cross-compilation
  \item Overriding cache variables
  \end{itemize}
}

\section{First usage of {\em autotools}}

\subsection{Download ethtool}

Go to the \code{$HOME/autotools-labs/usage/} directory.

To start our exploration of {\em autotools}, we'll use the {\bf
  ethtool} set of programs. Download the 3.18 version from the project
home page at
\url{https://www.kernel.org/pub/software/network/ethtool/} and extract
it.

\subsection{Exploration of the sources}

Inside the sources, look at the various files available, and pay
special attention to \code{configure.ac} and \code{Makefile.am}. Even
though we haven't yet discussed their syntax and contents, try to get
some idea about what they could be doing.

Compare their length with the length of their corresponding generated
files \code{configure} and \code{Makefile.in}. Also, check if there is
already a \code{Makefile} or not in the project.

Read the output of \code{./configure --help}.

\subsection{First build}

Run \code{./configure}, and look at the output. Especially check the
{\em build} and {\em host} systems that are detected. Also look at the
contents of the \code{config.log} file.

Then, run the build with \code{make}.

Finally, try to run the installation with \code{make install}. As you
can see, it fails with {\em Permission denied} messages, because it
tries to install to \code{/usr/local}, which is the default {\em
  prefix}.

Since we don't want to install to \code{/usr/local}, let's configure
{\em ethtool} to install to a different place:

\begin{verbatim}
mkdir $HOME/sys
./configure --prefix=$HOME/sys/
make
make install
\end{verbatim}

This time, in \code{$HOME/sys}, you should have the {\em ethtool}
program an man pages installed.

\subsection{Out of tree build and cross-compiling}

Now, create a separate build directory, say
\code{$HOME/autotools-labs/usage/ethtool-arm/}, and move to this
directory. Then, call the \code{configure} script of {\em ethtool}:

\begin{verbatim}
../ethtool-3.18/configure
\end{verbatim}

If you check the {\em build} and {\em host} system types, you can see
that we're still doing native compilation. To cross-compile {\em
  ethtool}, we'll first have to install a cross-compiler:

\begin{verbatim}
apt-get install gcc-arm-linux-gnueabihf
\end{verbatim}

Then you can do:

\begin{verbatim}
../ethtool-3.18/configure --host=arm-linux-gnueabihf
\end{verbatim}

Check again the detected {\em build} and {\em host} system types. Now
the {\em host} system type should be different!

Start the build, and verify that the cross-compiler is used.

\subsection{Overriding cache variables}

If you look at the {\em configure} output, you can see:

\begin{verbatim}
checking for strtol... yes
\end{verbatim}

The {\em configure} script is checking for the availability of the
\code{strtol()} function. Now let's pretend for some reason that the
detected value is not appropriate (it's obviously not the case for
such a simple test, but we need to take an example!). We want to
override this test by passing the appropriate value as an environment
variable to the \code{configure} script.

Bfore doing this change, look at the
\code{config.h} {\em configuration header} and see the value of the
definition related to \code{strtol}.

Read the \code{config.log}, and identify which variable can be used to
override the result of the \code{strtol} test. Pass it in the
environment of \code{./configure}, and check that {\em configure}
outputs:

\begin{verbatim}
checking for strtol... (cached) no
\end{verbatim}

You can also check in \code{config.h} the new value for definition
related to \code{strtol}.

However, it's interesting to notice that in practice the {\em ethtool}
source code assumes \code{strtol} is present, without taking into
account the \code{HAVE_STRTOL} definition. So if the function was
really absent, there would be a build failure.
