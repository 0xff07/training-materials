\subchapter{Application development}{Objective: Compile and run your
  own ncurses application on the target.}

\section{Setup}

Go to the \code{$HOME/embedded-linux-labs/appdev} directory.

\section{Compile your own application}

We will re-use the system built during the {\em Buildroot lab} and add
to it our own application.

In the lab directory the file \code{app.c} contains a very simple
ncurses application. It is a simple game where you need to reach a
target using the arrow keys of your keyboard.  We will compile and
integrate this simple application to our Linux system.

Buildroot has generated toolchain wrappers in
\code{output/host/usr/bin}, which make it easier to use the toolchain,
since these wrappers pass some mandatory flags (especially the
\code{--sysroot} gcc flag, which tells gcc where to look for the
headers and libraries).

Let's add this directory to our PATH:

\footnotesize
\begin{verbatim}
export PATH=$HOME/embedded-linux-labs/buildroot/buildroot-XXXX.YY/output/host/usr/bin:$PATH
\end{verbatim}
\normalsize

Let's try to compile the application:

\begin{verbatim}
arm-linux-gcc -o app app.c
\end{verbatim}

It complains about undefined references to some symbols. This is
normal, since we didn't tell the compiler to link with the necessary
libraries. So let's use \code{pkg-config} to query the {\em
pkg-config} database about the location of the header files and the
list of libraries needed to build an application against
ncurses:\footnote{Again, \code{output/host/usr/bin} has a special
\code{pkg-config} that automatically knows where to look, so it
already knows the right paths to find \code{.pc} files and their
sysroot.q}

\begin{verbatim}
arm-linux-gcc -o app app.c $(pkg-config --libs --cflags ncurses)
\end{verbatim}

You can see that \code{ncurses} doesn't need anything in particular for the
\code{CFLAGS} but you can have a look at what is needed for
\code{libvorbis} to get a feel of what it can look like:

\begin{verbatim}
pkg-config --libs --cflags vorbis
\end{verbatim}

Our application is now compiled! Copy the generated binary to the NFS
root filesystem (in the \code{root/} directory for example), start
your system, and run your application!

You can also try to run it over ssh if you added ssh support to your
target. Do you notice the difference ?
