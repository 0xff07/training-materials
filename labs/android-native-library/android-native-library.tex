\subchapter{Building a library}{Add a native library to the build system}

After this lab, you will be able to
\begin{itemize}
  \item Add an external library to the Android build system
  \item Compile it dynamically
  \item Add a component to a build
\end{itemize}

\section{Build a shared library}

To get the libusb source code, go to \url{http://libusb.org} and
download version 1.0.9. Extract the archive in the
\code{external/libusb} folder.

For this library, all the needed \code{.c} files are located in the
\code{libusb} folder and its subfolders. The headers are located in
the same folders. You shouldn't modify the \code{libusb} source code.

Remember that you have make functions in the Android build system that
allows you to list all the source files in a directory. You can also
use the \code{filter-out} make macro to remove files that match a
given pattern.

You will find one missing header that you will need to
generate. Indeed, the \code{config.h} generated by autoconf is not
generated at all, because the Android build system ignores other build
systems. You will have to generate it by yourself, by running the
\code{configure} script \footnote{You can have some hints about the
  available options available by passing the \code{--help} argument to
  \code{configure}} that you can find in the \code{libusb} source
code.

If successful, the build system should go through the build process
and you should have a directory generated in the \code{out} directory.

\section{Integrate the library into the Android image}

As you can see, your library has been compiled during the build
process, but if you boot the generated image or look inside the
folder, you can see that the shared object is not present.

Modify the appropriate files so that the library gets included in your
image.
